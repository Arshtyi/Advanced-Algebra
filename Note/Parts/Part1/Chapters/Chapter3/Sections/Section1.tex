\section{数域}
\dfn{数域与数环}{数域与数环}{
    设$\bb{K} \subseteq \bb{C} $且$\bb{K} $中元的个数不少于$2$.如果$\bb{K} $中任意两个元素作加、减、乘、除(除数不为零)得到的结果仍然属于$\bb{K} $,那么$\bb{K} $称为一个数域.

    特别地,如果$\bb{K} $仅对加、减、乘封闭而对除法不封闭,则称$\bb{K} $为一个数环.
}
\exa{}{}{
    回顾
    \[
        \bb{N}
        \subseteq \bb{Z}
        \subseteq \bb{Q}
        \subseteq \bb{R}
        \subseteq \bb{C}
    \]
    这其中,$\bb{N}$不是数域或数环,$\bb{Z} $是整数环,$\bb{Q} $是有理数域,$\bb{R} $是实数域,
    $\bb{C} $是复数域.
}
\lem{最小的数域}{最小的数域}{
    容易证明$\bb{C} $包含无穷多个数域,但任一数域$\bb{K}
        \subseteq \bb{C}$一定满足:
    \begin{enumerate}[label=\arabic*)]
        \item $0 \in \bb{K} ,
                  1 \in \bb{K} $
        \item $\bb{Q}
                  \subseteq \bb{K}$
    \end{enumerate}
    不难得知$\bb{Q}$是最“小”的数域.
}
\rem{}{}{
    一个包含数域(环)的数集不一定是 一个数域或数环.
}
\subsection{超越数与代数数}
\dfn{超越数与代数数}{超越数与代数数}{
    设$\alpha  \in \bb{C}$,如果存在一个多项式
    \[
        f\left(x\right) = a_n x^n +
        \cdots + a_1 x + a_0 \neq 0
    \]
    其中$a_i \neq 0\left(i = 0,\cdots,n\right)$,使得$f\left(\alpha\right) = 0$,则称$\alpha $是一个代数数,否则称$\alpha $是一个超越数.
}
\subsection{杂谈}
\rem{}{}{
    从抽象代数层面说,域(Field)是数域和四则运算的推广,是环的一种.最著名的域就是复数域、实数域、有理数域,此外还有有理函数域、代数函数域、代数数域(与数域有一些不同)等.

    域上定义有加法和乘法,一般地,记作$\oplus$和$\wedge$,最小的伽罗瓦域是$\bb{F}_2=\left\{\overline{0},\overline{1}\right\}$.

    因此,数域只是域的一种,在高等代数中,我们主要讨论数域,即复数域$\bb{C}$的子域.
}