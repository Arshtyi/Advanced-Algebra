\newpage
\section{向量组的秩}
\subsection{向量族与向量组}
\dfn{向量族与向量组}{向量族与向量组}{
    在向量空间$V_{\bbk}$中,向量族是$V$中的若干个向量的集合(可能有无穷多个且允许存在重复的元),向量组是$V$中有限个向量的集合.

    向量族或向量组$S$包含的向量个数一般记作$\#S$或者$\left|S\right|$.
}
\dfn{极大线性无关组}{极大线性无关组}{
    设$S$为$V_{\bbk}$中一个向量族,    若其中存在一组向量$\left\{
        \bm{\alpha}_1,\bm{\alpha}_2,\cdots,
        \bm{\alpha}_r
        \right\}$使得
    \begin{itemize}
        \item $\bm{\alpha}_1,\bm{\alpha}_2,\cdots,\bm{\alpha}_r$线性无关
        \item $\forall \bm{\beta} \in S,
                  \bm{\beta} = k_1\bm{\alpha}_1 +k_2\bm{\alpha}_2+
                  \cdots +k_r\bm{\alpha}_r$
    \end{itemize}
    则称向量组$\left\{
        \bm{\alpha}_1,\bm{\alpha}_2,\cdots,
        \bm{\alpha}_r
        \right\}$为$S$的一个极大(线性)无关组.
}
\cor{极大无关组的存在性}{极大无关组的存在性}{
    包含非零向量的向量组必有极大无关组.
    \begin{proof}
        对$S$包含的向量个数$k$作归纳.

        $k=1$时,设$S=\left\{\bm{\alpha}\right\}$且$
            \bm{\alpha}\neq \bm{0}$,那么$\left\{
            \bm{\alpha}\right\}$就是极大无关组.

        一般地,若$S$中$k$个向量时线性无关的,那么它们即是极大无关组.若是线性相关的,即至少存在一个向量可用其余向量线性表出,不妨设$\bm{\alpha}$可用$S\backslash\left\{\bm{\alpha}\right\}$(含有$k-1$个向量且至少一个非零向量)线性表出.于是$S\backslash \left\{\bm{\alpha}\right\}$存在极大无关组
        $\left\{\bm{\alpha}_1,
            \bm{\alpha}_2,\cdots,\bm{\alpha}_r\right\}$,那么由线性组合的传递性知$\bm{\alpha}$也可以用向量$\bm{\alpha}_1,\bm{\alpha}_2,\cdots,\bm{\alpha}_r$线性表出,于是$\left\{\bm{\alpha}_1,\bm{\alpha}_2,\cdots,\bm{\alpha}_r\right\}$就是$S$的极大无关组.
    \end{proof}
}
\lem{}{多少向量组的线性关系的制约}{
    $A$、$B$两个向量组分别包含$r$、$s$个向量,且$A$中的向量均可由$B$中的向量表示,若$A$中向量线性无关则必有$r\leqslant s$即$\#A \leqslant \#B$.\begin{proof}
        设\begin{align*}
             & A=\left\{\bm{\alpha}_1,\bm{\alpha}_2,\cdots
            ,\bm{\alpha}_r\right\},\#A=r                                            \\
             & B=\left\{\bm{\beta}_1,\bm{\beta}_2,\cdots,\bm{\beta}_s\right\},\#B=s
        \end{align*}
        考虑反证法,设$r>s$.设
        \[
            \bm{\alpha}_1=\lambda_1\bm{\beta}_1+\lambda_2\bm{\beta}_2+\cdots+\lambda_s\bm{\beta}_s
        \]
        因为$A$线性无关则$\bm{\alpha}_1\neq \bm{0}$,于是$\lambda_1,\lambda_2,\cdots,\lambda_s$不全为零.不妨设
        $\lambda_1\neq 0$,则
        \[
            \bm{\beta}_1=\frac{1}{\lambda_1}\bm{\alpha}_1-
            \frac{\lambda_2}{\lambda_1}\bm{\beta}_2-\cdots-
            \frac{\lambda_s}{\lambda_1}\bm{\beta}_s
        \]

        因为$A\backslash \left\{\bm{\alpha}_1\right\}$中的向量
        均可用$B$线性表示,且由上知$B$中的
        $\bm{\beta}_1$可用
        $\left\{\bm{\alpha}_1,\bm{\beta}_2,\cdots,\bm{\beta}_s\right\}$
        线性表示.于是$\forall 1<i\leqslant r$,$\bm{\alpha}_i$是
        $\bm{\alpha}_1,\bm{\beta}_2,\cdots,\bm{\beta}_s$的线性组合.

        考虑归纳法,设$\forall k < i \leqslant r$,
        $\bm{\alpha}_i$是
        $\left\{\bm{\alpha}_1,
            \bm{\alpha}_2,\cdots,\bm{\alpha}_k,
            \bm{\beta}_{k+1},\cdots,\bm{\beta}_s\right\}$的线性组合.
        不妨取
        \[
            \bm{\alpha}_{k+1}=\mu_{1}\bm{\alpha}_1+\mu_2\bm{\alpha}_2+\cdots+\mu_k\bm{\alpha}_k+
            \mu_{k+1}\bm{\beta}_{k+1}+\cdots+\mu_s\bm{\beta}_s
        \]
        若$\mu_{k+1}=\cdots=\mu_{s}=0$则
        $\bm{\alpha}_{k+1}$是
        $\bm{\alpha}_1,\bm{\alpha}_2,
            \cdots,\bm{\alpha}_k$的线性组合,与$A$线性无关
        矛盾,于是
        $\mu_{k+1},\cdots,\mu_s$不全为零.不妨设$\mu_{k+1}\neq 0$,则
        \begin{align*}
            \bm{\beta}_{k+1} & =
            -\frac{\mu_1}{\mu_{k+1}}\bm{\alpha}_1-
            \frac{\mu_2}{\mu_{k+1}}\bm{\alpha}_2-\cdots
            -\frac{\mu_k}{\mu_{k+1}}\bm{\alpha}_k+
            \frac{1}{\mu_{k+1}}\bm{\alpha}_{k+1}-
            \frac{\mu_{k+2}}{\mu_{k+1}}\bm{\beta}_{k+2}-
            \cdots-\frac{\mu_{s}}{\mu_{k+1}}\bm{\beta}_s
        \end{align*}
        那么
        \begin{align*}
            \left\{\bm{\alpha}_{k+2},\cdots,\bm{\alpha}_r\right\}
             & \hookrightarrow
            \left\{\bm{\alpha}_1,
            \bm{\alpha}_2,\cdots,\bm{\alpha}_k,
            \bm{\beta}_{k+1},\cdots,\bm{\beta}_s\right\} \\
             & \hookrightarrow
            \left\{\bm{\alpha}_1,
            \bm{\alpha}_2,\cdots,\bm{\alpha}_k,\bm{\alpha}_{k+1},
            \bm{\beta}_{k+2},\cdots,\bm{\beta}_s\right\}
        \end{align*}
        于是$\forall s < i \leqslant r$,$\bm{\alpha}_i$是$\left\{\bm{\alpha}_1,\bm{\alpha}_2,\cdots,\bm{\alpha}_s\right\}$
        的线性组合,与$A$线性无关矛盾.于是证毕.
    \end{proof}
}
\clm{}{}{
    \cref{lem:多少向量组的线性关系的制约}的逆否命题即若多的向量组可以用少的向量组线性表示,则多的向量组一定线性相关.
}
\lem{}{相互表出的线性无关向量组的向量个数关系}{
    设$A$、$B$为线性无关的向量组,且二者包含的向量可以互相表示,那么必有二者所含向量个数相同即$\#A=\#B$.
    \begin{proof}
        由\cref{lem:多少向量组的线性关系的制约}立得.
    \end{proof}
}
\cor{极大无关组的向量个数唯一}{极大无关组的向量个数唯一}{
    一个向量族极大线性无关组中向量的选取不一定是唯一的,但其中的向量的个数一定是唯一的.\begin{proof}
        由\cref{lem:多少向量组的线性关系的制约}和\cref{lem:相互表出的线性无关向量组的向量个数关系}立得.
    \end{proof}
}
\subsection{向量组的秩}
\dfn{向量组的秩}{向量组的秩}{
    向量族$S$的极大无关组包含的向量个数称为$S$的秩,记作$\rmr\left(S\right)$
    或者$\rank\left(S\right)$.
}
\clm{}{}{
    约定:\begin{itemize}
        \item 零向量组的秩为$0$
        \item 等价(可以互相表出)的向量组的秩相同,称为同秩
    \end{itemize}
}
\subsection{基}
\rem{}{}{
    如果$S = V_{\mathbb{K}}$,那么
    \[
        \begin{cases*}
            \text{极大无关组——线性空间的基}
            \\
            \text{极大无关组的秩——线性空间的维数}
        \end{cases*}
    \]
}
\dfn{基}{基}{
    设$V$是$\mathbb{K}$上的线性空间,若其中存在线性无关的一组向量$\left\{\bm{e}_1,\bm{e}_2
        ,\cdots,\bm{e}_n\right\}$
    使得$V$中任一向量均可表示为这组向量的线性组合,则称$\left\{
        \bm{e}_1,\bm{e}_2,\cdots
        ,\bm{e}_n
        \right\}$为$V$的一组基,$V$
    称为$n$维线性空间(具有维数$n$),记作
    \[
        \dim_{\mathbb{K}}V = n
    \]
    这组基记作
    \[
        \left\{
        \bm{e}_1,\cdots
        ,\bm{e}_n
        \right\}=
        V /
        \mathbb{K}
    \]
    若不存在这样的有限个向量组成的一组基,那么称$V$是无限维线性空间.
}
\clm{}{}{
    需要说明的是,虽然我们主要研究有限维线性空间,但无限维线性空间仍然具有基,只是需要修改线性表示和线性无关的定义并使用选择公理或Zorn引理.
}
\exa{}{}{
    \cref{def:n维行(列)向量空间}中的
    \[
        \dim_{\bbk}\bbk^n
        =\dim_{\bbk}\bbk_n
        = n
    \]
}
\exa{}{}{
    \[
        \left\{1,\mathrm{i} = \sqrt{-1}\right\} =
        \mathbb{C} /
        \mathbb{R}
        \Longrightarrow
        \dim_{\mathbb{R}}
        \mathbb{C} = 2
    \]
}
\cor{}{超过维数的向量组}{
    $n$维空间$V_{\mathbb{K}}$中任一包含向量个数多于$n$的向量组一定线性相关.
}
\thm{基的判定定理}{基的判定定理}{
    设$\bm{e}_1,\bm{e}_2,\cdots,\bm{e}_n $是$\mathbb{K}$上的$n$维线性空间$V$中的$n$个向量,若适合下面之一则$\left\{
        \bm{e}_1,\bm{e}_2,\cdots,\bm{e}_n
        \right\}$是$V$的一组基
    \begin{enumerate}[label=\arabic*)]
        \item $\bm{e}_1,\bm{e}_2,\cdots,\bm{e}_n$线性无关
        \item $V$中任一向量均可表示为这组向量
              $\left\{
                  \bm{e}_1,\bm{e}_2,\cdots,\bm{e}_n
                  \right\}$的线性组合
    \end{enumerate}
    \begin{proof}
        \begin{enumerate}[label=\arabic*)]
            \item $\forall \bm{\alpha}\in V$,$\bm{e}_1,\bm{e}_2,\cdots,\bm{e}_n,\bm{\alpha}$线性相关,故$\bm{\alpha}$是$\bm{e}_1,\bm{e}_2,\cdots,\bm{e}_n$的线性组合,证毕.
            \item 设$\left\{\bm{\alpha}_1,\bm{\alpha}_2,\cdots
                      ,\bm{\alpha}_r\right\}$为$\left\{
                      \bm{e}_1,\bm{e}_2,\cdots,\bm{e}_n
                      \right\}$的极大无关组.即
                  \[
                      V\hookrightarrow \left\{\bm{e}_1,\bm{e}_2,\cdots
                      ,\bm{e}_n\right\}\hookrightarrow
                      \left\{\bm{\alpha}_1,\bm{\alpha}_2,\cdots
                      ,\bm{\alpha}_r\right\}
                  \]
                  于是$\left\{\bm{\alpha}_1,\bm{\alpha}_2,\cdots,\bm{\alpha}_r\right\}$是$V$的一组基,$\dim V=r=n$.证毕.\qedhere
        \end{enumerate}
    \end{proof}
}
\thm{基扩张定理}{基扩张定理}{
设线性空间$V^n_{\mathbb{K}}$中的$m\left(m < n\right)$个线性无关的向量为$\bm{v}_1,\bm{v}_2
    ,\cdots,\bm{v}_m$,$\left\{
    \bm{e}_1,\bm{e}_2,\cdots
    ,\bm{e}_n
    \right\}$是该空间的一组基,则必定可以从这组基再选出$n - m$个向量来共同组成一组基
\[
    \left\{\bm{v}_1,\bm{v}_2,
    \cdots,
    \bm{v}_m,\bm{e}_{i_1},\bm{e}_{i_2},
    \cdots,\bm{e}_{i_{n - m}}\right\}
\]\begin{proof}
    首先证明$\exists 1\leqslant i \leqslant n $使得
    $\bm{v}_1,\bm{v}_2,\cdots,
        \bm{v}_m,\bm{e}_i$线性无关.

    考虑反证法,设$\forall 1
        \leqslant i \leqslant n $,
    $\bm{v}_1,\bm{v}_2,\cdots,
        \bm{v}_m,\bm{e}_i$线性相关,于是$\bm{e}_i$是
    $\bm{v}_1,\bm{v}_2,\cdots,
        \bm{v}_m$的线性组合.即
    \[
        \left\{\bm{e}_1,\bm{e}_2,\cdots,\bm{e}_n\right\}
        \hookrightarrow \left\{
        \bm{v}_1,\bm{v}_2,\cdots,\bm{v}_m
        \right\}
    \]
    且$\left\{\bm{e}_1,\bm{e}_2,\cdots,\bm{e}_n\right\}
    $是线性无关的,于是$n \leqslant m$,矛盾.得证.

    不妨设$\bm{v}_1,\bm{v}_2,\cdots,\bm{v}_m,\bm{e}_1$是线性无关的.
    \begin{enumerate}[label=\arabic*)]
        \item $m+1=n$时,这成为一组基

        \item $m+1<n$时,仿照上面不断做下去即可\qedhere
    \end{enumerate}
\end{proof}
}
\thm{基扩张定理}{基扩张定理的等价表述}{
    \begin{enumerate}[label=\arabic*)]
        \item $n$维线性空间$V_{\mathbb{K}}$中任意$m\left(m < n\right)$个线性无关的向量均可扩张为$V$的一组基
        \item $V$的任一子空间的基均可扩张为$V$的一组基
    \end{enumerate}
}