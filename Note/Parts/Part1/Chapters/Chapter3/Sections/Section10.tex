\newpage
\section{线性方程组的解}
\subsection{判定}
\thm{线性方程组解的判定定理}{线性方程组解的判定定理}{
    \[\bm{A}\bm{x} = \bm{\beta}
        \Longleftrightarrow
        \begin{cases*}
            a_{11}x_1+a_{12}x_2+\cdots+a_{1n}x_n = b_1 \\
            a_{21}x_1+a_{22}x_2+\cdots+a_{2n}x_n=b_2   \\
            \qquad\cdots\cdots\cdots\cdots\cdots
            \cdots\cdots                               \\
            a_{m1}x_1+a_{m2}x_2+\cdots+a_{mn}x_n=b_m
        \end{cases*}
    \]
    其增广矩阵
    \[
        \widetilde{\bm{A}}
        =\begin{pmatrix}
            \bm{A} & \bm{\beta}
        \end{pmatrix}
        =  \begin{pmatrix}
            a_{11} & a_{12} & \cdots & a_{1n} & b_1    \\
            a_{21} & a_{22} & \cdots & a_{2n} & b_2    \\
            \vdots & \vdots &        & \vdots & \vdots \\
            a_{m1} & a_{m2} & \cdots & a_{mn} & b_m
        \end{pmatrix}
    \]

    \begin{enumerate}[label=\arabic*)]
        \item $ \rmr \left(\bm{A}\right)
                  =\rmr \left(\widetilde{\bm{A}}
                  \right) = n\Longrightarrow
              $有唯一一组解
        \item $\rmr \left(\bm{A}\right)
                  =\rmr \left(\widetilde{\bm{A}}
                  \right) < n \Longrightarrow $
              有无穷多组解
        \item $\rmr \left(\bm{A}\right)
                  \neq \rmr \left(
                  \widetilde{\bm{A}}
                  \right)$(事实上,可以断言$\rmr \left(
                  \widetilde{\bm{A}
                  }\right) =
                  \rmr \left(\bm{A}\right) + 1$)$\Longrightarrow $无解
    \end{enumerate}\begin{proof}
        先证有解当且仅当
        $ \rmr \left(\bm{A}\right)
            =\rmr \left(\widetilde{\bm{A}}
            \right)$.作列分块$\bm{A}=\left(\bm{\alpha}_1,
            \bm{\alpha}_2,\cdots,\bm{\alpha}_n
            \right)$,于是方程化为
        \[
            x_1\bm{\alpha}_1+x_2\bm{\alpha}_2+\cdots+x_n\bm{\alpha}_n
            =\bm{\beta}
        \]
        因此方程有解等价于$\bm{\beta}$是$\bm{\alpha}_1,\bm{\alpha}_2,\cdots,\bm{\alpha}_n$的线性组合.

        先证必要性,设方程有解,则
        $\bm{\beta}$是$\bm{\alpha}_1,
            \bm{\alpha}_2,\cdots,\bm{\alpha}_n$的线性组合.设$\bm{A}$的列向量的极大无关组为
        $\left\{\bm{\alpha}_{i_1},\bm{\alpha}_{i_2}
            ,\cdots,\bm{\alpha}_{i_r}\right\}$,即$\rmr \left(\bm{A}\right)=r$.显然
        $\bm{\beta}$是$\bm{\alpha}_{i_1},
            \bm{\alpha}_{i_2},\cdots,\bm{\alpha}_{i_r}
        $的线性组合.于是
        $\left\{\bm{\alpha}_{i_1},
            \bm{\alpha}_{i_2},\cdots,\bm{\alpha}_{i_r}
            \right\}$也是$\widetilde{\bm{A}}$的
        列向量的极大无关组.于是$\rmr \left(\widetilde{\bm{A}}\right)=r=\rmr \left(\bm{A}\right).$

        再证充分性,设$\rmr \left(\widetilde{\bm{A}}\right)=
            \rmr \left(\bm{A}\right)=r$,并且
        $\left\{\bm{\alpha}_{i_1},\bm{\alpha}_{i_2},
            \cdots,\bm{\alpha}_{i_r}\right\}$是$\bm{A}$的列向量的极大无关组,即也是$\widetilde{\bm{A}}$的$
            r$个线性无关的向量,即是$\widetilde{\bm{A}}$的列向量
        的极大无关组.于是
        $\bm{\beta}$是$\bm{\alpha}_{i_1},
            \bm{\alpha}_{i_2},\cdots,\bm{\alpha}_{i_r}
        $的线性组合.

        有解条件证毕.

        若$\rmr \left(\bm{A}\right)=\rmr \left(
            \widetilde{\bm{A}}
            \right)=n$,则$\left(\bm{\alpha}_1,
            \bm{\alpha}_2,\cdots,\bm{\alpha}_n\right)$线
        性无关.又因为线性表示唯一,因此$\left(x_1,x_2,\cdots,x_n\right)$唯一确定.

        若$\rmr \left(\bm{A}\right)=\rmr \left(
            \widetilde{\bm{A}}
            \right)<n$,则$\left(\bm{\alpha}_1,
            \bm{\alpha}_2,\cdots,\bm{\alpha}_n\right)$线性相关即存在不全为零的$c_1,c_2,\cdots,c_n\in \bbk $使得
        \begin{align*}
            \bm{0}=c_1\bm{\alpha}_1
            +c_2\bm{\alpha}_2+\cdots+
            c_n\bm{\alpha}_n
        \end{align*}
        又因为方程有解等价于存在$k_1,k_2,\cdots,k_n\in \bbk $使得$
            \bm{\beta}=k_1\bm{\alpha}_1+k_2\bm{\alpha}_2+\cdots+k_n\bm{\alpha}_n
        $, 两式组合得
        \[
            \bm{\beta}=
            \left(k_1+kc_1\right)\bm{\alpha}_1+\left(
            k_2+kc_2
            \right)\bm{\alpha}_2+\cdots+\left(k_n+kc_n\right)
            \bm{\alpha}_n
        \]
        即$  x_i=k_i+kc_i,\forall k\in \bbk $,即有无穷多组解.
    \end{proof}
}
\subsection{解的结构与表达}
当线性方程组具有无穷多组解时,一个重要的问题就是如何用有限多组解来表达或者说生成这无穷多组解.
\lem{相伴方程组的解}{相伴方程组的解}{
    设$\bm{\gamma}$是$\bm{Ax}=\bm{\beta}$的一个特解,则$\bm{\alpha}$是$\bm{Ax}=\bm{\beta}$的解等价于$\bm{\alpha}-\bm{\gamma}$为相伴的
    齐次线性方程组$\bm{Ax}=\bm{0}$的解.
}
\lem{解空间}{解空间}{
    考虑相伴的齐次线性方程组$\bm{Ax}=\bm{0}$,其至少有零解.令
    \[
        V_{\bm{A}}=\left\{\bm{x}\in \bbk ^n\big|\bm{Ax}=\bm{0}\right\}
    \]
    即方程的解集,那么断言这是$\bbk ^n$的子空间,称为该齐次线性方程组的解空间.
}
\thm{基础解系}{基础解系}{
设$\rmr \left(\bm{A}
    \right)=r$,则解空间
$V_{\bm{A}} =
    \left\{
    \bm{x} \in \bbk ^n\mid
    \bm{Ax} =\bm{0}
    \right\}$是
$\bbk ^n$的一个$n  -r$维
子空间即
\[
    \dim_{\bbk }V_{\bm{A}} +
    \rmr \left(\bm{A}\right)
    = n
\]
从而存在一组基$\left\{\bm{\eta}_1,\bm{\eta}_2,\cdots
    ,\bm{\eta}_{n - r}\right\}$使得$\bm{Ax}=\bm{0}$的解都是$\bm{\eta}_1,\bm{\eta}_2,\cdots,
    \bm{\eta}_{n - r}$的线性组合,这组基$\left\{\bm{\eta}_1,\bm{\eta}_2,\cdots
    ,\bm{\eta}_{n - r}\right\}$(齐次线性方程组解空间的基)称为基础解系.\begin{proof}
    首先考虑高斯消去法与初等行变换,我们知道,对线性方程组进行同解基础上的化简等价于对系数矩阵$\bm{A}$做初等行变换.此外,列对换仅仅改变了未定元的位置,因此是允许的.

    由行变换,可将$\bm{A}$的行向量的极大无关组换至
    前$r$行.不妨设
    \[
        \bm{A}=\begin{pmatrix}
            \bm{\alpha}_1     \\
            \bm{\alpha}_2     \\
            \vdots            \\
            \bm{\alpha}_r     \\
            \bm{\alpha}_{r+1} \\
            \vdots            \\
            \bm{\alpha}_m
        \end{pmatrix}
    \]
    其中$\left\{\bm{\alpha}_1,
        \bm{\alpha}_2,
        \cdots,\bm{\alpha}_r\right\}$为行向量的极大无关组.于是有
    \[
        \bm{A}\longrightarrow
        \begin{pmatrix}
            \bm{\alpha}_1 \\
            \bm{\alpha}_2 \\
            \vdots        \\
            \bm{\alpha}_r \\
            \bm{0}        \\
            \vdots        \\
            \bm{0}
        \end{pmatrix}
    \]
    令\[
        \bm{A}_1=\begin{pmatrix}
            \bm{\alpha}_1 \\\bm{\alpha}_2\\\vdots\\\bm{\alpha}_r
        \end{pmatrix}\Longrightarrow
        \rmr \left(\bm{A}_1\right)=r
    \]
    在允许列对换(交换变量位置)的情况下,不妨设$\bm{A}_1$的列向量的极大无关组为前$r$列,则
    $\bm{A}_1=\left(\bm{B}_1^{r\times r}
        ,\bm{B}_2^{r\times \left(n-r\right)}\right)$,
    则$\rmr \left(\bm{B}_1\right)=r.$
    故$\bm{B}_1$非异,其可通过行变换成为单位阵,此时
    $\bm{A}\longrightarrow \left(\bm{I}_r,
        \bm{C}\right)$.

    总之,在初等行变换和列对换后$\bm{A}$化为
    \[
        \begin{bmatrix}
            \bm{I}_r & \bm{C} \\\bm{O} & \bm{O}
        \end{bmatrix}
        ,\bm{C}=\left(c_{ij}\right)_{r\times \left(n-r\right)}
    \]
    从而所给齐次线性方程组与下面的方程同解
    \[
        \begin{cases*}
            x_1+\qquad \qquad +c_{1,r+1}x_{r+1}+\cdots+c_{1,n}x_n=0      \\
            \qquad x_2+\qquad +c_{2,r+1}x_{r+1}+\cdots+c_{2,n}x_n=0      \\
            \qquad\qquad\cdots\cdots\cdots\cdots\cdots\cdots\cdots\cdots \\
            \qquad\qquad \quad x_r+c_{r,r+1}x_{r+1}+\cdots+c_{r,n}x_n=0
        \end{cases*}
    \]
    取$x_{r+1}=1$,$x_{r+2}=\cdots=x_n=0$得一个解
    \[
        \bm{\eta}_1=\begin{pmatrix}
            -c_{1,r+1} \\
            -c_{2,r+1} \\
            \vdots     \\
            -c_{r,r+1} \\
            1          \\
            0          \\
            \vdots     \\
            0
        \end{pmatrix}
    \]
    随后令
    $x_{r+2}=1$,其他为零,如此下去$\cdots$,得到
    $\bm{\eta}_1,\bm{\eta}_2,\cdots,\bm{\eta}_{n-r}
    $,并断言这就是该方程解空间的一组基.

    一方面,它们显然线性无关;另一方面,任取上述方程一解
    $\bm{\eta}=\left(a_1,a_2,\cdots,a_n\right)
        '$,那么
    \[
        \bm{\eta}=\begin{pmatrix}
            a_1 \\a_2\\\vdots\\a_n
        \end{pmatrix}
        =\begin{pmatrix}
            -c_{1,r+1}a_{r+1}-\cdots-c_{1,n}a_n \\
            -c_{2,r+1}a_{r+1}-\cdots-c_{2,n}a_n \\
            \vdots                              \\
            -c_{r,r+1}a_{r+1}-\cdots-c_{r,n}a_n \\
            a_{r+1}                             \\
            \vdots                              \\
            a_n
        \end{pmatrix}=a_{r+1}\bm{\eta}_1+\cdots+a_n\bm{\eta}
        _{n-r}
    \]
    线性组合亦成立.故这是的该方程解空间的一组基.同时也是齐次线性方程组的解空间$V_{\bm{A}}$的一组基.

    于是
    \[
        \dim V_{\bm{A}}=n-r=n-\rmr \left(\bm{A}
        \right)
        \qedhere
    \]
\end{proof}
}
\thm{结构定理}{结构定理}{
设$\rmr \left(\bm{A}\right)
    =\rmr \left(\widetilde{\bm{A}
    }\right) = r$,而$\bm{\gamma}$是$\bm{Ax}=\bm{\beta}$
的一个特解,$\left\{\bm{\eta}_1,\bm{\eta}_2,\cdots
    ,\bm{\eta}_{n - r}\right\}$是相伴齐次线性方程组$\bm{Ax}=\bm{0}$的一个基础解系,则$
    \bm{Ax}=\bm{\beta}$的通解为
\[
    \bm{\gamma}+
    k_1\bm{\eta}_1+k_2\bm{\eta}_2+\cdots
    +k_{n - r}\bm{\eta}_{n - r}
\]
其中$k_i \in \bbk $.\begin{proof}
    考虑$\bm{\alpha}-\bm{\gamma}$用基础解系线性组合立得.
\end{proof}
}
\rem{}{}{
    \cref{thm:结构定理} 强调数域$\bbk $求解.

    求解过程中,仅允许初等行变换和第三类初等列变换即列对换,其它变换不允许使用.
}
\subsection{求解\texorpdfstring{ $\bm{Ax}=\bm{\beta}$}{Ax = beta}}
\thm{求解$\bm{Ax}=\bm{\beta}$的方法}{求解Ax=beta的方法}{
一般地,求解方程
\[
    \begin{cases*}
        a_{11}x_1+a_{12}x_2+\cdots+a_{1n}x_n = b_1 \\
        a_{21}x_1+a_{22}x_2+\cdots+a_{2n}x_n=b_2   \\
        \qquad\cdots\cdots\cdots\cdots\cdots
        \cdots\cdots                               \\
        a_{m1}x_1+a_{m2}x_2+\cdots+a_{mn}x_n=b_m
    \end{cases*}\Longleftrightarrow
    \bm{Ax}=\bm{\beta}
\]
的方法如下:

$1.$施加行变换将$\widetilde{\bm{A}}
    =\begin{pmatrix}
        \bm{A} & \bm{\beta}
    \end{pmatrix}$化为阶梯形,判断$\rmr \left(\bm{A}\right)$与$\rmr \left(\widetilde{
        \bm{A}
    }\right)$的关系来确定解的存在性.

$2.$继续对$\widetilde{\bm{A}}$作初等行变换和列对换(不允许采用其他变换,不允许在列对换时操作常数列,列对换仅仅变换了未定元的次序,后面需要换回来)化为如下解方程组的标准型形式
\[
    \bm{D} =            \begin{pmatrix}
        \bm{I}_r                   & \bm{C}_{r
        \times \left(n - r\right)} & \bm{\gamma}          \\
        \bm{O}                     & \bm{O}      & \bm{O}
    \end{pmatrix}
\]
得到特解
\[
    \begin{bmatrix}
        \bm{\gamma} \\
        \bm{O}
    \end{bmatrix}
\]和基础解系
$\bm{\eta}_1,\bm{\eta}_2,\cdots,
    \bm{\eta}_{n - r}$,其中
\[
    \bm{\eta}_i
    =\begin{pmatrix}
        -c_{1,r+i} \\
        -c_{2,r+i} \\
        \vdots     \\
        -c_{r,r+i} \\
        0          \\
        \vdots     \\
        1          \\
        \vdots     \\
        0
    \end{pmatrix}
\]
其中$c_{i,r+i}$是$\bm{C}_{
        r\times\left(n - r\right)
    }$的第$\left(i,i\right)$元或者说$\bm{D}$的$\left(i,r+i\right)$元;元$1$位于$-c_{r,r+i}$下第$i$个位置或者说是总的第$r + i$个位置.

$3.$根据列对换的情况调整特解$\bm{\gamma}$和基础解系$\left\{\bm{\eta}_1,\bm{\eta}_2,\cdots,\bm{\eta}_{n - r}\right\}$的分量得到正确的特解$\bm{\delta }$和基础解系$\left\{\bm{\xi}_1,\bm{\xi}_2,\cdots,\bm{\xi}_{n - r}\right\}$,最后组合为通解
\[
    \bm{\delta}+k_1\bm{\xi}_1+k_2\bm{\xi}_2
    +\cdots+k_{n - r}\bm{\xi}_{n - r}
\]
其中$k_i \in \bbk ,\forall 1\leqslant i\leqslant n-r.$
}
\pro{}{对解的一些刻画}{
    设$\bm{Ax} = \bm{\beta}\left(
        \bm{\beta} \neq \bm{0}
        \right)$的特解为$\bm{\gamma}$,$\bm{Ax} = \bm{0}$的基础解系为$\left\{\bm{\eta}_1,\bm{\eta}_2,\cdots
        ,\bm{\eta}_{n - r}\right\}$,则\begin{enumerate}[label=\arabic*)]
        \item $\bm{\gamma},\bm{\gamma}+
                  \bm{\eta}_1,\bm{\gamma}+\bm{\eta}_2,\cdots,
                  \bm{\gamma}+\bm{\eta}_{n - r}$线性无关
        \item $\bm{Ax}=\bm{\beta}$的解必符合如下形式:
              \[c_0\bm{\gamma}+c_1\left(
                  \bm{\gamma}+\bm{\eta}_1
                  \right)+c_2\left(
                  \bm{\gamma}+\bm{\eta}_2
                  \right)+\cdots+
                  c_{n - r}\left(\bm{\gamma}+
                  \bm{\eta}_{n - r}\right)
              \]并且有$c_0+c_1+c_2+\cdots+c_{n - r}=1$.反之,这样形式的向量一定是方程的解.
    \end{enumerate}\begin{proof}
        \begin{enumerate}[label=\arabic*)]
            \item 设
                  \[
                      \lambda_0\bm{\gamma}
                      +\lambda_1\left(
                      \bm{\gamma}+\bm{\eta}_1
                      \right)+\cdots+
                      \lambda_{n - r}\left(\bm{\gamma}+
                      \bm{\eta}_{n - r}\right)=\bm{0}
                  \]
                  即
                  \[
                      \left(\sum_{i=0}^{n-r}
                      \lambda_i\right)\bm{\gamma}+\lambda_1\bm{\eta}_1+\cdots
                      +\lambda_{n-r}\bm{\eta}_{n-r}=
                      \bm{0}
                  \]
                  用$\bm{A}$左乘,即得
                  \[
                      \left(\sum_{i=0}^{n-r}\lambda_{i}\right)\bm{\beta}=\bm{0}
                  \]
                  故$\displaystyle
                      \sum_{i=0}^{n-r}\lambda_i=0$,反代即得
                  \[
                      \lambda_1\bm{\eta}_1+\cdots
                      +\lambda_{n-r}\bm{\eta}_{n-r}=
                      \bm{0}
                  \]
                  又因为线性无关,故$\lambda_1=\cdots=\lambda_{n-r}=0$,易推知$\lambda_0=0$.
            \item 由结构定理,任一解$\bm{\alpha}=\bm{\gamma}+k_1\bm{\eta}_1+\cdots
                      +k_{n - r}\bm{\eta}_{n - r},k_i\in \bbk .$
                  显然有
                  \[
                      \bm{\alpha}=\left(1-k_1-k_2-\cdots-k_{n-r}\right)
                      \bm{\gamma}+k_1\left(\bm{\gamma}+\bm{\eta}_1\right)
                      +k_2\left(\bm{\gamma}+\bm{\eta}_2\right)+
                      \cdots+k_{n-r}\left(\bm{\gamma}+\bm{\eta}_{n-r}\right)
                      \qedhere
                  \]
        \end{enumerate}
    \end{proof}
}
\subsection{求解线性方程组的应用}
\pro{}{求解线性方程组的应用}{
核心是
\[
    \dim_{\bbk }V_{\bm{A}} +
    \rmr \left(\bm{A}\right)=n
\]
其中$n$为未定元个数.主要可用于:\begin{enumerate}[label=\arabic*)]
    \item $n$阶方阵$\bm{A}$可逆等价于$\bm{Ax}=\bm{0}$仅有零解.
    \item 利用$\rmr \left(\bm{A}\right)$求$V_{\bm{A}}$.
    \item 利用$V_{\bm{A}}$求$\rmr \left(\bm{A}\right)$.
\end{enumerate}
}
\exa{}{}{
    已知$\bm{A}^2-\bm{A}-3\bm{I}_n=\bm{O}$,求证:$\bm{A}-2\bm{I}_n$非异.\begin{proof}
        只要证方程
        \[
            \left(\bm{A}-2\bm{I}_n\right)\bm{x}=\bm{0}
        \]
        仅有零解.设一解$\bm{x}_0$,则$\bm{Ax}_0=2\bm{x}_0$,于是$\bm{A}^2\bm{x}_0=2\bm{Ax}_0=4\bm{x}_0$,故$\bm{x}_0=\bm{0}$.
    \end{proof}
}
\exa{}{}{
    设$\lambda_1,\lambda_2,\cdots,\lambda_n$是数域$\bbk $中不同的数且有$1\leqslant k \leqslant n-1$.设
    \begin{flalign}
        \tag{$1$}
                                                                             & \begin{cases*}
                                                                                   x_1+x_2+\cdots+x_n=0                            \\
                                                                                   \lambda_1x_1+\lambda_2x_2+\cdots+\lambda_nx_n=0 \\
                                                                                   \qquad\cdots\cdots\cdots\cdots\cdots            \\
                                                                                   \lambda_{1}^{k-1}x_1+\lambda_2^{k-1}x_2+\cdots+\lambda_n^{k-1}x_n=0
                                                                               \end{cases*} \\
        \tag{$2$}                                                            &
        \begin{cases*}
            x_1^{k}+x_2^k+\cdots+x^k_n=0                                      \\
            \lambda_1^{k+1}x_1+\lambda_2^{k+1}x_2+\cdots+\lambda_n^{k+1}x_n=0 \\
            \qquad\cdots\cdots\cdots\cdots\cdots                              \\
            \lambda_{1}^{n-1}x_1+\lambda_2^{n-1}x_2+\cdots+\lambda_n^{n-1}x_n=0
        \end{cases*} &
    \end{flalign}
    设解空间分别为$V_1$、$V_2$,求证:$\bbk ^n=V_1\oplus V_2.$\begin{proof}
        考虑$V_1\cap V_2$,即$(1)$、$(2)$联立得到方程组的解空间
        \[
            \tag{$3$}
            \begin{cases*}
                x_1+x_2+\cdots+x_n=0                                                \\
                \lambda_1x_1+\lambda_2x_2+\cdots+\lambda_nx_n=0                     \\
                \qquad\cdots\cdots\cdots\cdots\cdots                                \\
                \lambda_{1}^{k-1}x_1+\lambda_2^{k-1}x_2+\cdots+\lambda_n^{k-1}x_n=0 \\
                x_1^{k}+x_2^k+\cdots+x^k_n=0                                        \\
                \lambda_1^{k+1}x_1+\lambda_2^{k+1}x_2+\cdots+\lambda_n^{k+1}x_n=0   \\
                \qquad\cdots\cdots\cdots\cdots\cdots                                \\
                \lambda_{1}^{n-1}x_1+\lambda_2^{n-1}x_2+\cdots+\lambda_n^{n-1}x_n=0
            \end{cases*}
        \]
        其系数矩阵的行列式为Vandermonde行列式,值为
        \[
            \prod_{1\leqslant i<j\leqslant n}
            \left(\lambda_j-\lambda_i\right)
            \neq 0
        \]
        故系数矩阵满秩.于是$V_1\cap V_2=0.$
        不妨将三个系数矩阵记作
        \[
            \bm{A}=\begin{pmatrix}
                \bm{A}_1 \\\bm{A}_2
            \end{pmatrix}
        \]
        且$\rmr \left(\bm{A}\right)=n$,则
        $\rmr \left(\bm{A}_1\right)=k,\rmr \left(\bm{A}_2\right)=n-k.$于是$\dim V_1=n-\rmr \left(\bm{A}_1\right)=n-k,
            \dim V_2=n-\rmr \left(\bm{A}_2\right)=k.$故$\dim\left(V_1\oplus V_2\right)=n=\dim \bbk ^n.$
    \end{proof}
}
\exa{}{}{
    设$\bm{A}\in M_{m\times n}\left(\bbr \right)$,求证:\[\rank \left(\bm{AA}'\right)=\rank \left(\bm{A}'\bm{A}\right)
        =\rank \left(\bm{A}\right)\]\begin{proof}
        设方程$\bm{Ax}=\bm{0}$的解空间为$V_{\bm{A}}$,方程$\bm{A}'\bm{Ax}=\bm{0}$的解空间为$V_{\bm{A}'\bm{A}}$.显然
        \[
            V_{\bm{A}}\subseteq
            V_{\bm{A}'\bm{A}}
        \]
        任取$\bm{x}_0\in V_{\bm{A}'\bm{A}}$,此时
        $\bm{x}_0\in \bbr ^n$且$\bm{A}'\bm{Ax}_0=\bm{0}.$令$\bm{Ax}_0=\left(a_1,a_2,\cdots,a_m\right)'\in \bbr ^m$.
        对$\bm{A}'\bm{Ax}_0=\bm{0}$左乘$\bm{x}_0'$得
        \[
            \left(\bm{Ax}_0\right)'\left(\bm{Ax}_0\right)
            =\bm{0}
        \]
        即$\displaystyle
            \sum_{i=1}^{m}a_i^2=0.$故$\forall 1\leqslant i\leqslant m,a_i=0$即
        $\bm{Ax}_0=\bm{0}.$则
        $
            V_{\bm{A}}\supseteq
            V_{\bm{A}'\bm{A}}
        $
        故$V_{\bm{A}}=V_{\bm{A}'\bm{A}}$,利用维数公式即得.
    \end{proof}
}