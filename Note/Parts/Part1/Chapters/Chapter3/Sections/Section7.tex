\newpage
\section{坐标向量}
\subsection{映射}
\lem{线性表出唯一性}{线性表出唯一性}{
    设$\left\{
        \bm{e}_1,\bm{e}_2,\cdots,
        \bm{e}_n
        \right\}$为$V_{
                \mathbb{K}
            }$的一组基且对于
    $\bm{\alpha}\in V$有
    \[
        \bm{\alpha} =
        a_1\bm{e}_1+a_2\bm{e}_2+\cdots+a_n\bm{e}_n
        =
        b_1\bm{e}_1 + b_2\bm{e}_2+\cdots +
        b_n\bm{e}_n
    \]
    则\[
        a_i = b_i \quad
        \left(i = 1,2,\cdots,n\right)
    \]
}
\dfn{坐标向量}{坐标向量}{
    固定$V_{\mathbb{K}}$的一组基$\left\{\bm{e}_1,\bm{e}_2,\cdots
        ,\bm{e}_n\right\}$及其顺序,定义映射$\varphi $
    \begin{align*}
        \varphi :
        V_{\mathbb{K}} &
        \longrightarrow
        \mathbb{K}^n                                \\
        \bm{\alpha} = \sum_{i = 1}
        ^{n}a_i\bm{e}_i
                       & \longmapsto \begin{pmatrix}
                                         a_1    \\
                                         a_2    \\
                                         \vdots \\
                                         a_n
                                     \end{pmatrix}
    \end{align*}
    其中$\left(a_1,a_2,\cdots,a_n\right)^{\prime}$是$\bm{\alpha}$在$\left\{
        \bm{e}_1,\bm{e}_2,\cdots,\bm{e}_n
        \right\}$下的坐标向量.容易证明$\varphi$是双射.
}
\subsection{同构}
\dfn{线性同构}{线性同构}{
设$V_{\mathbb{K}},U_{\mathbb{K}}$,若存在一个双射$\varphi    : V \longrightarrow U$使得
$\forall \bm{\alpha},\bm{\beta}
    \in V,k \in \mathbb{K}$,满足
\[
    \varphi \left(\bm{\alpha}
    +\bm{\beta}
    \right) =
    \varphi \left(\bm{\alpha}
    \right)+\varphi \left(
    \bm{\beta}
    \right),
    \varphi \left(k\bm{\alpha}
    \right)=k\varphi \left(
    \bm{\alpha}
    \right)
\]
即$\varphi $保持了加法和数乘也就是保持了线性组合,则称$\bm{\varphi}:V\longrightarrow U$为(线性)同构,简称$V$同构于$U$,记作$V\cong U$.
}
\thm{同构的存在性}{同构的存在性}{
$V_{\mathbb{K}},U_{\mathbb{K}}$同构等价于
\[
    \dim_{\mathbb{K}}V
    = \dim_{\mathbb{K}}U
\]
}
\thm{}{与标准行(列)向量空间同构}{
    $\mathbb{K}$上任一$n$维线性空间$V$均与$n$维行(列)向量空间$\bbk^n\left(\mathbb{K}^n\right)$同构.\begin{proof}
        设$\bm{\alpha}$、$\bm{\beta}\in V$,$\displaystyle
            \bm{\alpha}=\sum_{i=1}^{n}a_i\bm{e}_i$,$\displaystyle
            \bm{\beta}=\sum_{i=1}^{n}b_i\bm{e}_i$,则
        \[
            \bm{\varphi}\left(\bm{\alpha}\right)=\begin{pmatrix}
                a_1 \\a_2\\\vdots\\a_n
            \end{pmatrix},\bm{\varphi}\left(\bm{\beta}\right)=
            \begin{pmatrix}
                b_1 \\b_2\\\vdots\\b_n
            \end{pmatrix}
        \]
        又$\displaystyle
            \bm{\alpha}+\bm{\beta}=\sum_{i=1}^{n}\left(a_i+b_i\right)\bm{e}_i$,于是
        \[
            \bm{\varphi}\left(\bm{\alpha}+\bm{\beta}\right)=
            \begin{pmatrix}
                a_1+b_1 \\a_2+b_2\\\vdots\\a_n+b_n
            \end{pmatrix}=\bm{\varphi}\left(\bm{\alpha}\right)+\bm{\varphi}
            \left(\bm{\beta}\right)
        \]
        又因为$\displaystyle k\bm{\alpha}=\sum_{i=1}^{n}ka_i\bm{e}_i$,于是
        \[
            \bm{\varphi}\left(k\bm{\alpha}\right)=\begin{pmatrix}
                ka_1 \\ka_2\\\vdots\\ka_n
            \end{pmatrix}=k\bm{\varphi}\left(\bm{\alpha}\right)
            \qedhere
        \]
    \end{proof}
}
\thm{同构对线性结构的保持}{同构对线性结构的保持}{
    线性同构保持向量组的线性关系.
    \begin{enumerate}[label=\arabic*)]
        \item $\varphi :V \longrightarrow
                  U$为线性空间的同构,则$
                  \varphi \left(\bm{0}\right)
                  = \bm{0}$
        \item $\varphi$ 将向量的线性组合映射为对应向量的线性组合,将线性相关的向量映射为线性相关的向量,将线性无关的向量映射为线性无关的向量
    \end{enumerate}\begin{proof}
        \begin{enumerate}[label=\arabic*)]
            \item 首先
                  \[
                      \bm{\varphi}\left(\bm{0}+\bm{0}\right)=\bm{\varphi}\left(\bm{0}\right)+\bm{\varphi}\left(
                      \bm{0}\right)=\bm{\varphi}\left(\bm{0}\right)
                  \]
                  进一步
                  \[
                      \bm{\varphi}^{-1}\left(\bm{0}\right)=
                      \left\{\bm{\alpha}\in V\big| \bm{\varphi}\left(\bm{\alpha}\right)=\bm{0}\right\}
                      =\left\{\bm{0}\right\}
                  \]
            \item 先证线性组合,设$\bm{\beta}=\lambda_1\bm{\alpha}_1+\lambda_2\bm{\alpha}_2+\cdots+\lambda_m\bm{\alpha}_m$,则
                  \begin{align*}
                      \bm{\varphi}\left(\bm{\beta}\right) & =\bm{\varphi}\left(\lambda_1\bm{\alpha}_1+\lambda_2\bm{\alpha}_2+\cdots+\lambda_m\bm{\alpha}_m\right)                               \\
                                                          & =\lambda_1\bm{\varphi}\left(\bm{\alpha}_1\right)+\lambda_2\bm{\varphi}\left(\bm{\alpha}_2\right)+\cdots+\lambda_m\bm{\varphi}\left(
                      \bm{\alpha}_m\right)
                  \end{align*}

                  然后证明线性相关,设$\lambda_1\bm{\alpha}_1+\lambda_2\bm{\alpha}_2+\cdots+\lambda_m\bm{\alpha}_m=\bm{0}$,其中$\lambda_i$不全为零,于是
                  \begin{align*}
                      \bm{0} & =\bm{\varphi}\left(\bm{0}\right)                                                                                                                 \\
                             & =\bm{\varphi}\left(\lambda_1\bm{\alpha}_1+\lambda_2\bm{\alpha}_2+\cdots+\lambda_m\bm{\alpha}_m\right)                                            \\
                             & =\lambda_1\bm{\varphi}\left(\bm{\alpha}_1\right)+\lambda_2\bm{\varphi}\left(\bm{\alpha}_2\right)+\cdots+\lambda_m\bm{\varphi}\left(\bm{\alpha}_m
                      \right)
                  \end{align*}

                  最后是线性无关,设$\bm{\alpha}_1,\bm{\alpha}_2,\cdots,\bm{\alpha}_m$线性无关,则设
                  \[\lambda_1\bm{\varphi}\left(\bm{\alpha}_1\right)+\lambda_2\bm{\varphi}\left(\bm{\alpha}_2\right)+\cdots+\lambda_m\bm{\varphi}\left(\bm{\alpha}_m
                      \right)=\bm{0}\]即
                  \[
                      \bm{0}=\bm{\varphi}\left(\lambda_1\bm{\alpha}_1+\lambda_2\bm{\alpha}_2+\cdots
                      +\lambda_m\bm{\alpha}_m\right)
                  \]
                  由\begin{align*}
                      \bm{\varphi}^{-1}\left(\bm{0}\right) & =
                      \left\{\bm{\alpha}\in V\big| \bm{\varphi}\left(\bm{\alpha}\right)=\bm{0}\right\} \\
                                                           & =\left\{\bm{0}\right\}
                  \end{align*}
                  得到
                  \[
                      \lambda_1\bm{\alpha}_1+\lambda_2\bm{\alpha}_2+\cdots
                      +\lambda_m\bm{\alpha}_m=\bm{0}
                  \]
                  又因为$\bm{\alpha}_1,\bm{\alpha}_2,\cdots,\bm{\alpha}_m$线性无关,故
                  $\lambda_1=\lambda_2=\cdots=\lambda_m=0.$\qedhere
        \end{enumerate}
    \end{proof}
}
\thm{}{秩与极大无关组指标的保持}{
    设$\varphi :
        V \longrightarrow
        \mathbb{K}^n$,$\bm{\alpha}_1,\bm{\alpha}_2
        ,\cdots,\bm{\alpha}_m\in V$,
    作
    \[
        \widetilde{\bm{\alpha}}_i:=
        \varphi\left(\bm{\alpha}_i
        \right)
    \]
    则\begin{itemize}
        \item $\rank\left(\left\{\bm{\alpha}_1,\bm{\alpha}_2,\cdots,\bm{\alpha}_m\right\}\right)=\rank\left(\left\{\widetilde{\bm{\alpha}}_1,\widetilde{\bm{\alpha}}_2,\cdots,\widetilde{\bm{\alpha}}_m  \right\}\right)$
        \item 若$\left\{
                  \bm{\alpha}_{i_1},
                  \bm{\alpha}_{i_2},
                  \cdots
                  ,\bm{\alpha}_{i_r}
                  \right\}$为
              $\left\{\bm{\alpha}_1,\bm{\alpha}_2
                  ,\cdots,\bm{\alpha}_m
                  \right\}$的极大线性无关组,那么
              $\left\{\widetilde{\bm
                      {\alpha}
                  }_{i_1},
                  \widetilde{\bm{\alpha}}_{i_2},
                  \cdots,\widetilde{
                      \bm{\alpha}
                  }_{i_r}  \right\}$
              是$\left\{\widetilde{\bm{\alpha}
                  }_1,
                  \widetilde{\bm{\alpha}}_2,
                  \cdots,\widetilde{
                      \bm{\alpha}
                  }_m \right\}$的极大无关组.
    \end{itemize}

    即同构保持秩和极大无关组的指标.
}