\newpage
\section{基变换与过渡矩阵}
\subsection{定义}
\dfn{过渡矩阵}{过渡矩阵}{
    设$V_{\bbk }$的两组基
    $\left\{\bm{e}_1,\bm{e}_2,\cdots,
        \bm{e}_n\right\}$,$
        \left\{\bm{f}_1,\bm{f}_2
        ,\cdots,\bm{f}_n\right\}$
    则有\[
        \begin{cases*}
            \bm{f}_1 = a_{11}\bm{e}_1+a_{21}\bm{e}_2
            +\cdots+a_{n1}\bm{e}_n   \\
            \bm{f}_2=a_{12}\bm{e}_1+a_{22}\bm{e}_2+\cdots+
            a_{n2}\bm{e}_n           \\
            \qquad\cdots\cdots
            \cdots\cdots\cdots\cdots \\
            \bm{f}_n=a_{1n}\bm{e}_1+a_{2n}\bm{e}_2+\cdots
            +a_{nn}\bm{e}_n
        \end{cases*}
    \]
    这里的系数构成一个$n$阶方阵的转置
    \[
        \bm{A} = \begin{pmatrix}
            a_{11} & a_{21} & \cdots & a_{n1} \\
            a_{12} & a_{22} & \cdots & a_{n2} \\
            \vdots & \vdots &        & \vdots \\
            a_{1n} & a_{2n} & \cdots & a_{nn}
        \end{pmatrix}
    \]
    称为从基$\left\{\bm{e}_1,\bm{e}_2,\cdots,\bm{e}_n\right\}$到基$\left\{\bm{f}_1,\bm{f}_2,\cdots,\bm{f}_n\right\}$的过渡矩阵.
}
\dfn{形式向量的运算定义}{形式向量的运算定义}{
    \begin{enumerate}[label=\arabic*)]
        \item 相等:
              \[
                  \left(\bm{\alpha}_1,\bm{\alpha}_2,\cdots
                  ,\bm{\alpha}_n\right)
                  =\left(\bm{\beta}_1,\bm{\beta}_2,\cdots
                  ,\bm{\beta}_n\right)
                  \Longleftrightarrow
                  \bm{\alpha}_i =
                  \bm{\beta}_i,\forall
                  1\leqslant i\leqslant n
              \]
        \item 加法\[
                  \left(\bm{\alpha}_1,\bm{\alpha}_2,\cdots
                  ,\bm{\alpha}_n\right)
                  +\left(\bm{\beta}_1,\bm{\beta}_2,\cdots
                  ,\bm{\beta}_n\right)
                  =\left(
                  \bm{\alpha}_1+\bm{\beta}_1
                  ,\bm{\alpha}_1+\bm{\beta}_2,
                  \cdots,\bm{\alpha}_n+
                  \bm{\beta}_n
                  \right)
              \]
        \item 数乘($k\in\bbk $)\[
                  k\left(\bm{\alpha}_1,\bm{\alpha}_2,\cdots
                  ,\bm{\alpha}_n\right)=
                  \left(k\bm{\alpha}_1,k\bm{\alpha}_2,\cdots
                  ,k\bm{\alpha}_n\right)
              \]
        \item 与矩阵的乘法
              \[
                  \left(\bm{\alpha}_1,\bm{\alpha}_2,\cdots
                  ,\bm{\alpha}_n\right)\bm{A}_{
                      n \times m
                  } =
                  \left(
                  \sum_{i = 1}^{n}a_{i1}\bm{\alpha}_i,
                  \sum_{i=1}^{n}a_{i2}\bm{\alpha}_i
                  \cdots,
                  \sum_{i = 1}^{n}a_{im}\bm{\alpha}_i
                  \right)
              \]
    \end{enumerate}
}
\clm{过渡矩阵的形式向量表出}{}{
    \cref{def:过渡矩阵}的基变换也可以写为
    \[
        \left(\bm{f}_1,\bm{f}_2,\cdots,\bm{f}_n\right)
        =\left(\bm{e}_1,\bm{e}_2,\cdots,\bm{e}_n\right)\bm{A}
    \]
    其中$\bm{A}$即为过渡矩阵.
}
\lem{过渡矩阵的唯一性}{过渡矩阵的唯一性}{
    设$V_{\bbk }$的一组基
    $\left\{
        \bm{e}_1,\bm{e}_2,\cdots
        ,\bm{e}_n
        \right\}$,  $\bm{A}=\left(a_{ij}\right)_{n \times m}$,$\bm{B} =
        \left(b_{ij}\right)_{n \times m}$使得
    \[
        \left(
        \bm{e}_1,\bm{e}_2,\cdots
        ,\bm{e}_n
        \right)\bm{A}=
        \left(
        \bm{e}_1,\bm{e}_2,\cdots
        ,\bm{e}_n
        \right)\bm{B}
    \]
    则$\bm{A}=\bm{B}$.
}
\lem{}{过渡矩阵的确定}{
    设数域$\bbk $上的$n$维线性空间$V$,取定$V$的两组基为$\left\{\bm{e}_1,\bm{e}_2,\cdots,
        \bm{e}_n\right\}$,且$
        \left\{\bm{f}_1,\bm{f}_2
        ,\cdots,\bm{f}_n\right\}$,$\forall
        \bm{\alpha} \in V$有
    \[
        \bm{\alpha}=
        \lambda _1\bm{e}_1+\lambda_2\bm{e}_2+\cdots
        +\lambda _n\bm{e}_n
        =\mu _1\bm{f}_1+\mu_2\bm{f}_2+\cdots
        +\mu _n\bm{f}_n
    \]
    则有
    \[
        \begin{pmatrix}
            \lambda _1 \\
            \lambda_2  \\
            \vdots     \\
            \lambda _n
        \end{pmatrix}
        = \begin{pmatrix}
            a_{11} & a_{12} & \cdots & a_{n1} \\
            a_{21} & a_{22} & \cdots & a_{2n} \\
            \vdots & \vdots &        & \vdots \\
            a_{1n} & a_{n2} & \cdots & a_{nn}
        \end{pmatrix}
        \begin{pmatrix}
            \mu _1 \\
            \mu_2  \\
            \vdots \\
            \mu_n
        \end{pmatrix}
    \]
    这是矩阵\[
        \bm{A} = \begin{pmatrix}
            a_{11} & a_{21} & \cdots & a_{n1} \\
            a_{12} & a_{22} & \cdots & a_{n2} \\
            \vdots & \vdots &        & \vdots \\
            a_{1n} & a_{2n} & \cdots & a_{nn}
        \end{pmatrix}
    \]是从$\left\{\bm{e}_1,\bm{e}_2,\cdots,\bm{e}_n\right\}$到$ \left\{\bm{f}_1,\bm{f}_2,\cdots,\bm{f}_n\right\}$的过渡矩阵的充分必要条件.\begin{proof}
        一方面,设
        \[
            \left(\bm{f}_1,\bm{f}_2,\cdots,\bm{f}_n\right)
            =\left(\bm{e}_1,\bm{e}_2,\cdots,\bm{e}_n\right)\bm{A}
        \]因为
        \begin{align*}
            \bm{\alpha} & =\left(\bm{e}_1,\bm{e}_2,\cdots,\bm{e}_n\right)\begin{pmatrix}
                                                                             \lambda_1 \\\lambda_2\\\vdots\\\lambda_n
                                                                         \end{pmatrix} \\
                        & =\left(\bm{f}_1,\bm{f}_2,\cdots,\bm{f}_n\right)
            \begin{pmatrix}
                \mu_1 \\\mu_2\\\vdots\\\mu_n
            \end{pmatrix}                                                                          \\
                        & =\left(\bm{e}_1,\bm{e}_2,\cdots,\bm{e}_n\right)\bm{A}
            \begin{pmatrix}
                \mu_1 \\\mu_2\\\vdots\\\mu_n
            \end{pmatrix}
        \end{align*}

        另一方面,考虑$\bm{f}_i$的新坐标向量为$\left(0,0,\cdots,1,\cdots,0\right)^{\prime}\in \bbk ^n$,那么由上式得其旧坐标向量为$\bm{A}\left(0,0,\cdots,1,\cdots,0
            \right)^{\prime}
        $
        即$\bm{A}$的第$i$列$\left(
            a_{1i},a_{2i},\cdots,a_{ni}
            \right)^{\prime}$,符合过渡矩阵定义.
    \end{proof}
}
\subsection{复合}
\thm{过渡矩阵的可逆性与复合性质}{过渡矩阵的可逆性与复合性质}{
    设数域$\bbk $上的线性空间$V$的三组基为$\left\{
        \bm{e}_1,\bm{e}_2,\cdots
        ,\bm{e}_n
        \right\}$、$
        \left\{
        \bm{f}_1,\bm{f}_2,\cdots,
        \bm{f}_n
        \right\}$、$
        \left\{
        \bm{g}_1,\bm{g}_2,\cdots
        ,\bm{g}_n
        \right\}$,且从基$\left\{
        \bm{e}_1,\bm{e}_2,\cdots
        ,\bm{e}_n
        \right\}$到基$
        \left\{
        \bm{f}_1,\bm{f}_2,\cdots,
        \bm{f}_n
        \right\}$的过渡矩阵为$\bm{A}$,从基$\left\{
        \bm{f}_1,\bm{f}_2,\cdots,
        \bm{f}_n
        \right\}$到基$
        \left\{
        \bm{g}_1,\bm{g}_2,\cdots
        ,\bm{g}_n
        \right\}$的过渡矩阵为$
        \bm{B}$,从基$\left\{
        \bm{e}_1,\bm{e}_2,\cdots,
        \bm{e}_n
        \right\}$到基$
        \left\{
        \bm{g}_1,\bm{g}_2,\cdots
        ,\bm{g}_n
        \right\}$的过渡矩阵为$
        \bm{C}$,则
    \begin{enumerate}[label=\arabic*)]
        \item 任何过渡矩阵都是可逆的
        \item $\bm{C} = \bm{AB}$
    \end{enumerate}\begin{proof}
        \begin{enumerate}[label=\arabic*)]
            \item 设从基$\left\{
                      \bm{f}_1,\bm{f}_2,\cdots,
                      \bm{f}_n
                      \right\}$到基$\left\{
                      \bm{e}_1,\bm{e}_2,\cdots,
                      \bm{e}_n
                      \right\}$的过渡矩阵为$\bm{P}$,则
                  \begin{align*}
                      \left(\bm{f}_1,\bm{f}_2,\cdots,\bm{f}_n\right)=\left(\bm{e}_1,\bm{e}_2,\cdots,\bm{e}_n\right)\bm{A} \\
                      \left(\bm{e}_1,\bm{e}_2,\cdots,\bm{e}_n
                      \right)=\left(\bm{f}_1,\bm{f}_2,\cdots,\bm{f}_n\right)\bm{P}
                  \end{align*}
                  于是
                  \[
                      \left(\bm{e}_1,\bm{e}_2,\cdots,\bm{e}_n
                      \right)=\left(\bm{e}_1,\bm{e}_2,\cdots,\bm{e}_n
                      \right)\bm{AP}
                  \]
                  即$\bm{AP}=\bm{I}_n$.
            \item 显然
                  \begin{align*}
                      \left(\bm{f}_1,\bm{f}_2,\cdots,\bm{f}_n\right)=\left(\bm{e}_1,\bm{e}_2,\cdots,\bm{e}_n\right)\bm{A} \\
                      \left(\bm{g}_1,\bm{g}_2,\cdots,\bm{g}_n\right)=\left(\bm{f}_1,\bm{f}_2,\cdots,\bm{f}_n\right)\bm{B} \\
                      \left(\bm{g}_1,\bm{g}_2,\cdots,\bm{g}_n\right)=\left(\bm{e}_1,\bm{e}_2,\cdots,\bm{e}_n\right)\bm{C}
                  \end{align*}
                  转换立刻得到.\qedhere
        \end{enumerate}
    \end{proof}
}
\cor{反方向的过渡矩阵}{反方向的过渡矩阵}{
    设从基$\left\{
        \bm{e}_1,\bm{e}_2,\cdots
        ,\bm{e}_n
        \right\}$到基$
        \left\{
        \bm{f}_1,\bm{f}_2,\cdots,
        \bm{f}_n
        \right\}$的过渡矩阵为$\bm{A}$,从基
    $\left\{
        \bm{f}_1,\bm{f}_2,\cdots
        ,\bm{f}_n
        \right\}$到基$
        \left\{
        \bm{e}_1,\bm{e}_2,\cdots,
        \bm{e}_n
        \right\}$的过渡矩阵为$\bm{B}$,则二者是逆阵关系即
    \[
        \bm{AB} =  \bm{I}_n
    \]
}
\rem{}{}{
    \cref{thm:过渡矩阵的可逆性与复合性质}的重要意义在于,对于非标准基之间的过渡矩阵,不再需要去解出一个包含$n^2$个未定元的方程,而是通过与标准基之间的过渡矩阵和简单的复合表示,转化为解矩阵方程的问题,进一步地,转换为求逆阵这一简单问题.
}
\exa{}{}{
    设$\bbk $上的线性空间$V$的两组基为
    \[
        \left\{
        \bm{f}_1=\left(1,0,-1\right),\bm{f}_2=\left(2,1,1\right),\bm{f}_3=\left(1,1,1\right)
        \right\}
    \]\[\left\{
        \bm{g}_1=\left(0,1,1\right),\bm{g}_2=\left(-1,1,0\right),\bm{g}_3=\left(1,2,1\right)\right\}
    \]求基$\bm{f}$到基$\bm{g}$的过渡矩阵.\begin{solution}
        设标准基为$\left\{\bm{e}_1=\left(1,0,0\right),\bm{e}_2=\left(0,1,0\right),\bm{e}_3=\left(0,0,1\right)\right\}$,则基$\bm{e}$到基$\bm{f},\bm{g}$的过渡矩阵分别为\[
            \begin{pmatrix}
                1  & 2 & 1 \\
                0  & 1 & 1 \\
                -1 & 1 & 1
            \end{pmatrix},\begin{pmatrix}
                0 & -1 & 1 \\
                1 & 1  & 2 \\
                1 & 0  & 1
            \end{pmatrix}
        \]并设基$\bm{f}$到基$\bm{g}$的过渡矩阵为$\bm{X}$.根据\cref{thm:过渡矩阵的可逆性与复合性质}有\[
            \bm{AX}=\bm{B}\Longrightarrow \bm{X}=\bm{A}^{-1}\bm{B}
        \]显然\[
            \bm{X}=\begin{pmatrix}
                0 & 1 & 1 \\-1&-3&-2\\2 & 4 & 4
            \end{pmatrix}
        \]
    \end{solution}
}