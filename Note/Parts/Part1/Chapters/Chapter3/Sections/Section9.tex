\newpage
\section{子空间}
\subsection{子空间}
\dfn{子空间}{子空间}{
    设线性空间$V_{\mathbb{K}}$,$V_0$为$V$的非空子集,若
    $\forall \bm{\alpha},\bm{\beta}
        \in V_0,k \in \mathbb{K}$
    \[
        \bm{\alpha}+\bm{\beta} \in V_0,k\bm{\alpha
        } \in V_0
    \]
    称$V_0$是$V$的(线性)子空间(准确地说是子线性空间).
}
\lem{}{子空间仍然是线性空间}{
    $V_0$在$V$的加法和数乘下是$\mathbb{K}$的线性空间
    \begin{align*}
        + :
        V_0 \times V_0
         & \longrightarrow
        V_0                \\
        \left(
        \bm{\alpha},\bm{\beta}
        \right)
         & \longmapsto
        \bm{\alpha}+\bm{\beta}
        \\
        \cdot : \mathbb{K} \times
        V_0
         & \longrightarrow
        V_0                \\
        \left(k,\bm{\alpha}
        \right)
         & \longmapsto
        k \cdot \bm{\alpha}
    \end{align*}\begin{proof}
        加法交换律、结合律显然成立,又$\forall\bm{\alpha}\in V_0$,$\bm{0}=\bm{\alpha}+\left(-\bm{\alpha}\right)\in V_0$.即零向量在$V_0$中,且就是$V$中的零向量.显然负元存在.

        显然数乘的性质均满足.
    \end{proof}
}
\cor{}{子空间对线性运算的封闭}{
    子空间对线性组合封闭即若$\bm{\alpha}_1,\bm{\alpha}_2,\cdots,\bm{\alpha}_n\in V_0$,
    $\lambda_1,\lambda_2,\cdots,\lambda_n\in \mathbb{K}$,则
    \[
        \lambda_1\bm{\alpha}_1+\lambda_2\bm{\alpha}_2+\cdots+\lambda_n\bm{\alpha}_n\in V_0
    \]
}
\dfn{平凡子空间}{平凡子空间}{
    一般地,任一线性空间$V$至少有下面两个子空间,被称为平凡子空间.
    \[
        \text{平凡子空间}
        \begin{cases*}
            \text{零子空间:}\left\{
            \bm{0}_V
            \right\} \\
            \text{全子空间:}
            V
        \end{cases*}
    \]
}
\lem{子空间的维数}{子空间的维数}{
    对于$\mathbb{K}$上的线性空间$V$的任一子空间$V_0$
    \[
        0 \leqslant
        \dim_{\mathbb{K}}V_0
        \leqslant
        \dim_{\mathbb{K}}V
    \]
    特别地,非平凡子空间取到严格不等号.\begin{proof}
        由\cref{thm:基扩张定理}基扩张定理,这是显然的.仅证明:$\dim_{\mathbb{K}}V_0=\dim_{\mathbb{K}}V=n$时$V=V_0$.

        取$V_0$的一组基$\left\{\bm{e}_1,\bm{e}_2,\cdots,\bm{e}_n\right\}$,则其是$V$中$n$个线性无关的向量,又$\dim_{\mathbb{K}}V=n$,故这是$V$的一组基.

        又$\forall \bm{\alpha}\in V$,$\lambda_1\bm{\alpha}_1+\lambda_2\bm{\alpha}_2+\cdots+\lambda_n\bm{\alpha}_n\in V_0$,则$V_0=V$.
    \end{proof}
}
\thm{子空间的相等}{子空间的相等}{
    设$V$是$\mathbb{K}$上的线性空间,$V_1,V_2$是$V$的两个子空间,则
    \[
        V_1=V_2 \iff
        \dim V_1=
        \dim V_2
    \]
}
\subsection{交空间与和空间}
\dfn{交空间与和空间}{交空间与和空间}{
    设$V_1,V_2\subseteq V$,定义它们的交为集合的交
    \[
        V_1 \cap V_2
        = \left\{
        \bm{\alpha} \mid
        \bm{\alpha} \in V_1,
        \bm{\alpha} \in V_2
        \right\}
    \]
    和为
    \[
        V_1 + V_2 =\left\{
        \bm{\alpha} + \bm{\beta} \mid
        \bm{\alpha} \in V_1,
        \bm{\beta} \in V_2
        \right\}
    \]
    且二者均是$V$的子空间即交空间、和空间,但
    \[
        V_1\cup V_2 =\left\{
        \bm{\alpha} |
        \bm{\alpha} \in V_1 \text{或}
        \bm{\alpha} \in V_2
        \right\}
    \]
    不是$V$的子空间(一般来说).\begin{proof}
        先证交空间,设$\bm{\alpha},\bm{\beta}\in V_1\cap V_2$,则
        $\bm{\alpha}+\bm{\beta}\in V_1$且$\bm{\alpha}+\bm{\beta}\in V_2$,故
        $\bm{\alpha}+\bm{\beta}\in V_1\cap V_2$.$\forall k \in\mathbb{K}$,
        $\bm{\alpha}\in V_1\cap V_2$,均有$k\bm{\alpha}\in V_1$且
        $k\bm{\alpha}\in V_2$,故$k\bm{\alpha}\in V_1\cap V_2.$

        再证和空间,设$\bm{\alpha},\bm{\beta}\in V_1+V_2$,且
        \begin{align*}
             & \bm{\alpha}=\bm{\alpha}_1+\bm{\alpha}_2,\bm{\alpha}_1\in V_1,\bm{\alpha}_2\in V_2 \\
             & \bm{\beta}=\bm{\beta}_1+\bm{\beta}_2,\bm{\beta}_1\in V_1,\bm{\beta}_2\in V_2
        \end{align*}
        则
        \[
            \bm{\alpha}+\bm{\beta}=\left(\bm{\alpha}_1+\bm{\beta}_1\right)+
            \left(\bm{\alpha}_2+\bm{\beta}_2\right)\in V_1+V_2
        \]
        考虑$\forall k \in \mathbb{K}$,$k\bm{\alpha}=k\bm{\alpha}_1+k\bm{\alpha}_2\in V_1+V_2$.
    \end{proof}
}
\clm{}{}{
    两个线性空间的交空间不可能为空,因为至少包含零元,也就是说,下限就是零子空间$0=\left\{\bm{0}\right\}$.
}
\cor{}{多个子空间的交空间与和空间}{
    设$V_1,V_2,\cdots,V_m$均为$V$的子空间,则它们的交空间
    \[
        V_1 \cap V_2\cap\cdots \cap V_m
        = \left\{
        \bm{\alpha} \mid
        \bm{\alpha} \in V_1 ,
        \bm{\alpha}\in V_2,
        \cdots,
        \bm{\alpha} \in V_m
        \right\}
    \]
    与和空间
    \[
        V_1 + V_2+\cdots + V_m =
        \left\{
        \bm{\alpha}_1 + \bm{\alpha}_2+\cdots +
        \bm{\alpha}_m\mid
        \bm{\alpha}_1 \in V_1,
        \bm{\alpha}_2\in V_2,
        \cdots,
        \bm{\alpha}_m \in V_m
        \right\}
    \]
    均为$V$的子空间.
}
\subsection{张成的子空间}
\dfn{张成的子空间}{张成的子空间}{
    设$S$为$V_{
                \mathbb{K}
            }$的非空子集,记$L\left(S\right)$为$S$中所有可能的线性组合构成的集合即
    \[
        L\left(S\right)
        =
        \left\{
        \lambda _1 \bm{\alpha}_1+\lambda_2\bm{\alpha}_2+
        +\cdots+
        \lambda _m\bm{\alpha}_m
        \Bigg |
        \begin{matrix}
            \lambda _1,\lambda_2,\cdots,
            \lambda _m \in \mathbb{K} \\
            \bm{\alpha}_1,\bm{\alpha}_2,\cdots,
            \bm{\alpha}_m \in S
        \end{matrix}
        \quad
        \left(
        m \geqslant 0\textup{但有限}
        \right)
        \right\}
    \]
    其在加法、数乘下封闭,是$V$的一个子空间,称为由$S$生成/张成的子空间.有时也记作$\Span\left(S\right)$.
}
\pro{}{张成的子空间的性质}{
    $S$为$V_{\mathbb{K}}$的子集,$L\left(S\right)$是$S$张成的子空间,则
    \begin{enumerate}[label=\arabic*)]
        \item $L\left(S\right)$是包含$S$的$V$的最小子空间
        \item 若$S$存在极大无关组$
                  \left\{\bm{\alpha}_1,\bm{\alpha}_2,\cdots
                  ,\bm{\alpha}_r\right\}$,则
              \[
                  L\left(S\right)
                  = L\left(\bm{\alpha}_1,\bm{\alpha}_2,
                  \cdots
                  ,\bm{\alpha}_r\right)
              \]
              且$
                  \left\{\bm{\alpha}_1,\bm{\alpha}_2,
                  \cdots
                  ,\bm{\alpha}_r\right\}$
              是$
                  L\left(S
                  \right)
              $的一组基即\[
                  \dim_{\mathbb{K}}L\left(S
                  \right) = \rmr \left(
                  S\right)
              \]
    \end{enumerate}\begin{proof}
        \begin{enumerate}[label=\arabic*)]
            \item 任取子空间$V_0\supseteq S$,任取$\lambda_1\bm{\alpha}_1+\lambda_2\bm{\alpha}_2+\cdots+\lambda_m\bm{\alpha}_m\in L\left(S\right)$, $\bm{\alpha}_i\in S$,故$\bm{\alpha}_i\in V_0$,又因为子空间
                  保持线性组合,于是$\lambda_1\bm{\alpha}_1+\lambda_2\bm{\alpha}_2+\cdots+\lambda_m\bm{\alpha}_m\in V_0$,于是$L\left(S\right)\subseteq V_0.$
            \item 因为
                  \[
                      L\left(S\right)\hookrightarrow S\hookrightarrow
                      \left\{\bm{\alpha}_1,\bm{\alpha}_2,\cdots,
                      \bm{\alpha}_r\right\}
                  \]
                  显然
                  \[
                      L\left(S\right)
                      = L\left(\bm{\alpha}_1,\bm{\alpha}_2,
                      \cdots
                      ,\bm{\alpha}_r\right)
                  \]
                  且$\left\{\bm{\alpha}_1,\bm{\alpha}_2,\cdots,\bm{\alpha}_r\right\}$是$L\left(S\right)$的一组基.于是
                  \[
                      \dim_{\mathbb{K}}L\left(S\right)=r=\rmr \left(S\right)
                      \qedhere
                  \]
        \end{enumerate}
    \end{proof}
}
\cor{}{并集张成的空间}{
    若$V_1,V_2,\cdots,V_m$为$V$的子空间,则
    \[
        L\left(
        V_1 \cup V_2\cup \cdots \cup
        V_m
        \right)=
        V_1 +V_2+\cdots
        + V_m
    \]\begin{proof}
        考虑$\forall \bm{\alpha}=\bm{\alpha}_1+\bm{\alpha}_2+
            \cdots+\bm{\alpha}_m\in V_1+V_2+\cdots+V_m$,其中$\bm{\alpha}_i\in V_i$,从而
        \[
            \bm{\alpha}=\bm{\alpha}_1+\bm{\alpha}_2+
            \cdots+\bm{\alpha}_m\in L\left(
            V_1 \cup V_2\cup \cdots \cup
            V_m
            \right)
        \]即有
        \[
            L\left(
            V_1 \cup V_2\cup \cdots \cup
            V_m
            \right)\supseteq
            V_1 +V_2+\cdots
            + V_m
        \]
        又因为$V_i\subseteq V_1 + V_2+ \cdots +V_m $,则$\bm{\alpha}_i\in V_1+V_2+\cdots+V_m$,于是
        \[
            V_1\cup V_2\cup \cdots\cup V_m\subseteq V_1+V_2+\cdots+V_m
        \]
        由最小性即得
        \[
            L\left(
            V_1 \cup V_2\cup \cdots \cup
            V_m
            \right)\subseteq
            V_1 +V_2+\cdots
            + V_m
        \]
        于是有\[
            L\left(
            V_1 \cup V_2\cup \cdots \cup
            V_m
            \right)=
            V_1 +V_2+\cdots
            + V_m
            \qedhere
        \]
    \end{proof}
}
\subsection{维数公式}
\thm{子空间的维数公式}{子空间的维数公式}{
    设$V_1,V_2$为$V_{\mathbb{K}}$的子空间,则\[
        \dim_{\mathbb{K}}\left(V_1 +
        V_2\right)
        =
        \dim_{\mathbb{K}}V_1+
        \dim_{\mathbb{K}}V_2 -
        \dim_{\mathbb{K}}\left(
        V_1\cap V_2
        \right)
    \]
    特别地,$V_1 \cap V_2 =0=\left\{
        \bm{0}
        \right\}$时\[
        \dim_{\mathbb{K}}\left(V_1 +
        V_2\right)
        =
        \dim_{\mathbb{K}}V_1+
        \dim_{\mathbb{K}}V_2
    \]\begin{proof}
        设右式依次为$m$、$n$、$r$,取$V_1\cap V_2$的一组基$\left\{\bm{\alpha}_1,\bm{\alpha}_2,\cdots,\bm{\alpha}_r
            \right\}$,因为$V_1\cap V_2\subseteq V_1$,根据基扩张定理得到$V_1$的一组基\[\left\{\bm{\alpha}_1,\bm{\alpha}_2,\cdots,\bm{\alpha}_r
            ,\bm{\beta}_1,\bm{\beta}_2,\cdots,\bm{\beta}_{m-r}\right\}\]同理得到$V_2$的一组基
        \[\left\{\bm{\alpha}_1,\bm{\alpha}_2,\cdots,\bm{\alpha}_r
            ,\bm{\gamma}_1,\bm{\gamma}_2,\cdots,\bm{\gamma}_{n-r}
            \right\}\]所以只需要证明$
            \left\{\bm{\alpha}_1,\bm{\alpha}_2,\cdots,\bm{\alpha}_r
            ,\bm{\beta}_1,\bm{\beta}_2,\cdots,\bm{\beta}_{m-r},
            \bm{\gamma}_1,\bm{\gamma}_2,\cdots,\bm{\gamma}_{n-r}\right\}$是$V_1+V_2$的基即可.

        $\forall \bm{v}=\bm{v}_1+\bm{v}_2\in V_1+V_2 ,
            \bm{v}_1\in V_1,\bm{v}_2\in V_2$,显然,分成两部分分别考虑,容易得出$\bm{v}$是
        \[\left\{\bm{\alpha}_1,\bm{\alpha}_2,\cdots,\bm{\alpha}_r
            ,\bm{\beta}_1,\bm{\beta}_2,\cdots,\bm{\beta}_{m-r},
            \bm{\gamma}_1,\bm{\gamma}_2,\cdots,\bm{\gamma}_{n-r}\right\}
        \]的线性组合.

        再证线性无关.设
        \begin{align*}
              & \lambda_1\bm{\alpha}_1+\lambda_2\bm{\alpha}_2+\cdots+\lambda_r
            \bm{\alpha}_r
            \\
            + & \mu_1\bm{\beta}_1+\mu_2
            \bm{\beta}_2+\cdots+\mu_{m-r}
            \bm{\beta}_{m-r}                                                   \\
            + & k_1
            \bm{\gamma}_1+k_2
            \bm{\gamma}_2+\cdots+k_{n-r}\bm{\gamma}_{n-r}
            =\bm{0}
        \end{align*}
        移项,有
        \begin{align*}
            V_1\ni & \lambda_1\bm{\alpha}_1+\lambda_2\bm{\alpha}_2+\cdots+\lambda_r
            \bm{\alpha}_r
            \\
                   & +\mu_1\bm{\beta}_1+\mu_2
            \bm{\beta}_2+\cdots+\mu_{m-r}
            \bm{\beta}_{m-r}                                                        \\
            =      & -\left(k_1
            \bm{\gamma}_1+k_2
            \bm{\gamma}_2+\cdots+k_{n-r}\bm{\gamma}_{n-r}
            \right)\in V_2
        \end{align*}
        故左右两边都属于$V_1\cap V_2.$不妨记作
        \[
            l_1\bm{\alpha}_1+l_2\bm{\alpha}_2+\cdots+l_r\bm{\alpha}_r
            =-\left(k_1
            \bm{\gamma}_1+k_2
            \bm{\gamma}_2+\cdots+k_{n-r}\bm{\gamma}_{n-r}
            \right)
        \]
        移项并由线性无关得$l_1=l_2=\cdots=l_r=k_1=k_2=\cdots=k_{n-r}=0$,代回又得$\lambda_1=\lambda_2=\cdots=\lambda_r=\mu_1=\mu_2=\cdots=\mu_{m-r}=0$.
    \end{proof}
}
\subsection{直和}
\dfn{直和}{直和}{
    设$V_1,V_2,\cdots,V_m$为$V$的子空间,若$\forall 1
        \leqslant i\leqslant m$\[
        V_i \cap \left(
        V_1 + V_2+\cdots +\widehat{V_i}
        +\cdots +V_m
        \right)
        = 0
    \]则称和$V_1
        +V_2+\cdots +V_m$
    为直接和,简称直和,记作\[
        V_1 \oplus V_2\oplus \cdots
        \oplus V_m
    \]
}
\exa{}{}{
    $m = 2$时,$V_1 + V_2$为直和等价于$V_1 \cap V_2 = 0$,比如$V =\mathbb{R}^3 = V_1 \oplus V_{23}$
}
\exa{}{}{
    $m \geqslant 3$时$V_i$两两的交为$0$不能保证直和,比如
    \[
        V_1\oplus V_2 \oplus V_3
        \Longleftrightarrow
        \begin{cases*}
            V_1 \cap \left(V_2+V_3\right) = 0 \\
            V_2 \cap \left(V_1+V_3\right) = 0 \\
            V_3 \cap \left(V_1+V_2\right) = 0
        \end{cases*}
    \]
}
\clm{}{}{当子空间数大于2,子空间
    两两的交为零子空间并不能确定直和,直和的条件比两两的交为零子空间要严格得多.
}
\exa{}{}{
    考虑笛卡尔坐标系,$V=\mathbb{R}^2$,$V_1:x$轴,$V_2:y$轴,$V_3:y=x$.即使两两的交为零子空间,但
    \[
        V_3\cap\left(V_1+V_2\right)=V_3\neq 0
    \]
}
\thm{直和的判定}{直和的判定}{
    若$V_0 = V_1 +V_2+\cdots+V_m$,则下列结论等价
    \begin{enumerate}[label=\arabic*)]
        \item $V_0 =V_1\oplus V_2\oplus
                  \cdots\oplus V_m$
        \item $\forall 2\leqslant i\leqslant m,
                  V_i \cap \left(V_1 +\cdots
                  +V_{i - 1}\right)
                  = 0$
        \item $\dim\left( V_1 +V_2+\cdots+V_m\right)= \dim V_1 +\dim V_2+\cdots
                  +\dim V_m$
        \item $V_1,V_2,\cdots,V_m$的各一组基可以拼成$V_0$的一组基
        \item $V_0$中向量用$V_1,V_2,\cdots,V_m$中向量和表出方式唯一,即若
              $\bm{\alpha}\in V_0 $且$
                  \bm{\alpha} =
                  \bm{v}_1+\bm{v}_2+\cdots
                  +\bm{v}_m =
                  \bm{u}_1 + \bm{u}_2+\cdots
                  + \bm{u}_m
              $其中$\bm{v}_i,\bm{u}_i
                  \in V_i$则$\bm{v}_i = \bm{u}_i
                  \left(
                  1\leqslant i\leqslant m
                  \right)$,称为分块表
              示唯一.事实上,这等价于
              零向量表示唯一.
    \end{enumerate}
    \begin{proof}
        $(1)\Longrightarrow (2)$
        \[
            V_i\cap \left(V_1+\cdots+V_{i-1}\right)\subseteq
            V_i\cap \left(V_1+\cdots+V_{i-1}+V_{i+1}+\cdots+
            V_m\right)=0
        \]
        故
        \[
            V_i \cap \left(V_1 +\cdots
            +V_{i - 1}\right)
            = 0\left(\forall 2\leqslant i\leqslant
            m\right)
        \]

        $(2)\Longrightarrow (3)$对$V_m\cap\left(V_1+\cdots+V_{m-1}\right)=0$使用\cref{thm:子空间的维数公式}维数公式
        得$\dim\left(V_1+\cdots+V_m\right)=\dim\left(V_1+\cdots+V_{m-1}\right)+\dim V_m$,一直如此便得到
        \[
            \dim\left( V_1 +V_2+\cdots
            +V_m\right)= \dim V_1 +\dim V_2+\cdots
            +\dim V_m
        \]

        $(3)\Longrightarrow(4)$设
        $\dim V_i=n_i$,$\dim V_0=n_1+\cdots+n_m
            =n$.取$V_i$基
        $\left\{\bm{e}_{i1},\bm{e}_{i2},\cdots,
            \bm{e}_{in_i}\right\}\left(1\leqslant i \leqslant
            m\right)$,拼在一起即
        \[
            \left\{
            \bm{e}_{11},\bm{e}_{12},\cdots,\bm{e}_{1n_1},
            \bm{e}_{21},\bm{e}_{22},\cdots,\bm{e}_{2n_2},
            \cdots,
            \bm{e}_{m1},\bm{e}_{m2},\cdots,\bm{e}_{mn_m}
            \right\}
        \]
        共$n_1+n_2+\cdots+n_m=n$个.又$\dim V_0=n.$

        $\forall \bm{\alpha}=\bm{\alpha}_1+\cdots+\bm{\alpha}_m\in V_0=
            V_1+\cdots+V_m$,其中$\bm{\alpha}_i\in V_i.$
        设$\bm{\alpha}_i=
            \lambda_{i1}\bm{e}_{i1}+\lambda_{i2}\bm{e}_{i2}+\cdots+\lambda_{in_i}
            \bm{e}_{in_i}\left(1\leqslant i \leqslant m\right)$.容易证明线性组合,以此判定基.

        $(4)\Longrightarrow (5)$,我们先证其与推论等价,由$(5)$推出推论显然.反之
        \[
            \bm{0}=\left(\bm{v}_1-\bm{u}_1\right)+\left(
            \bm{v}_2-\bm{u}_2
            \right)+\cdots+\left(\bm{v}_m-\bm{u}_m\right)\in V_1+V_2+\cdots+V_m
        \]
        由零向量表示唯一立得$\bm{v}_i=\bm{u}_i.$
        于是只需要证明$(4)$推得$(5)$推论.

        设$\bm{0}=\bm{v}_1+\bm{v}_2+\cdots+\bm{v}_m$,其中$\bm{v}_i\in V_i.$因为
        $\bm{v}_i=
            \lambda_{i1}\bm{e}_{i1}+\lambda_{i2}\bm{e}_{i2}+\cdots+\lambda_{in_i}
            \bm{e}_{in_i}\left(
            1\leqslant i\leqslant m
            \right)$.故
        $\displaystyle
            \bm{0}=\sum_{i=1}^{m}\left(
            \lambda_{i1}\bm{e}_{i1}+\lambda_{i2}\bm{e}_{i2}+\cdots+\lambda_{in_i}
            \bm{e}_{in_i}
            \right)$,由基的线性无关性得
        \[
            \lambda_{ij}=0,\forall 1\leqslant i\leqslant m,1\leqslant j\leqslant n_i
        \]
        即$\forall 1\leqslant i\leqslant m,\bm{v}_i=\bm{0}$.

        最后为该推论推得$(1)$.任取$
            \bm{\alpha}\in V_i\cap \left(V_1+\cdots+V_{i-1}+V_{i+1}+\cdots+
            V_m\right)$,即$\bm{\alpha}=\bm{\alpha}_1+\cdots+\bm{\alpha}_{i-1}+\bm{\alpha}_{i+1}+\cdots+\bm{\alpha}_m$.即有对零向量分解
        \[
            \bm{0}=\bm{\alpha}_1+\cdots
            +\bm{\alpha}_{i-1}+\left(-\bm{\alpha}
            \right)+\bm{\alpha}_{i+1}+\cdots+\bm{\alpha}_m
        \]
        则$\bm{\alpha}_1=\cdots=\bm{\alpha}_{i-1}=\bm{\alpha}=
            \bm{\alpha}_{i+1}=\cdots=\bm{\alpha}_m=\bm{0}.$
        即$\bm{\alpha}=0$,于是
        \[
            V_i\cap \left(V_1+\cdots+V_{i-1}+V_{i+1}+\cdots+
            V_m\right)=0\left(1\leqslant i\leqslant
            m\right)
            \qedhere
        \]
    \end{proof}
}