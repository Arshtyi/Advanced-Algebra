\newpage
\section{向量的线性关系}
\subsection{线性表示}
\dfn{线性表示}{线性表示}{
    设$\bb{K}$上的线性空间$V$,$\bm{\alpha}_1,\bm{\alpha}_2,\cdots,
        \bm{\alpha}_n \in
        V_{\bbk}$,如果存在$
        k_1,k_2,\cdots,k_n \in \bbk$,使得
    \[
        \bm{\beta} =
        k_1\bm{\alpha}_1+\cdots+
        k_n\bm{\alpha}_n
    \]
    称$\bm{\beta}$是$\bm{\alpha}_1,
        \bm{\alpha}_2,\cdots,\bm{\alpha}_n$的线性组合,或者说$\bm{\beta}$可以用$\bm{\alpha}_1,\bm{\alpha}_2,\cdots,\bm{\alpha}_n$线性表示/表出.记作
    \[
        \bm{\beta}\hookrightarrow \left\{\bm{\alpha}_1,\bm{\alpha}_2,\cdots,\bm{\alpha}_n\right\}
    \]
}
\exa{}{}{
    $\bbk$上的$n$维标准单位行向量
    \[
        \bm{e}_i = \left(0,\cdots,1,\cdots,0\right)
        \quad
        \left(1\leqslant i\leqslant n\right)
    \]
    (其中$1$位于第$i$位)构成的向量组$\left\{\bm{e}_1,\bm{e}_2,\cdots,\bm{e}_n\right\}$称为$\bbk$上的标准单位行向量组.显然任一$n$维行向量$\bm{\alpha}=
        \left(a_1,a_2,\cdots,a_n\right)$都可用其表示
    \[
        \bm{\alpha}=a_1\bm{e}_1+a_2\bm{e}_2+\cdots+a_n\bm{e}_n
    \]
}
\subsection{线性相关与线性无关}
\dfn{线性相关}{线性相关}{
    设$\bbk$上的线性空间$V$,    $\bm{\alpha}_1,\bm{\alpha}_2,\cdots,
        \bm{\alpha}_n \in
        V_{\bbk}$,若存在不全为零的$k_1,k_2,\cdots,k_n \in\bbk$使得\[
        k_1\bm{\alpha}_1+k_2\bm{\alpha}_2+\cdots
        +k_n\bm{\alpha}_n = \bm{0}
    \]
    称$\bm{\alpha}_1,\bm{\alpha}_2,\cdots,
        \bm{\alpha}_n $是线性相关的;反之若不存在这样的一组
    $k_1,k_2,\cdots,k_n\in\bbk$(或者说只有可能是$k_1=k_2=\cdots=k_n=0$),则称
    $\bm{\alpha}_1,\bm{\alpha}_2,\cdots,\bm{\alpha}_n $是线性无关的.
}
\clm{}{}{
    向量的线性关系取决于基域的选取.
}
\exa{}{}{
    $\bbr \subseteq\bbc$,易证$\left\{1,\mathrm{i}\right\}$在$\bbr$线性无关,在$\bbc$线性相关.
}
\lem{}{含有零向量的向量组一定线性相关}{
    包含$\bm{0}$的向量组一定线性相关.
}
\lem{}{只有一个向量的向量组}{
    \[
        A = \left\{\bm{\alpha}\right\}\begin{cases*}
            \text{线性无关} & $\bm{\alpha} \neq \bm{0}$ \\
            \text{线性相关} & $\bm{\alpha} = \bm{0}$
        \end{cases*}
    \]
}
\thm{}{大小向量组线性关系的约束}{
    $n > m > s$,如果$\left\{\bm{\alpha}_1
        ,\bm{\alpha}_2,\cdots,\bm{\alpha}_m\right\}$线
    性相关,那么$\left\{\bm{\alpha}_1,\bm{\alpha}_2,
        \cdots,\bm{\alpha}_m,\cdots,
        \bm{\alpha}_n\right\}$也线性相关;
    如果$\left\{\bm{\alpha}_1,\bm{\alpha}_2
        ,\cdots,\bm{\alpha}_m\right\}$,线
    性无关,那么$\left\{\bm{\alpha}_1,\bm{\alpha}_2,
        \cdots,\bm{\alpha}_s\right\}$也
    线性无关.
}
\cor{}{向量组内部的线性关系}{
    $\bm{\alpha}_1,\bm{\alpha}_2,\cdots,
        \bm{\alpha}_n$线性相关等价于存在$\bm{\alpha}_i\left(
        1 \leqslant i
        \leqslant n
        \right)$是$\left\{
        \bm{\alpha}_1,\bm{\alpha}_2,\cdots
        ,\widehat{\bm{\alpha}}_i,\bm{\alpha}_{i+1},
        \cdots,\bm{\alpha}_n
        \right\}$的线性组合.
}
\cor{}{加入线性无关的向量组的向量的线性关系}{
    设$\bm{\alpha}_1,\bm{\alpha}_2,\cdots,
        \bm{\alpha}_n,\bm{\beta}
        \in  V_{\bbk}$,且$\bm{\alpha}_1,\bm{\alpha}_2,\cdots
        ,\bm{\alpha}_n$是线性无关的,那么一定有要么$\bm{\alpha}_1,\bm{\alpha}_2,\cdots,
        \bm{\alpha}_n,\bm{\beta}
    $线性无关,要么$\bm{\beta}$ 是$\bm{\alpha}_1,\bm{\alpha}_2,\cdots,
        \bm{\alpha}_n$的线性组合.
}
\thm{线性表示唯一}{线性表示唯一}{
    已知$\bm{\alpha}_1,\bm{\alpha}_2,\cdots,
        \bm{\alpha}_n,\bm{\beta}
        \in  V_{\bbk}$且
    \[
        \bm{\beta} = k_1\bm{\alpha}_1+k_2\bm{\alpha}_2
        +\cdots+k_n\bm{\alpha}_n
    \]
    则该表示形式唯一当且仅当$\bm{\alpha}_1,\bm{\alpha}_2,\cdots,\bm{\alpha}_n$线性无关.
}
\thm{}{线性组合的传递性}{
    设有向量组$A = \left\{
        \bm{\alpha}_1,\bm{\alpha}_2,\cdots,
        \bm{\alpha}_m
        \right\}$、$
        B = \left\{
        \bm{\beta}_1,\bm{\beta}_2,\cdots,
        \bm{\beta}_n
        \right\}$、$C =
        \left\{\bm{\gamma}_1,\bm{\gamma}_2,\cdots
        ,\bm{\gamma}_p\right\}$,若$B$中的向量均为$A$中的向量的线性组合且$C$中的向量均为$B$中的向量的线性组合,那么$C$中的向量均可用$A$中的向量线性表示.即若
    \[
        C\hookrightarrow B \hookrightarrow A
    \]
    则
    \[
        C\hookrightarrow A
    \]
}
\exa{}{}{
    $\bm{\alpha} = \left(a_1,a_2,
        \cdots,a_n\right)$与$
        \bm{\beta} = \left(b_1,b_2,\cdots,
        b_n\right)$线性相关的充分必要条件是$a_i = c b_i(i = 1,2,\cdots,n,c \in\bbr)$成立.
}
\exa{}{}{
    $V = \bbr^2$中
    $\overrightarrow{OA} =\left(
        a_1,a_2
        \right)$、$\overrightarrow{OB}=
        \left(b_1,b_2\right)
    $线性相关等价于$O$、$A$、$B$共线;
    $\overrightarrow{OA}
        =\left(a_1,a_2\right) $、
    $\overrightarrow{OB} = \left(b_1
        ,b_2\right)
    $线性无关等价于
    $\bigtriangleup OAB$非退化,这时
    \[
        S_{\bigtriangleup OAB} =
        \frac{\left|\overrightarrow{OA}
            \times \overrightarrow{OB} \right|
        }{2} = \frac{
            \begin{vmatrix}
                a_1 & a_2 \\
                b_1 & b_2
            \end{vmatrix}
        }{2}
        \neq 0
    \]

    $V = \bbr^3$中
    $\overrightarrow{OA} =\left(
        a_1,a_2,a_3
        \right)$、$\overrightarrow{OB}=
        \left(b_1,b_2,b_3\right)
    $、
    $\overrightarrow{OC} =
        \left(c_1,c_2,c_3\right) $线性相关等价于$O$、$A$、$B$、$C$共面 ;
    $\overrightarrow{OA}
        =\left(a_1,a_2,a_3\right) $、
    $\overrightarrow{OB} = \left(b_1
        ,b_2,b_3\right)
    $、$
        \overrightarrow{OC} =
        \left(c_1,c_2,c_3\right) $线性无关等价于四面体$ OABC$非退化  (或者说该平行六面体非退化),这时
    \[
        \begin{vmatrix}
            a_1 & a_2 & a_3 \\
            b_1 & b_2 & b_3 \\
            c_1 & c_2 & c_3
        \end{vmatrix}
        \neq 0
    \]
}
\cor{}{向量组线性关系的行列式判断}{
    $\bm{\alpha}_1 = \left(a_{11},a_{21},
        \cdots,a_{n1}\right),\bm{\alpha}_2=
        \left(a_{12},a_{22},\cdots,a_{2n}\right),\cdots,\bm{\alpha}_n
        =\left(a_{1n},a_{2n},\cdots,a_{nn}\right)
        \in \bbk^n$线性无关等价于
    \[
        \left|\bm{\alpha}_1,\bm{\alpha}_2,\cdots
        ,\bm{\alpha}_n\right|=
        \begin{vmatrix}
            a_{11} & a_{12} & \cdots & a_{1n} \\
            a_{21} & a_{22} & \cdots & a_{2n} \\
            \vdots & \vdots & s      & \vdots \\
            a_{n1} & a_{n2} & \cdots & a_{nn}
        \end{vmatrix}
        \neq 0
    \]
    即是矩阵可逆,也就是满秩.反之则是缺秩,对应线性相关.
}