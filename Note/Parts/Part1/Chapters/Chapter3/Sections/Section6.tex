\newpage
\section{矩阵的秩}
\subsection{定义}
\dfn{行秩与列秩}{行秩与列秩}{
    设矩阵$\bm{A}_{m \times n}$,有行分块\[
        \bm{A}=\begin{pmatrix}
            \bm{\alpha}_1 \\
            \bm{\alpha}_2 \\
            \vdots        \\
            \bm{\alpha}_m
        \end{pmatrix}
    \]
    和列分块
    \[
        \bm{A}=\left(
        \bm{\beta}_1,\bm{\beta}_2,\cdots,\bm{\beta}_n
        \right)
    \]
    那么这$m$个行向量组成的向量组的秩$\rank \left(
        \bm{\alpha}_1,\bm{\alpha}_2,\cdots,\bm{\alpha}_m
        \right)$称为矩阵$\bm{A}$的行秩,$n$个列向量组成的向量组的秩$\rank \left(
        \bm{\beta}_1,\bm{\beta}_2,\cdots,\bm{\beta}_n
        \right)$称为矩阵$\bm{A}$的列秩.
}
\lem{}{极大无关组的指标保持}{
    设矩阵$\bm{A}_{m \times n} =
        \left(\bm{\beta}_1,\bm{\beta}_2,\cdots,
        \bm{\beta}_n\right)$,$\bm{Q}$为
    $m$阶非异阵,若$\left\{\bm{\beta}
        _{i_1},\cdots,
        \bm{\beta}_{i_r}\right\}$是$\bm{A}$
    的列向量的极大无关组,则$
        \left\{
        \bm{Q}\bm{\beta}_{i_1},\bm{Q\beta}_{i_2}
        ,\cdots,
        \bm{Q}\bm{\beta}_{i_r}
        \right\}$是$\bm{QA}
        = \left(\bm{Q}\bm{\beta}_1,\bm{Q\beta}_{2},\cdots
        ,\bm{Q}\bm{\beta}_n\right)$
    的列向量的极大无关组.

    即初等行变换保持矩阵列向量极大无关组的列指标.
    \begin{proof}
        参考\cref{thm:初等变换不改变秩}的证明.
    \end{proof}
}
\thm{初等变换不改变秩}{初等变换不改变秩}{
    初等变换不改变矩阵的行秩和列秩.同时,分块初等变换也不改变分块矩阵的秩.
    \begin{proof}
        下面证明列秩.

        记$\rmr _c\left(\bm{A}\right)=
            \rmr \left(\bm{\beta}_1,\bm{\beta}_2,
            \cdots,\bm{\beta}_n\right)$为$\bm{A}$列秩.

        先证明初等列变换,设初等矩阵$\bm{Q}$,因为
        \begin{enumerate}[label=\arabic*)]
            \item $\bm{AP}_{ij}=\left(\bm{\beta}_1,\cdots,\bm{\beta}_j,\cdots,\bm{\beta}_i,\cdots,\bm{\beta}_n\right)$
            \item $\bm{AP}_i\left(c\right)=\left(\bm{\beta}_1,\cdots,c\bm{\beta}_i,\cdots,\bm{\beta}_n\right)\left(c \neq 0\right)$
            \item $\bm{AT}_{ji}\left(c\right)=\left(\bm{\beta}_1,\cdots,\bm{\beta}_i,\cdots,\bm{\beta}_j+c\bm{\beta}_i,\cdots,\bm{\beta}_n\right)$
        \end{enumerate}

        容易看出$\bm{AQ}$的列向量均为$\bm{A}$的列向量的线性组合.我们考虑
        \[
            \bm{A}=\left(\bm{AQ}\right)\bm{Q}^{-1}
        \]
        那么上面的论断反过来也是正确的.
        从而$\bm{A}$和$\bm{AQ}$的列向量组一定
        是等价的.于是
        $\rmr _c\left(\bm{A}\right)=\rmr _c\left(\bm{AQ}\right)$,证毕.

        然后证明初等行变换.我们证明更强的结论即\cref{lem:极大无关组的指标保持}.

        首先证明$\bm{Q\beta}_{i_1},\bm{Q\beta}_{i_2},\cdots,\bm{Q\beta}_{i_r}$线性无关.
        设
        \[
            \lambda_1\bm{Q\beta}_{i_1}+\lambda_2\bm{Q\beta}_{i_2}+\cdots+\lambda_r\bm{Q\beta}_{i_r}=\bm{0}
        \]
        即
        \[
            \bm{Q}\left(\lambda_1\bm{\beta}_{i_1}+\lambda_2\bm{\beta}_{i_2}+\cdots+\lambda_r\bm{\beta}_{i_r}\right)=\bm{0}
        \]
        此处$\bm{Q}$非异,可约
        \[
            \lambda_1\bm{\beta}_{i_1}+\lambda_2\bm{\beta}_{i_2}+\cdots+\lambda_r\bm{\beta}_{i_r}=\bm{0}
        \]
        于是$\lambda_1=\lambda_2=\cdots=\lambda_r=0$.于是结论成立.

        再证$\bm{Q\beta}_j$均为$\bm{Q\beta}_{i_1},\bm{Q\beta}_{i_2},\cdots,\bm{Q\beta}_{i_r}$的
        线性组合.因为$\bm{\beta}_j$是
        $\bm{\beta}_{i_1},\bm{\beta}_{i_2},\cdots,\bm{\beta}_{i_r}$
        的线性组合.

        \cref{lem:极大无关组的指标保持}证毕.

        令$\bm{Q}$为初等矩阵,由\cref{lem:极大无关组的指标保持}立刻得证
        $\rmr _c\left(\bm{A}\right)=
            \rmr _c\left(\bm{QA}\right)$.注意此时考虑特例
        $\bm{A}=\bm{O}$亦成立.

        综上,证毕.
    \end{proof}
}
\dfn{矩阵的秩}{矩阵的秩}{
    矩阵的行秩等于列秩,称为矩阵的秩,记作$\rmr\left(\bm{A}\right)$或$\rank\left(\bm{A}\right)$.\begin{proof}
        因为任一矩阵均有相抵标准型
        \[
            \begin{pmatrix}
                \bm{I}_r & \bm{O} \\
                \bm{O}   & \bm{O}
            \end{pmatrix}
        \]再结合\cref{thm:初等变换不改变秩},证毕.
    \end{proof}
}
\cor{}{与可逆阵积的秩}{
    $\bm{A} \in
        M_{m \times n}\left(
        \bbk
        \right)$,$\bm{P}$、
    $\bm{Q}$分别为$m$、$n$阶可逆阵,则\[
        \rmr \left(\bm{PAQ}
        \right) = \rmr \left(
        \bm{A}
        \right)
    \]\begin{proof}
        考虑到\cref{thm:初等变换与初等矩阵},再根据\cref{thm:初等变换不改变秩},证毕.
    \end{proof}
}
\cor{}{相抵的矩阵具有相同的秩}{
    设$\bm{A}$、$\bm{B}$为$m$阶矩阵,$\bm{A} \sim \bm{B}$,则\[
        \rank\left(\bm{A}\right) = \rank \left(\bm{B}\right)
    \]
}
\thm{}{矩阵性质与秩}{
    特别地,对于$n$阶方阵$\bm{A}$
    \[
        \left|\bm{A}\right| \neq 0
        \Longleftrightarrow
        \bm{A}\text{可逆}\Longleftrightarrow
        \rmr \left(\bm{A}\right)
        = n\Longleftrightarrow
        \bm{A}\text{是满秩阵}\begin{cases*}
            \text{行满秩}\Longleftrightarrow
            \text{行向量线性无关} \\
            \text{列满秩}\Longleftrightarrow
            \text{列向量线性无关}
        \end{cases*}
    \]
}
\lem{极大无关组的判定定理}{极大无关组的判定定理}{
    设$\rank \left(\bm{A}\right)=r$,
    $\bm{A}=\left\{\bm{\beta}_1,\bm{\beta}_2,\cdots,
        \bm{\beta}_n\right\}$,当向量组$\left\{
        \bm{\beta}_{i_1},\bm{\beta}_{i_2},\cdots,
        \bm{\beta}_{i_r}\right\}$满足下面之一:
    \begin{enumerate}[label=\arabic*)]
        \item $\bm{\beta}_{i_1},\bm{\beta}_{i_2},\cdots,
                  \bm{\beta}_{i_r}$线性无关.
        \item $\forall \bm{\beta}_j$都是
              $\bm{\beta}_{i_1},\bm{\beta}_{i_2},\cdots,
                  \bm{\beta}_{i_r}$的线性组合.
    \end{enumerate}
    那么$\bm{\beta}_{i_1},\bm{\beta}_{i_2},\cdots,
        \bm{\beta}_{i_r}$是$\bm{A}$的极大无关组.
}
\lem{}{阶梯形与秩}{
设$\bm{A}$为阶梯形矩阵,$a_{1k_1},a_{2k_2},\cdots,
    a_{rk_r}$是阶梯点,则$\rank \left(\bm{A}\right)=r=\bm{A}$的非零行数, 且阶梯点所在列向量构成$\bm{A}$的列向量的极大无关组.
\begin{proof}
    设阶梯形矩阵
    \[
        \bm{A}=
        \begin{bmatrix}
            0      & \cdots & 0      & a_{1k_1} & *        & *        & \cdots &        &        & \\
            0      & \cdots & \cdots & 0        & a_{2k_2} & *        & *      & \cdots &        & \\
            \cdots & \cdots & \cdots & \cdots   & \cdots   &          &        &        &        & \\
            0      & \cdots & \cdots & \cdots   & 0        & a_{rk_r} & *      & *      & \cdots & \\
                   &        &        &          &          &          &        &        &        & \\
        \end{bmatrix}
    \]
    明显可以化为相抵标准型\[
        \begin{pmatrix}
            \bm{I}_r & \bm{O} \\
            \bm{O}   & \bm{O}
        \end{pmatrix}
    \]
    那么得到秩.然后将阶梯点所在列向量取出构成新矩阵
    \[
        \bm{B}=
        \begin{bmatrix}
            a_{1k_1} & *        & *      & *        \\
                     & a_{2k_2} & *      & *        \\
                     &          & \ddots & *        \\
                     &          &        & a_{rk_r} \\
                     &          &        &          \\
        \end{bmatrix}
    \]
    该矩阵仍为阶梯形,$r$个列向量线性无关.由\cref{lem:极大无关组的判定定理}知此为极大无关组.
\end{proof}
}
\subsection{求秩相关问题}
\exa{}{}{
    常见的三类问题为:

    \paragraph{Case\,1}
    求矩阵$\bm{A}$的秩及列向量极大无关组的方法.
    \begin{enumerate}[label=\arabic*)]
        \item 施加初等行变换化为阶梯形矩阵$\bm{B}
                  ,b_{1k_{1}},b_{2k_2},\cdots,b_{rk_{r}}$
              为阶梯点
        \item $\rmr \left(\bm{A}\right)
                  =\rmr \left(\bm{B}\right)
                  =\bm{B}$的非零行数$ = r$
        \item $\bm{A} = \left(\bm{\beta}_1,\bm{\beta}_2,
                  \cdots,\bm{\beta}_n\right)$
              的极大无关组为
              $\left(
                  \bm{\beta}_{k_1},\cdots,\bm{\beta}_{k_r}
                  ,\bm{\beta}_{k_r}
                  \right)$
    \end{enumerate}

    \paragraph{Case\,2}
    行、列向量组秩的计算及线性关系判定.
    \begin{enumerate}[label=\arabic*)]
        \item 把向量组拼成矩阵求秩$\rmr \left(\bm{A}
                  \right)$
        \item $\displaystyle
                  \rmr \left(\bm{A}\right)
                  \begin{cases*}
                      = \text{向量个数}\Longrightarrow \text{
                          线性无关
                      } \\
                      < \text{向量个数}\Longrightarrow
                      \text{线性相关(阶梯形矩阵至少有一行零)}
                  \end{cases*}
              $
    \end{enumerate}

    \paragraph{Case\,3}
    求行、列向量组的一组极大无关组.
    \begin{enumerate}[label=\arabic*)]
        \item 按照列向量(行向量需转置)方式拼成矩阵
              $\bm{A}$
        \item 求出秩和极大无关组
    \end{enumerate}
}
\subsection{子式判别法}
\lem{}{截断的线性关系}{
    设$\bm{A}\in M_{m\times n}\left(\bbk \right),\bm{A}=\left\{
        \bm{\beta}_1,\bm{\beta}_2,\cdots,\bm{\beta}_n
        \right\}$线性相关,且$0 < r < m$,定义截断$\tau_{\leqslant r}\bm{\beta}=\left(
        a_1,a_2,\cdots,a_r
        \right)
    $,那么这些截断组成的向量组也是线性相关的即
    \[\left\{\tau_{\leqslant r}\bm{\beta}_1,\tau_{\leqslant r}\bm{\beta}_2,\cdots,
        \tau_{\leqslant r}\bm{\beta}_n\right\}\]线性相关.
}
\thm{子式判别法}{子式判别法}{
    设$\bm{A} \in M_{m \times n}
        \left(\bbk \right)$
    \[
        \rmr \left(\bm{A}\right)
        = r \Longleftrightarrow
        \bm{A}\text{有一个非零}
        r\text{阶子式且所有}r+1
        \text{阶子式均为零}
    \]\begin{proof}
        先证必要性,$\rmr \left(\bm{A}\right)=r$,不妨设前$r$行线性无关.取出为
        \[
            \bm{B}=\begin{pmatrix}
                \bm{\alpha}_1 \\
                \bm{\alpha}_2 \\
                \vdots        \\
                \bm{\alpha}_r
            \end{pmatrix}
            \Longrightarrow \rmr \left(\bm{B}\right)=\bm{B}\text{行秩}=r
        \]
        那么列秩也等于$r$.不妨设前$r$列线性无关.
        取出为
        \[
            \bm{C}
            =
            \begin{pmatrix}
                a_{11} & a_{12} & \cdots & a_{1r} \\
                a_{21} & a_{22} & \cdots & a_{2r} \\
                \vdots & \vdots &        & \vdots \\
                a_{r1} & a_{r2} & \cdots & a_{rr}
            \end{pmatrix}\Longrightarrow \rmr \left(\bm{C}\right)=r
        \]
        故$\bm{C}$满秩,那么它非异.于是证明了存在这样一个非零
        $r$阶子式.

        再来证明$r+1$阶子式为零,不妨证
        \[
            \bm{A}\begin{pmatrix}
                1 & 2 & \cdots & r & r+1 \\
                1 & 2 & \cdots & r & r+1
            \end{pmatrix}=0
        \]
        因为$\rmr \left(\bm{A}\right)=r$,所以$\bm{\alpha}_1,\bm{\alpha}_{2},\cdots,\bm{\alpha}_{r+1}$线
        性相关.设$\bm{\alpha}=\left(a_1,a_2,\cdots,a_n\right)\in \bbk _n$,定义一个切断$\tau _{\leqslant r}\bm{\alpha}=\left(a_1,a_2,\cdots,
            a_r\right)$,那么由\cref{lem:截断的线性关系}得知
        \[
            \tau_{\leqslant r+1}\bm{\alpha}_1,\tau_{\leqslant r+1}\bm{\alpha}_2,
            \cdots,\tau_{\leqslant r+1}\bm{\alpha}_{r+1}
        \]
        线性相关.又
        \[
            \begin{pmatrix}
                \tau_{\leqslant r+1}\bm{\alpha}_1 \\
                \tau_{\leqslant r+1}\bm{\alpha}_2 \\
                \vdots                            \\
                \tau_{\leqslant r+1}\bm{\alpha}_{r+1}
            \end{pmatrix}=
            \begin{pmatrix}
                a_{11}    & a_{12}    & \cdots & a_{1,r+1}   \\
                a_{21}    & a_{22}    & \cdots & a_{2,r+1}   \\
                \vdots    & \vdots    &        & \vdots      \\
                a_{r+1,1} & a_{r+1,2} & \cdots & a_{r+1,r+1}
            \end{pmatrix}
        \]
        这个矩阵缺秩.于是
        \[
            \bm{A}\begin{pmatrix}
                1 & 2 & \cdots & r & r+1 \\
                1 & 2 & \cdots & r & r+1
            \end{pmatrix}=\left|\bm{C}\right|=0
        \]
        从而必要性证毕,下面证明充分性.

        因为$r+1$阶子式全为零,用\cref{thm:Laplace定理}Laplace定理可以证明所有高于$r$阶的子式全为零.设$\rmr \left(\bm{A}\right)=t$,由上可知,$\bm{A}$有一个非零$t$阶子式,高于$t$阶子式均为零.

        首先,$t>r$时,显然与所有高于$r$阶子式为零矛盾.$t<r$类似.于是$t=r$,证毕.
    \end{proof}
}
\subsection{关于秩的式子}
\thm{}{分块对角阵和分块三角阵的秩}{
    \begin{enumerate}[label=\arabic*)]
        \item  \[
                  \bm{C} = \begin{pmatrix}
                      \bm{A} & \bm{O} \\
                      \bm{O} & \bm{B}
                  \end{pmatrix}
                  \Longrightarrow \rmr \left(\bm{C}
                  \right) = \rmr \left(\bm{A}
                  \right)+\rmr \left(\bm{B}
                  \right)
              \]\begin{proof}
                  设存在非异阵$\bm{P}_1$、$\bm{P}_2$、$\bm{Q}_1$、
                  $\bm{Q}_2$使得
                  \begin{align*}
                      \bm{P}_1\bm{AQ}_1=\begin{pmatrix}
                                            \bm{I}_{r_1} & \bm{O} \\
                                            \bm{O}       & \bm{O}
                                        \end{pmatrix} \\
                      \bm{P}_2\bm{BQ}_2=\begin{pmatrix}
                                            \bm{I}_{r_2} & \bm{O} \\\bm{O} & \bm{O}
                                        \end{pmatrix}
                  \end{align*}
                  于是
                  \begin{align*}
                      \begin{pmatrix}
                          \bm{P}_1 & \bm{O} \\\bm{O} & \bm{Q}_1
                      \end{pmatrix}
                      \begin{pmatrix}
                          \bm{A} & \bm{O} \\
                          \bm{O} & \bm{B}
                      \end{pmatrix}
                      \begin{pmatrix}
                          \bm{P}_2 & \bm{O} \\\bm{O} & \bm{Q}_2
                      \end{pmatrix}
                       & =\begin{pmatrix}
                              \bm{P}_1\bm{AQ}_1 & \bm{O}            \\
                              \bm{O}            & \bm{P}_2\bm{BQ}_2
                          \end{pmatrix}         \\
                       & =\begin{pmatrix}
                              \bm{I}_{r_1} & \bm{O} & \bm{O}       & \bm{O} \\
                              \bm{O}       & \bm{O} & \bm{O}       & \bm{O} \\
                              \bm{O}       & \bm{O} & \bm{I}_{r_2} & \bm{O} \\
                              \bm{O}       & \bm{O} & \bm{O}       & \bm{O}
                          \end{pmatrix}
                  \end{align*}
                  由于分块初等变换不改变分块矩阵的秩,上面的矩阵化为
                  \[
                      \begin{pmatrix}
                          \bm{I}_{r_1+r_2} & \bm{O} \\
                          \bm{O}           & \bm{O}
                      \end{pmatrix}
                  \]
                  于是得到$\rmr \left(\bm{C}
                      \right)=r_1+r_2$.
              \end{proof}
        \item \[
                  \bm{C}=
                  \begin{pmatrix}
                      \bm{A} & \bm{D} \\
                      \bm{O} & \bm{B}
                  \end{pmatrix}
                  \text{或}
                  \begin{pmatrix}
                      \bm{A} & \bm{O} \\
                      \bm{D} & \bm{B}
                  \end{pmatrix}
                  \Longrightarrow
                  \rmr \left(\bm{C}
                  \right) \geqslant  \rmr \left(\bm{A}
                  \right)+\rmr \left(\bm{B}
                  \right)
              \]
              等号当且仅当矩阵方程
              \[
                  \bm{AX}+\bm{BY} = \bm{D}
              \]
              有解时取到\begin{proof}
                  仿照作
                  \begin{align*}
                      \begin{pmatrix}
                          \bm{P}_1 & \bm{O} \\\bm{O} & \bm{Q}_1
                      \end{pmatrix}
                      \begin{pmatrix}
                          \bm{A} & \bm{D} \\
                          \bm{O} & \bm{B}
                      \end{pmatrix}
                      \begin{pmatrix}
                          \bm{P}_2 & \bm{O} \\\bm{O} & \bm{Q}_2
                      \end{pmatrix}
                       & =\begin{pmatrix}
                              \bm{P}_1\bm{AQ}_1 & \bm{P}_1\bm{DQ}_2 \\
                              \bm{O}            & \bm{P}_2\bm{BQ}_2
                          \end{pmatrix}              \\
                       & =\begin{pmatrix}
                              \bm{I}_{r_1} & \bm{O} & \bm{D}_{11}  & \bm{D}_{12} \\
                              \bm{O}       & \bm{O} & \bm{D}_{21}  & \bm{D}_{22} \\
                              \bm{O}       & \bm{O} & \bm{I}_{r_2} & \bm{O}      \\
                              \bm{O}       & \bm{O} & \bm{O}       & \bm{O}
                          \end{pmatrix} \\
                       & \longrightarrow
                      \begin{pmatrix}
                          \bm{I}_{r_1} & \bm{O} & \bm{O}       & \bm{O}      \\
                          \bm{O}       & \bm{O} & \bm{O}       & \bm{D}_{22} \\
                          \bm{O}       & \bm{O} & \bm{I}_{r_2} & \bm{O}      \\
                          \bm{O}       & \bm{O} & \bm{O}       & \bm{O}
                      \end{pmatrix}     \\
                       & \longrightarrow
                      \begin{pmatrix}
                          \bm{I}_{r_1} & \bm{O}      & \bm{O}       & \bm{O} \\
                          \bm{O}       & \bm{D}_{22} & \bm{O}       & \bm{O} \\
                          \bm{O}       & \bm{O}      & \bm{I}_{r_2} & \bm{O} \\
                          \bm{O}       & \bm{O}      & \bm{O}       & \bm{O}
                      \end{pmatrix}
                  \end{align*}
                  得证.取等条件为$\bm{D}_{22}=\bm{O}$,经过优化为当且仅当矩阵方程
                  \[
                      \bm{AX}+\bm{YB}=\bm{D}\]
                  有解.
              \end{proof}
    \end{enumerate}
}
\thm{秩的降阶公式}{秩的降阶公式}{
    设
    \[
        \bm{M} = \begin{pmatrix}
            \bm{A} & \bm{B} \\
            \bm{C} & \bm{D}
        \end{pmatrix}
    \]
    那么有
    \begin{enumerate}[label=\arabic*)]
        \item $\bm{A}$可逆则
              \[\rmr \left(\bm{M}\right)
                  = \rmr \left(\bm{A}\right)
                  + \rmr \left(
                  \bm{D} - \bm{C}\bm{A}^{-1}
                  \bm{B}
                  \right) \]
        \item $\bm{D}$可逆则\[
                  \rmr \left(\bm{M}\right)
                  =\rmr \left(\bm{D}\right)
                  +\rmr \left(
                  \bm{A}-
                  \bm{B}\bm{D}^{-1}\bm{C}
                  \right)\]
        \item $\bm{A}$、$\bm{D}$均可逆则\[
                  \rmr \left(\bm{A}\right)
                  + \rmr \left(
                  \bm{D} - \bm{C}\bm{A}^{-1}
                  \bm{B}
                  \right)=
                  \rmr \left(\bm{D}\right)
                  +\rmr \left(
                  \bm{A}-
                  \bm{B}\bm{D}^{-1}\bm{C}
                  \right)\]
    \end{enumerate}
    \begin{proof}
        仅证$(1)$.
        \begin{align*}
            \bm{M} & \longrightarrow\begin{pmatrix}
                                        \bm{A} & \bm{B}
                                        \\
                                        \bm{O} & \bm{D}-\bm{CA}^{-1}\bm{B}
                                    \end{pmatrix}
            \longrightarrow
            \begin{pmatrix}
                \bm{A} & \bm{O}                    \\
                \bm{O} & \bm{D}-\bm{CA}^{-1}\bm{B}
            \end{pmatrix}\qedhere
        \end{align*}
    \end{proof}
}
\thm{运算的秩}{运算的秩}{
    \begin{enumerate}[label=\arabic*)]
        \item  $\rmr \left(k\bm{A}\right)
                  =\rmr \left(\bm{A}\right)
                  \left(k \neq 0\right)$
        \item $\rmr \left(\bm{A}+\bm{B}
                  \right)\leqslant \rmr
                  \left(\bm{A}\right) + \rmr
                  \left(\bm{B}\right)$
        \item $\left|
                  \rmr \left(\bm{A}\right)
                  -
                  \rmr \left(\bm{B}\right)
                  \right|
                  \leqslant
                  \rmr \left(\bm{A}
                  -\bm{B}\right)
                  \leqslant
                  \rmr \left(\bm{A}\right)+
                  \rmr \left(\bm{B}\right)$
        \item $\rmr
                  \begin{pmatrix}
                      \bm{A} & \bm{B}
                  \end{pmatrix}\leqslant
                  \rmr \left(\bm{A}\right)+
                  \rmr \left(\bm{B}\right)$
        \item $\rmr
                  \begin{pmatrix}
                      \bm{A} \\\bm{B}
                  \end{pmatrix}
                  \leqslant
                  \rmr \left(\bm{A}\right)+
                  \rmr \left(\bm{B}\right)$
        \item $\bm{A} \in M_{m \times n}\left(
                  \bbk
                  \right) ,\bm{B}\in
                  M_{n \times p}
                  \left(\bbk \right)$,则\[
                  \rmr \left(\bm{A}\right)+
                  \rmr \left(\bm{B}\right)
                  - n\leqslant
                  \rmr \left(
                  \bm{AB}\right)
                  \leqslant
                  \min\left\{
                  \rmr \left(\bm{A}\right),
                  \rmr \left(\bm{B}\right)
                  \right\}\]
              左侧称为Sylvester不等式
        \item $\rmr \left(\bm{A}\right)
                  +\rmr \left(\bm{I}_n
                  +\bm{A}\right)\geqslant
                  n$
        \item 对于实矩阵$
                  \rmr \left(\bm{AA}^*\right)
                  =\rmr \left(\bm{A}^*\bm{A}
                  \right)=\rmr \left(\bm{A}\right)$
    \end{enumerate}
    \begin{proof}
        证明$(6)$.先证明上界,先证$\rmr \left(\bm{AB}\right)\leqslant \rmr \left(\bm{B}
            \right)$.

        令$\bm{B}=\left(
            \bm{\beta}_1,\bm{\beta}_2,\cdots,\bm{\beta}_p
            \right)$
        ,其极大无关组为$\left\{
            \bm{\beta}_{i_1},\bm{\beta}_{i_2},\cdots,\bm{\beta}_{i_r}
            \right\}$,可以断言$\forall 1\leqslant j\leqslant p$,
        $\bm{A\beta}_j$是$
            \bm{A\beta}_{i_1},\bm{A\beta}_{i_2},
            \cdots,\bm{A\beta}_{i_r}
        $的线性组合.前者正是$\bm{AB}$的列向量,于是
        $\rmr \left(\bm{AB}\right)\leqslant r=\rmr
            \left(\bm{B}\right).$一方面证毕.
        另一方面,考虑转置,$\rmr \left(\bm{AB}\right)=
            \rmr \left(\bm{B}^{\prime}\bm{A}^{\prime}\right)\leqslant
            \rmr \left(\bm{A}^{\prime}\right)=\rmr \left(\bm{A}
            \right)$.
        综上,上界证毕.下面证明下界Sylvester不等式.
        \begin{align*}
            \begin{pmatrix}
                \bm{A}    & \bm{O} \\
                -\bm{I}_n & \bm{B}
            \end{pmatrix}
             & \longrightarrow
            \begin{pmatrix}
                \bm{O}    & \bm{AB} \\
                -\bm{I}_n & \bm{B}
            \end{pmatrix} \\
             & \longrightarrow
            \begin{pmatrix}
                \bm{O}    & \bm{AB} \\
                -\bm{I}_n & \bm{O}
            \end{pmatrix} \\
             & \longrightarrow
            \begin{pmatrix}
                \bm{AB} & \bm{O}   \\
                \bm{O}  & \bm{I}_n
            \end{pmatrix}
        \end{align*}
        于是
        \[
            \rmr \left(\bm{A}\right)+\rmr \left(\bm{B}\right)\leqslant
            \rmr \left(\bm{AB}\right)+n\qedhere
        \]
    \end{proof}
}
\thm{伴随的秩}{伴随的秩}{
    对于矩阵$\bm{A}\in M_n\left(\bbk \right)$,其伴随矩阵$\bm{A}^*$的秩为\[
        \rmr \left(\bm{A}^*\right)=\begin{cases*}
            n & ,$\rmr \left(\bm{A}\right)=n$   \\
            1 & ,$\rmr \left(\bm{A}\right)=n-1$ \\
            0 & ,$\rmr \left(\bm{A}\right)<n-1$
        \end{cases*}
    \]\begin{proof}
        根据\cref{prop:伴随阵的性质总结}即得.
    \end{proof}
}
\exa{几种特殊的矩阵的秩}{}{
    \paragraph{幂等阵}幂等阵$\bm{A}$
    满足$\bm{A}^2=\bm{A}$,其秩满足
    \[
        \rmr \left(\bm{A}\right)
        +\rmr \left(\bm{I}_n
        -\bm{A}\right) =
        n
    \]\begin{proof}
        考虑
        \begin{align*}
            \begin{pmatrix}
                \bm{A} & \bm{O}          \\
                \bm{O} & \bm{I}_n-\bm{A}
            \end{pmatrix} & \longrightarrow
            \begin{pmatrix}
                \bm{A} & \bm{O}          \\
                \bm{A} & \bm{I}_n-\bm{A}
            \end{pmatrix}                      \\
                                        & \longrightarrow
            \begin{pmatrix}
                \bm{A} & \bm{A}   \\
                \bm{A} & \bm{I}_n
            \end{pmatrix}                             \\
                                        & \longrightarrow
            \begin{pmatrix}
                \bm{A}-\bm{A}^2 & \bm{O}   \\
                \bm{A}          & \bm{I}_n
            \end{pmatrix}                    \\
                                        & \longrightarrow
            \begin{pmatrix}
                \bm{A}-\bm{A}^2 & \bm{O}   \\
                \bm{O}          & \bm{I}_n
            \end{pmatrix}                    \\
                                        & \longrightarrow
            \begin{pmatrix}
                \bm{I}_n & \bm{O} \\\bm{O} & \bm{O}
            \end{pmatrix}\qedhere
        \end{align*}
    \end{proof}
    \paragraph{对合阵}对合阵$\bm{A}$满足$\bm{A}^2 = \bm{I}_n$,其秩满足
    \[
        \rmr \left(
        \bm{I}_n - \bm{A}\right)
        +\rmr \left(\bm{I}_n
        -\bm{A}\right) =
        n
    \]
}