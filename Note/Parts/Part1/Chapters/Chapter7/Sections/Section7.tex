\newpage
\section{矩阵函数}
\subsection{级数相关知识}
\clm{}{}{
    定义矩阵函数至少有三种方法:其一,将函数作幂级数展开并利用Jordan标准型理论定义;其二,利用多项式定义;其三,利用Cauchy积分定理定义.这里采用第一种定义.

    这里我们主要讨论级数(无穷),这与多项式(有限多项)有些许相似但并不相同.
}
下面先给出一些级数相关的定义和定理.
\dfn{级数收敛}{级数收敛}{
    无穷级数$\displaystyle \sum_{n=1}^{\infty}a_n$收敛当且仅当部分和数列$\left\{s_n\right\}$收敛,其中$s_n=a_1+a_2+\cdots+a_n.$若\[
        \lim_{n\to \infty}s_n=S
    \]则记级数$\displaystyle \sum_{n=1}^{\infty}a_n=S$.
}
\thm{Cauchy审敛法}{Cauchy审敛法}{
    考虑正项级数$\displaystyle \sum_{n=1}^{\infty}a_n$,令$\displaystyle r=\overline{\lim_{n\to\infty}}\sqrt[n]{a_n}$则\begin{enumerate}[label=\arabic*)]
        \item $\displaystyle \sum_{n=1}^{\infty}a_n$收敛当且仅当$r<1$
        \item $\displaystyle \sum_{n=1}^{\infty}a_n$发散当且仅当$r>1$
        \item 当$r=1$时无法判断
    \end{enumerate}
}
\dfn{幂级数收敛半径}{幂级数收敛半径}{
    考虑幂级数$\displaystyle \sum_{n=0}^{\infty}a_nx^n$,令\[A=\overline{\lim_{n\to\infty}}\sqrt[n]{\left|a_n\right| }\]定义该幂级数收敛半径\[
        R=\begin{cases*}
            +\infty      & ,$A=0$         \\
            \dfrac{1}{A} & ,$0<A<+\infty$ \\
            0            & ,$A=+\infty$
        \end{cases*}
    \]
}
\thm{Cauchy-Hadamard定理}{Cauchy-Hadamard定理}{
    在\cref{def:幂级数收敛半径}的基础上,幂级数$\displaystyle \sum_{n=0}^{\infty}a_nx^n$的敛散性为\begin{enumerate}[label=\arabic*)]
        \item $\left|x\right|<R$时,$\displaystyle \sum_{n=0}^{\infty}a_nx^n$绝对收敛
        \item $\left|x\right|>R$时,$\displaystyle \sum_{n=0}^{\infty}a_nx^n$发散
        \item $\left|x\right|=R$时,$\displaystyle \sum_{n=0}^{\infty}a_nx^n$敛散性不确定
    \end{enumerate}
}
\thm{逐项可导性}{逐项可导性}{
幂级数$\displaystyle f\left(x\right)=\sum_{n=0}^{\infty}a_nx^n$在$\left|x\right|<R$时逐项可导即\[
    f'\left(x\right)=\frac{\rmd }{\rmd x}\left(
    \sum_{n=0}^{\infty}a_nx^n
    \right)=\sum_{n=0}^{\infty}\frac{
        \rmd
    }{\rmd x}\left(
    a_nx^n
    \right)=\sum_{n=0}^{\infty}na_nx^{n-1}
\]并且$\displaystyle f'\left(x\right)=
    \sum_{n=0}^{\infty}na_nx^{n-1}
$的收敛半径也是$R$.
}
\rem{}{}{
    下面考虑复幂级数
    \[
        f\left(z\right)=\sum_{n=0}^{\infty}a_nz^n
        ,a_i\in\bbc ,z\in\bbc
    \]值得注意的是$\left|z\right|$表示复数的模长而不是实数的绝对值.其余表述与实幂级数类似.
}
\pro{常见复幂级数及其收敛半径}{常见复幂级数及其收敛半径}{
    一些常见的复幂级数及其收敛半径如下

    \begin{enumerate}[label=\arabic*)]
        \item $\displaystyle \rme ^z=1+z+\frac{z^2}{2!}+\frac{z^3}{3!}+\cdots,R=+\infty$
        \item $\displaystyle \sin z=z-\frac{z^3}{3!}
                  +\frac{z^5}{5!}-\frac{z^7}{7!}+\cdots,R=+\infty
              $
        \item $\displaystyle \cos z=1-\frac{z^2}{2!}+\frac{z^4}{4!}-\frac{z^6}{6!}+\cdots,R=+\infty$
        \item $\ln \left(1+z\right)=
                  z-\frac{z^2}{2}+\frac{z^3}{3}-\frac{z^4}{4}+\cdots,R=1
              $
    \end{enumerate}
}
\subsection{矩阵幂级数}
考虑$f\left(z\right)=a_0+a_1z+\cdots+a_pz^p\in \bbc \left[z\right]$,矩阵$\bm{A}\in M_n\left(\bbc \right)$则\[
    f\left(\bm{A}\right)\coloneqq a_0\bm{I}_n+a_1\bm{A}+\cdots+a_p\bm{A}^p
\]

设非异阵$\bm{P}$使得\[
    \bm{P}^{-1}\bm{A}\bm{P}=\bm{J}= \diag \,\left\{
    \bm{J}_{r_1}\left(\lambda_1\right),\bm{J}_{r_2}\left(\lambda_2\right),\cdots,\bm{J}_{r_k}\left(\lambda_k\right)
    \right\}
\]并且$f\left(\bm{A}\right)=\bm{P}f\left(\bm{J}\right)\bm{P}^{-1}=\bm{P}\diag \left\{
    f\left(
    \bm{J}_{r_1}\left(\lambda_1\right)
    \right),f\left(
    \bm{J}_{r_2}\left(\lambda_2\right)
    \right),\cdots,f\left(
    \bm{J}_{r_k}\left(\lambda_k\right)
    \right)
    \right\}\bm{P}^{-1}$.

于是考虑每一个Jordan块,由\cref{thm:Jordan-Chevalley分解定理}知有分解$\bm{J}_{r_i}\left(\lambda_i\right)=\lambda_i\bm{I}+\bm{N}$,其中$\bm{N}=\bm{J}_{r_i}\left(0\right),\bm{N}^{r_i}=\bm{O}$ .

考虑$f\left(z\right)$在$\lambda_i$处的Taylor展开\[
    f\left(z\right)=f\left(\lambda_i\right)+\frac{f'\left(\lambda_i\right)}{1!}\left(z-\lambda_i\right)+\frac{f''\left(\lambda_i\right)}{2!}\left(z-\lambda_i\right)^2+\cdots+\frac{
        f^{\left(p\right)}\left(\lambda_i\right)
    }{p!}\left(z-\lambda_i\right)^p
\]代入$\bm{J}_{r_i}\left(\lambda_i\right)=\lambda_i\bm{I}+\bm{N}$得到\begin{align*}
    f\left(
    \bm{J}_{r_i}\left(
        \lambda_i
        \right)
    \right) & =f\left(\lambda_i\right)\bm{I}+\frac{f'\left(\lambda_i\right)}{1!}\bm{N}+\frac{f''\left(\lambda_i\right)}{2!}\bm{N}^2+\cdots+\frac{
        f^{\left(r_i-1\right)}\left(\lambda_i\right)
    }{\left(r_i-1\right)!}\bm{N}^{r_i-1}                                                                                                                                                                              \\
            & =\begin{pmatrix}
                   f\left(\lambda_i\right) & \cfrac{f'\left(\lambda_i\right)}{1!} & \cfrac{f''\left(\lambda_i\right)}{2!} & \cdots & \cfrac{f^{\left(r_i-1\right)}\left(\lambda_i\right)}{\left(r_i-1\right)!} \\
                                           & f\left(\lambda_i\right)              & \cfrac{f'\left(\lambda_i\right)}{1!}  & \cdots & \cfrac{f^{\left(r_i-2\right)}\left(\lambda_i\right)}{\left(r_i-2\right)!} \\
                                           &                                      & f\left(\lambda_i\right)               & \cdots & \cfrac{f^{\left(r_i-3\right)}\left(\lambda_i\right)}{\left(r_i-3\right)!} \\
                                           &                                      &                                       & \ddots & \vdots                                                                    \\
                                           &                                      &                                       &        & f\left(\lambda_i\right)
               \end{pmatrix}
\end{align*}
\dfn{矩阵幂级数收敛}{矩阵幂级数收敛}{
设矩阵序列$\left\{\bm{A}_p\right\}$,其中\[
    \bm{A}_p=\begin{pmatrix}
        a_{11}^{\left(p\right)} & a_{12}^{\left(p\right)} & \cdots & a_{1n}^{\left(p\right)} \\
        a_{21}^{\left(p\right)} & a_{22}^{\left(p\right)} & \cdots & a_{2n}^{\left(p\right)} \\
        \vdots                  & \vdots                  &        & \vdots                  \\
        a_{n1}^{\left(p\right)} & a_{n2}^{\left(p\right)} & \cdots & a_{nn}^{\left(p\right)}
    \end{pmatrix}
\]和矩阵$\bm{B}=\left(b_{ij}\right)_{n\times n}.$若$\forall 1\leqslant i,j\leqslant n,$数列$\left\{
    a_{ij}^{\left(p\right)}
    \right\}$收敛到$b_{ij}$即$\displaystyle \lim_{p\to\infty}a_{ij}^{\left(p\right)}=b_{ij},$则称矩阵序列$\left\{\bm{A}_p\right\}$收敛到矩阵$\bm{B}$,记作$\displaystyle \lim_{p\to\infty}\bm{A}_p=\bm{B},$否则称矩阵序列$\left\{\bm{A}_p\right\}$发散.
}
\exa{}{}{
    \[
        \bm{A}_p=\begin{pmatrix}
            1+\cfrac{1}{p} & \cfrac{1}{p^2} \\
            2              & 3
        \end{pmatrix}
    \]收敛于\[
        \bm{B}=\begin{pmatrix}
            1 & 0 \\
            2 & 3
        \end{pmatrix}
    \]
}
\dfn{部分和定义收敛}{部分和定义收敛}{
设复幂级数$\displaystyle f\left(z\right)=\sum_{i=0}^{\infty}a_iz^i,$其部分和\[
    f_p\left(z\right)=a_0+a_1z+\cdots+a_pz^p
\]设$\bm{A}\in M_n\left(\bbc \right)$得到矩阵序列$\displaystyle \left\{
    f_p\left(\bm{A}\right)
    \right\}_{p=0}^{\infty}$.若该矩阵序列收敛到矩阵$\bm{B}$,则矩阵幂级数\[
    \sum_{i=0}^{\infty}a_i\bm{A}^i=a_0\bm{I}_n+a_1\bm{A}+a_2\bm{A}^2+\cdots
\]收敛到$\bm{B}$,记作$f\left(\bm{A}\right)=\bm{B}.$
}
\dfn{矩阵函数}{矩阵函数}{
    将\cref{def:部分和定义收敛}中取未定元$\bm{X}$则得到一般的矩阵函数\[
        f\left(\bm{X}\right)=\sum_{i=0}^{\infty}a_i\bm{X}^i=a_0\bm{I}_n+a_1\bm{X}+a_2\bm{X}^2+\cdots
    \]
}
\thm{}{矩阵幂级数收敛的等价命题}{
    设复幂级数$\displaystyle f\left(z\right)=\sum_{i=0}^{\infty}a_iz^i$,收敛半径$R$,则\begin{enumerate}[label=\arabic*)]
        \item $f\left(\bm{X}\right)$收敛当且仅当对于任意非异阵$\bm{P}$,$f\left(
                  \bm{P}^{-1}\bm{X}\bm{P}
                  \right)$都收敛并且有\[
                  f\left(\bm{P}^{-1}\bm{X}\bm{P}\right)=\bm{P}^{-1}f\left(\bm{X}\right)\bm{P}
                  \Longrightarrow f\left(\bm{X}\right)=\bm{P}f\left(
                  \bm{P}^{-1}\bm{X}\bm{P}
                  \right)\bm{P}^{-1}
              \]
        \item 设\[\bm{X}=\diag \left\{
                  \bm{X}_1,\bm{X}_2,\cdots,\bm{X}_k
                  \right\}\]则$f\left(\bm{X}\right)$收敛当且仅当$f\left(
                  \bm{X}_1
                  \right),f\left(
                  \bm{X}_2
                  \right),\cdots,f\left(
                  \bm{X}_k
                  \right)$收敛并且\[
                  f\left(\bm{X}\right)=\diag \,\left\{
                  f\left(
                  \bm{X}_1
                  \right),f\left(
                  \bm{X}_2
                  \right),\cdots,f\left(
                  \bm{X}_k
                  \right)
                  \right\}
              \]
        \item 对Jordan块$\bm{J}_{r_i}\left(\lambda_i\right)$,若$\left|\lambda_i\right|<R$,则$f\left(
                  \bm{J}_{r_i}\left(\lambda_i\right)
                  \right)$收敛到\[
                  f\left(
                  \bm{J}_{r_i}\left(\lambda_i\right)
                  \right)=\begin{pmatrix}
                      f\left(\lambda_i\right) & \cfrac{f'\left(\lambda_i\right)}{1!} & \cfrac{f''\left(\lambda_i\right)}{2!} & \cdots & \cfrac{f^{\left(r_i-1\right)}\left(\lambda_i\right)}{\left(r_i-1\right)!} \\
                                              & f\left(\lambda_i\right)              & \cfrac{f'\left(\lambda_i\right)}{1!}  & \cdots & \cfrac{f^{\left(r_i-2\right)}\left(\lambda_i\right)}{\left(r_i-2\right)!} \\
                                              &                                      & f\left(\lambda_i\right)               & \cdots & \cfrac{f^{\left(r_i-3\right)}\left(\lambda_i\right)}{\left(r_i-3\right)!} \\
                                              &                                      &                                       & \ddots & \vdots                                                                    \\
                                              &                                      &                                       &        & f\left(\lambda_i\right)
                  \end{pmatrix}
              \]
    \end{enumerate}\begin{proof}
        设部分和\[
            f_p\left(z\right)=a_0+a_1z+\cdots+a_pz^p
        \]

        \begin{enumerate}[label=\arabic*)]
            \item 因为$f\left(
                      \bm{P}^{-1}\bm{X}\bm{P}
                      \right)=
                      \bm{P}^{-1}f\left(\bm{X}\right)\bm{P}
                  $而$f\left(\bm{X}\right)$收敛当且仅当$\displaystyle \lim_{p\to\infty}f_p\left(\bm{X}\right)$存在,于是这又当且仅当$\displaystyle \lim_{p\to\infty}\bm{P}^{-1}f_p\left(\bm{X}\right)\bm{P}$存在即$f\left(
                      \bm{P}^{-1}\bm{X}\bm{P}
                      \right)$收敛并且\begin{align*}
                      f\left(
                      \bm{P}^{-1}\bm{X}\bm{P}
                      \right)=\lim_{p\to\infty}\bm{P}^{-1}f_p\left(\bm{X}\right)\bm{P}=\bm{P}^{-1}\lim_{p\to\infty}f_p\left(\bm{X}\right)\bm{P}=\bm{P}^{-1}f\left(\bm{X}\right)\bm{P}
                  \end{align*}
            \item 首先\[
                      f_p\left(\bm{X}\right)=\diag \left\{
                      f_p\left(
                      \bm{X}_1
                      \right),f_p\left(
                      \bm{X}_2
                      \right),\cdots,f_p\left(
                      \bm{X}_k
                      \right)
                      \right\}
                  \]因为$f\left(\bm{X}\right)$收敛当且仅当$\displaystyle \lim_{p\to\infty}f_p\left(\bm{X}\right)$存在当且仅当$\displaystyle \lim_{p\to\infty}f_p\left(
                      \bm{X}_i
                      \right)$存在$\left(\forall
                      1\leqslant i\leqslant k
                      \right)$即$f\left(
                      \bm{X}_1
                      \right),f\left(
                      \bm{X}_2
                      \right),\cdots,f\left(
                      \bm{X}_k
                      \right)$收敛并且\begin{align*}
                      f\left(\bm{X}\right) & =\lim_{p\to\infty}f_p\left(\bm{X}\right) \\
                                           & =\lim_{p\to\infty}\diag \left\{
                      f_p\left(
                      \bm{X}_1
                      \right),f_p\left(
                      \bm{X}_2
                      \right),\cdots,f_p\left(
                      \bm{X}_k
                      \right)
                      \right\}                                                        \\
                                           & =\diag \left\{
                      \lim_{p\to\infty}f_p\left(
                      \bm{X}_1
                      \right),\lim_{p\to\infty}f_p\left(
                      \bm{X}_2
                      \right),\cdots,\lim_{p\to\infty}f_p\left(
                      \bm{X}_k
                      \right)
                      \right\}                                                        \\
                                           & =\diag \left\{
                      f\left(
                      \bm{X}_1
                      \right),f\left(
                      \bm{X}_2
                      \right),\cdots,f\left(
                      \bm{X}_k
                      \right)
                      \right\}
                  \end{align*}
            \item 由上文计算结果知\[
                      f_p\left(
                      \bm{J}_{r_i}\left(\lambda_i\right)
                      \right)=\begin{pmatrix}
                          f_p\left(\lambda_i\right) & \cfrac{f_p'\left(\lambda_i\right)}{1!} & \cfrac{f_p''\left(\lambda_i\right)}{2!} & \cdots & \cfrac{f_p^{\left(r_i-1\right)}\left(\lambda_i\right)}{\left(r_i-1\right)!} \\
                                                    & f_p\left(\lambda_i\right)              & \cfrac{f_p'\left(\lambda_i\right)}{1!}  & \cdots & \cfrac{f_p^{\left(r_i-2\right)}\left(\lambda_i\right)}{\left(r_i-2\right)!} \\
                                                    &                                        & f_p\left(\lambda_i\right)               & \cdots & \cfrac{f_p^{\left(r_i-3\right)}\left(\lambda_i\right)}{\left(r_i-3\right)!} \\
                                                    &                                        &                                         & \ddots & \vdots                                                                      \\
                                                    &                                        &                                         &        & f_p\left(\lambda_i\right)
                      \end{pmatrix}
                  \]因为$f\left(\bm{J}_{r_i}\left(\lambda_i\right)\right)$收敛当且仅当$\displaystyle
                      \lim_{p\to\infty}f_p\left(
                      \bm{J}_{r_i}\left(\lambda_i\right)
                      \right)$存在,即等价于\[
                      \lim_{p\to\infty}f_p\left(
                      \lambda_i
                      \right),\lim_{p\to\infty}f_p'\left(
                      \lambda_i
                      \right),\cdots,\lim_{p\to\infty}f_p^{\left(r_i-1\right)}\left(
                      \lambda_i
                      \right)
                  \]均存在,因为$\left|\lambda_i\right|<R$,则\[
                      \lim_{p\to\infty}f_p\left(\lambda_i\right)=f\left(\lambda_i\right)
                  \]根据\cref{thm:逐项可导性}知\begin{align*}
                       & \lim_{p\to\infty}f_p'\left(\lambda_i\right)=f'\left(\lambda_i\right)                                         \\
                       & \lim_{p\to\infty}f_p''\left(\lambda_i\right)=f''\left(\lambda_i\right)                                       \\
                       & \cdots\cdots\cdots\cdots                                                                                     \\
                       & \lim_{p\to\infty}f_p^{\left(r_i-1\right)}\left(\lambda_i\right)=f^{\left(r_i-1\right)}\left(\lambda_i\right)
                  \end{align*}于是\begin{align*}
                      f\left(
                      \bm{J}_{r_i}\left(\lambda_i\right)
                      \right)=\lim_{p\to\infty}f_p\left(
                      \bm{J}_{r_i}\left(\lambda_i\right)
                      \right) & =\begin{pmatrix}
                                     f\left(\lambda_i\right) & \cfrac{f'\left(\lambda_i\right)}{1!} & \cfrac{f''\left(\lambda_i\right)}{2!} & \cdots & \cfrac{f^{\left(r_i-1\right)}\left(\lambda_i\right)}{\left(r_i-1\right)!} \\
                                                             & f\left(\lambda_i\right)              & \cfrac{f'\left(\lambda_i\right)}{1!}  & \cdots & \cfrac{f^{\left(r_i-2\right)}\left(\lambda_i\right)}{\left(r_i-2\right)!} \\
                                                             &                                      & f\left(\lambda_i\right)               & \cdots & \cfrac{f^{\left(r_i-3\right)}\left(\lambda_i\right)}{\left(r_i-3\right)!} \\
                                                             &                                      &                                       & \ddots & \vdots                                                                    \\
                                                             &                                      &                                       &        & f\left(\lambda_i\right)
                                 \end{pmatrix}\qedhere
                  \end{align*}
        \end{enumerate}
    \end{proof}
}
\thm{审敛法}{审敛法}{
    设复幂级数$\displaystyle f\left(z\right)=\sum_{i=0}^{\infty}a_iz^i$的收敛半径为$R$,$\bm{A}\in M_n\left(\bbc \right),$其谱集$\sigma\left(\bm{A}\right)=\left\{
        \lambda_1,\lambda_2,\cdots,\lambda_n
        \right\}$,其中$\lambda_i$为$\bm{A}$的特征值.定义谱半径$\displaystyle \rho\left(\bm{A}\right)=\max_{
            1\leqslant i\leqslant n
        }\left|\lambda_i\right|$.则
    \begin{enumerate}[label=\arabic*)]
        \item 若$\rho\left(\bm{A}\right)<R$,则$f\left(\bm{A}\right)$收敛
        \item 若$\rho\left(\bm{A}\right)>R$,则$f\left(\bm{A}\right)$发散
        \item 若$\rho\left(\bm{A}\right)=R$,$\forall\left|\lambda_i\right|=R,$设属于$\lambda_i$的初等因子的最大幂次为$n_i$,则$f\left(\bm{A}\right)$收敛当且仅当\[
                  f\left(\lambda_i\right),
                  f'\left(\lambda_i\right),
                  \cdots,
                  f^{\left(n_i-1\right)}\left(\lambda_i\right)
              \]收敛($\forall i$)
        \item 若$f\left(\bm{A}\right)$收敛,则$f\left(\bm{A}\right)$的特征值为\[
                  f\left(\lambda_1\right),f\left(\lambda_2\right),\cdots,f\left(\lambda_n\right)
              \]
    \end{enumerate}\begin{proof}
        首先存在非异阵$\bm{P}$使得\[
            \bm{P}^{-1}\bm{A}\bm{P}=\bm{J}=\diag \left\{
            \bm{J}_{r_1}\left(\lambda_1\right),\bm{J}_{r_2}\left(\lambda_2\right),\cdots,\bm{J}_{r_k}\left(\lambda_k\right)
            \right\}
        \]根据\cref{thm:矩阵幂级数收敛的等价命题}知$f\left(\bm{A}\right)$收敛当且仅当$f\left(\bm{J}\right)$收敛当且仅当$f\left(
            \bm{J}_{r_i}\left(\lambda_i\right)
            \right)\left(
            \forall 1\leqslant i\leqslant k
            \right)$都收敛,则\begin{enumerate}[label=\arabic*)]
            \item 若$\rho\left(\bm{A}\right)<R$,\cref{thm:矩阵幂级数收敛的等价命题}知$f\left(
                      \bm{J}_{r_i}\left(\lambda_i\right)
                      \right)$收敛
            \item 若$\rho\left(\bm{A}\right)>R$,即$\exists \lambda_i,\left|\lambda_i\right|>R$,根据\cref{thm:Cauchy-Hadamard定理}知$f\left(\lambda_i\right)$发散,则根据\cref{thm:矩阵幂级数收敛的等价命题}知$f\left(
                      \bm{J}_{r_i}\left(\lambda_i\right)
                      \right)$发散,则$f\left(\bm{A}\right)$发散
            \item 若$\rho\left(\bm{A}\right)=R$,则其中$\left|\lambda_i\right|<R$的部分收敛.于是考虑模长等于$R$的特征值,考虑到$n_i$为最大值,于是\cref{thm:矩阵幂级数收敛的等价命题}中取得更精细的结论即$f\left(
                      \bm{J}_{n_i}\left(\lambda_i\right)
                      \right)$收敛当且仅当\[
                      \lim_{p\to\infty}f_p\left(\lambda_i\right),\lim_{p\to\infty}f_p'\left(\lambda_i\right),\cdots,\lim_{p\to\infty}f_p^{\left(n_i-1\right)}\left(\lambda_i\right)
                  \]均存在,即$f_p\left(\lambda_i\right),f_p'\left(\lambda_i\right),\cdots,f_p^{\left(n_i-1\right)}\left(\lambda_i\right)$收敛.

            \item 因为\begin{align*}f\left(\bm{A}\right) & =f\left(\bm{PJP}^{-1}\right)=
              \bm{P}f\left(\bm{J}\right)\bm{P}^{-1}                \\
                                   & =\bm{P}\diag \,\left\{
              f\left(
              \bm{J}_{r_1}\left(\lambda_1\right)
              \right),f\left(
              \bm{J}_{r_2}\left(\lambda_2\right)
              \right),\cdots,f\left(
              \bm{J}_{r_k}\left(\lambda_k\right)
              \right)
              \right\}\bm{P}^{-1}
                  \end{align*}于是$f\left(\bm{A}\right)$的特征值与该对角阵的特征值相同,即\[
                      f\left(\lambda_1\right),f\left(\lambda_2\right),\cdots,f\left(\lambda_n\right)\qedhere
                  \]
        \end{enumerate}
    \end{proof}
}
\subsection{矩阵函数}
\dfn{简单矩阵函数}{简单矩阵函数}{
    设$\bm{A}\in M_n\left(\bm{A}\right)$,则\begin{enumerate}[label=\arabic*)]
        \item $\displaystyle \rme ^{\bm{A}}=\bm{I}_n+\frac{\bm{A}}{1!}+\frac{\bm{A}^2}{2!}+\frac{\bm{A}^3}{3!}+\cdots$
        \item $\displaystyle
                  \sin \bm{A}= \bm{A}-\frac{\bm{A}^3}{3!}+\frac{\bm{A}^5}{5!}-\frac{\bm{A}^7}{7!}+\cdots
              $
        \item $\displaystyle
                  \cos \bm{A}=\bm{I}_n-\frac{\bm{A}^2}{2!}+\frac{\bm{A}^4}{4!}-\frac{\bm{A}^6}{6!}+\cdots
              $
        \item $\displaystyle
                  \ln\left(
                  \bm{I}_n+\bm{A}
                  \right)=\bm{A}-\frac{\bm{A}^2}{2}+\frac{\bm{A}^3}{3}-\frac{\bm{A}^4}{4}+\cdots,\rho\left(\bm{A}\right)<1
              $
    \end{enumerate}
}
\rem{}{}{
    矩阵函数不能随意套用数值函数的性质.例如\[
        \rme ^x\cdot \rme ^y=
        \rme ^y\cdot \rme ^x
        =\rme ^{x+y}
    \]但\[
        \rme ^{\bm{A}+\bm{B}}
        \neq
        \rme ^{\bm{A}}\cdot \rme ^{\bm{B}}\neq \rme ^{\bm{A}}\cdot \rme ^{\bm{B}}
    \]其中\[
        \bm{A}=\begin{bmatrix}
            1 & 1 \\0 & 0
        \end{bmatrix},\bm{B}=\begin{bmatrix}
            0 & 0 \\1 & 1
        \end{bmatrix}
    \]特别地,$\bm{A},\bm{B}$可交换即$\bm{AB}=\bm{BA}$时才有$\rme ^{\bm{A}+\bm{B}}=\rme ^{\bm{A}}\cdot \rme ^{\bm{B} }=\rme ^{\bm{B}}\cdot \rme ^{\bm{A}}$(第一个等号证明需要引入范数).
}
\exa{}{}{
    设$\bm{A}\in M_n\left(\bbc \right)$,求$\rme ^{t\bm{A}},t$为变量.\begin{solution}
        复幂级数$f\left(z\right)=\rme ^{tz}$,$f^{\left(i\right)}\left(z\right)=t^i\rme ^{tz}$,设\[
            \bm{P}^{-1}\bm{A}\bm{P}=\bm{J}=\diag \left\{
            \bm{J}_{r_1}\left(\lambda_1\right),\bm{J}_{r_2}\left(\lambda_2\right),\cdots,\bm{J}_{r_k}\left(\lambda_k\right)
            \right\}
        \]于是\begin{align*}
            \rme ^{t\bm{A}}=f\left(\bm{A}\right)=\bm{P}\diag \left\{
            f\left(
            \bm{J}_{r_1}\left(\lambda_1\right)
            \right),f\left(
            \bm{J}_{r_2}\left(\lambda_2\right)
            \right),\cdots,f\left(
            \bm{J}_{r_k}\left(\lambda_k\right)
            \right)
            \right\}\bm{P}^{-1}
        \end{align*}其中\begin{align*}
            \rme ^{t\bm{J}_{r_i}\left(\lambda_i\right)} & =\begin{pmatrix}
                                                               \rme ^{t\lambda_i} & \cfrac{t\rme ^{t\lambda_i}}{1!} & \cfrac{t^2\rme ^{t\lambda_i}}{2!} & \cdots & \cfrac{t^{r_i-1}\rme ^{t\lambda_i}}{\left(r_i-1\right)!} \\
                                                                                  & \rme ^{t\lambda_i}              & \cfrac{t\rme ^{t\lambda_i}}{1!}   & \cdots & \cfrac{t^{r_i-2}\rme ^{t\lambda_i}}{\left(r_i-2\right)!} \\
                                                                                  &                                 & \rme ^{t\lambda_i}                & \cdots & \cfrac{t^{r_i-3}\rme ^{t\lambda_i}}{\left(r_i-3\right)!} \\
                                                                                  &                                 &                                   & \ddots & \vdots                                                   \\
                                                                                  &                                 &                                   &        & \rme ^{t\lambda_i}
                                                           \end{pmatrix} \\
                                                        & =\rme ^{t\lambda_i}\begin{pmatrix}
                                                                                 1 & t & \cfrac{t^2}{2!} & \cdots & \cfrac{t^{r_i-1}}{\left(r_i-1\right)!}  \\
                                                                                   & 1 & t               & \cdots & \cfrac{t^{r_i-2}}{\left(r_i-2\right)!}  \\
                                                                                   &   & 1               & \cdots & \cfrac{t^{r_i-3}}{\left(r_i-3 \right)!} \\
                                                                                   &   &                 & \ddots & \vdots                                  \\
                                                                                   &   &                 &        & 1
                                                                             \end{pmatrix}
        \end{align*}
    \end{solution}
}