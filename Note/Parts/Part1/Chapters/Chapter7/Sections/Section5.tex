\newpage
\section{初等因子}
\dfn{准素因子}{准素因子}{
    设$f\left(\lambda\right)\in\mathbb{K}\left[\lambda\right],p\left(\lambda\right)$为不可约多项式,若$\exists e\in \mathbb{Z}^+\st$\[
        p\left(\lambda\right)^e\mid f\left(\lambda\right),p\left(\lambda\right)^{e+1}\nmid f\left(\lambda\right)
    \]则称$p\left(
        \lambda
        \right)^e$为$f\left(\lambda\right)$的准素因子.
}
\exa{}{}{
    由\cref{thm:标准因式分解}标准因式分解\[
        f\left(\lambda\right)=cp_1\left(\lambda\right)^{e_1}p_2\left(\lambda\right)^{e_2}\cdots p_m\left(\lambda\right)^{e_m}
    \]其中$c\neq 0,p_i\left(\lambda\right)$为互异的首一不可约多项式,$e_i\geqslant 1\left(
        \forall 1\leqslant i\leqslant m
        \right)$.则$f\left(\lambda\right)$的准素因子为
    \[
        p_1\left(\lambda\right)^{e_1},p_2\left(\lambda\right)^{e_2},\cdots,p_m\left(\lambda\right)^{e_m}
    \]因此\cref{thm:标准因式分解}也可以叫做准素分解.
}
\lem{}{不变因子的公共因式分解}{
考虑$\bm{A}\in M_n\left(\mathbb{K}\right)$的非零不变因子为\[d_1\left(\lambda\right),d_2\left(\lambda\right),\cdots,d_k\left(\lambda\right)\]
考虑\cref{cor:公共因式分解}公共因式分解
\[
    d_1\left(\lambda\right)=p_1\left(\lambda\right)^{e_{11}}p_2\left(\lambda\right)^{e_{12}}\cdots p_t\left(\lambda\right)^{e_{1t}}
\]\[
    d_2\left(\lambda\right)=p_1\left(\lambda\right)^{e_{21}}p_2\left(\lambda\right)^{e_{22}}\cdots p_t\left(\lambda\right)^{e_{2t}}
\]
\[
    \cdots\cdots\cdots\cdots
\]
\[
    d_r\left(\lambda\right)=p_1\left(\lambda\right)^{e_{r1}}p_2\left(\lambda\right)^{e_{r2}}\cdots p_t\left(\lambda\right)^{e_{rt}}
\]
\[
    \cdots\cdots\cdots\cdots
\]
\[
    d_k\left(\lambda\right)=p_1\left(\lambda\right)^{e_{k1}}p_2\left(\lambda\right)^{e_{k2}}\cdots p_t\left(\lambda\right)^{e_{kt}}
\]由不变因子的整除关系知$\forall 1\leqslant j\leqslant t:$\[
    0\leqslant e_{1j}\leqslant e_{2j}\leqslant \cdots\leqslant e_{kj}\]
}
\dfn{初等因子}{初等因子}{
若\cref{lem:不变因子的公共因式分解}中某一个$e_{ij}>0$,则称$p_{j}\left(\lambda\right)^{e_{ij}}$为$\bm{A}$的一个初等因子.

$\bm{A}$的所有初等因子构成了$\bm{A}$的初等因子组.
}
\cor{}{初等因子与准素因子}{
    等价地,$\bm{A}$的非常数不变因子的准素因子即为$\bm{A}$的初等因子.从而$\bm{A}$的初等因子组就是$\bm{A}$的非常数不变因子的准素因子全体.
}
\thm{初等因子确定不变因子}{初等因子确定不变因子}{
给出$\bm{A}$的初等因子组,设其中涉及到的不可约多项式为$p_1\left(\lambda\right),p_2\left(\lambda\right),\cdots,p_t\left(\lambda\right)$,在添加一些不可约多项式的零次幂后,将同一不可约多项式的幂次降幂排列:
\[
    \begin{matrix}
        p_1\left(\lambda\right)^{e_{k1}} & p_1\left(\lambda\right)^{e_{k-1,1}} & \cdots & p_1\left(\lambda\right)^{e_{11}} \\
        p_2\left(\lambda\right)^{e_{k2}} & p_2\left(\lambda\right)^{e_{k-1,2}} & \cdots & p_2\left(\lambda\right)^{e_{12}} \\
        \vdots                           & \vdots                              &        & \vdots                           \\
        p_t\left(\lambda\right)^{e_{kt}} & p_t\left(\lambda\right)^{e_{k-1,t}} & \cdots & p_t\left(\lambda\right)^{e_{1t}}
    \end{matrix}
\]其中$e_{1j}\leqslant e_{2j}\leqslant \cdots\leqslant e_{kj},\forall 1\leqslant j\leqslant t.$考虑到不变因子的整除关系,令\[
    d_k\left(\lambda\right)=p_1\left(\lambda\right)^{e_{k1}}p_2\left(\lambda\right)^{e_{k2}}\cdots p_t\left(\lambda\right)^{e_{kt}}
\]
\[
    d_{k-1}\left(\lambda\right)=p_1\left(\lambda\right)^{e_{k-1,1}}p_2\left(\lambda\right)^{e_{k-1,2}}\cdots p_t\left(\lambda\right)^{e_{k-1,t}}
\]\[
    \cdots\cdots\cdots\cdots
\]
\[
    d_1\left(\lambda\right)=p_1\left(\lambda\right)^{e_{11}}p_2\left(\lambda\right)^{e_{12}}\cdots p_t\left(\lambda\right)^{e_{1t}}
\]于是由初等因子组确定了不变因子组.
}
\thm{唯一确定关系}{唯一确定关系}{
    由\cref{lem:不变因子的公共因式分解},\cref{def:初等因子},\cref{thm:初等因子确定不变因子}和\cref{thm:初等因子确定不变因子}所表明的这一相互确定的操作是一一对应.
}
\thm{相似关系的全系不变量}{相似关系的全系不变量}{
    初等因子组是相似关系下的全系不变量.
}
\rem{}{}{
    初等因子依赖于因式分解,后者严重依赖于基域的选取,于是初等因子是依赖于基域的.
}
\exa{}{}{
    设$\bm{A}$的不变因子为\[
        1,\cdots,1,\left(\lambda-1\right)\left(\lambda^2+1\right),\left(\lambda-1\right)^2\left(\lambda^2+1\right)\left(\lambda^2-2\right)
    \]求$\bm{A}$在$\mathbb{Q},\mathbb{R},\mathbb{C}$上的初等因子.\begin{solution}
        \[
            /\mathbb{Q}:\lambda-1,\left(\lambda-1\right)^2,\lambda^2+1,\lambda^2+1,\lambda^2-2
        \]
        \[
            /\mathbb{R}:\lambda-1,\left(\lambda-1\right)^2,\lambda^2+1,\lambda^2+1,\lambda-\sqrt{2},\lambda+\sqrt{2}
        \]
        \[
            /\mathbb{C}:\lambda-1,\left(\lambda-1\right)^2,\lambda-\mathrm{i},\lambda-\mathrm{i},\lambda+\mathrm{i},\lambda+\mathrm{i},\lambda-\sqrt{2},\lambda+\sqrt{2}
        \]
    \end{solution}
}
\exa{}{}{
    设$\bm{A}^{10}$的初等因子组为\[
        \lambda-1,\lambda-1,\left(\lambda-1\right)^2,\left(\lambda+1\right)^2,\left(\lambda+1\right)^3,\lambda-2
    \]求它的不变因子组.\begin{solution}
        容易有\[
            \begin{matrix}
                \left(\lambda-1\right)^2 & \lambda-1                & \lambda-1 \\
                \left(\lambda+1\right)^3 & \left(\lambda+1\right)^2 & 1         \\
                \lambda-2                & 1                        & 1
            \end{matrix}
        \]于是\[
            d_3\left(\lambda\right)=\left(\lambda-1\right)^2\left(\lambda+1\right)^3\left(\lambda-2\right),d_2\left(\lambda\right)=\left(\lambda-1\right)\left(\lambda+1\right)^2,d_1\left(\lambda\right)=\lambda-1
        \]于是不变因子组为
        \[
            1,\cdots,1, \lambda-1,\left(\lambda-1\right)\left(\lambda+1\right)^2,\left(\lambda-1\right)^2\left(\lambda+1\right)^3\left(\lambda-2\right)
        \]其中有$7$个$1$.
    \end{solution}
}
下面简单说明一下下一节对更精细的标准型的构造.

矩阵$\bm{A}$在$\mathbb{K}$上的初等因子为\[
    p_1\left(\lambda\right)^{e_1},p_2\left(\lambda\right)^{e_2},\cdots,p_m\left(\lambda\right)^{e_m}
\]其中$p_i\left(\lambda\right)$是$\mathbb{K}$上的不可约多项式,$e_i\geqslant 1.$设其不变因子组为\[
    1,\cdots,1,d_1\left(\lambda\right),\cdots,d_k\left(\lambda\right)
\]则其有理标准型\[
    \bm{F}=\mathrm{diag}\,\left\{
    \bm{F}_1\left(d_1\left(\lambda\right)\right),\bm{F}_2\left(d_2\left(\lambda\right)\right),\cdots,\bm{F}_k\left(d_k\left(\lambda\right)\right)
    \right\}
\]而构造一个相对简单的矩阵$\bm{J}\left(p_i\left(\lambda\right)^{e_i}\right)$使得其初等因子仅有$p_i\left(\lambda\right)^{e_i}$.于是得到
\[
    \bm{J}=\mathrm{diag}\,\left\{
    \bm{J}_1\left(p_1\left(\lambda\right)^{e_1}\right),\bm{J}_2\left(p_2\left(\lambda\right)^{e_2}\right),\cdots,\bm{J}_m\left(p_m\left(\lambda\right)^{e_m}\right)
    \right\}
\]然后验证其相似于$\bm{A}$,这就是下一节需要讨论的问题.
\rem{}{}{
    上述问题在一般数域上是超出本门课的内容的,于是我们只会讨论在复数域$\mathbb{C}$上的情况.
}