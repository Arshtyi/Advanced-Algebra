\newpage
\section{有理标准型}
\subsection{引入}
\lem{友阵}{友阵}{
    考虑$r$阶矩阵\[
        \bm{F}=\begin{pmatrix}
            0      & 1        & 0        & \cdots & 0      \\
            0      & 0        & 1        & \cdots & 0      \\
            \vdots & \vdots   & \vdots   &        & \vdots \\
            0      & 0        & 0        & \cdots & 1      \\
            -a_r   & -a_{r-1} & -a_{r-2} & \cdots & -a_1
        \end{pmatrix}
    \]\begin{enumerate}[label=\arabic*)]
        \item $\bm{F}$的行列式因子组为\[
                  \overbrace{1, \ldots, 1}^{r-1 \text{个}}, f\left(\lambda\right)
              \]其中\[
                  f\left(\lambda\right)=\lambda^r+a_1\lambda^{r-1}+\cdots+a_{r-1}\lambda+a_r
              \]不变因子组相同
        \item $\bm{F}$的特征多项式与极小多项式相同,均为
              \[
                  f\left(\lambda\right)=\lambda^r+a_1\lambda^{r-1}+\cdots+a_{r-1}\lambda+a_r
              \]
    \end{enumerate}\begin{proof}
        \begin{align*}
            \lambda\bm{I}_r-\bm{F} & =\begin{pmatrix}
                                          \lambda & -1      & 0       & \cdots & 0           \\
                                          0       & \lambda & -1      & \cdots & 0           \\
                                          \vdots  & \vdots  & \vdots  &        & \vdots      \\
                                          0       & 0       & 0       & \cdots & -1          \\
                                          a_r     & a_{r-1} & a_{r-2} & \cdots & a_1+\lambda
                                      \end{pmatrix}
        \end{align*}
        有$D_r\left(\lambda\right)=\left|
            \lambda\bm{I}_r-\bm{F}
            \right|=f\left(\lambda\right).$再考虑$1\leqslant r\leqslant r-1$的情况,$\lambda\bm{I}_n-\bm{F}$有一个$k$阶子式$\left(-1\right)^k\Longrightarrow D_k\left(\lambda\right)=1,\forall 1\leqslant k\leqslant r-1$.

        设极小多项式$m\left(\lambda\right)$,只要证$m\left(\lambda\right)=f\left(\lambda\right)$.由\cref{thm:Cayley-Hamilton定理}Cayley-Hamilton定理知$m\left(\lambda\right)\mid f\left(\lambda\right)\Longrightarrow
            \deg m\left(\lambda\right)\leqslant r.$一方面,若$\deg m\left(\lambda\right)=r\Longrightarrow m\left(\lambda\right)=f\left(\lambda\right).$另一方面,设$\deg m\left(\lambda\right)<r$,设\[
            m\left(\lambda\right)=c_{r-1}\lambda^{r-1}+\cdots+c_1\lambda+c_0
        \]其中$c_i$不完全为零.考虑标准单位行向量,则有
        \[
            \bm{e}_1\bm{F}=\bm{e}_2,\bm{e}_2\bm{F}=\bm{e}_3,\cdots,\bm{e}_{r-1}\bm{F}=\bm{e}_r
        \]即\[
            \bm{e}_1\bm{F}^i=\bm{e}_{i+1},\forall 1\leqslant i\leqslant r-1
        \]由极小多项式定义有$\bm{O}=m\left(\bm{F}\right)=
            c_{r-1}\bm{F}^{r-1}+\cdots+c_1\bm{F}+c_0\bm{I}_r
        $同时左乘$\bm{e}_1$
        \[
            \bm{O}=c_{r-1}\bm{e}_1\bm{F}^{r-1}+\cdots+c_1\bm{e}_1\bm{F}+c_0\bm{e}_1\bm{I}_r
        \]利用上述迭代即得
        \[
            \bm{O}=c_{r-1}\bm{e}_r+\cdots+c_1\bm{e}_2+c_0\bm{e}_1
        \]即$\left(
            c_{r-1},\cdots,c_1,c_0
            \right)=\bm{0}\Longrightarrow
            c_i=0,\forall 1\leqslant i\leqslant r-1
        $.这与假设矛盾.故$m\left(\lambda\right)=f\left(\lambda\right)$.
    \end{proof}
}
\rem{}{}{
    \cref{lem:友阵}中实际上是多项式相伴的友阵的转置(一些书本上也可能不取转置).一般记作$\bm{F}\left(f\left(\lambda\right)\right)=\bm{F}.$
}
\lem{}{利用多项式矩阵刻画相抵}{
设$\lambda$-阵$\bm{A}\left(\lambda\right)$相抵于\[
    \diag \left\{
    d_1\left(\lambda\right),d_2\left(\lambda\right),\cdots,d_n\left(\lambda\right)
    \right\}
\]$\lambda$-阵$\bm{B}\left(\lambda\right)$相抵于\[
    \diag \left\{
    d_{i_1}\left(\lambda\right),d_{i_2}\left(\lambda\right),\cdots,d_{i_n}\left(\lambda\right)
    \right\}
\]其中$\left(
    i_1,i_2,\cdots,i_n
    \right)$是$\left(1,2,\cdots,n\right)$的一个置换,则$\bm{A}\left(\lambda\right)\sim\bm{B}\left(\lambda\right)$.\begin{proof}
    事实上,根据相抵关系传递,只需证明两个对角阵相抵即可.

    先考虑这样简单的情况:对角阵主对角线上两个元素对换时它们是否相抵.考虑如下对角阵
    \[
        \diag \left\{
        d_1\left(\lambda\right),\cdots,d_i\left(\lambda\right),\cdots,d_j\left(\lambda\right),\cdots,d_n\left(\lambda\right)
        \right\}
    \]
    要将第$i$和第$j$个对角元素对换,只需先对换第$i$行和第$j$行,再对换第$i$列和第$j$列即可.对于一般的情况,只需不断地对换相邻的两个对角元素即可.
\end{proof}
}
\subsection{有理标准型}
\thm{}{矩阵相似于友阵组成的对角阵}{
    设$\bm{A}\in M_n\left(\mathbb{K}\right)$的不变因子组为\[
        1,1,\cdots,1;
        d_1\left(\lambda\right),d_2\left(\lambda\right),\cdots,d_k\left(\lambda\right)
    \]则$\bm{A}$相似于
    \[\bm{F}=
        \diag \left\{
        \bm{F}\left(d_1\left(\lambda\right)\right),\bm{F}\left(d_2\left(\lambda\right)\right),\cdots,\bm{F}\left(d_k\left(\lambda\right)\right)
        \right\}
    \]\begin{proof}
        因为$\lambda\bm{I}_n-\bm{A}$相抵于
        \[
            \diag \left\{1,1,\cdots,1;
            d_1\left(\lambda\right),d_2\left(\lambda\right),\cdots,d_k\left(\lambda\right)
            \right\}
        \]因为他们有相同的行列式因子组,故他们有相同的$n$阶行列式因子即行列式即
        \[
            \left|
            \lambda\bm{I}_n-\bm{A}
            \right|=d_1\left(\lambda\right)d_2\left(\lambda\right)\cdots d_k\left(\lambda\right)
        \]设$\deg d_i\left(\lambda\right)=n_i\geqslant 1,\forall 1\leqslant i\leqslant k$,显然$n_1+n_2+\cdots+n_k.$

        因为$\lambda\bm{I}-\bm{F}\left(d_i\left(\lambda\right)\right)$的法式为\[\diag \{
            \overbrace{1, \ldots, 1}^{n_i-1 \text{个}}, d_i\left(\lambda\right)
            \}\]考虑到
        \[
            \lambda\bm{I}-\bm{F}=\begin{pmatrix}
                \lambda\bm{I}-\bm{F}\left(d_1\left(\lambda\right)\right) &                                                          &        &                                                          \\
                                                                         & \lambda\bm{I}-\bm{F}\left(d_2\left(\lambda\right)\right) &        &                                                          \\
                                                                         &                                                          & \ddots &                                                          \\
                                                                         &                                                          &        & \lambda\bm{I}-\bm{F}\left(d_k\left(\lambda\right)\right)
            \end{pmatrix}
        \]通过初等变换一定得到一个对角阵
        \[
            \diag \left\{
            1,\cdots,1,d_1\left(\lambda\right),1,\cdots,1,d_2\left(\lambda\right),\cdots,1,\cdots,1,d_k\left(\lambda\right)
            \right\}
        \]显然$1$一共有\begin{align*}n_1-1+\cdots+n_k-1 & =n_1+\cdots+n_k-k \\
                                 & =n-k\end{align*}个,根据\cref{lem:利用多项式矩阵刻画相抵},$\lambda\bm{I}-\bm{A}\sim\lambda\bm{I}-\bm{F}$.于是数字矩阵具有相似关系即$\bm{A}\approx\bm{F}$.
    \end{proof}
}
\dfn{有理标准型}{有理标准型}{
    \cref{thm:矩阵相似于友阵组成的对角阵}中的分块对角阵
    \[\bm{F}=
        \diag \left\{
        \bm{F}\left(d_1\left(\lambda\right)\right),\bm{F}\left(d_2\left(\lambda\right)\right),\cdots,\bm{F}\left(d_k\left(\lambda\right)\right)
        \right\}
    \]称为$\bm{A}$的Frobenius标准型或有理标准型.每一个$\bm{F}\left(d_i\left(\lambda\right)\right)$称为Frobenius块.
}
\exa{}{}{
    设$6$阶矩阵$\bm{A}$的不变因子组为
    \[
        1,1,1,\lambda-1,\left(\lambda-1\right)^2,\left(\lambda-1\right)^2\left(\lambda+1\right)
    \]求它的有理标准型.\begin{proof}
        将非常数的不变因子展开
        \[
            \left(\lambda-1\right)^2=\lambda^2-2\lambda+1,\left(\lambda-1\right)^2\left(\lambda+1\right)=\lambda^3-\lambda^2-\lambda+1
        \]
        对应Frobenius块为其非首项系数取负倒序,得到\[
            \bm{F}=\begin{pmatrix}
                1 &    &   &    &   &   \\
                  & 0  & 1 &    &   &   \\
                  & -1 & 2 &    &   &   \\
                  &    &   & 0  & 1 & 0 \\
                  &    &   & 0  & 0 & 1 \\
                  &    &   & -1 & 1 & 1
            \end{pmatrix}
        \]
    \end{proof}
}
\thm{}{不变因子组导出极小多项式}{
    设$\bm{A}\in M_n\left(\mathbb{K}\right)$的不变因子组为\[
        1,\cdots,1,d_1\left(\lambda\right),\cdots,d_k\left(\lambda\right)
    \]其中非零首一多项式$d_1\left(\lambda\right)\mid d_2\left(\lambda\right)\mid \cdots\mid d_k\left(\lambda\right)$,则$\bm{A}$的极小多项式为$m\left(\lambda\right)=d_k\left(\lambda\right).$\begin{proof}
        因为$\bm{A}$相似于其有理标准型$\bm{F}=\diag \left\{
            \bm{F}\left(d_1\left(\lambda\right)\right),\bm{F}\left(d_2\left(\lambda\right)\right),\cdots,\bm{F}\left(d_k\left(\lambda\right)\right)
            \right\}$且相似的矩阵具有相同的极小多项式即$m_{\bm{A}}\left(\lambda\right)=m_{\bm{F}}\left(\lambda\right)$,记$\bm{F}_i=\bm{F}\left(d_i\left(\lambda\right)\right),\forall 1\leqslant i\leqslant k$,则
        \[
            m_{\bm{F}}\left(\lambda\right)=\left[
            m_{\bm{F}_1}\left(\lambda\right),m_{\bm{F}_2}\left(\lambda\right),\cdots,m_{\bm{F}_k}\left(\lambda\right)
            \right]
        \]\cref{lem:友阵}告诉我们
        \[
            m_{\bm{F}_i}\left(\lambda\right)=d_i\left(\lambda\right),\forall 1\leqslant i\leqslant k
        \]于是
        \[
            m_{\bm{A}}\left(\lambda\right)=\left[
                d_1\left(\lambda\right),d_2\left(\lambda\right),\cdots,d_k\left(\lambda\right)
                \right]=d_k\left(\lambda\right)\qedhere
        \]
    \end{proof}
}
\cor{}{极小多项式的扩张}{
设$\bm{A}\in M_n\left(\bbf \right)$,其极小多项式$m_{\bbf }\left(\lambda\right)$.若$\bbf \subseteq\mathbb{K}$,将$\bm{A}$视为$\mathbb{K}$上的矩阵,其极小多项式$m_{\mathbb{K}}\left(\lambda\right)$,则$m_{\bbf }\left(\lambda\right)=m_{\mathbb{K}}\left(\lambda\right).$\begin{proof}
    由\cref{cor:不变因子的扩张}知不变因子在基域扩张下不改变及\cref{thm:不变因子组导出极小多项式}可知.
\end{proof}
}
\exa{}{}{
    考虑四阶矩阵
    \[
        \bm{A}=\begin{pmatrix}
            0 & 0 & 0 & 0 \\
            0 & 0 & 0 & 0 \\
            0 & 0 & 0 & 1 \\
            0 & 0 & 0 & 0
        \end{pmatrix},\bm{B}=\begin{pmatrix}
            0 & 1 & 0 & 0 \\
            0 & 0 & 0 & 0 \\
            0 & 0 & 0 & 1 \\
            0 & 0 & 0 & 0
        \end{pmatrix}
    \]\begin{solution}
        它们的不变因子组为
        \begin{align*}
             & \bm{A}:1,\lambda,\lambda,\lambda^2 \\
             & \bm{B}:1,1,\lambda^2,\lambda^2
        \end{align*}
        在本章开始时曾说过,即使是极小多项式与特征多项式放在一起也不能成为相似关系下的全系不变量,这里就直观地给出了这个例子:极小多项式仅仅是不变因子组的最后一项,而特征多项式是不变因子组的乘积,这两个相比于整个不变因子组来说都是不完整的,都是很弱的条件,哪怕合在一起也是弱的.
    \end{solution}
}
\rem{}{}{
    我们曾说要寻找尽可能简单的矩阵来描述相似关系,但是这里的Frobenius标准型在$d_i\left(\lambda\right)$的次数比较高时其Frobenius块的阶数也会变得很大,这让它并不足够简单,所以我们的一个想法是对$d_i\left(\lambda\right)$做一些简化如因式分解.
}