\newpage
\section{矩阵的法式}
\subsection{法式}
\lem{}{任意多项式矩阵必定相抵于首元素为其他元素因子的多项式矩阵}{
    设非零$\lambda$-阵$\bm{A}\left(\lambda\right)=\left(a_{ij}\left(\lambda\right)\right)_{m\times n}$,则$\bm{A}\left(\lambda\right)$必定相抵于$\bm{B}\left(\lambda\right)=\left(b_{ij}\left(\lambda\right)\right)_{m\times n}$,其中$b_{11}\left(\lambda\right)\neq 0$且$b_{11}\left(\lambda\right)\mid b_{ij}\left(\lambda\right),\forall 1\leqslant i\leqslant m,1\leqslant j\leqslant n.$\begin{proof}
        设$k=\min\left\{
            \deg a_{ij}\left(\lambda\right)\mid
            a_{ij}\left(\lambda\right)\neq 0,1\leqslant i\leqslant m,1\leqslant j\leqslant n
            \right\}\geqslant 0$,对$k$进行归纳.若$k=0$则存在$a_{ij}\left(\lambda\right)$是非零常数,则通过行列对换即可.

        设非零元次数最小值小于$k$时结论成立,下面证明等于$k$时的情况.

        取$a_{ij}\left(\lambda\right)$使得$\deg a_{ij}\left(\lambda\right)=k$则通过行列对换将其换到$\left(1,1\right)$位置.不妨设$a_{11}\left(\lambda\right)$达到非零元次数最小值.

        其一,设$\exists a_{i1}\left(\lambda\right)\st a_{11}\left(\lambda\right)\nmid a_{i1}\left(\lambda\right)$则有带余除法
        \[
            a_{i1}\left(\lambda\right)=
            a_{11}\left(\lambda\right)q\left(\lambda\right)+r\left(\lambda\right)
        \]其中$\deg r\left(\lambda\right)<\deg a_{11}\left(\lambda\right)=k$且$r\left(\lambda\right)\neq 0.$那么此时将第一行乘上$-q\left(\lambda\right)$加到第$i$行,于是得到一个具有更低的最低次数元($r\left(\lambda\right)$)的矩阵,根据归纳假设得证.

        其二,若$\exists a_{1j}\left(\lambda\right)\st a_{11}\left(\lambda\right)\nmid a_{1j}\left(\lambda\right)$,同理可得.

        其三,若$a_{11}\left(\lambda\right)\mid a_{i1}\left(\lambda\right),a_{11}\left(\lambda\right)\mid a_{1j}\left(\lambda\right),\forall 1\leqslant i\leqslant m,1\leqslant j\leqslant n.$此时利用初等变换将同行同列其他元素变为零.然后考虑是否有$a_{11}\left(\lambda\right)\mid a_{ij}'\left(\lambda\right),\forall 2\leqslant i\leqslant m,2\leqslant j\leqslant n$(做过变换已不等于原矩阵元素),若是,则证毕.

        设$\exists a_{ij}'\left(\lambda\right)\st a_{11}\left(\lambda\right)\nmid a_{ij}'\left(\lambda\right).$将第$i$行加到第一行,此为第一种情况,得证.
    \end{proof}
}
\thm{}{任意多项式矩阵相抵于对角多项式矩阵}{
    设$n$阶$\lambda$-矩阵$\bm{A}\left(\lambda\right)$,则其相抵于
    \[
        \bm{B}\left(\lambda\right)=\diag \left\{
        d_1\left(\lambda\right),d_2\left(\lambda\right),\cdots,d_r\left(\lambda\right);0,\cdots,0
        \right\}
    \]其中$d_i\left(\lambda\right)$为非零首一多项式且$d_i\left(\lambda\right)\mid d_{i+1}\left(\lambda\right),\forall 1\leqslant i\leqslant r-1$\begin{proof}
        对阶数$n$归纳,$n=1$时显然.设小于$n$阶时成立,来证明等于$n$阶的情形.$\bm{A}\left(\lambda\right)=\bm{O}$显然,下设$\bm{A}\left(\lambda\right)\neq \bm{O}$.

        由\cref{lem:任意多项式矩阵必定相抵于首元素为其他元素因子的多项式矩阵}知,$\bm{A}\left(\lambda\right)$必定相抵于$\bm{B}\left(\lambda\right)$,其中$b_{11}\left(\lambda\right)\neq 0$且$b_{11}\left(\lambda\right)\mid b_{ij}\left(\lambda\right),\forall 1\leqslant i,j\leqslant n.$然后利用初等变换得到
        \[
            \bm{B}'\left(\lambda\right)=\begin{pmatrix}
                b_{11}\left(\lambda\right) & 0                           & \cdots & 0                           \\
                0                          & b_{22}'\left(\lambda\right) & \cdots & b_{2n}'\left(\lambda\right) \\
                \vdots                     & \vdots                      &        & \vdots                      \\
                0                          & b_{n2}'\left(\lambda\right) & \cdots & b_{nn}'\left(\lambda\right)
            \end{pmatrix}
        \]显然$b_{11}\left(\lambda\right)\mid b_{ij}'\left(\lambda\right),\forall 2\leqslant i,j\leqslant n.$将该矩阵右下记作$\bm{B}_{n-1}'\left(\lambda\right)$

        由归纳假设,存在$\bm{P}\left(\lambda\right),\bm{Q}\left(\lambda\right)$作为若干个$n-1$阶初等$\lambda$-阵的乘积,使得
        \[
            \bm{P}\left(\lambda\right)\bm{B}_{n-1}'\left(\lambda\right)\bm{Q}\left(\lambda\right)=\diag \left\{
            d_2\left(\lambda\right),\cdots,d_r\left(\lambda\right);0,\cdots,0
            \right\}
        \]
        其中$d_i\left(\lambda\right)$为非零首一多项式且$d_i\left(\lambda\right)\mid d_{i+1}\left(\lambda\right),\forall 2\leqslant i\leqslant r-1$.设$b_{11}\left(\lambda\right)$的首项系数为$c$,记$d_1\left(\lambda\right)=c^{-1}b_{11}\left(\lambda\right).$作
        \[
            \begin{pmatrix}
                c^{-1} & 0 \\0 & \bm{I}_{n-1}
            \end{pmatrix}\begin{pmatrix}
                1 & 0 \\0 & \bm{P}\left(\lambda\right)
            \end{pmatrix}\bm{B}'\left(\lambda\right)\begin{pmatrix}
                1 & 0 \\0 & \bm{Q}\left(\lambda\right)
            \end{pmatrix}=
            \begin{pmatrix}
                d_1\left(\lambda\right) &                         &        &                         &   &        &   \\
                                        & d_2\left(\lambda\right) &        &                         &   &        &   \\
                                        &                         & \ddots &                         &   &        &   \\
                                        &                         &        & d_r\left(\lambda\right) &   &        &   \\
                                        &                         &        &                         & 0 &        &   \\
                                        &                         &        &                         &   & \ddots &   \\
                                        &                         &        &                         &   &        & 0
            \end{pmatrix}
        \]因为$d_1\left(\lambda\right)$整除$\bm{B}_{n-1}'\left(\lambda\right)$的任一元素,故$d_1\left(\lambda\right)$整除$\bm{P}\left(\lambda\right)\bm{B}_{n-1}'\left(\lambda\right)\bm{Q}\left(\lambda\right)$的任一元素,于是$d_1\left(\lambda\right)\mid d_2\left(\lambda\right)$.
    \end{proof}
}
\clm{}{}{
    由\cref{thm:任意多项式矩阵相抵于对角多项式矩阵}对于任意矩阵$\bm{A}\left(\lambda\right)_{m\times n}$,必定相抵于
    \[
        \begin{pmatrix}
            d_1\left(\lambda\right) & 0                       & \cdots & 0                       & \bm{O} \\
            0                       & d_2\left(\lambda\right) & \cdots & 0                       & \bm{O} \\
            \vdots                  & \vdots                  & \ddots & \vdots                  & \vdots \\
            0                       & 0                       & \cdots & d_r\left(\lambda\right) & \bm{O} \\
            \bm{O}                  & \bm{O}                  & \cdots & \bm{O}                  & \bm{O}
        \end{pmatrix}
    \]
}
\rem{}{}{
    \cref{thm:任意多项式矩阵相抵于对角多项式矩阵}得到的矩阵的主对角线上非零元素个数$r$可以称为该$\lambda$-矩阵的秩,但是$\lambda$-阵的秩要比数字矩阵的秩来得弱的多,满秩并不意味着可逆.
}
\dfn{法式}{法式}{
    \cref{thm:任意多项式矩阵相抵于对角多项式矩阵}得到矩阵称为$\bm{A}\left(\lambda\right)$的法式或者相抵标准型.
}
\rem{}{}{
    把\cref{def:法式}叫作相抵标准型过早,还需要证明相抵标准型不依赖于$\lambda$-矩阵初等变换的选取.
}
\lem{$\lambda$-阵乘积的行列式}{多项式矩阵的乘积的行列式}{
    设$n$阶$\lambda$-矩阵$\bm{A}\left(\lambda\right),\bm{B}\left(\lambda\right)$,则
    \begin{enumerate}[label=\arabic*)]
        \item $\left|\bm{A}\left(\lambda\right)\cdot\bm{B}\left(\lambda\right)\right|=\left|\bm{A}\left(\lambda\right)\right|\cdot\left|\bm{B}\left(\lambda\right)\right|$
        \item 定义伴随阵:
              \[
                  \bm{A}\left(\lambda\right)\bm{A}\left(\lambda\right)^*=\bm{A}\left(\lambda\right)^*\bm{A}\left(\lambda\right)=\left|\bm{A}\left(\lambda\right)\right|\bm{I}_n
              \]
    \end{enumerate}\begin{proof}
        \begin{enumerate}[label=\arabic*)]
            \item 令$f\left(\lambda\right)=
                      \left|\bm{A}\left(\lambda\right)\cdot\bm{B}\left(\lambda\right)\right|-\left|\bm{A}\left(\lambda\right)\right|\cdot\left|\bm{B}\left(\lambda\right)\right|
                  $是关于$\lambda$的多项式.$\forall a\in \mathbb{K}$,$f\left(a\right)=0\Longrightarrow f\left(\lambda\right)=0.$
            \item 令$\left(f_{ij}\left(\lambda\right)\right)_{n\times n}=
                      \bm{A}\left(\lambda\right)\bm{A}\left(\lambda\right)^*-\left|
                      \bm{A}\left(\lambda\right)
                      \right|\bm{I}_n
                  $其中$f_{ij}\left(\lambda\right)\in \mathbb{K}\left[\lambda\right]$.$\forall a\in \mathbb{K},
                      \left(f_{ij}\left(a\right)\right)_{n\times n}=\bm{A}\left(a\right)
                      \bm{A}\left(a\right)^*-\left|\bm{A}\left(a\right)\right|\bm{I}_n=\bm{O}\Longrightarrow
                      f_{ij}\left(\lambda\right)=0,\forall 1\leqslant i,j\leqslant n.$\qedhere
        \end{enumerate}
    \end{proof}
}
\subsection{逆阵}
\thm{$\lambda$-阵可逆的充要条件}{多项式矩阵可逆的充要条件}{
    设$n$阶$\lambda$-阵,则以下等价:\begin{enumerate}[label=\arabic*)]
        \item $\bm{A}\left(\lambda\right)$是可逆$\lambda$-阵
        \item $\left|\bm{A}\left(\lambda\right)\right|$是非零常数
        \item $\bm{A}\left(\lambda\right)$的相抵标准型为单位阵
        \item $\bm{A}\left(\lambda\right)$只通过行变换或列变换就能化为$\bm{I}_n$
        \item $\bm{A}\left(\lambda\right)$是若干$\lambda$-初等阵的乘积
    \end{enumerate}\begin{proof}
        $(1)\Longrightarrow (2)$因为存在$\bm{B}\left(\lambda\right)$使得$\bm{A}\left(\lambda\right)\bm{B}\left(\lambda\right)=\bm{B}\left(\lambda\right)\bm{A}\left(\lambda\right)=\left|\bm{A}\left(\lambda\right)\right|\bm{I}_n$,取行列式立得.

        $(2)\Longrightarrow (3)$由\cref{thm:任意多项式矩阵相抵于对角多项式矩阵},存在$\bm{P}\left(\lambda\right),\bm{Q}\left(\lambda\right)$(均为初等$\lambda$-阵的乘积)使得
        \[
            \bm{P}\left(\lambda\right)\bm{A}\left(\lambda\right)\bm{Q}\left(\lambda\right)
            =
            \diag \left\{
            d_1\left(\lambda\right),d_2\left(\lambda\right),\cdots,d_r\left(\lambda\right);0,\cdots,0
            \right\}
        \]
        其中,$d_i\left(\lambda\right)$为非零首一多项式且$d_i\left(\lambda\right)\mid d_{i+1}\left(\lambda\right),\forall 1\leqslant i\leqslant r-1.$两边取行列式得到$r=n$.并且\[d_1\left(\lambda\right)d_2\left(\lambda\right)\cdots d_n\left(\lambda\right)=c\neq 0\]再由首一得到\[d_1\left(\lambda\right)=d_2\left(\lambda\right)=\cdots=d_n\left(\lambda\right)=1\]于是
        $\bm{A}\left(\lambda\right)$相抵于$\bm{I}_n.$

        $(3)\Longrightarrow (4)$因为\cref{thm:任意多项式矩阵相抵于对角多项式矩阵},存在$\bm{P}\left(\lambda\right),\bm{Q}\left(\lambda\right)$(均为初等$\lambda$-阵的乘积)使得
        \[
            \bm{P}\left(\lambda\right)\bm{A}\left(\lambda\right)\bm{Q}\left(\lambda\right)
            =\bm{I}_n
        \]于是$\bm{Q}\left(\lambda\right)\bm{P}\left(\lambda\right)\bm{A}\left(\lambda\right)=\bm{I}_n$于是$\bm{A}\left(\lambda\right)$只需要行变换就可以化为相抵标准型$\bm{I}_n$,同理列变换.

        $(4)\Longrightarrow (5)$考虑行变换,即
        \[
            \bm{P}_r\left(\lambda\right)
            \cdots\bm{P}_1\left(\lambda\right)\bm{A}\left(\lambda\right)=\bm{I}_n
        \]
        其中$\bm{P}_i\left(\lambda\right)$都是初等$\lambda$-阵.于是
        \[
            \bm{A}\left(\lambda\right)=
            \bm{P}_1\left(\lambda\right)^{-1}
            \cdots\bm{P}_r\left(\lambda\right)^{-1}
        \]

        $(5)\Longrightarrow (1)$根据\cref{thm:初等多项式矩阵的逆阵},初等$\lambda$-阵是可逆阵,其乘积也一定可逆.
    \end{proof}
}
\cor{$\lambda$-阵的逆阵}{多项式矩阵的逆阵}{
    在\cref{lem:多项式矩阵的乘积的行列式}下
    \[
        \bm{A}\left(\lambda\right)^{-1}=\frac{1}{\left|\bm{A}\left(\lambda\right)\right|}\bm{A}\left(\lambda\right)^*
    \]
}
\thm{特征矩阵的法式}{特征矩阵的法式}{
    设$\bm{A}\in M_n\left(\mathbb{K}\right)$,则$\lambda\bm{I}_n-\bm{A}$相抵于
    \[
        \diag \left\{
        1,\cdots,1;d_1\left(\lambda\right),\cdots,d_m\left(\lambda\right)
        \right\}
    \]其中$d_i\left(\lambda\right)$是非常数的首一多项式,且$d_1\left(\lambda\right)\mid d_2\left(\lambda\right)\mid \cdots\mid d_m\left(\lambda\right).$这就是$\lambda\bm{I}_n-\bm{A}$的法式.\begin{proof}
        因为\cref{thm:任意多项式矩阵相抵于对角多项式矩阵},存在$\bm{P}\left(\lambda\right),\bm{Q}\left(\lambda\right)$(均为初等$\lambda$-阵的乘积)使得
        \[
            \bm{P}\left(\lambda\right)\left(\lambda\bm{I}_n-\bm{A}\right)\bm{Q}\left(\lambda\right)
            =
            \diag \left\{
            d_1\left(\lambda\right),d_2\left(\lambda\right),\cdots,d_r\left(\lambda\right);0,\cdots,0
            \right\}
        \]取行列式得$r=n$且
        \[
            c\left|\lambda\bm{I}_n-\bm{A}\right|=d_1\left(\lambda\right)d_2\left(\lambda\right)\cdots d_n\left(\lambda\right)
        \]其中$c\neq 0.$考虑到首一,立得$c=1.$最后调整顺序即可.
    \end{proof}
}
\exa{}{}{
    在\cref{thm:特征矩阵的法式}中,若$\deg d_i\left(\lambda\right)\geqslant 1$,则$\bm{A}=c\bm{I}_n$\begin{proof}
        因为\[\left|\lambda\bm{I}_n-\bm{A}\right|=d_1\left(\lambda\right)\cdots d_n\left(\lambda\right)\]两边取次数得到\[
            n=\sum_{i=1}^{n}\deg d_i\left(\lambda\right)\geqslant n\]即有\[\deg d_i\left(\lambda\right)=1,\forall 1\leqslant i\leqslant n\]所以$d_i\left(\lambda\right)$是相互整除的首一一次多项式,于是他们只能全部相等.

        于是$\lambda\bm{I}_n-\bm{A}$相抵于\[
            \diag \left\{
            \lambda-c, \cdots,\lambda-c
            \right\}=\lambda\bm{I}_n-c\bm{I}_n
        \]因为\cref{thm:矩阵相似等价于特征矩阵相抵},于是\[\bm{A}\approx c\bm{I}_n\]后者两边乘上$\bm{P}^{-1}$、$\bm{P}$是交换的,即\[\bm{A}=c\bm{I}_n\qedhere\]
    \end{proof}
}
\clm{}{}{
    于是,\cref{thm:特征矩阵的法式}中的$1$,只有在$\bm{A}$是纯量阵时才会不存在.
}
\exa{}{}{
    计算矩阵
    \[
        \begin{pmatrix}
            0  & 1  & -1 \\
            3  & -2 & 0  \\
            -1 & 1  & -1
        \end{pmatrix}
    \]对应的特征矩阵的相抵标准型.\begin{solution}
        \begin{align*}
            \lambda\bm{I}_3-\bm{A} & =\begin{pmatrix}
                                          \lambda & -1        & 1         \\
                                          -3      & \lambda+2 & 0         \\
                                          1       & -1        & \lambda+1
                                      \end{pmatrix} \\
                                   & \longrightarrow
            \begin{pmatrix}
                1       & -1        & \lambda+1 \\
                -3      & \lambda+2 & 0         \\
                \lambda & -1        & 1
            \end{pmatrix}                           \\
                                   & \longrightarrow
            \begin{pmatrix}
                1 & -1        & \lambda+1            \\
                0 & \lambda-1 & 3\lambda+3           \\
                0 & \lambda-1 & -\lambda^2-\lambda+1
            \end{pmatrix}                      \\
                                   & \longrightarrow
            \begin{pmatrix}
                1 & 0         & 0                    \\
                0 & \lambda-1 & 3\lambda+3           \\
                0 & \lambda-1 & -\lambda^2-\lambda+1
            \end{pmatrix}                      \\
                                   & \longrightarrow
            \begin{pmatrix}
                1 & 0         & 0                     \\
                0 & \lambda-1 & 6                     \\
                0 & \lambda-1 & -\lambda^2-4\lambda+7
            \end{pmatrix}                     \\
                                   & \longrightarrow
            \begin{pmatrix}
                1 & 0 & 0                                                       \\
                0 & 1 & 0                                                       \\
                0 & 0 & \left(\lambda-1\right)\left(\lambda^2+4\lambda+2\right)
            \end{pmatrix}
        \end{align*}
    \end{solution}
}