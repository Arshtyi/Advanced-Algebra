\newpage
\section{因式分解}
\subsection{可约与不可约}
\dfn{可约多项式与不可约多项式}{可约多项式与不可约多项式}{
    设$f\left(x\right)\in\mathbb{K}\left[x\right]$,
    $\deg f\left(x\right)\geqslant 1$.若存在$g\left(x\right),h\left(x\right)\in\mathbb{K}\left[x\right]$,且
    $\deg g\left(x\right)<\deg f\left(x\right),\deg h\left(x\right)<\deg f\left(x\right)$\st
    \[f\left(x\right)=g\left(x\right)h\left(x\right)\]
    那么称$f\left(x\right)$在$\mathbb{K}\left[x\right]$上是可约的.否则称为是不可约的.
}
\clm{一次多项式的不可约性}{}{
    多项式的可约与否取决于数域的选择.但不管在哪个域上,一次多项式总是不可约的.
}
\dfn{可约的等价定义}{可约的等价定义}{
    设$f\left(x\right)\in\mathbb{K}\left[x\right]$,$\deg f\left(x\right)\geqslant 1$.若存在多项式$g\left(x\right),h\left(x\right)\in\mathbb{K}\left[x\right]$,且
    $\deg g\left(x\right)>0,\deg h\left(x\right)>0$\st
    \[f\left(x\right)=g\left(x\right)h\left(x\right)\]那么称$f\left(x\right)$在$\mathbb{K}\left[x\right]$上是可约的.
}
\dfn{不可约的等价定义}{不可约的等价定义}{
    设$f\left(x\right)\in\mathbb{K}\left[x\right],\deg f\left(x\right)\geqslant 1$,则$f\left(x\right)$不可约等价于$f\left(x\right)$的因式有且仅有$c\neq 0$和$cf\left(x\right)$.
}
\exa{}{}{
    $f\left(x\right)=x^2-2$在$\mathbb{Q}\left[x\right]$不可约,在$\bbr \left[x\right]$可约.
}
\lem{}{不可约多项式与其他多项式}{
    设$p\left(x\right)$为$\mathbb{K}\left[x\right]$
    上的一个不可约多项式,$f\left(x\right)\in \mathbb{K}\left[x\right]$,则要么$\left(p\left(x\right),f\left(x\right)\right)=1$,要么
    $p\left(x\right)\mid f\left(x\right)$.\begin{proof}
        设$\left(p\left(x\right),f\left(x\right)\right)=d\left(x\right)$,则$d\left(x\right)|p\left(x\right)$.
        那么显然,$d\left(x\right)=1$或者$d\left(x\right)=cp\left(x\right)$(首一多项式).
    \end{proof}
}
\thm{}{不可约多项式作为因子不会是两个多项式的因子乘法得到的}{
    设不可约多项式$p\left(x\right)\in\mathbb{K}\left[x\right]$,
    且有多项式$f\left(x\right),g\left(x\right)\in\mathbb{K}\left[x\right]$且$p\left(x\right)\mid f\left(x\right)g\left(x\right)$,则
    $p\left(x\right)\mid f\left(x\right)$或$p\left(x\right)\mid g\left(x\right)$.
}
\clm{}{}{
    根据\cref{lem:不可约多项式与其他多项式},对于首一不可约多项式,互异即是互素.
}
\cor{}{不可约多项式作为因子不会是多个多项式的因子乘法得到的}{
    设不可约多项式$p\left(x\right)$且\[
        p\left(x\right)\mid f_1\left(x\right)f_2\left(x\right)\cdots f_m\left(x\right)
    \]
    则$\exists 1\leqslant i\leqslant m$,\st$p\left(x\right)\mid f_i\left(x\right)$.
}
\subsection{因式分解}
\thm{因式分解定理}{因式分解定理}{
    设$f\left(x\right)\in\mathbb{K}\left[x\right]$且$\deg f\left(x\right)\leqslant 1$,则\begin{enumerate}[label=\arabic*)]
        \item 存在性:$f\left(x\right)$一定可分解为$\mathbb{K}$上有限多个不可约多项式之积
        \item 唯一性:设\[
                  f\left(x\right)=p_1\left(x\right)p_2\left(x\right)\cdots p_s\left(x\right)=q_1\left(x\right)q_2\left(x\right)\cdots q_t\left(x\right)
              \]
              是$f\left(x\right)$的两个不可约分解,即$p_i\left(x\right),q_j\left(x\right)$均为$\mathbb{K}$上的次数大于零的不可约多项式,则一定有$s=t$且在适当调换因式的次序后,有\[
                  p_i\left(x\right)\sim q_i\left(x\right)\left(\forall 1\leqslant i\leqslant s=t\right)
              \]
    \end{enumerate}\begin{proof}
        \begin{enumerate}[label=\arabic*)]
            \item 对$\deg f\left(x\right)=n$归纳,$n=1$显然成立.

                  设$\deg f\left(x\right)<n$时成立,下证$\deg f\left(x\right)=n$.$f\left(x\right)$不可约时显然成立.设$f\left(x\right)$可约.则
                  \[f\left(x\right)=g\left(x\right)h\left(x\right)\]其中$g\left(x\right),h\left(x\right)\in\mathbb{K}\left[x\right],\deg g\left(x\right)<n,\deg h\left(x\right)<n$.
                  则由归纳假设有
                  \[g\left(x\right)=p_1\left(x\right)\cdots p_s\left(x\right),h\left(x\right)=q_1\left(x\right)q_2\left(x\right)\cdots q_t\left(x\right)\]其中$p_i\left(x\right),q_j\left(x\right)$均为$\mathbb{K}\left[x\right]$上的不可约多项式.于是存在性证毕.

            \item 对不可约因式个数$s$归纳,$s=1$时$f\left(x\right)=p_1\left(x\right)=q_1\left(x\right)\cdots q_t\left(
                      x\right)$,$t\geqslant 2$时$p_1\left(x\right)$可约,矛盾.于是$t = 1.$
                  即$s=t$且$p_1\left(x\right)=q_1\left(x\right)$.

                  设不可约因式个数小于$s$时成立,下证等于时成立.因为\[
                      f\left(x\right)=p_1\left(x\right)p_2\left(x\right)\cdots p_s\left(x\right)=q_1\left(x\right)q_2\left(x\right)\cdots q_t\left(x\right)
                  \]
                  故$p_1\left(x\right)\mid q_1\left(x\right)q_2\left(x\right)\cdots q_t\left(x\right)$,由素性知,$\exists 1\leqslant i\leqslant t$,\st
                  $p_1\left(x\right)\mid q_i\left(x\right).$不妨调整为$i=1$.即$q_1\left(x\right)=p_1\left(x\right)h\left(x\right).$由$p_1\left(x\right)$的不可约性知$h\left(x\right)=c\neq 0.$
                  即$q_1\left(x\right)=cp_1\left(x\right)\Longrightarrow p_1\left(x\right)\sim q_1\left(x\right)$.于是得到
                  \[
                      p_2\left(x\right)\cdots p_s\left(x\right)=cq_2\left(x\right)\cdots q_t\left(x\right)
                  \]
                  则由归纳假设有,$s-1=t-1\Longrightarrow s=t$且调换次序下$p_i\left(x\right)\sim q_i\left(x\right)\left(\forall 2\leqslant i \leqslant s=t\right)$.

                  于是得证.\qedhere
        \end{enumerate}
    \end{proof}
}
\thm{标准因式分解}{标准因式分解}{
    设$f\left(x\right)\in\mathbb{K}\left[x\right],\deg f\left(x\right)\geqslant 1$,则有如下标准因式分解
    \[
        f\left(x\right)=cp_1\left(x\right)^{e_1}p_2\left(x\right)^{e_2}\cdots p_m\left(x\right)^{e_m}
    \]
    其中$c\neq 0\in\mathbb{K}$,$p_i\left(x\right)\left(1\leqslant i\leqslant m\right)$为互异的首一不可约多项式,$ e_i\geqslant 1\left(\forall1\leqslant i\leqslant m\right)$.
}
\dfn{重因式}{重因式}{
    若标准因式分解中$e_1>1\left(\forall 1\leqslant i\leqslant m\right)$,则$p_i\left(x\right)$称为$f\left(x\right)$的$e_i$重因式.$e_i=1$时称为单因式.
}
\cor{公共因式分解}{公共因式分解}{
    设$f\left(x\right),g\left(x\right)\in\mathbb{K}\left[x\right]$,则添加某些不可约多项式的零次幂后,有如下公共因式分解
    \begin{align*}
         & f\left(x\right)=c_1p_1\left(x\right)^{e_1}p_2\left(x\right)^{e_2}\cdots p_n\left(x\right)^{e_n};
        \\
         & g\left(x\right)=c_2p_1\left(x\right)^{f_1}p_2\left(x\right)^{f_2}\cdots p_n\left(x\right)^{f_n}.
    \end{align*}
    其中$c_1\neq 0,c_2\neq 0\in\mathbb{K}$,$p_i\left(x\right)\left(1\leqslant i\leqslant n\right)$为互异的首一不可约多项式,$e_i\geqslant 0,f_i\geqslant 0\left(\forall 1\leqslant i\leqslant n\right)$.
}
\pro{公共因式分解的若干性质}{公共因式分解的若干性质}{
    \begin{enumerate}[label=\arabic*)]
        \item  $g\left(x\right)\mid f\left(x\right)\Longleftrightarrow f_i\leqslant e_i\left(\forall 1\leqslant i\leqslant
                  n\right)$
        \item 令$k_i=\min\left\{e_i,f_i\right\},l_i=\max\left\{e_i,f_i\right\}\left(\forall 1\leqslant i\leqslant n\right)$,则有
              \begin{align*}
                   & \left(f\left(x\right),g\left(x\right)\right)=p_1\left(x\right)^{k_1}p_2\left(x\right)^{k_2}\cdots p_n\left(x\right)^{k_n} \\
                   & \left[f\left(x\right),g\left(x\right)\right]=p_1\left(x\right)^{l_1}p_2\left(x\right)^{l_2}\cdots p_n\left(x\right)^{l_n}
              \end{align*}
        \item           \[f\left(x\right)g\left(x\right)\sim \left(f\left(x\right),g\left(x\right)\right)\left[f\left(x\right),g\left(x\right)\right]\]
    \end{enumerate}
}
\rem{}{}{
    考虑计算某一多项式是否具有重因式的问题,尽管我们可以计算出它的标准因式分解,但前提是找到合适的不可约多项式,这是极其困难的一件事.
}
\subsection{形式导数与重因式判定}
\dfn{形式导数}{形式导数}{
设$f\left(x\right)=a_nx^n+a_{n-1}x^{n-1}+\cdots+a_1x+a_0\in\mathbb{K}\left[x\right]$,则
\[
    f'\left(x\right)=na_{n}x^{n-1}+\left(n-1\right)a_{n-1}x^{n-2}+\cdots+a_1
\]
称为其形式导数.
}
\pro{形式导数的若干性质}{形式导数的若干性质}{
    \begin{enumerate}[label=\arabic*)]
        \item $\left(f\left(x\right)+g\left(x\right)\right)'=f'\left(x\right)+g'\left(x\right)$
        \item $\left(cf\left(x\right)\right)'=cf'\left(x\right)$
        \item $\left(f\left(x\right)g\left(x\right)\right)'=f'\left(x\right)g\left(x\right)+f\left(x\right)g'\left(x\right)$
        \item $\left(f\left(x\right)^m\right)'=mf\left(x\right)^{m-1}f'\left(x\right)$
        \item 若$\deg f\left(x\right)\geqslant 1$,则$\deg f'\left(x\right)=\deg f\left(x\right)-1$,那么
              \[
                  f\left(x\right)\nmid f'\left(x\right)
              \]
    \end{enumerate}
}
\cor{}{相同的不可约因子}{
    设多项式$f\left(x\right)\in\mathbb{K}\left[x\right],\deg f\left(x\right)\geqslant 1$,其形式导数为$f'\left(x\right)$,$d\left(x\right)=\left(f\left(x\right),f'\left(x\right)\right)$,则$\displaystyle
        \frac{f\left(x\right)}{d\left(x\right)}$一定没有重因式且与$f\left(x\right)$有相同的不可约因子(不计重数).\begin{proof}
        设标准分解
        \[
            f\left(x\right)=cp_1\left(x\right)^{e_1}p_2\left(x\right)^{e_2}\cdots p_m\left(x\right)^{e_m}
        \]
        其中$c\neq 0\in\mathbb{K}$,$p_i\left(x\right)\left(1\leqslant i\leqslant m\right)$为互异的首一不可约多项式,$e_i\geqslant 1\left(1\leqslant i\leqslant m\right)$.求导即有
        \begin{align*}
            f'\left(x\right)= & ce_1p_1\left(x\right)^{e_1-1}p_2\left(x\right)^{e_2}\cdots p_m\left(x\right)^{e_m}p_1'\left(x\right)  \\
                              & +ce_2p_1\left(x\right)^{e_1}p_2\left(x\right)^{e_2-1}\cdots p_m\left(x\right)^{e_m}p_2'\left(x\right) \\
                              & +\cdots                                                                                               \\
                              & +ce_mp_1\left(x\right)^{e_1}p_2\left(x\right)^{e_2}\cdots p_m\left(x\right)^{e_m-1}p_m'\left(x\right)
        \end{align*}
        故$p_1\left(x\right)^{e_1-1}p_2\left(x\right)^{e_2-1}\cdots p_m\left(x\right)^{e_m-1}$是$f\left(x\right),f'\left(x\right)$的一个公因式.下面证明它是最大公因式.

        设$d\left(x\right)=\left(f\left(x\right),f'\left(x\right)\right)=
            p_1\left(x\right)^{k_1}p_2\left(x\right)^{k_2}\cdots p_m\left(x\right)^{k_m}$
        .$e_i-1\leqslant k_i\leqslant
            e_i\left(\forall 1\leqslant i
            \leqslant m\right)$,断言全部取下界.

        考虑反证法,设$k_1=e_1$,即$p_1\left(x\right)^{e_1}\mid f'\left(x\right)$,显然矛盾.于是
        \[
            d\left(x\right)=p_1\left(x\right)^{e_1-1}p_2\left(x\right)^{e_2-1}\cdots p_m\left(x\right)^{e_m-1}
        \]
        则$\displaystyle
            \frac{f\left(x\right)}{d\left(x\right)}=cp_1\left(x\right)p_2\left(x\right)
            \cdots p_m\left(x\right)$无重因式且与$f\left(x\right)$有相同的不可约因子.
    \end{proof}
}
\thm{重因式判定}{重因式判定}{
    设$f\left(x\right)\in\mathbb{K}\left[x\right],\deg f\left(x\right)\geqslant 1$,则$f\left(x\right)$无重因式等价于
    \[\left(f\left(x\right),f'\left(x\right)\right)=1\]
}
\rem{}{}{
    \cref{thm:重因式判定}表明,计算一个多项式是否存在重因式时,可以
    避开标准分解(实际上找到标准分解不见得简单,甚至会非常难,因为想要证明或判定
    一个多项式在域上不可约实际上异常困难,目前也仅有一些充分条件而没有必要条件),转而利用辗转相除法求得
    该多项式与其导数的公因式从而根据
    定理\cref{thm:重因式判定}判定.
}