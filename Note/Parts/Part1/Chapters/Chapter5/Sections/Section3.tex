\newpage
\section{最大公因式}
\subsection{定义}
\dfn{公因式与公倍式}{公因式与公倍式}{
    设$f\left(x\right),g\left(x\right)\in\mathbb{K}\left[x\right]$,若$d\left(x\right)\in\mathbb{K}\left[x\right]$\st
    \[
        d\left(x\right)\mid f\left(x\right),d\left(x\right)\mid g\left(x\right)
    \]
    则称$d\left(x\right)$为$f\left(x\right)$与$g\left(x\right)$的一个公因式(公因子);
    若$m\left(x\right)\in\mathbb{K}\left[x\right]$\st
    \[
        f\left(x\right)\mid m\left(x\right),g\left(x\right)\mid m\left(x\right)
    \]
    则称$m\left(x\right)$是$f\left(x\right)$与$g\left(x\right)$的一个公倍式.
}
\dfn{最大公因式与最小公倍式}{最大公因式与最小公倍式}{
    若$d\left(x\right)$是$f\left(x\right)$与$g\left(x\right)$的公因式,且对于$f\left(x\right)$与$g\left(x\right)$的任一公因式$h\left(x\right)$,均有
    \[
        h\left(x\right)\mid d\left(x\right)
    \]
    则称$d\left(x\right)$是$f\left(x\right)$与$g\left(x\right)$的最大公因式,记作
    \[
        d\left(x\right)=\left(f\left(x\right),g\left(x\right)\right)=\gcd\left(f\left(x\right),g\left(x\right)\right)
    \]

    若$m\left(x\right)$是$f\left(x\right)$与$g\left(x\right)$的公倍式,且对于$f\left(x\right)$与$g\left(x\right)$的任一公倍式$l\left(x\right)$,均有
    \[
        m\left(x\right)\mid l\left(x\right)
    \]
    则称$m\left(x\right)$是$f\left(x\right)$与$g\left(x\right)$的最小公倍式,记作
    \[
        m\left(x\right)=\left[f\left(x\right),g\left(x\right)\right]=\lcm\left(f\left(x\right),g\left(x\right)\right)
    \]
}
\clm{零多项式的最大公因式}{}{
    约定零多项式的最大公因式为零多项式.
}
\thm{最大公因式存在性}{最大公因式存在性}{
    设$f\left(x\right),g\left(x\right)\in\mathbb{K}\left[x\right]$,且二者不全为零,则$f\left(x\right),g\left(x\right)$的最大公因式
    $d\left(x\right)$必定存在且存在
    $u\left(x\right),v\left(x\right)\in\mathbb{K}\left[x\right]$\st
    \[
        d\left(x\right)=f\left(x\right)u\left(x\right)+g\left(x\right)v\left(x\right)
    \]\begin{proof}
        利用Euclid辗转相除法证明.

        其一,若$f\left(x\right)=0$,则$\left(f\left(x\right),g\left(x\right)\right)=g\left(x\right)$,反之亦然.

        设$f\left(x\right)\neq 0,g\left(x\right)\neq 0$.作带余除法
        \[
            f\left(x\right)=g\left(x\right)q_1\left(x\right)+r_1\left(x\right)
        \]
        其中$\deg r_1\left(x\right)<\deg g\left(x\right)$.于是\[
            g\left(x\right)=r_1\left(x\right)q_2\left(x\right)+r_2\left(x0\right)
        \]
        其中$\deg r_2\left(x\right)<\deg r_1\left(x\right)$.如此重复即
        \[
            r_{s-2}\left(x\right)=r_{s-1}\left(x\right)q_s\left(x\right)+r_s\left(x\right)
        \]
        其中$\deg r_s\left(x\right)<\deg r_{s-1}\left(x\right)$.再做一次即得
        \[
            r_{s-1}\left(x\right)=r_s\left(x\right)q_{s+1}\left(x\right)+r_{s+1}\left(x\right)
        \]
        其中$\deg r_{s+1}\left(x\right)<\deg r_s\left(x\right)$.考虑到余式次数严格递减, 这个过程必定在某一步得到一个为零的余式,不妨设$r_{s+1}\left(x\right)=0,r_{s}\left(x\right)\neq 0.$断言
        \[
            r_s\left(x\right)=\gcd\left(f\left(x\right),g\left(x\right)\right)
        \]
        约定$r_{-1}\left(x\right)=f\left(x\right),r_0\left(x\right)=
            g\left(x\right)$.
        只需倒回验证即可.存在性证毕.

        因为\[
            r_s\left(x\right)=r_{s-2}\left(x\right)-r_{s-1}\left(x\right)q_s\left(x\right)
        \]
        且\[
            r_{s-1}\left(x\right)=r_{s-3}\left(x\right)-r_{s-2}\left(x\right)q_{s-1}\left(x\right)
        \]
        则\[r_s\left(x\right)=
            r_{s-3}\left(x\right)\left(-q_s\left(x\right)\right)+\left(1+q_{s-1}\left(x\right)q_s\left(x\right)\right)
            r_{s-2}\left(x\right)
        \]
        迭代即得
        \[
            r_s\left(x\right)=r_{-1}\left(x\right)u\left(x\right)+r_0\left(x\right)v\left(x\right)
            \qedhere\]
    \end{proof}
}
\rem{}{}{
    反过来,若存在$u\left(x\right),v\left(x\right)$\st
    \[
        d\left(x\right)=f\left(x\right)u\left(x\right)+g\left(x\right)v\left(x\right)
    \]
    却不能推出$d\left(x\right)=\left(f\left(x\right),g\left(x\right)\right)$.除非加上$d\left(x\right)\mid f\left(x\right)$且$d\left(x\right)\mid g\left(x\right)$的条件.
}
\subsection{唯一性}
\thm{最大公因式不唯一}{最大公因式不唯一}{
    设$f\left(x\right),g\left(x\right)\in\mathbb{K}\left[x\right]$的两个最大公因式为
    $d_1\left(x\right),d_2\left(x\right)$,
    则$d_1\left(x\right)\mid
        d_2\left(x\right),d_2
        \left(x\right)\mid d_1\left(x\right)$.那么必定存在$c\neq 0\in\mathbb{K}$\st
    \[d_2\left(x\right)=cd_1\left(x\right)\]
}
\dfn{首一多项式}{首一多项式}{
    称首项系数为$1$的多项式为首一多项式.
}
\clm{}{}{
    一般地,约定最大公因式均指首一多项式,如此约定下最大公因式存在且唯一.
}
\thm{最小公倍式不唯一}{最小公倍式不唯一}{
    设多项式$f\left(x\right),g\left(x\right)\in\mathbb{K}\left[x\right]$的两个最小公倍式为
    $m_1\left(x\right),m_2\left(x\right)$,
    则$m_1\left(x\right)\mid
        m_2\left(x\right),m_2
        \left(x\right)\mid m_1\left(x\right)$.那么必
    定存在$c\neq 0\in\mathbb{K}$\st
    \[m_2\left(x\right)=cm_1\left(x\right)\]
}
\clm{}{}{
    一般地,约定最小公倍式均指首一多项式,如此约定下最小公倍式存在且唯一(存在性证明见).
}
\subsection{有限多个多项式}
\dfn{有限多个多项式的公因式与最大公因式}{有限多个多项式的公因式与最大公因式}{
    设存在有限多个多项式$f_1\left(x\right),f_2\left(x\right),\cdots,f_m\left(x\right)\in\mathbb{K}\left[x\right]\left(m\geqslant 2\right)$,若$d\left(x\right)\mid f_i\left(x\right)\left(\forall 1\leqslant i\leqslant m\right)$,则称$d\left(x\right)$为
    $f_1\left(x\right),f_2\left(x\right),\cdots,f_m\left(x\right)$的一个公因式.
    进一步地,若$f_1\left(x\right),f_2\left(x\right),\cdots,f_m\left(x\right)$的任一公因式$h\left(x\right)$均有
    \[
        h\left(x\right)\mid d\left(x\right)
    \]
    则称$d\left(x\right)$为$f_1\left(x\right),f_2\left(x\right),\cdots,f_m\left(x\right)$的最大公因式,记作\[
        d\left(x\right)=\left(
        f_1\left(x\right),f_2\left(x\right),\cdots,f_m\left(x\right)
        \right)\left(m\geqslant 2\right)
    \]
}
\dfn{有限多个多项式的公倍式与最小公倍式}{有限多个多项式的公倍式与最小公倍式}{
    设存在有限多个多项式$f_1\left(x\right),f_2\left(x\right),\cdots,f_m\left(x\right)\in\mathbb{K}\left[x\right]\left(m\geqslant 2\right)$,若$f_i\left(x\right)|m\left(x\right)\left(\forall 1\leqslant i\leqslant m\right)$,则称$m\left(x\right)$为
    $f_1\left(x\right),f_2\left(x\right),\cdots,f_m\left(x\right)$的一个公倍式.
    进一步的,若$f_1\left(x\right),f_2\left(x\right),\cdots,f_m\left(x\right)$的任一公倍式$l\left(x\right)$均有
    \[
        m\left(x\right)\mid l\left(x\right)
    \]
    则称$m\left(x\right)$为
    $f_1\left(x\right),f_2\left(x\right),\cdots,f_m\left(x\right)$的最小公倍式,记作\[
        m\left(x\right)=\left[
            f_1\left(x\right),f_2\left(x\right),\cdots,f_m\left(x\right)
            \right]\left(m\geqslant 2\right)
    \]
}
\lem{}{多个多项式的最大公因式的求法}{
    设多项式$f_1\left(x\right),f_2\left(x\right),\cdots,f_m\left(x\right)\in\mathbb{K}\left[x\right]\left(m\geqslant 3\right)$,则\[
        \left(f_1\left(x\right),f_2\left(x\right),\cdots,f_m\left(x\right)\right)=\left(\left(f_1\left(x\right),f_2\left(x\right)\right),f_3\left(x\right),\cdots,f_m\left(x\right)\right)
    \]\begin{proof}
        设$d\left(x\right)=\left(f_1\left(x\right),f_2\left(x\right),\cdots,f_m\left(x\right)\right),d_{12}\left(x\right)=\left(f_1\left(x\right),f_2\left(x\right)\right)$.
        只需证\[
            d\left(x\right)=\left(d_{12}\left(x\right),f_3\left(x\right),\cdots,f_m\left(x\right)\right)
        \]
        一方面,$d\left(x\right)|f_i\left(x\right)\left(\forall 1\leqslant i\leqslant m\right)$,故$d\left(x\right)|d_{12}\left(x\right),d\left(x\right)|f_i\left(x\right)\left(\forall 3\leqslant i\leqslant m\right)$.另一方面,任取$h\left(x\right)$为$d_{12}\left(x\right),f_3\left(x\right),\cdots,f_m\left(x\right)$,则
        $h\left(x\right)\mid d_{12}\left(x\right)$,故$h\left(x\right)\mid f_1\left(x\right)$且
        $h\left(x\right)\mid f_2\left(x\right)$,于是$h\left(x\right)\mid f_i\left(x\right)\left(\forall 1\leqslant i\leqslant m\right)$,因此$h\left(x\right)\mid d\left(x\right)$.
    \end{proof}
}
\subsection{互素}
\dfn{互素}{互素}{
    设$f\left(x\right),g\left(x\right)\in\mathbb{K}\left[x\right]$,若$\gcd\left(f\left(x\right),g\left(x\right)\right)=1$,称$f\left(x\right),g\left(x\right)$互素.
}
\thm{互素的判定}{互素的判定}{
    设$f\left(x\right),g\left(x\right)\in\mathbb{K}\left[x\right]$,则$f\left(x\right)$与$g\left(x\right)$互素等价于存在$u\left(x\right),v\left(x\right)\in\mathbb{K}\left[x\right]$\st
    \[
        f\left(x\right)u\left(x\right)+g\left(x\right)v\left(x\right)=1
    \]\begin{proof}
        必要性,即已知互素,$d\left(x\right)=1$,由\cref{thm:最大公因式存在性}得证.

        充分性,设$d\left(x\right)=\left(f\left(x\right),g\left(x\right)\right)$,则
        $d\left(x\right)\mid f\left(x\right)u\left(x\right)+g\left(x\right)v\left(x\right)=1$,即$\exists
            t\left(x\right)\in\mathbb{K}\left[x\right]$\st$1=d\left(x\right)t\left(x\right)$.两边同时取次数即得
        $d\left(x\right)=a\neq 0$,考虑到首一多项式,$d\left(x\right)=1$.
    \end{proof}
}
\cor{}{互素因子之积仍是因子}{
    设$f_1\left(x\right)\mid f\left(x\right)$且$f_2\left(x\right)\mid f\left(x\right)$,若$\left(f_1\left(x\right),f_2\left(x\right)\right)=1$,则\[
        f_1\left(x\right)f_2\left(x\right)\mid f\left(x\right)
    \]\begin{proof}
        设$u\left(x\right),v\left(x\right)\in\mathbb{K}\left[x\right]$\st
        \[
            f_1\left(x\right)u\left(x\right)+f_2\left(x\right)v\left(x\right)=1
        \]
        即\[
            f_1\left(x\right)u\left(x\right)f\left(x\right)+f_2\left(x\right)v\left(x\right)f\left(x\right)=f\left(x\right)
        \]
        明显$f_1\left(x\right)f_2\left(x\right)\mid f_1\left(x\right)u\left(x\right)f\left(x\right)$,$f_1\left(x\right)f_2\left(x\right)\mid
            f_2\left(x\right)v\left(x\right)f\left(x\right)$.
    \end{proof}
}
\cor{}{互素多项式之积的因子也是每个的因子}{
    若$\left(f\left(x\right),g\left(x\right)\right)=1$,且$f\left(x\right)\mid g\left(x\right)h\left(x\right)$,则\[
        f\left(x\right)\mid h\left(x\right)
    \]\begin{proof}
        类似地
        \[
            f\left(x\right)u\left(x\right)h\left(x\right)+g\left(x\right)v\left(x\right)h\left(x\right)=h\left(x\right)
            \qedhere
        \]
    \end{proof}
}
\cor{}{素多项式之积仍然是素多项式}{
    若$\left(f_1\left(x\right),g\left(x\right)\right)=1,\left(f_2\left(x\right),g\left(x\right)\right)=1$,则
    \[
        \left(f_1\left(x\right)f_2\left(x\right),g\left(x\right)\right)=1
    \]\begin{proof}
        因为\[
            f_1\left(x\right)u_1\left(x\right)+g\left(x\right)v_1\left(x\right)=1
        \]
        \[
            f_2\left(x\right)u_2\left(x\right)+g\left(x\right)v_2\left(x\right)=1
        \]
        于是\begin{align*}
              & \left(f_1\left(x\right)f_2\left(x\right)\right)\left(u_1\left(x\right)u_2\left(x\right)\right)                                                                                          \\
            + & g\left(x\right)\left[f_1\left(x\right)u_1\left(x\right)v_2\left(x\right)+f_2\left(x\right)u_2\left(x\right)v_1\left(x\right)+g\left(x\right)v_1\left(x\right)v_2\left(x\right)\right]=1
            \qedhere
        \end{align*}
    \end{proof}
}
\cor{}{提取互素部分}{
    设$d\left(x\right)=\left(f\left(x\right),g\left(x\right)\right)$,且$f\left(x\right)=
        f_1\left(x\right)d\left(x\right),g\left(x\right)=g_1\left(x\right)d\left(x\right)$,则\[
        \left(f_1\left(x\right),g_1\left(x\right)\right)=1
    \]
}
\cor{}{因子的叠加}{
    设$\left(f\left(x\right),g\left(x\right)\right)=d\left(x\right)$,则\[
        \left(t\left(x\right)f\left(x\right),t\left(x\right)g\left(x\right)\right)=t\left(x\right)d\left(x\right)
    \]\begin{proof}
        因为\[
            f\left(x\right)u\left(x\right)+g\left(x\right)v\left(x\right)=d\left(x\right)
        \]
        则\[
            f\left(x\right)t\left(x\right)u\left(x\right)+g\left(x\right)t\left(x\right)v\left(x\right)=t\left(x\right)d\left(x\right)
        \]
        任取$f\left(x\right)t\left(x\right),g\left(x\right)t\left(x\right)$公因子为$h\left(x\right)$,则
        \[
            h\left(x\right)\mid t\left(x\right)d\left(x\right)
            \qedhere
        \]
    \end{proof}
}
\subsection{最小公倍式}
\thm{最小公倍式存在性}{最小公倍式存在性}{
    设非零多项式$f\left(x\right),g\left(x\right)$,则\[
        f\left(x\right)g\left(x\right)\sim \left(f\left(x\right)g\left(x\right)\right)\left[f\left(x\right),g\left(x\right)\right]
    \]\begin{proof}
        设最大公因式$d\left(x\right)=\left(f\left(x\right),g\left(x\right)\right),
            f\left(x\right)=f_1\left(x\right)d\left(x\right),g\left(x\right)=g_1\left(x\right)d\left(x\right)$,于是$\left(
            f_1\left(x\right),g_1\left(x\right)
            \right)=1.$于是\[
            \frac{
                f\left(x\right)g\left(x\right)
            }{\left(f\left(x\right),g\left(x\right)\right)}=f_1\left(x\right)d\left(x\right)g_1\left(x\right)
        \]
        只需证明$f_1\left(x\right)d\left(x\right)g_1\left(x\right)$为最小公倍式.

        一方面,显然这是$f\left(x\right),g\left(x\right)$的一个公倍式.另一方面,任取$l\left(x\right)$为$f\left(x\right),g\left(x\right)$的一个公倍式.
        因为$l\left(x\right)=f\left(x\right)u\left(x\right)=g\left(x\right)
            v\left(x\right)$.于是
        \[
            f_1\left(x\right)u\left(x\right)=g_1\left(x\right)v\left(x\right)
        \]因为$\left(f_1\left(x\right),g_1\left(x\right)\right)=1$,故\[
            g_1\left(x\right)\mid u\left(x\right)
        \]设$u\left(x\right)=g_1\left(x\right)p\left(x\right)$,于是$l\left(x\right)=f_1\left(x\right)d\left(x\right)g_1\left(x\right)p\left(x\right)$.
    \end{proof}
}
\subsection{Chinese Remainden Theorem}
\thm{Chinese Remainden Theorem}{Chinese Remainden Theorem}{ 中国剩余定理,又叫孙子定理.

    设$g_1\left(x\right),\cdots,g_m\left(x\right)$为两两互素的多项式,
    $r_1\left(x\right),\cdots,r_m\left(x\right)\in\mathbb{K}\left[x\right]$,则存在$f\left(x\right)\in\mathbb{K}\left[x\right]$使得
    \[
        f\left(x\right)=g_i\left(x\right)q_i\left(x\right)+r_i\left(x\right)\left(\forall 1\leqslant i\leqslant m\right)
    \]\begin{proof}
        构造\[f_i\left(x\right)=g_i\left(x\right)p_i\left(x\right)+1\left(\forall 1\leqslant i\leqslant m\right)\]
        且\[
            g_j\left(x\right)\mid f_i\left(x\right)\left(j\neq i\right)
        \]
        那么$\displaystyle
            f\left(x\right)=\sum_{i=1}^{m}r_i\left(x\right)f_i\left(x\right)
            .$

        $f_1\left(x\right)$满足如下条件:
        \begin{align*}
             & f_1\left(x\right)\equiv 1\pmod {g_1\left(x\right)}         \\
             & f_1\left(x\right)\equiv 0\pmod {g_j\left(x\right),j\neq 1}
        \end{align*}
        因为$\left(g_1\left(x\right),g_j\left(x\right)\right)=1\left(j\neq 1\right)$.故$\exists u_j\left(x\right),v_j\left(x\right)\in\mathbb{K}\left[x\right]$\st
        \[
            g_1\left(x\right)u_j\left(x\right)+g_j\left(x\right)v_j\left(x\right)=1
        \]
        于是构造$f_1\left(x\right)$如下:
        \begin{align*}
            f_1\left(x\right) & =g_2\left(x\right)v_2\left(x\right)\cdots g_n\left(x\right)v_n\left(x\right)                              \\
                              & =\left(1-g_1\left(x\right)u_2\left(x\right)\right)\cdots\left(1-g_1\left(x\right)u_n\left(x\right)\right)
        \end{align*}
        其余同理,证毕.
    \end{proof}
}