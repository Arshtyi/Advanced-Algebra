\newpage
\section{多项式函数}
\subsection{根}
\dfn{多项式函数}{多项式函数}{
设$f\left(x\right)=a_nx^n+a_{n-1}x^{n-1}+\cdots+a_1x+a_0\in\bbk \left[x\right]$,则其是一个函数
\begin{align*}
    f:\bbk & \longrightarrow \bbk        \\
    b      & \longmapsto f\left(b\right)
\end{align*}
$f\left(b\right)\coloneqq
    a_nb^n+a_{n-1}b^{n-1}+\cdots a_1b+a_0$称为$f\left(x\right)$在$b$点的取值.

于是对于一个多项式$f$,我们可以自然地视作一个多项式函数,记作$\overline{f}$.
}
\rem{}{}{
    一个问题是,两个(一元)多项式作为多项式相等与作为多项式函数相等是否是一回事?前者推后者显然, 因为我们给出的多项式形式定义足够强.我们下面来考虑多项式相等得出多项式函数相等这件事.

    一般地,多项式$f\in\bbk \left[x\right]$诱导的多项式函数记作$\overline{f}$,即我们需要证明
    \[
        f=g\Longleftrightarrow \overline{f}=\overline{g}
    \]

    在数域上,这一问题可能有一点“显然”,但实际上,数域只是一种特殊的域,后面我们将说明,这个等价关系在一般的域上不见得成立.
}
\dfn{根}{根}{
    设$f\left(x\right)\neq 0\in\bbk
        \left[x\right]$,若$b\in\bbk $,\st$f\left(b\right)=0$,则称$b$是$f\left(x\right)$的根或零点.
}
\thm{余数定理}{余数定理}{
    设$f\left(x\right)\in\bbk \left[x\right],b\in\bbk $,则存在$g\left(x\right)\in\bbk $,\st
    \[
        f\left(x\right)=\left(x-b\right)g\left(x\right)+f\left(b\right)
    \]
    特别地,$b$是$f\left(x\right)$的根当且仅当$\left(x-b\right)\mid f\left(x\right)$.\begin{proof}
        首先有带余除法
        \[
            f\left(x\right)=\left(x-b\right)q\left(x\right)+r\left(x\right)
        \]
        其中$\deg r\left(x\right)<\deg \left(x-b\right)=1\Longrightarrow r\left(x\right)$为非零常数,
        令上式$x=b\Longrightarrow r\left(x\right)=f\left(b\right)$.
    \end{proof}
}
\dfn{重根}{重根}{
    设$f\left(x\right)\in\bbk \left[x\right],b\in\bbk $,若存在
    $k\in\mathbb{Z}^+$\st$\left(x-b\right)^k\mid f\left(x\right)$但$\left(x-b\right)^{k+1}\nmid f\left(x\right).$称$b$是$f\left(x\right)$的$k$重根.特别地,$k=1$时称为单根.
}
\dfn{重根的等价定义}{重根的等价定义}{
$k$重根等价于
\[
    f\left(b\right)=f'\left(b\right)=\cdots=f^{\left(k-1\right)}\left(b\right)=0,f^{\left(k\right)}\left(b\right)\neq 0
\]
}
\clm{}{}{
    $k$重根视作$k$个根.
}
\lem{不可约多项式的根}{不可约多项式的根}{
    设$\bbk $上的不可约多项式$f\left(x\right)$,若$\deg f\left(x\right)\geqslant 2$,则$f\left(x\right)$在$\bbk $上没有根.\begin{proof}
        考虑反证法,设$b\in\bbk $是$f\left(x\right)$的根,从而$\left(x-b\right)\mid f\left(x\right)$即
        \[
            f\left(x\right)=\left(x-b\right)q\left(x\right)
        \]
        于是$\deg q\left(x\right)\geqslant 1.$这与不可约矛盾.
    \end{proof}
}
\thm{根的个数}{根的个数}{
    设数域$\bbk $上的$n$次多项式$f\left(x\right)$,
    则$f\left(x\right)$在数域$\bbk $上至多有$n$个根.\begin{proof}
        作标准因式分解
        \[
            f\left(x\right)=cp_1\left(x\right)^{e_1}p_2\left(x\right)^{e_2}\cdots p_m\left(x\right)^{e_m}
        \]
        将一次项单独拿出
        \[
            f\left(x\right)=c\left(x-b_1\right)^{n_1}\left(x-b_2\right)^{n_2}\cdots\left(x-b_r\right)^{n_r}p_1\left(x\right)^{e_1}p_2\left(x\right)^{e_2}\cdots p_s\left(x\right)^{e_s}
        \]
        其中$c\neq 0,b_1,b_2,\cdots,b_r$互不相同,$p_1\left(x\right),p_2\left(x\right),\cdots,p_s\left(x\right)$是互异的首一不可约多项式且\[\deg p_i\left(x\right)\geqslant 1\left(\forall 1\leqslant i\leqslant
            s\right)\]于是$f\left(x\right)$的根为$b_1,b_2,\cdots,b_r$,其分别为$n_1,n_2,\cdots,n_r$重.两边同时取次数即得
        \[
            n=n_1+n_2+\cdots+n_r+e_1\deg p_1\left(x\right)+e_2\deg p_2\left(x\right)+\cdots+e_s\deg p_s\left(x\right)
        \]于是$f\left(x\right)$在数域$\bbk $上的根一共有$n_1+n_2+\cdots+n_r\leqslant n$个.
    \end{proof}
}
\lem{}{多项式相等则多项式函数相等}{
设多项式$f\left(x\right),g\left(x\right)\in\bbk \left[x\right],\deg f\left(x\right)\leqslant n,\deg g\left(x\right)\leqslant n$,若$\bbk $中存在$n+1$个数$b_1,b_2,\cdots,b_{n+1}$\st
\[
    f\left(b_i\right)=g\left(b_i\right)\left(\forall 1\leqslant i\leqslant n+1\right)
\]
则$f\left(x\right)=g\left(x\right)$(作为多项式而言).\begin{proof}
    设$h\left(x\right)=f\left(x\right)-g\left(x\right)$,由\cref{thm:根的个数}显然.
\end{proof}
}
至此,我们得到了作为多项式相等则作为多项式函数相等的结论.下面证明作为多项式函数相等则作为多项式相等.
\begin{proof}
    设$\deg f\left(x\right)\leqslant n,\deg g\left(x\right)\leqslant n.$因为$\bbk \supseteq \bbq  $,故能取出$n+1$个不同的数
    $b_1,b_2,\cdots,b_{n+1}\in\bbk $
    \st$f\left(b_i\right)=g\left(b_i\right)\left(\forall 1\leqslant i\leqslant n+1
        \right)$,则二者作为多项式恒等.
\end{proof}
\subsection{域}
\rem{}{}{
    上述证明的关键在于“数域”$\bbk $是无穷集.但对于一般的域而言,却不见得成立.

    在一般的域上,一般来说,作为多项式相等和作为多项式函数相等,是不一样的.
}
\exa{}{}{
    考虑伽罗瓦域$\bbf _2=\left\{\overline{0},\overline{1}\right\}$,这里的$\overline{0}$可以理解为全体偶数,$\overline{1}$理解为全体奇
    数.定义加法
    \begin{align*}
        \overline{0}+\overline{0}=\overline{0},\overline{1}+\overline{1}=\overline{0},\overline{1}+\overline{0}=\overline{1}
    \end{align*}
    和乘法
    \[
        \overline{0}\cdot \overline{0}=\overline{0},
        \overline{0}\cdot \overline{1}=\overline{0},
        \overline{1}\cdot\overline{1}=\overline{1}
    \]
    容易验证这个域对加减乘除封闭.同时这是一个有限域,那么上面的定理就不成立了.

    比如,考虑$f\left(x\right)=x^2,g\left(x\right)=x$,二者作为多项式不相等,但是作为$\bbf _2$上的
    多项式函数是相等的.
}