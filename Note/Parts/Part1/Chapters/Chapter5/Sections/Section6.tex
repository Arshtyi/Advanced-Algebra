\newpage
\section{复系数多项式}
\subsection{二元连续函数}
我们需要二元连续函数的一些相关知识作为铺垫.
\dfn{连续}{连续}{
    设$f\left(x,y\right)$是$\bbr ^2$上的函数, 则$f\left(x,y\right)$在$\left(x_0,y_0\right)$连续定义为$\forall \epsilon >0,\exists \delta >0,\forall \left(x,y\right)\in O_{\left(x_0,y_0\right),\delta}$
    \[
        \left|f\left(x,y\right)-f\left(x_0,y_0\right)\right|<\epsilon
    \]
}
\thm{二元多项式的连续性}{二元多项式的连续性}{
    二元多项式都是二元连续函数.
}
\lem{二元连续函数的最值}{二元连续函数的最值}{
    任一二元连续函数$f\left(x,y\right)$一定能在闭圆盘$D=\left\{
        \left(x,y\right)\big|\left(x-
        x_0\right)^2+\left(y-y_0\right)^2\leqslant r^2
        \right\}$上取到最大值和最小值.
}
在承认该命题的前提下,我们来证明代数学基本定理(其有多种等价表述,证明思路同样多样).
\subsection{代数学基本定理}
\thm{代数学基本定理}{代数学基本定理}{
    任何次数大于等于$1$的复系数多项式至少有一个复根.或者说,复数域上的不可约多项式只能是一次的.\begin{proof}
        设\[f\left(z\right)=a_nz^n+a_{n-1}z^{n-1}+\cdots+a_1z+a_0\in\bbc \left[z\right],a_n\neq 0,n\geqslant 1\]

        第一步,证明:$\exists z_0\in\bbc $\,\textup{s.t.}$\forall z\in\bbc ,
            \left|f\left(z\right)\right|\geqslant \left|f\left(z_0\right)\right|$.
        设$z=x+y\mathrm{i},x,y\in\bbr $,则$f\left(z\right)=f\left(x+y\mathrm{i}\right)=
            P\left(x,y\right)+Q\left(x,y\right)\mathrm{i}$,其中$P\left(x,y\right),Q\left(x,y\right)$是二元多项式.那么\[
            \left|f\left(z\right)\right|
            =\sqrt{P\left(x,y\right)^2+
                Q\left(x,y\right)^2}\]
        故$\left|f\left(z\right)\right|$是关于$x,y$
        的二元连续函数.

        考虑估计
        \[
            \left|z_1\right|-\left|z_2\right|\leqslant \left|z_1+z_2\right|\leqslant \left|z_1\right|+\left|z_2\right|,\forall z_1,z_2\in\bbc
        \]
        那么
        \begin{align*}
            \left|f\left(z\right)\right| & =\left|a_nz^n+a_{n-1}z^{n-1}+\cdots+a_1z+a_0\right|                                    \\
                                         & \geqslant \left|a_n\right|\left|z\right|^n-\left|a_{n-1}z^{n-1}+\cdots+a_1z+a_0\right| \\
                                         & \geqslant\cdots                                                                        \\
                                         & \geqslant \left|z\right|^n\left(
            \left|a_n\right|-
            \left(
                \frac{\left|a_{n-1}\right|}{\left|z\right|}+
                \cdots+
                \frac{\left|a_1\right|}{\left|z^{n-1}\right|}+
                \frac{\left|a_0\right|}{\left|z^n\right|}
                \right)
            \right)
        \end{align*}
        $\left|z\right|\to \infty$时,上式\[
            \geqslant \frac{1}{2}\left|a_n\right|\left|z\right|^n\to
            \infty\]

        任取$z_1\in\bbc ,\exists R\in\bbr ,\st\left|f\left(z\right)\right|\geqslant \left|f\left(z_1\right)\right|\left(
            \left|z\right|>R\right)$,由二元连续函数的性质知,
        $\left|f\left(z\right)\right|$在闭圆盘
        $D=\left\{z\mid \left|z\right|\leqslant R\right\}$
        取得最小值.不妨设为$z_0$,即
        \[
            \left|f\left(z\right)\right|\geqslant \left|f\left(z_0\right)\right|,\forall z\in
            D
        \]
        于是$\left|f\left(z\right)\right|\left(z\in\bbc
            \right)$在$z_0$取
        到最小值.

        第二步,证明$f\left(z_0\right)=0.$考虑反证法,设$f\left(z_0\right)\neq 0.$ 作一微小摄动$z=z_0+h$,则
        \[
            f\left(z_0+h\right)=b_nh^n+b_{n-1}h^{n-1}+\cdots+b_1h+b_0
        \]
        其中,$b_n=a_n\neq 0,b_0=f\left(z_0\right)\neq 0.$则
        \begin{align*}
            \frac{f\left(z_0+h\right)}{f\left(z_0\right)} & =1+\frac{b_1}{f\left(z_0\right)}h+
            \cdots+\frac{b_{n-1}}{f\left(z_0\right)}h^{n-1}+\frac{b_n}{f\left(z_0\right)}h^n     \\
                                                          & =1+c_1h+\cdots+c_{n-1}h^{n-1}+c_nh^n
        \end{align*}
        设存在$1\leqslant k\leqslant n\st c_1=\cdots=c_{k-1}=0,c_k\neq 0.
        $于是
        \begin{align*}
            \frac{f\left(z_0+h\right)}{f\left(z_0\right)}=1+c_kh^k+\cdots+c_nh^n
        \end{align*}
        令\[
            h=e\cdot d,d=\sqrt[k]{-\frac{1}{c_k}},0<e\ll 1\]于是
        \begin{align*}
            \left|
            \frac{f\left(z_0+h\right)}{f\left(z_0\right)}\right| & =
            \left|1-e^k+c_{k+1}e^{k+1}d^{k+1}+\cdots+c_ne^nd^n
            \right|
            \\
                                                                 & \leqslant
            \left|1-e^k\right|+\left|c_{k+1}e^{k+1}d^{k+1}+\cdots+c_ne^nd^n
            \right|                                                                                                                                  \\
                                                                 & =1-e^k+e^{k+1}\left(\left|c_{k+1}d^{k+1}\right|+\cdots+\left|c_nd^n\right|\right)
        \end{align*}
        当$e$充分小时\[
            LHS\leqslant 1-e^k+\frac{1}{2}e^k<1\]与$\left|f\left(z_0\right)\right|$最小矛盾.
    \end{proof}
}
\cor{代数学基本定理的等价命题}{代数学基本定理的等价命题}{
    设$n\geqslant 1$,则下列结论等价\begin{enumerate}[label=\arabic*)]
        \item 任一$n$次复系数多项式至少有一个复根
        \item 复数域上的不可约多项式都是一次多项式
        \item 任一复系数多项式都是一次多项式的乘积
        \item 任一$n$次复系数多项式恰好有$n$个复根(计重数)
    \end{enumerate}\begin{proof}
        $(1)\Longrightarrow (2)$设$p\left(x\right)$在$\bbc $不可约,设$p\left(b\right)=0$则由前文知$\left(x-b\right)|p\left(x\right)\Longrightarrow p\left(x\right)=\left(x-b\right)q\left(x\right)\Longrightarrow q\left(x\right)=c\neq 0.$

        $(2)\Longrightarrow (3)$由因式分解定理立得.

        $(3)\Longrightarrow (4)$$f\left(x\right)=c\left(x-b_1\right)\left(x-b_2\right)\cdots\left(x-b_n\right)$.则$(4)\Longrightarrow (1)$显然.
    \end{proof}
}
\subsection{Vieta定理}
\thm{Vieta定理}{Vieta定理}{
设$f\left(x\right)=a_0x^n+a_1x^{n-1}+\cdots+a_{n-1}x+a_n\in\mathbb{K}\left[x
        \right]$且其复根为$x_1,x_2,\cdots,x_n$,则
\[
    \sum_{i=1}^{n}x_i=-\frac{a_1}{a_0},
\]
\[
    \sum_{1\leqslant i<j\leqslant n}x_ix_j=\frac{a_2}{a_0},
\]
\[
    \sum_{1\leqslant i<j<k\leqslant n}x_ix_jx_k=-\frac{a_3}{a_0},
\]
\[
    \cdots\cdots\cdots\cdots
\]
\[
    \sum_{1\leqslant i_1<i_2<\cdots<i_r\leqslant n}x_{i_1}x_{i_2}\cdots x_{i_r}=\left(-1\right)^r\frac{a_r}{a_0},
\]
\[
    \cdots\cdots\cdots\cdots
\]
\[
    \prod_{i=1}^{n}x_i=\left(-1\right)^n\frac{a_n}{a_0}
\]\begin{proof}
    $f\left(x\right)=c\left(x-x_1\right)\left(x-x_2\right)\cdots\left(x-x_n\right)$,展开即得.
\end{proof}
}
\subsection{一元方程的求解}
一元$n$次方程的求解在历史上是持久的问题,下面都是在复数域$\bbc $讨论.
\subsubsection{一元二次方程}
一元二次方程$ax^2+bx+c=0$的求根公式为
\[
    x_{1,2}=\frac{
        -b\pm \sqrt{b^2-4ac}
    }{2a}
\]
\subsubsection{一元三次方程}
设方程
\[
    x^3+ax^2+bx+c=0
\]
作$\displaystyle
    x=y-\frac{1}{3}a$得
\[
    y^3+py+q=0
\]

于是问题转为对形如
$f\left(x\right)=x^3+px+q=0$的方程的解的讨论.

\begin{enumerate}[label=\arabic*)]
    \item $q=0$时$x_1=0,x_2=\sqrt{-p},x_3=-\sqrt{-p}.p=0$时$x_1=\sqrt[3]{-q},x_2=\sqrt[3]{-q}\omega,x_3=\sqrt[3]{-q}\omega^2.$其中$\omega$为三次单位根
          \[
              \omega ^3=1,\omega = -\frac{1}{2}+\frac{\sqrt{3}}{2}\mathrm{i}
          \]
          因此解的形式也可以是
          $x_i=\sqrt[3]{-q}\omega^i\left(i=1,2,3
              \right)$.
    \item 考虑$p\neq 0,q\neq 0$.引$x=u+v$,则
          \[
              x^3-3uvx-\left(u^3+v^3\right)=0
          \]
          于是\[
              \begin{cases*}
                  uv=-\frac{1}{3}p \\
                  u^3+v^3=-q
              \end{cases*}
          \]
          则\[
              u^3,v^3=-\frac{q}{2}\pm \sqrt{\frac{q^2}{4}+\frac{p^3}{27}}
          \]
          令\[
              \Delta = \frac{q^2}{4}+\frac{p^3}{27}
          \]
          则
          \[
              \begin{cases*}
                  x_1=\sqrt[3]{-\cfrac{q}{2}+\sqrt{\Delta}}+\sqrt[3]{-\cfrac{q}{2}-\sqrt{\Delta}}                \\
                  x_2=\omega \sqrt[3]{-\cfrac{q}{2}+\sqrt{\Delta}}+\omega^2\sqrt[3]{-\cfrac{q}{2}-\sqrt{\Delta}} \\
                  x_3=\omega^2 \sqrt[3]{-\cfrac{q}{2}+\sqrt{\Delta}}+\omega\sqrt[3]{-\cfrac{q}{2}-\sqrt{\Delta}}
              \end{cases*}
          \]
          此式称为Cardano(卡尔达诺公式)
\end{enumerate}

\subsubsection{一元四次方程——Ferrari解法}
考虑\[
    x^4+ax^3+bx^2+cx+d=0
\]
作$\displaystyle
    x=y-\frac{1}{4}a$转为求解方程
\[
    x^4+ax^2+bx+c=0
\]
引未知数$u$并加减$\displaystyle
    ux^2+\frac{u^2}{4}$项得
\[
    \left(x^2+\frac{u}{2}\right)^2=\left[
        \left(u-a\right)x^2-bx+\frac{u^2}{4}-c
        \right]
\]
右边是完全平方的条件为
\[
    b^2-4\left(u-a\right)\left(\frac{u^2}{4}-c\right)=0
\]
称为预解式.解出$u$后(考虑到对称性,此三次方程的根取其一即可)得\[
    \left(x^2+\frac{u}{2}\right)^2=
    \left(
    \sqrt{u-a}x-\frac{b}{2\sqrt{u-a}}
    \right)^2
\]
得到两个二次方程,求解即可.

这一求解方法被称为Ferrari(费拉里
)解法.
\subsubsection{更高次方程}
数学家们早期发现高于四次的方程没有一般形式的根式解,但无法证明.这一结论在19世纪30年代被天才数学家Galois(伽罗瓦)证明.即五次及以上的方程没有求根公式,但并非不可解.

对于具体的一个方程,Galois也给出了一套判定,即一个多项式的根可以以根式形式表示,等价于
这个多项式的Galois群是可解群.