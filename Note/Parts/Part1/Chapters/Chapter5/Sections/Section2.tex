\newpage
\section{整除}
\subsection{定义}
\dfn{整除}{整除}{
    设$f\left(x\right),g\left(x\right)\in \mathbb{K}\left[x\right]$,若存在$h\left(x\right)\in \mathbb{K}\left[x\right]$,\st
    \[
        f\left(x\right)=g\left(x\right)h\left(x\right)
    \]
    则称$g\left(x\right)$是$f\left(x\right)$的因式,或称$g\left(x\right)$可以整除$f\left(x\right)$,或称$f\left(x\right)$可以被$g\left(x\right)$整除,记作$g\left(x\right)\mid f\left(x\right)$;反之则称$g\left(x\right)$不能整除$f\left(x\right)$,或称$f\left(x\right)$不能被$g\left(x\right)$整除,记作 $g\left(x\right)\nmid f\left(x\right)$.
}
\clm{}{}{
    零多项式的因式可以是任一多项式,但任一多项式的因式不能是零多项式.
}
\pro{整除的若干性质}{整除的若干性质}{
    设$f\left(x\right),g\left(x\right),h\left(x\right)\in \mathbb{K}\left[x\right]$,$c\neq 0\in\mathbb{K}$,则\begin{enumerate}[label=\arabic*)]
        \item 若$f\left(x\right)\mid g\left(x\right)$,则$cf\left(x\right)\mid g\left(x\right)$
        \item $f\left(x\right)\mid f\left(x\right)$
        \item 若$f\left(x\right)\mid g\left(x\right),g\left(x\right)\mid h\left(x\right)$,则$f\left(x\right)\mid h\left(x\right)$
        \item 若$f\left(x\right)\mid g\left(x\right),f\left(x\right)\mid h\left(x\right)$,则$\forall u\left(x\right),v\left(x\right)\in \mathbb{K}\left[x\right]$,\st
              \[
                  f\left(x\right)\mid g\left(x\right)u\left(x\right)+h\left(x\right)v\left(x\right)
              \]
        \item 若$f\left(x\right)\mid g\left(x\right)$且$g\left(x\right)\mid f\left(x\right)$,则$\exists k\in\mathbb{K}$,\st\[
                  f\left(x\right)=kg\left(x\right)\]
              此时$f\left(x\right),g\left(x\right)$称为相伴多项式,记作\[f\left(x\right)\sim g\left(x\right)\]
    \end{enumerate}\begin{proof}
        \begin{enumerate}[label=\arabic*)]
            \item $g\left(x\right)=f\left(x\right)h\left(x\right)=\left(cf\left(x\right)\right)\left(c^{-1}h\left(x\right)\right)$
            \item $f\left(x\right)=1\cdot f\left(x\right)$
            \item 代入即得
            \item 设$g\left(x\right)=f\left(x\right)p\left(x\right),h\left(x\right)=f\left(x\right)q\left(x\right)$,则
                  \[
                      g\left(x\right)u\left(x\right)+h\left(x\right)v\left(x\right)=f\left(x\right)\left(p\left(x\right)u\left(x\right)+q\left(x\right)v\left(x\right)\right)
                  \]
            \item 设$g\left(x\right)=f\left(x\right)p\left(x\right),h\left(x\right)=f\left(x\right)q\left(x\right)$,则
                  \[
                      f\left(x\right)=f\left(x\right)p\left(x\right)q\left(x\right)
                  \]
                  $f\left(x\right)=0$显然成立.$f\left(x\right)\neq 0$时,两边取次数
                  \[
                      \deg f\left(x\right)=\deg f\left(x\right)+\deg \left(p\left(x\right)q\left(x\right)\right)
                  \]
                  于是$\deg p\left(x\right)=\deg q\left(x\right)=0$即二者均为非零常数多项式
                  \qedhere
        \end{enumerate}
    \end{proof}
}
\dfn{相伴多项式}{相伴多项式}{
    若$f\left(x\right)\mid g\left(x\right)$且$g\left(x\right)\mid f\left(x\right)$,则
    此时$f\left(x\right),g\left(x\right)$称为相伴多项式,记作\[f\left(x\right)\sim
        g\left(x\right)\]
}
\subsection{带余除法}
\thm{带余除法}{带余除法}{
    设多项式$f\left(x\right),g\left(x\right)\in \mathbb{K}\left[x\right],g\left(x\right)\neq 0$,则必定存在唯一的$q\left(x\right),r\left(x\right)\in \mathbb{K}\left[x\right]$,\st
    \[
        f\left(x\right)=g\left(x\right)q\left(x\right)+r\left(x\right)
    \]
    且有$\deg r\left(x\right)<\deg g\left(x\right)$.\begin{proof}
        先证存在性.若$\deg f\left(x\right)<\deg g\left(x\right)$,则$q\left(x\right)=0,r\left(x\right)=f\left(x\right)$.

        以下设$\deg f\left(x\right)\geqslant \deg g\left(x\right)\geqslant 0$,考虑对$\deg f\left(x\right)$归纳.
        其一,若$\deg f\left(x\right)=\deg g\left(x\right)=0$,即$f\left(x\right)=a\neq 0,g\left(x\right)=b\neq 0$,则$q\left(x\right)=ab^{-1},r\left(x\right)=0$符合.

        设$\deg f\left(x\right)<n$时结论成立,来证明$\deg f\left(x\right)=n$的情况.设\begin{align*}
             & f\left(x\right)=a_nx^n+a_{n-1}x^{n-1}+\cdots+a_1x+a_0,a_n\neq 0 \\
             & g\left(x\right)=b_mx^m+b_{m-1}x^{m-1}+\cdots+b_1x+b_0,b_m\neq 0
        \end{align*}
        且$n\geqslant m.$设$f_1\left(x\right)=f\left(x\right)-a_nb_m^{-1}x^{n-m}g\left(x\right)$,于是
        $\deg f_1\left(x\right)<n.$由归纳假设得知存在$q_1\left(x\right),r\left(x\right)$,\st
        \[
            f_1\left(x\right)=g\left(x\right)q_1\left(x\right)+r\left(x\right)
        \]
        其中$\deg r\left(x\right)< \deg g\left(x\right).$即有\[
            f\left(x\right)=g\left(x\right)\left(a_nb_m^{-1}x^{n-m}+q_1\left(x\right)\right)+r\left(x\right)
        \]
        则$q\left(x\right)=a_nb_m^{-1}x^{n-m}+q_1\left(x\right)$.
        存在性证毕.

        下证唯一性.考虑设出两个带余除法,作差即可.即设$f\left(x\right)=g\left(x\right)p\left(x\right)+t\left(x\right)=g\left(x\right)q\left(x\right)+r\left(x\right)$.
        于是\[
            g\left(x\right)\left(p\left(x\right)-q\left(x\right)\right)=t\left(x\right)-r\left(x\right)
        \]
        设$p\left(x\right)\neq q\left(x\right)$,对上式两边取次数
        得$LHS = \deg g\left(x\right)+\deg \left(p\left(x\right)-q\left(x\right)\right)\geqslant \deg g\left(x\right)$.
        $RHS=\deg\left(t\left(x\right)-r\left(x\right)\right)\leqslant \max\left\{\deg t\left(x\right),\deg r\left(x\right)\right\}<\deg g\left(x\right)$.矛盾.
    \end{proof}
}