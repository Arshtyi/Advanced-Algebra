\newpage
\section{结式与判别式}
\subsection{结式}
\thm{基域扩张}{基域扩张}{
    设$f\left(x\right),g\left(x\right)\in\mathbb{K}\left[x\right]$,基于基域扩张下的不变性,有:$\left(
        f\left(x\right),g\left(x\right)
        \right)_{\mathbb{K}}=1\Longleftrightarrow
        \left(
        f\left(x\right),g\left(x\right)
        \right)_{\mathbb{C}}=1\Longleftrightarrow
        f\left(x\right),g\left(x\right)$在$\mathbb{C}$无公共根.$\left(
        f\left(x\right),g\left(x\right)
        \right)\neq 1\Longleftrightarrow
        f\left(x\right),g\left(x\right)$在$\mathbb{C}$有公共根.
}
\lem{}{不互素的条件}{
    设$f\left(x\right),g\left(x\right)\in\mathbb{K}\left[x\right],d\left(x\right)=\left(
        f\left(x\right),g\left(x\right)
        \right)$,则$d\left(x\right)\neq 1$当且仅当$\exists u\left(x\right)\neq 0,v\left(x\right)\neq 0\in\mathbb{K}\left[x\right]\st$
    \[
        f\left(x\right)u\left(x\right)=g\left(x\right)v\left(x\right)
    \]其中
    $\deg u\left(x\right)<\deg g\left(x\right),\deg v\left(x\right)<\deg f\left(x\right)$.\begin{proof}
        先考虑必要性,即设$d\left(x\right)\neq 1$,则
        $f\left(x\right)=d\left(x\right)v\left(x\right),g\left(x\right)=d\left(x\right)u\left(x\right)$,显然成立.

        再证充分性,即设$f\left(x\right)u\left(x\right)=g\left(x\right)v\left(x\right)$,则考虑反证法,设$\left(
            f\left(x\right),g\left(x\right)
            \right)=1$,由
        \[
            f\left(x\right)\mid f\left(x\right)u\left(x\right)=g\left(x\right)v\left(x\right)
        \]知$f\left(x\right)\mid v\left(x\right)$,这与
        $v\left(x\right)\neq 0$且$\deg v\left(x\right)<
            \deg f\left(x\right)$矛盾,证毕.
    \end{proof}
}
\rem{}{}{
    下设
    \begin{align*}
         & f\left(x\right)=a_0x^n +a_1x^{n-1}+\cdots+a_{n-1}x+a_n,a_0\neq 0 \\
         & g\left(x\right)=b_0x^m+b_1x^{m-1}+\cdots+b_{m-1}x+b_m,b_0\neq 0  \\
         & u\left(x\right)=x_0x^{m-1}+x_1x^{m-2}+\cdots+x_{m-2}x+x_{m-1}    \\
         & v\left(x\right)=y_0x^{n-1}+y_1x^{n-2}+\cdots+y_{n-2}x+y_{n-1}
    \end{align*}
}
代入\cref{lem:不互素的条件}中的\[f\left(x\right)u\left(x\right)=g\left(x\right)v\left(x\right)\]得到一个包含$x_0,x_1,\cdots,x_{m-1},y_0,y_1,\cdots,y_{n-1}$共$m+n$个未知数的线性方程组
\[
    \begin{cases*}
        a_0x_0=  b_0y_0                                      \\
        a_1x_0+a_0x_1=b_1y_0+b_0y_1                          \\
        a_2x_0+a_1x_1+a_0x_2=b_2y_0+b_1y_1+b_0y_2            \\
        \qquad \qquad   \qquad\qquad\cdots\cdots\cdots\cdots \\
        a_nx_{m-3}+a_{n-1}x_{m-2}+a_{n-2}x_{m-1}=
        b_m y_{n-3}+b_{m-1}y_{n-2}+b_{m-2}y_{n-1}            \\
        a_nx_{m-2}+a_{n-1}x_{m-1}=b_m y_{n-2}+b_{m-1}y_{n-1} \\
        a_nx_{m-1}=b_m y_{n-1}
    \end{cases*}
\]
其系数矩阵的转置
\[
    \bm{A}'=\begin{pmatrix}
        a_0    & a_1    & a_2    & \cdots & \cdots & a_n     & 0        & \cdots & 0      \\
        0      & a_0    & a_1    & \cdots & \cdots & a_{n-1} & a_n      & \cdots & 0      \\
        0      & 0      & a_0    & \cdots & \cdots & a_{n-2} & a_{n-1}  & \cdots & 0      \\
        \vdots & \vdots & \vdots & \vdots & \vdots & \vdots  & \vdots   & \vdots & \vdots \\
        0      & 0      & \cdots & 0      & a_0    & \cdots  & \cdots   & \cdots & a_n    \\
        -b_0   & -b_1   & -b_2   & \cdots & \cdots & \cdots  & -b_{m}   & \cdots & 0      \\
        0      & -b_0   & -b_1   & \cdots & \cdots & \cdots  & -b_{m-1} & -b_m   & 0      \\
        \vdots & \vdots & \vdots & \vdots & \vdots & \vdots  & \vdots   & \vdots & \vdots \\
        0      & \cdots & 0      & -b_0   & -b_1   & \cdots  & \cdots   & \cdots & -b_m
    \end{pmatrix}
\]
因为$\left|\bm{A}'\right|=0$等价于该线性方程组存在非零解等价于
\[u\left(x\right)\neq 0,v\left(x\right)\neq 0\in\mathbb{K}\left[x\right]\]存在即等价于\[
    \left(
    f\left(x\right),g\left(x\right)
    \right)\neq 1
\]另一方面,若$\left|\bm{A}'\right|\neq 0$,则该线性方程组只有零解,即$f\left(x\right),g\left(x\right)$互素.
\dfn{结式/Sylvester行列式}{结式}{
    设
    \begin{align*}
         & f\left(x\right)=a_0x^n +a_1x^{n-1}+\cdots+a_{n-1}x+a_n,a_0\neq 0 \\
         & g\left(x\right)=b_0x^m+b_1x^{m-1}+\cdots+b_{m-1}x+b_m,b_0\neq 0
    \end{align*}
    则称下列$m+n$阶行列式

    \[
        \left|R\left(f\left(x\right),g\left(x\right)\right)\right|=
        \begin{vmatrix}
            a_0    & a_1    & a_2    & \cdots & \cdots & a_n     & 0       & \cdots & 0      \\
            0      & a_0    & a_1    & \cdots & \cdots & a_{n-1} & a_n     & \cdots & 0      \\
            0      & 0      & a_0    & \cdots & \cdots & a_{n-2} & a_{n-1} & \cdots & 0      \\
            \vdots & \vdots & \vdots & \vdots & \vdots & \vdots  & \vdots  & \vdots & \vdots \\
            0      & 0      & \cdots & 0      & a_0    & \cdots  & \cdots  & \cdots & a_n    \\
            b_0    & b_1    & b_2    & \cdots & \cdots & \cdots  & b_{m}   & \cdots & 0      \\
            0      & b_0    & b_1    & \cdots & \cdots & \cdots  & b_{m-1} & b_m    & 0      \\
            \vdots & \vdots & \vdots & \vdots & \vdots & \vdots  & \vdots  & \vdots & \vdots \\
            0      & \cdots & 0      & b_0    & b_1    & \cdots  & \cdots  & \cdots & b_m
        \end{vmatrix}
    \]为$f\left(x\right)$与$g\left(x\right)$的结式(resultant)或Sylvester行列式.

    若设
    \begin{align*}
         & \bm{\alpha} = (a_0, a_1, \ldots, a_{m-1}, a_m, \overbrace{0, \ldots, 0}^{\mathclap{n-1 \text{个}}}) \\
         & \bm{\beta} = (b_0, b_1, \ldots, b_{m-1}, b_m, \underbrace{0, \ldots, 0}_{n-1 \text{个}})
    \end{align*}
    定义一个循环映射
    \begin{align*}
        r:\mathbb{K}_{n+m}                            & \longrightarrow\mathbb{K}_{n+m}                                \\
        \left(c_1,c_2,\cdots,c_{n+m-1},c_{n+m}\right) & \longmapsto\left(c_{n+m},c_1,\cdots,c_{n+m-2},c_{n+m-1}\right)
    \end{align*}于是该结式可以写成
    \[
        R\left(f\left(x\right),g\left(x\right)\right)=\det\,(\begin{bmatrix}
            \bm{\alpha} \\r\left(\bm{\alpha}\right)\\\vdots\\r^{m-1}\left(\bm{\alpha}\right)\\\bm{\beta}\\r\left(\bm{\beta}\right)\\\vdots\\r^{n-1}\left(\bm{\beta}\right)
        \end{bmatrix})
    \]
}
\rem{}{}{
    此处记号的确混乱,更多时候,$R\left(
        f\left(x\right),g\left(x\right)
        \right)$是一个行列式.
}
\thm{结式判定互素}{结式判定互素}{
    $f\left(x\right),g\left(x\right)$在$\mathbb{C}$有公根当且仅当$R\left(f\left(x\right),g\left(x\right)\right)=0$.
}
\lem{}{结式的两个变式}{
    设常数$\lambda$则
    \[
        \left|
        R\left(
        f\left(x\right),g\left(x\right)\left(x-\lambda\right)
        \right)
        \right|=
        \left(-1\right)^nf\left(\lambda\right)R\left(f\left(x\right),g\left(x\right)\right)
    \]及
    \[
        \left|R\left(
        f\left(x\right),x-\lambda
        \right)\right|=\left(-1\right)^nf\left(\lambda\right)
    \]
}
\thm{结式的根表示}{结式的根表示}{
    设
    \begin{align*}
         & f\left(x\right)=a_0x^n +a_1x^{n-1}+\cdots+a_{n-1}x+a_n,a_0\neq 0 \\
         & g\left(x\right)=b_0x^m+b_1x^{m-1}+\cdots+b_{m-1}x+b_m,b_0\neq 0
    \end{align*}
    且$f\left(x\right)$的根为$x_1,x_2,\cdots,x_n$,$g\left(x\right)$的根为$y_1,y_2,\cdots,y_m$,则
    \[\left|
        R\left(f\left(x\right),g\left(x\right)\right)\right|=
        a_0^mb_0^n\prod_{i=1}^{n}\prod_{j=1}^{m}\left(
        x_i-y_j
        \right)
    \]\begin{proof}
        采用摄动法证明.

        第一步,首一化.$f\left(x\right)=a_0f_1\left(x\right),g\left(x\right)=b_0g_1\left(x\right)$,那么
        \[\left|
            R\left(f\left(x\right),g\left(x\right)\right)\right|=
            a_0^mb_0^nR\left(f_1\left(x\right),g_1\left(x\right)\right)
        \]于是考虑首一多项式的情况
        \[\left|
            R\left(f_1\left(x\right),g_1\left(x\right)\right)\right|
            =
            \prod_{i=1}^{n}\prod_{j=1}^{m}\left(
            x_i-y_j
            \right)
        \]
        即可.

        第二步,设$f\left(x\right)=\left(x-x_1\right)
            \left(x-x_2\right)\cdots\left(x-x_n\right),g\left(x\right)=\left(x-y_1\right)
            \left(x-y_2\right)\cdots\left(x-y_m\right)$.

        则一方面,若$\exists 1\leqslant i\leqslant n,1\leqslant j\leqslant m\,\textup{s.t.}x_i=y_j,LHS=0=RHS.$

        另一方面,设$x_i\neq y_j,\forall 1\leqslant i\leqslant n,1\leqslant j\leqslant m$且$x_1,x_2,\cdots,x_n$互不相同,$y_1,y_2,\cdots,y_m$互不相同.那么
        \begin{align*}
              & R\left(f\left(x\right),g\left(x\right)\right)\begin{bmatrix}
                                                                 x_1^{n+m-1} & \cdots & x_n^{n+m-1} & y_1^{n+m-1} & \cdots & y_m^{n+m-1} \\
                                                                 x_1^{n+m-2} & \cdots & x_n^{n+m-2} & y_1^{n+m-2} & \cdots & y_m^{n+m-2} \\
                                                                 \vdots      &        & \vdots      & \vdots      &        & \vdots      \\
                                                                 x_1         & \cdots & x_n         & y_1         & \cdots & y_m         \\
                                                                 1           & \cdots & 1           & 1           & \cdots & 1
                                                             \end{bmatrix} \\
            = &
            \begin{bmatrix}
                0                          & \cdots & 0                          & y_1^{m-1}f\left(y_1\right) & \cdots & y_m^{m-1}f\left(y_m\right) \\
                \vdots                     &        & \vdots                     & \vdots                     &        & \vdots                     \\
                0                          & \cdots & 0                          & f\left(y_1\right)          & \cdots & f\left(y_m\right)          \\
                x_1^{n-1}g\left(x_1\right) & \cdots & x_n^{n-1}g\left(x_n\right) & 0                          & \cdots & 0                          \\
                \vdots                     &        & \vdots                     & \vdots                     &        & \vdots                     \\
                g\left(x_1\right)          & \cdots & g\left(x_n\right)          & 0                          & \cdots & 0
            \end{bmatrix}
        \end{align*}
        取行列式得
        \[
            LHS = \left|
            R \left(
            f_1\left(x\right),g_1\left(x\right)
            \right)
            \right|\prod_{1\leqslant i<k\leqslant n}\left(
            x_i-x_k
            \right)\prod_{1\leqslant j<l\leqslant m}\left(
            y_j-y_l
            \right)
            \prod_{i=1}^{n}\prod_{j=1}^{m}\left(
            x_i-y_j
            \right)
        \]
        利用\cref{thm:Laplace定理}Laplace定理得到
        \[
            RHS=\left(-1\right)^{mn}\prod_{i=1}^{n}
            g\left(x_i\right)\prod_{j=1}^{m}f\left(y_j\right)\prod_{1\leqslant i<k\leqslant n}\left(
            x_i-x_k
            \right)\prod_{1\leqslant j<l\leqslant m}\left(
            y_j-y_l
            \right)
        \]
        注意到
        \begin{align*}
            \left(-1\right)^{mn}\prod_{j=1}^{m}f\left(y_j\right) & =\prod_{j=1}^{m}\left(
            \left(-1\right)^n\left(y_j-x_1\right)\left(y_j-x_2\right)\cdots\left(y_j-x_n\right)
            \right)                                                                                      \\
                                                                 & =\prod_{j=1}^{m}\prod_{i=1}^{n}\left(
            x_i-y_j
            \right)
        \end{align*}
        于是
        \begin{align*}
            \left|
            R\left(f_1\left(x\right),g_1\left(x\right)\right)
            \right| & =\prod_{i=1}^{n}g\left(x_i\right) \\
                    & =
            \prod_{j=1}^{m}\prod_{i=1}^{n}\left(
            x_i-y_j
            \right)
        \end{align*}至此两函数无同根及无同根且无重根的情况均证毕.

        第三步,摄动法证明两个函数没有同根但是有重根的情况.设$\exists c_1,c_2,\cdots,c_n,d_1,d_2,\cdots,d_m\in\mathbb{C}\st\forall 0<t\ll 1$\[x_1+c_1t,x_2+c_2t,\cdots,x_n+c_nt,y_1+d_1t,y_2+d_2t,\cdots,y_m+d_mt\]两两互不相同.

        构造
        \begin{align*}
             & f_t\left(x\right)=\left(
            x-x_1-c_1t
            \right)\left(
            x-x_2-c_2t
            \right)\cdots\left(
            x-x_n-c_nt
            \right)                     \\
             & g_t\left(x\right)=\left(
            x-y_1-d_1t
            \right)\left(
            x-y_2-d_2t
            \right)\cdots\left(
            x-y_m-d_mt
            \right)
        \end{align*}
        二者的系数均是$t$的多项式,特别地,$f_0\left(x\right)=f\left(x\right),g_0\left(x\right)=g\left(x\right)$.由上一步,$\forall 0<t\ll 1$
        \[
            \left|R\left(
            f_t\left(x\right),g_t\left(x\right)
            \right)\right|=\prod_{i=1}^{n}
            \prod_{j=1}^{m}
            \left(
            x_i+c_it-y_j-d_jt
            \right)
        \]
        此为关于$t$的连续函数,令$t\to 0^+$,则
        \[
            R\left(f\left(x\right),g\left(x\right)\right)=\prod_{i=1}^{n}
            \prod_{j=1}^{m}
            \left(
            x_i-y_j
            \right)\qedhere
        \]
    \end{proof}
}
\subsection{判别式}
\dfn{判别式}{判别式}{
设$f\left(x\right)=a_0x^n
    +a_1x^{n-1}+\cdots+a_{n-1}x+a_n\left(
    a_0\neq 0
    \right)$,则称\[
    \Delta\left(f\right)\coloneqq
    \left(-1\right)^{
    \frac{
        n\left(n-1\right)
    }{2}
    }a_0^{-1}R\left(
    f,f'
    \right)
\]
为$f\left(x\right)$的判别式.
}
\thm{判别式的根定义}{判别式的根定义}{
设 $f\left(x\right)=a_0x^n
    +a_1x^{n-1}+\cdots+a_{n-1}x+a_n\left(
    a_0\neq 0
    \right)$的根为
$x_1,x_2,\cdots,x_n$
,则其判别式为
\[
    \Delta \left(f\right)=
    a_0^{2n-2}\prod_{1\leqslant i<j\leqslant n}\left(
    x_i-x_j
    \right)^2
\]\begin{proof}
    因为
    \begin{align*}
        R\left(
        f,g
        \right) & =
        a_0^m\prod_{i=1}^{n}\left(
        b_0\left(x_i-y_1\right)\left(x_i-y_2\right)\cdots\left(x_i-y_m\right)
        \right)                                          \\
                & =a_0^m\prod_{i=1}^{n}g\left(x_i\right)
    \end{align*}
    令$g\left(x\right)=f'\left(x\right)\Longrightarrow m=n-1$,则
    \[
        f'\left(x_i\right)=a_0\left(x_i-x_1\right)\cdots\left(x_i-x_{i-1}\right)\left(x_i-x_{i+1}\right)\cdots\left(x_i-x_n\right)
    \]于是
    \begin{align*}
        R\left(f,f'\right) & =a_0^{n-1}\prod_{i=1}^{n}\left(a_0
        \left(x_i-x_1\right)\cdots\left(x_i-x_{i-1}\right)\left(x_i-x_{i+1}\right)\cdots\left(x_i-x_n\right)
        \right)                                                 \\
                           & =\left(-1\right)^{
        \frac{n\left(n-1\right)}{2}
        }a_0^{2n-1}\prod_{1\leqslant i<j\leqslant n}\left(
        x_i-x_j
        \right)^2
    \end{align*}
    代入\cref{def:判别式}即得.
\end{proof}
}
\cor{重根判定}{重根判定}{
    多项式$f\left(x\right)$在$\mathbb{C}$有重根当且仅当其判别式$\Delta\left(f\right)=0$.
}
之所以结式出现在多元多项式之后,是因为结式可以用于求解多元高次方程组.
\exa{}{}{
    考虑求解
    \[
        \begin{cases*}
            f\left(x,y\right)=0 \\
            g\left(x,y\right)=0
        \end{cases*}
    \]\begin{solution}
        设\begin{align*}
             & f(x, y) = a_0(y)x^n + a_1(y)x^{n-1} + \cdots + a_n(y) \\
             & g(x, y) = b_0(y)x^m + b_1(y)x^{m-1} + \cdots + b_m(y)
        \end{align*}
        其中$a_i\left(y\right),b_j\left(y\right)$是$y$的多项式且$a_0\left(y\right)\neq 0,b_0\left(y\right)\neq 0.$
        并设一解为$\left(
            \alpha,\beta
            \right)$.令$y=\beta$,则$f\left(x,\beta\right)=0$与$g\left(x,\beta\right)=0$有公共根$x=\alpha$,则二者结式为零即
        \begin{align*}
              & R_x(f\left(x,\beta\right), g\left(x,\beta\right))                                      \\ = &
            \begin{vmatrix}
                a_0(y) & a_1(y) & a_2(y) & \cdots & \cdots     & a_n(y)     & 0          & \cdots & 0      \\
                0      & a_0(y) & a_1(y) & \cdots & \cdots     & a_{n-1}(y) & a_n(y)     & \cdots & 0      \\
                0      & 0      & a_0(y) & \cdots & \cdots     & a_{n-2}(y) & a_{n-1}(y) & \cdots & 0      \\
                \vdots & \vdots & \vdots & \vdots & \vdots     & \vdots     & \vdots     & \vdots & \vdots \\
                0      & 0      & \cdots & 0      & a_0(y)     & \cdots     & \cdots     & \cdots & a_n(y) \\
                b_0(y) & b_1(y) & b_2(y) & \cdots & \cdots     & \cdots     & b_m(y)     & \cdots & 0      \\
                0      & b_0(y) & b_1(y) & \cdots & \cdots     & \cdots     & b_{m-1}(y) & b_m(y) & \cdots \\
                0      & 0      & b_0(y) & \cdots & b_{m-2}(y) & b_{m-1}(y) & \cdots     & b_m(y)          \\
                \vdots & \vdots & \vdots & \vdots & \vdots     & \vdots     & \vdots     & \vdots & \vdots \\
                0      & \cdots & 0      & b_0(y) & b_1(y)     & \cdots     & \cdots     & \cdots & b_m(y)
            \end{vmatrix} \\
            = & 0
        \end{align*}
        这是一个关于$y$的多项式,一定有根$y=\beta$,反代回去就完成了未定元的减少.
    \end{solution}
}
\clm{}{}{
    从上面可以看出一般的思路,即先求$R_x\left(f,g\right)$(一个关于$y$的多项式),再求其根$\beta_1,\cdots,\beta_r$,再代入$f\left(x,\beta_i\right)=0$解得$x=\alpha_1,\cdots,\alpha_s$,最后,需要验证$\left(
        \alpha_i,\beta_j\right)$是否是合理解.
}