\newpage
\section{多元多项式}
\subsection{定义}
\dfn{单项式}{单项式}{
设$\mathbb{K}$为数域,$x_1,x_2,\cdots,x_n$是未定元,则
\[
    ax_1^{i_1}x_2^{i_2}\cdots x_n^{i_n}
\]
(其中$a\in\mathbb{K}$称为系数,$i_1,i_2,\cdots,i_n\in\bbn $)称为一个单项式.

单项式的次数定义为
\[
    \deg \left(ax_1^{i_1}x_2^{i_2}\cdots x_n^{i_n} \right)\coloneqq
    \begin{cases*}
        -\infty            & ,$a=0$     \\
        i_1+i_2+\cdots+i_n & ,$a\neq 0$
    \end{cases*}
\]
}
\dfn{同类项}{同类项}{
若 $ax_1^{i_1}x_2^{i_2}\cdots x_n^{i_n}\left(a\neq 0\right)$ 与 $bx_1^{j_1}x_2^{j_2}\cdots x_n^{j_n}\left(b\neq 0\right)$ 为同类项,则 $i_1=j_1,i_2=j_2,\cdots,i_n=j_n$.
}
\dfn{合并同类项}{合并同类项}{
定义两个同类项之间的加法就是系数的加法,即若 $ax_1^{i_1}x_2^{i_2}\cdots x_n^{i_n}$ 与 $bx_1^{i_1}x_2^{i_2}\cdots x_n^{i_n}$ 为同类项,则
\[ ax_1^{i_1}x_2^{i_2}\cdots x_n^{i_n}+bx_1^{i_1}x_2^{i_2}\cdots x_n^{i_n}=\left(a+b\right)x_1^{i_1}x_2^{i_2}\cdots x_n^{i_n} \]
}
\dfn{单项式的运算}{单项式的运算}{
同类项的加法即是合并同类项,非同类项的加法即是形式和;$\forall k \in\mathbb{K}$,数乘定义为
\[ kax_1^{i_1}x_2^{i_2}\cdots x_n^{i_n}=\left(ka\right)x_1^{i_1}x_2^{i_2}\cdots x_n^{i_n} \]
乘法定义为
\[ ax_1^{i_1}x_2^{i_2}\cdots x_n^{i_n}\cdot bx_1^{j_1}x_2^{j_2}\cdots x_n^{j_n}=\left(ab\right)x_1^{i_1+j_1}x_2^{i_2+j_2}\cdots x_n^{i_n+j_n} \]
}
\dfn{多元多项式}{多元多项式}{
有限个单项式的形式和称为多元多项式即
\[
    f\left(x_1,x_2,\cdots,x_n\right)
    =
    \sum_{\left(i_1,i_2,\cdots,i_n\right)\in \bbn ^n}a_{i_1i_2\cdots i_n}x_1^{i_1}x_2^{i_2}\cdots x_n^{i_n}\left(\textup{有限形式和}\right)
\]
其次数定义为
\[
    \deg\,f\left(x\right)\coloneqq
    \begin{cases*}
        \max\left\{
        i_1+i_2+\cdots+i_n\mid a_{i_1i_2\cdots i_n}\neq 0
        \right\} & ,$f\left(x\right)\neq 0$ \\
        -\infty  & ,$f\left(x\right)=0$
    \end{cases*}
\]
}
\dfn{多元多项式的相等}{多元多项式的相等}{
两个多元多项式 \begin{align*}
     & f\left(x_1,x_2,\cdots,x_n\right)=\sum
    _{\left(i_1,i_2,\cdots,i_n\right)}
    a_{i_1i_2\cdots i_n}x_1^{i_1}x_2^{i_2}\cdots x_n^{i_n} \\
     & g\left(x_1,x_2,\cdots,x_n\right)=
    \sum_{\left(i_1,i_2,\cdots,i_n\right)}b_{i_1i_2\cdots i_n}x_1^{i_1}x_2^{i_2}\cdots x_n^{i_n}
\end{align*} 相等当且仅当$a_{i_1i_2\cdots i_n}=b_{i_1i_2\cdots i_n},\forall \left(i_1,i_2,\cdots,i_n\right)\in\bbn ^n$.
}
\dfn{多元多项式全体}{多元多项式全体}{
记$\mathbb{K}$上的$n$元多形式全体为\[\mathbb{K}\left[x_1,x_2,\cdots,x_n\right]=\left\{
    f\left(x_1,x_2,\cdots,x_n\right)=\sum_{\left(i_1,i_2,\cdots,i_n\right)\in\mathbb{
            N
        }^n}a_{i_1i_2\cdots i_n}x_1^{i_1}x_2^{i_2}\cdots x_n^{i_n}
    \right\}\]
这是一个$\mathbb{K}$-代数,称为关于$x_1,x_2,\cdots,x_n$的$n$元多项式代数(环).
}
\rem{}{}{
    一个自然的问题是,$\mathbb{K}\left[x\right]$的各种性质是否可以推广到$\mathbb{K}\left[x_1,x_2,\cdots,x_n\right]$.
}
\dfn{多元多项式的运算}{多元多项式的运算}{
加法定义为
\[
    f\left(x\right)+g\left(x\right)=\sum_{\left(i_1,i_2,\cdots,i_n\right)}\left(a_{i_1i_2\cdots i_n}+b_{i_1i_2\cdots i_n}\right)x_1^{i_1}x_2^{i_2}\cdots x_n^{i_n}\]
$\forall k\in\mathbb{K}$,数乘定义为
\[
    kf\left(x\right)=\sum_{\left(i_1,i_2,\cdots,i_n\right)}\left(ka_{i_1i_2\cdots i_n}\right)x_1^{i_1}x_2^{i_2}\cdots x_n^{i_n} \]
定义乘法则利用分配律转换为单项式的乘积之和并合并同类项.
}
\subsection{\texorpdfstring{$n$}{n}元多项式代数环}
\dfn{字典序}{字典序}{
将未定元按自然足标排序,即
\[
    x_1\succ
    x_2\succ\cdots\succ x_n
\]
然后对单项式排序,定义
$\displaystyle
    ax_1^{i_1}x_2^{i_2}\cdots x_n^{i_n}\succ bx_{1}^{j_1}x_2^{j_2}\cdots x_n^{j_n}\left(ab\neq 0\right)$当且仅当
$\exists 1\leqslant k\leqslant n,\st i_1=j_1,i_2=j_2,\cdots,i_{k-1}=j_{k-1},i_k>j_k.$
}
\clm{}{}{
    \cref{def:字典序}的定义方式并不是唯一的.

    容易证明,在字典序下,每个多项式有唯一的排序,称为字典排序.但是,首项未必是次数最高的单项式,而是字典序最大的单项式;末项未必是次数最低的单项式,而是字典序最小的单项式.
}
\lem{}{字典序下多项式乘积的首项保持}{
    在字典序下,$f\cdot g$的首项是$f$的首项与$g$的首项的乘积.\begin{proof}
        若$f=0$或$g=0$,显然.

        设$f$的首项为$ax_1^{i_1}x_2^{i_2}\cdots x_n^{i_n}$,$g$的首项为$bx_1^{j_1}x_2^{j_2}\cdots x_n^{j_n}$,则任取
        $f\left(x\right)$的一个单项$cx_1^{k_1}x_2^{k_2}\cdots x_n^{k_n}$和
        $g\left(x\right)$的一个单项$dx_1^{l_1}x_2^{l_2}\cdots x_n^{l_n}$,则$\exists 1\leqslant r,s\leqslant n\st
            i_1=k_1,i_2=k_2,\cdots,i_{r-1}=k_{r-1},i_r>k_r\Longleftrightarrow
            a\succ c$且$j_1=l_1,j_2=l_2,\cdots,j_{s-1}=l_{s-1},j_s>l_s
            \Longleftrightarrow b\succ d
        $(其中系数指代整个单项),于是不妨设$r\leqslant s$则
        \[
            i_1+j_1=k_1+l_1,i_2+j_2=k_2+l_2,
            \cdots,i_{r-1}+j_{r-1}=k_{r-1}+l_{r-1},i_r+j_r>k_r+l_r
        \]
        即
        \[
            ab x_1^{i_1+j_1}x_2^{i_2+j_2}\cdots x_n^{i_n+j_n}\succ cd x_1^{k_1+l_1}x_2^{k_2+l_2}\cdots x_n^{k_n+l_n}
        \]
        同理可证$ab\succ cb,ab\succ ad$,考虑到不可合并,则一定是首项.
    \end{proof}
}
\thm{}{多元多项式乘法的整性}{
    设$0\neq f,0\neq g\in\mathbb{K}\left[
            x_1,x_2,\cdots,x_n
            \right]$,则$f\cdot g\neq 0$.\begin{proof}
        考虑首项即得.
    \end{proof}
}
\cor{多元多项式环的乘法消去律}{}{
    由\cref{thm:多元多项式乘法的整性}整性立刻得到,多元多项式环满足乘法消去律:设 $f,g,h\in\mathbb{K}\left[x_1,x_2,\cdots,x_n\right]$,且 $f\neq 0$,则
    \[  f\cdot g=f\cdot h\Longrightarrow g=h \]
}
\subsection{齐次分解}
字典排序的缺点在于排序与单项式次数没有很好的关系,因此引入齐次分解.
\dfn{齐次多项式}{齐次多项式}{
    若多项式$f$的所有单项式均为$m$次单项式,则称$f$为$m$次齐次多项式或$m$次型.
}
\thm{齐次分解}{齐次分解}{
    \begin{enumerate}[label=\arabic*)]
        \item  若$f$、$g$均是$m$次型且$f+g\neq 0$,则$f+g$也是$m$次型
        \item 设$m$次型$f\neq 0 $和$k$次型$g\neq 0$,则$f\cdot g\neq 0$且$f\cdot g$是$m+k$次型
        \item 设$f\in\mathbb{K}\left[x_1,x_2,\cdots,x_n\right]$,则一定可以通过限制定义域得到与\[
                  f=f_d+f_{d-1}+ \cdots+f_1+f_0
              \]
              其中$d=\deg f\geqslant 0,f_d\neq 0$为$d$次型,$f_i=0$或为$i$次型$\left(\forall 0\leqslant i\leqslant d-1\right)$
    \end{enumerate}\begin{proof}
        \begin{enumerate}[label=\arabic*)]
            \item 若$f+g\neq 0$,则$f+g$的首项为$f$的首项与$g$的首项之和,故为$m$次型.
            \item 若$f\cdot g\neq 0$,则$f\cdot g$的首项为$f$的首项与$g$的首项之积,故为$m+k$次型.
            \item 逐步考查$d$次单项式、$d-1$次单项式$\cdots$即可.\qedhere
        \end{enumerate}
    \end{proof}
}
\lem{}{多元多项式运算结果的次数}{
    设$f$、$g\in\mathbb{K}\left[x_1,x_2,\cdots,x_n\right]$,则\begin{enumerate}[label=\arabic*)]
        \item $\deg\left(f\cdot g\right)=\deg f+\deg g$
        \item $\deg \left(
                  f\pm g
                  \right)\leqslant \max\left\{
                  \deg f,\deg g\right\}$
    \end{enumerate}
}
\subsection{多元多项式与多元函数}
正如\cref{def:多项式函数},多元多项式$f\left(
    x_1,x_2,\cdots,x_n
    \right)\in
    \mathbb{K}\left[
        x_1,x_2,\cdots,x_n\right]
$作赋值后成为一个多元函数$\overline{f}\left(
    x_1,x_2,\cdots,x_n
    \right)$
\begin{align*}
    f:\mathbb{K}^n & \longrightarrow \mathbb{K} \\
    \begin{pmatrix}
        a_1    \\
        a_2    \\
        \vdots \\
        a_n
    \end{pmatrix}
                   & \longmapsto f\left(
    a_1,a_2,\cdots,a_n
    \right)
\end{align*}
一个自然的问题是,$f$、$g\in\mathbb{K}
    \left[
        x_1,x_2,\cdots,x_n
        \right]$作为多元多项式相等与作为多元函数相等是否等价.即下述关系是否成立
\[
    f\left(
    x_1,x_2,\cdots,x_n
    \right)=g\left(
    x_1,x_2,\cdots,x_n
    \right)\Longleftrightarrow \overline{f}\left(
    x_1,x_2,\cdots,x_n
    \right)=\overline{g}\left(
    x_1,x_2,\cdots,x_n
    \right)
\]
\clm{多元多项式相等得到多元函数相等}{}{
    由于对多元多项式相等的定义足够强,于是作为多元多项式相等可推得作为多元函数相等.
}
\lem{非零多项式一定是非零函数}{非零多项式一定是非零函数}{
    设$0\neq f\in\mathbb{K}\left[
            x_1,x_2,\cdots,x_n
            \right]$,则存在$a_1,a_2,\cdots,a_n\in\mathbb{K}\st$
    \[  f\left(
        a_1,a_2,\cdots,a_n
        \right)\neq 0 \]\begin{proof}
        考虑归纳法,$n=1$显然,假设$n-1$成立,那么只把$x_n$视为未定元,那么
        \[
            f=b_m\left(
            x_1,x_2,\cdots,x_{n-1}
            \right)x_n^m+\cdots+b_1\left(
            x_1,x_2,\cdots,x_{n-1}
            \right)x_n+b_0\left(
            x_1,x_2,\cdots,x_{n-1}
            \right)
        \]
        其中$b_m\left(
            x_1,x_2,\cdots,x_{n-1}
            \right)\neq 0$,那么由归纳假设,$\exists a_1,a_2,\cdots,a_{n-1}\in\mathbb{K}\st$
        \[
            b_m\left(
            a_1,a_2,\cdots,a_{n-1}
            \right)x_n^m+\cdots+b_1\left(
            a_1,a_2,\cdots,a_{n-1}
            \right)x_n+b_0\left(
            a_1,a_2,\cdots,a_{n-1}
            \right)\neq 0
        \]
        是一个关于$x_n$的$m$次多项式,故有根.
    \end{proof}
}
\cor{多元多项式相等等价于多元函数相等}{多元多项式相等等价于多元函数相等}{
    设$f$、$g\in\mathbb{K}\left[
            x_1,x_2,\cdots,x_n
            \right]$,则它们作为多元多项式相等当且仅当它们作为多元函数相等.\begin{proof}
        作为多元多项式相等推得作为多元函数相等显然,作为多元函数相等推得作为多元多项式相等考虑反证法.

        设$f$、$g\in\mathbb{K}\left[
                x_1,x_2,\cdots,x_n
                \right]$且$\overline{f}=\overline{g},f\neq g$则$f-g\neq 0$,由\cref{lem:非零多项式一定是非零函数}$\exists a_1,a_2,\cdots,a_n\in\mathbb{K}\st$
        \[
            \left(f-g\right)
            \left(
            a_1,a_2,\cdots,a_n
            \right)=0\Longrightarrow
            f\left(
            a_1,a_2,\cdots,a_n
            \right)=g\left(
            a_1,a_2,\cdots,a_n
            \right)
        \]
        矛盾,证毕.
    \end{proof}
}
\clm{}{}{
    同样的,可以定义多元多项式的整除关系、公因式、最大公因式、公倍式、最小公倍式、可约与不可约、辗转相除法、带余除法等等.但是,并不是所有的性质都可以推广到多元多项式.
}
\exa{}{}{
    考虑$f=x,g=y\in\mathbb{K}\left[x,y\right]$,容易有$\left(f,g\right)=1$且$\forall
        u\left(x,y\right),v\left(x,y\right)\in\mathbb{K}\left[x,y\right]$
    \[
        fu\left(x,y\right) +gv\left(x,y\right)\neq 1
    \]
}
\thm{多元多项式的因式分解}{多元多项式的因式分解}{
    设$f\left(
        x_1,x_2,\cdots,x_n
        \right)\in\mathbb{K}\left[
            x_1,x_2,\cdots,x_n
            \right]$,则有如下分解\[
        f\left(
        x_1,x_2,\cdots,x_n
        \right)=
        cP_1\left(
        x_1,x_2,\cdots,x_n
        \right)^{m_1}P_2\left(
        x_1,x_2,\cdots,x_n
        \right)^{m_2}\cdots P_k\left(
        x_1,x_2,\cdots,x_n
        \right)^{m_k}
    \]
    其中
    $
        c\in\mathbb{K},P_i \left(
        x_1,x_2,\cdots,x_n
        \right)\left(
        \forall 1\leqslant i\leqslant k
        \right)$是$\mathbb{K}$上的 $n$ 元不可约多项式.并且这一分解在相伴定义下是唯一的.\begin{proof}
        该证明超过高等代数范畴.
    \end{proof}
}
\clm{}{}{
    基于\cref{thm:多元多项式的因式分解},我们将$n$ 元多项式环$\mathbb{K}\left[
            x_1,x_2,\cdots,x_n
            \right]$称为唯一分解整区(UFD).
}