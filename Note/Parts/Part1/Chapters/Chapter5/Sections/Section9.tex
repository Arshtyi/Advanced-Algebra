\newpage
\section{对称多项式}
\subsection{定义}
\dfn{对称多项式}{对称多项式}{
    设 $f\left(
        x_1,x_2,\cdots,x_n
        \right)\in\mathbb{K}\left[
            x_1,x_2,\cdots,x_n
            \right]$,若$\forall 1\leqslant i<j\leqslant n$,都有
    \[
        f\left(
        x_1,\cdots,x_i,\cdots,x_j,\cdots,x_n
        \right)=
        f\left(
        x_1,\cdots,x_j,\cdots,x_i,\cdots,x_n
        \right)
    \]则称$f\left(
        x_1,x_2,\cdots,x_n
        \right)$是$\mathbb{K}$上的$n$元对称多项式.
}
\lem{}{置换定义的对称多项式}{
    设$\left(
        k_1,k_2,\cdots,k_n
        \right)
    $是 $\left(
        1,2,\cdots,n
        \right)$的一个置换,且有对称多项式
    $f\left(
        x_1,x_2,\cdots,x_n
        \right)$,则
    \[
        f\left(
        x_{k_1},x_{k_2},\cdots,x_{k_n}
        \right)=f\left(
        x_1,x_2,\cdots,x_n
        \right)
    \]\begin{proof}
        根据\cref{thm:逆序数的意义},任意一个置换经过其逆序数次相邻对换就成为常排列.
    \end{proof}
}
\cor{}{对称多项式的封闭}{
    对称多项式的和与积仍然是对称多项式.
}
\lem{对称多项式的复合}{对称多项式的复合}{
    设对称多项式$f_1,f_2,\cdots,f_m\in\mathbb{K}\left[
            x_1,x_2,\cdots,x_n
            \right]$,且有$g\left(
        y_1,y_2,\cdots,y_m
        \right)\in\mathbb{K}\left[
            y_1,y_2,\cdots,y_m
            \right]$,则$
        g\left(
        f_1,f_2,\cdots,f_m
        \right)
    $
    仍然是关于
    $x_1,x_2,\cdots,x_n$
    的对称多项式.
}
一个自然的问题是,我们能否用尽可能少的对称多项式去控制所有的对称多项式.
\subsection{初等对称多项式}
\dfn{初等对称多项式}{初等对称多项式}{
设 $f\left(
    x_1,x_2,\cdots,x_n
    \right)\in\mathbb{K}\left[
        x_1,x_2,\cdots,x_n
        \right]$,则称
\[
    \sigma_1\left(
    x_1,x_2,\cdots,x_n
    \right)=x_1+x_2+\cdots+x_n=
    \sum_{i=1}^{n}x_i\]\[
    \sigma_2\left(
    x_1,x_2,\cdots,x_n
    \right)=x_1x_2+x_1x_3+\cdots+x_{n-1}x_n=
    \sum_{1\leqslant i<\leqslant n}x_ix_j\]\[
    \cdots\cdots\cdots\cdots\cdots\cdots\]
\[
    \sigma_r\left(
    x_1,x_2,\cdots,x_n
    \right)=
    \sum_{1\leqslant i_1<i_2<\cdots<i_r\leqslant n}x_{i_1}x_{i_2}\cdots x_{i_r}
\]
\[
    \cdots\cdots \cdots\cdots\cdots\cdots
\]
\[\sigma_n\left(
    x_1,x_2,\cdots,x_n
    \right)=x_1x_2\cdots x_n=
    \prod_{i=1}^{n}x_i
\]
为初等对称多项式.
}
\thm{对称多项式基本定理}{对称多项式基本定理}{
    设$f\left(
        x_1,x_2,\cdots,x_n
        \right)$是$\mathbb{K}$上的$n$元对称多项式,则存在唯一的多项式
    $g\left(
        y_1,y_2,\cdots,y_n
        \right)$使得
    \[
        f\left(
        x_1,x_2,\cdots,x_n
        \right)=g\left(
        \sigma_1,\sigma_2,\cdots,\sigma_n
        \right)
    \]即任意一个对称多项式都可以唯一地表示为初等对称多项式的多项式.\begin{proof}
        先证明存在性,设$f\neq 0$,则取字典排序下首项为$ax_1^{i_1}x_2^{i_2}\cdots x_n^{i_n}\left(
            a\neq 0
            \right)$,则断言$
            i_1\geqslant i_2\geqslant\cdots\geqslant i_n
        $(反证法,对换显然).构造
        \[
            g_1\left(
            x_1,x_2,\cdots,x_n
            \right)=
            a\sigma_1^{i_1-i_2}\sigma_2^{i_2-i_3}\cdots\sigma_{n-1}^{i_{n-1}-i_n}\sigma_n^{i_n}
        \]这是一个对称多项式,且其首项(首项的乘积等于乘积的首项)为\[
            ax_1^{i_1-i_2}\left(
            x_1x_2
            \right)^{i_2-i_3}\cdots\left(
            x_1x_2\cdots x_n\right)^{i_n}=ax_1^{i_1}x_2^{i_2}\cdots x_n^{i_n}
        \]

        接下来做这样一件事,首先定义$f_0=f$,找出其首项$a
            x_1^{i_1}x_2^{i_2}\cdots x_n^{i_n}$,然后构造$g_1$,然后令$f_1=f_0-g_1$,则$f_1$的首项次数小于$f_0$,重复上述过程,对于一般的$f_i$,根据其首项
        \[bx_1^{k_1}
            x_2^{k_2}\cdots x_n^{k_n}\left(
            i_1\geqslant k_1\geqslant k_2\geqslant\cdots\geqslant k_n\geqslant 0
            \right)\]构造$g_{i+1}$,$f_ {i+1}=f_i-g_{i+1}$
        这里的指数组$
            \left(
            k_1,k_2,\cdots,k_n
            \right)$
        的取法只有有限种.从而存在$s\in\mathbb{Z}^+\st f_{s+1}=0.$反递推得到
        \[
            f=g_1+g_2+\cdots+g_s+g_{s+1}
        \]存在性证毕.

        再证明唯一性,设$h\left(
            y_1,y_2,\cdots,y_n
            \right)$使得
        \[
            f\left(
            x_1,x_2,\cdots,x_n
            \right)=h\left(
            \sigma_1,\sigma_2,\cdots,\sigma_n
            \right)
        \]令
        \[
            \varphi\left(
            y_1,y_2,\cdots,y_n
            \right)=g\left(
            y_1,y_2,\cdots,y_n
            \right)-h\left(
            y_1,y_2,\cdots,y_n
            \right)
        \]则$
            \varphi\left(
            \sigma_1,\sigma_2,\cdots,\sigma_n
            \right)=0
        $,只要证明$\varphi=0$即可.考虑反证法,$\varphi\neq 0$,因为
        \[
            \varphi =
            \cdots+c y_1^{j_1}y_2^{j_2}\cdots y_n^{j_n}+d y_1^{k_1}y_2^{k_2}\cdots y_n^{k_n}+\cdots
        \]
        无同类项且系数$\cdots,c,d,\cdots\neq 0$.则
        \begin{align*}
            0 & =\varphi\left(
            \sigma_1,\sigma_2,\cdots,\sigma_n
            \right)                                                                                                                \\
              & =\cdots+c\sigma_1^{j_1}\sigma_2^{j_2}\cdots\sigma_n^{j_n}+d\sigma_1^{k_1}\sigma_2^{k_2}\cdots\sigma_n^{k_n}+\cdots
        \end{align*}
        其中$c\sigma_1^{j_1}
            \sigma_2^{j_2}\cdots\sigma_n^{j_n}
        $的首项是\[
            cx_1^{j_1+j_2+\cdots+j_n}x_2^{j_2+\cdots+j_n}\cdots x_n^{j_n}
        \]$d\sigma_1^{k_1}\sigma_2^{k_2}\cdots\sigma_n^{k_n}$的首项是\[
            dx_1^{k_1+k_2+\cdots+k_n}x_2^{k_2+\cdots+k_n}\cdots x_n^{k_n}\]断言所有首项均不是同类项.考虑反证法,即有
        \[
            \begin{cases*}
                j_1+j_2+\cdots+j_n=k_1+k_2+\cdots+k_n \\
                j_2+\cdots+j_n=k_2+\cdots+k_n         \\
                \qquad\cdots\cdots\cdots\cdots        \\
                j_n=k_n
            \end{cases*}
        \]则$j_r=k_r,\forall 1\leqslant r\leqslant n$.

        所有的首项均不是同类项,则一定能找到一个字典序最先的首项,其先于$\varphi
            \left(
            \sigma_1,\sigma_2,\cdots,\sigma_n
            \right)$的所有其它单项,于是它不能够被消去,因此\[
            \varphi\left(
            \sigma_1,\sigma_2,\cdots,\sigma_n
            \right)\neq 0
            \qedhere\]
    \end{proof}
}
\exa{}{}{
    设$f\left(x_1,x_2,x_3\right)=x_1^2x_2
        +x_1^2x_3
        +x_2^2x_1+x_2^2x_3+x_3^2x_1+x_3^2x_2$,求其初等对称多项式表示.\begin{solution}
        显然,$f$的首项为$x_1^2x_2$,则$g_1=
            \sigma_1^{2-1}\sigma_2^{1-0}\sigma_3^{0}=f+3x_1x_2x_3=f+3\sigma_3
            \Longrightarrow f=\sigma_1\sigma_2-3\sigma_3.$
    \end{solution}
}
\exa{齐次对称多项式的初等对称多项式表示示例}{齐次对称多项式的初等对称多项式表示示例}{
    设$6$次齐次对称
    多项式$f=\left(
        x_1^2+x_2^2
        \right)\left(
        x_2^2+x_3^2
        \right)\left(
        x_3^2+x_1^2
        \right)$,求其初等对称多项式表示.\begin{solution}
        显然$f$的首项为$x_1^4x_2^2$.进行上述构造时,某一$f_i$的首项
        $bx_1^{k_1}x_2^{k_2}x_3^{k_3}$满足:
        \[
            \begin{cases*}
                k_1+    k_2+k_3=6 \\
                4\geqslant k_1\geqslant k_2\geqslant k_3\geqslant 0
            \end{cases*}
        \]
        于是该指数组$\left(
            k_1,k_2,k_3
            \right)$的有限取值及其$g_i$为
        \begin{align*}
             & \left(
            4,2,0
            \right)\Longrightarrow
            \sigma_1^{4-2}\sigma_2^{2-0}\sigma_3^{0}=
            \sigma_1^2
            \sigma_2^2,                           \\
             & \left(4,1,1\right) \Longrightarrow
            \sigma_1^{4-1}\sigma_2^{1-1}\sigma_3^{1}=
            \sigma_1^3\sigma_3,                   \\
             & \left(
            3,3,0
            \right)
            \Longrightarrow
            \sigma_1^{3-3}\sigma_2^{3-0}\sigma_3^{0}=
            \sigma_2^3,                           \\
             & \left(
            3,2,1
            \right)
            \Longrightarrow
            \sigma_1^{3-2}\sigma_2^{2-1}\sigma_3^{1}=
            \sigma_1\sigma_2\sigma_3,             \\
             & \left(
            2,2,2
            \right) \Longrightarrow
            \sigma_1^{2-2}\sigma_2^{2-2}\sigma_3^{2}=
            \sigma_3^2
        \end{align*}
        且容易知道
        \[
            f=\sigma_1^2\sigma_2^2+a\sigma_1^3\sigma_3+b\sigma_2^3+c\sigma_1\sigma_2\sigma_3+d\sigma_3^2
        \]取特殊值得到
        \[
            a=-2,b=-2,c=4,d=-1
        \]
        于是
        \[
            f  =\sigma_1^2\sigma_2^2-2\sigma_1^3\sigma_3-2\sigma_2^3+4\sigma_1\sigma_2\sigma_3-\sigma_3^2
        \]
    \end{solution}
}
\rem{}{}{
    \cref{ex:齐次对称多项式的初等对称多项式表示示例}是齐次对称
    多项式的初等对称多项式表示,对于一般的对称多项式,可以进行齐次分解(容易证明分解得到的多项式均是对称的)后按\cref{ex:齐次对称多项式的初等对称多项式表示示例}方法进行.
}
\subsection{Newton公式}
\dfn{Fermal和}{Fermal和}{
    定义
    \[
        s_k\left(
        x_1,x_2,\cdots,x_n
        \right)=x_1^k+x_2^k+\cdots+x_n^k\left(
        k\geqslant 1
        \right)
    \]为Fermat和.显然是一个对称多项式.

    约定$S_0=n.$
}
\lem{}{Fermat与形式导数}{
    设\begin{align*}
        f\left(x\right) & =
        \left(x-x_1\right)\left(x-x_2\right)\cdots\left(x-x_n\right)                         \\
                        & =x^n-\sigma_1x^{n-1} +\sigma_2x^{n-2}+\cdots+\left(-1\right)^{n-1}
        \sigma_{n-1}x+\left(-1\right)^n\sigma_n
    \end{align*}则$\forall k\geqslant 1$
    \[
        x^{k+1}f'\left(x\right)=\left(
        s_0x^k+s_1 x^{k-1}+\cdots+s_k
        \right)f\left(x\right)+g\left(x\right)\]
    其中$\deg g\left(x\right)<n.$
    \begin{proof}
        显然\begin{align*}
            f'\left(x\right) & =\sum_{i=1}^{n}\left(x-x_1\right)\cdots\widehat{
                \left(x-x_i\right)
            } \cdots\left(x-x_n\right)                                          \\
                             & =\sum_{i=1}^{n}\prod_{i\neq j}\left(x-x_j\right) \\
                             & =\sum_{i=1}^{n}\frac{
                f\left(x\right)
            }{x-x_i}
        \end{align*}
        则
        \begin{align*}
            x^{k+1}f'\left(x\right) & =\sum_{i=1}^{n}x^{k+1}\frac{
                f\left(x\right)
            }{x-x_i}                                                                        \\
                                    & =\sum_{i=1}^{n}\frac{
            x^{k+1}-x_i^{k+1}
            }{x-x_i}f\left(x\right)                                                         \\
                                    & +\sum_{i=1}^{n}\frac{x_i^{k+1}f\left(x\right)}{x-x_i} \\
                                    & \quad+\sum_{i=1}^{n}\prod_{j\neq i}(x-x_j)            \\
        \end{align*}
        则定义\[
            g\left(x\right)=\sum_{i=1}^{n}\frac{x_i^{k+1}f\left(x\right)}{x-x_i}
        \]$\deg g\left(x\right)<n$.于是上式等于
        \begin{align*}
              & \sum_{i=1}^{n}
            \left(
            x^{k}+x_ix^{k-1}+\cdots+x_i^k
            \right)f\left(x\right)+g\left(x\right) \\
            = &
            \left(
            s_0x^k+s_1 x^{k-1}+\cdots+s_k
            \right)f\left(x\right)+g\left(x\right)\qedhere
        \end{align*}
    \end{proof}
}
\thm{Newton公式}{Newton公式}{
    记号同\cref{lem:Fermat与形式导数}\begin{enumerate}[label=\arabic*)]
        \item $k\leqslant n-1$,则
              \[
                  s_k-s_{k-1}\sigma_1+\cdots+\left(-1\right)^{k-1}s_1\sigma_{k-1}+\left(-1\right)^k k\sigma_k=0\]
        \item $k\geqslant n$,则
              \[
                  s_k-s_{k-1}\sigma_1+\cdots+
                  \left(-1\right)^{n-1}s_{k - n+1}\sigma_{n-1}+
                  \left(-1\right)^{n}s_{k-n}\sigma_n=0
              \]
    \end{enumerate}\begin{proof}
        因为
        \[
            f'\left(x\right)=
            nx^{n-1}-\sigma_1\left(
            n-1
            \right)x^{n-2}+\cdots+\left(-1\right)^{n-1}\sigma_{n-1}
        \]
        则
        \[
            x^{k+1} f'\left(x\right)=
            nx^{n+k} -\sigma_1\left(
            n-1
            \right)x^{n+k-1}+\cdots+\left(-1\right)^{n-1}\sigma_{n-1}x^{k+1}
        \]
        其中第$i$项是
        \[
            \left(-1\right)^{n+k-i}\left(i-k\right)\sigma_{n+k-i} x^{i}
        \]又因为
        \[
            x^{k+1}  f'\left(x\right)=
            \left(
            s_0x^k+s_1x^{k-1}+\cdots+s_k
            \right)\left(
            x^n-\sigma_1x^{n-1}+\cdots+\left(-1\right)^n\sigma_n
            \right)+g\left(x\right)
        \]接下来比较$x^n$的系数,$k\leqslant n-1$时为
        \[
            \left(-1\right)^k\left(n-k\right)\sigma_k
            =s_k-s_{k-1}\sigma_1+\cdots+\left(-1\right)^{k}s_0\sigma_{k}
        \]
        其中$s_0=n$,于是
        \[
            s_k-s_{k-1}\sigma_1+\cdots+\left(-1\right)^{k-1}s_1\sigma_{k-1}+\left(-1\right)^k k\sigma_k=0
        \]
        $k\geqslant n$时,$x^n$的系数为
        \[
            0= s_k-s_{k-1}\sigma_1+\cdots+
            \left(-1\right)^{n-1}s_{k - n+1}\sigma_{n-1}+
            \left(-1\right)^{n}s_{k-n}\sigma_n\qedhere
        \]
    \end{proof}
}