\section{一元多项式代数}
\subsection{定义}
\dfn{一元多项式}{一元多项式}{
设数域$\mathbb{K}$,$x$为未定元,$a_i\in\mathbb{K}\left(0\leqslant i\leqslant n\right)$,则称如下形式表达式
\[f\left(x\right)=
    a_nx^n+a_{n-1}x^{n-1}+\cdots+a_1x+a_0
\]
为$\mathbb{K}$上关于$x$的一元多项式.$\mathbb{K}$上的一元多项式全体记作$\mathbb{K}\left[x\right]$.
}
\rem{}{}{
    所谓“形式”二字,其一,只是形式写成和式,但加法尚未定义;其二,$x$是未定元,不是变元,因为未定元不是数,变元是数.
}
\dfn{一元多项式的次数}{一元多项式的次数}{
    一般地,若$a_n\neq 0$,则$\deg f\left(x\right)=n.$

    特别地,对于常数多项式$f\left(x\right)=a\in\mathbb{K}$,若$a\neq 0$,则$\deg f\left(x\right)=0$,称为零次多项式;若$a=0$,称为零多项式,约定其次数为$-\infty$.
}
\subsection{运算}
\dfn{一元多项式的相等}{一元多项式的相等}{
特别地,一个多项式等于零多项式当且仅当$a_n=a_{n-1}=\cdots=a_0=0$.

一般地,考虑两个非零多项式
\begin{align*}
     & f\left(x\right)=a_nx^n+a_{n-1}x^{n-1}+\cdots+a_1x+a_0,a_n\neq 0 \\
     & g\left(x\right)=b_mx^m+b_{m-1}x^{m-1}+\cdots+b_1x+b_0,b_m\neq 0
\end{align*}
其中$m,n\geqslant 0.$则$f\left(x\right)=g\left(x\right)$当且仅当$n=m$且$a_i=b_i\left(\forall 0\leqslant i\leqslant n=m\right)$.
}
\dfn{一元多项式的加法与数乘}{一元多项式的加法与数乘}{
一般地,考虑两个多项式
\begin{align*}
     & f\left(x\right)=a_nx^n+a_{n-1}x^{n-1}+\cdots+a_1x+a_0 \\
     & g\left(x\right)=b_mx^m+b_{m-1}x^{m-1}+\cdots+b_1x+b_0
\end{align*}
不妨设$n\geqslant m$.定义加法
为\[
    f\left(x\right)+g\left(x\right)=a_nx^n+\cdots+a_{m+1}x^{m+1}+\left(a_m+b_m\right)x^m+\cdot+\left(a_1+b_1\right)x+\left(a_0+b_0\right)
\]
$\forall k\in\mathbb{K}$,定义数乘
\[
    k\cdot f\left(x\right)=ka_nx^n+ka_{n-1}x^{n-1}+\cdots+ka_1x+ka_0
\]
}
\thm{}{一元多项式全体是线性空间}{
    $\mathbb{K}\left[x\right]$在加法和数乘下成为$\mathbb{K}$-线性空间.
}
\dfn{一元多项式的乘法}{一元多项式的乘法}{
    特别地,任一多项式与零多项式的乘积均为零多项式.

    一般地,对于两个非零多项式
    \begin{align*}
         & f\left(x\right)=a_nx^n+a_{n-1}x^{n-1}+\cdots+a_1x+a_0,a_n\neq 0 \\
         & g\left(x\right)=b_mx^m+b_{m-1}x^{m-1}+\cdots+b_1x+b_0,b_m\neq 0
    \end{align*}
    定义它们的乘积为一个$n+m$次多项式
    \begin{align*}
        h\left(x\right) & =f\left(x\right)\cdot g\left(x\right) \\
                        & =c_{n+m}x^{n+m}+\cdots+c_1x+c_0
    \end{align*}
    其中\[
        c_{n+m}=a_nb_m,\cdots,c_k=\sum_{i+j=k}a_ib_j,\cdots,c_0=a_0b_0\]
}
\pro{一元多项式的乘法的性质}{一元多项式的乘法的性质}{
    \begin{enumerate}[label=\arabic*)]
        \item 交换律:$f\left(x\right)g\left(x\right)=
                  g\left(x\right) f\left(x\right)$
        \item 结合律:$\left(f\left(x\right)g\left(x\right)\right)h\left(x\right)=
                  f\left(x\right)\left(g\left(x\right)h\left(x\right)\right)$
        \item 分配律:$f\left(x\right)\left(g\left(x\right)+h\left(x\right)\right)=f\left(x\right)g\left(x\right)+f\left(x\right)h\left(x\right)$
        \item 乘法与数乘相容:$k\left(f\left(x\right)g\left(x\right)\right)=\left(kf\left(x\right)\right)g\left(x\right)=f\left(x\right)
                  \left(kg\left(x\right)\right)$
        \item 乘法单位元:$1\cdot f\left(x\right)=f\left(x\right)$
    \end{enumerate}
}
\subsection{交换代数}
\cor{}{一元多项式全体是交换代数}{
    $\mathbb{K}\left[x\right]$是$\mathbb{K}$-交换代数.
}
\cor{}{一元多项式全体是交换环}{
    $\mathbb{K}\left[x\right]$又称$\bbk$上的一元多项式环,进一步地,他还是一个交换环.
}
\thm{一元多项式积的次数}{一元多项式积的次数}{
    若$f\left(x\right),g\left(x\right)\in \mathbb{K}\left[x\right]$,则\[
        \deg \left(f\left(x\right)g\left(x\right)\right)=\deg f\left(x\right)+\deg g\left(x\right)
    \]
}
\clm{}{}{
    \cref{thm:一元多项式积的次数}也适用于零多项式.
}
\cor{}{一元多项式是整环}{
    若$f\left(x\right),g\left(x\right)\in \mathbb{K}\left[x\right]$,且$f\left(x\right)\neq 0,g\left(x\right)\neq 0$,则
    \[
        f\left(x\right)g\left(x\right)\neq 0
    \]\begin{proof}
        \[
            \deg \left(f\left(x\right)g\left(x\right)\right)=\deg f\left(x\right)+\deg g\left(x\right)\geqslant 0
            \qedhere
        \]
    \end{proof}
}
\exa{}{}{
    $\mathbb{R}$上的代数$C\left[0,1\right]$不具有整性.

    $\mathbb{K}$上的代数$M_n\left(\mathbb{K}\right)$不具有整性.
}
\cor{一元多项式的乘法消去律}{一元多项式的乘法消去律}{
    设$f\left(x\right)\neq 0 ,g\left(x\right),
        h\left(x\right)\in \mathbb{K}\left[x\right]$,且$f\left(x\right)g\left(x\right)=f\left(x\right)h\left(x\right)$,则\[
        g\left(x\right)=h\left(x\right)
    \]
}
\pro{关于次数的若干结论}{关于次数的若干结论}{
设$f\left(x\right)=a_nx^n+a_{n-1}x^{n-1}+\cdots+a_1x+a_0,g\left(x\right)=b_mx^m+b_{m-1}x^{m-1}+\cdots+b_1x+b_0
    \in \mathbb{K}\left[x\right],c\neq 0\in \mathbb{K}$,则\begin{enumerate}[label=\arabic*)]
    \item $\deg\left(cf\left(x\right)\right)=\deg f\left(x\right)$
    \item $\deg\left(f\left(x\right)\pm g\left(x\right)\right)\leqslant \max\left\{
              \deg f\left(x\right),\deg g\left(x\right)
              \right\}$ ,等号有两种情况取到
          \[
              \begin{cases*}
                  \deg f\left(x\right)\neq \deg g\left(x\right) \\
                  \deg f\left(x\right)=\deg g\left(x\right),a_n+b_m\neq 0
              \end{cases*}
          \]
\end{enumerate}
}