\newpage
\section{线性映射与矩阵}
\subsection{表示矩阵}
\lem{线性扩张定理}{线性扩张定理}{
    设线性空间$V_{\bbk}$、$U_{\bbk}$,取定$V$的一组基
    $\left\{\bm{e}_1,\bm{e}_2,\cdots,
        \bm{e}_n\right\}$,设$U$中的一组向量
    $\left\{\bm{\beta}_1,\bm{\beta}_2,\cdots
        ,\bm{\beta}_n\right\}$,
    则存在唯一一个线性映射$\bm{\varphi}:V\longrightarrow U$满足
    \[
        \bm{\varphi}\left(
        \bm{e}_i
        \right)=\bm{\beta}_i\left(
        \forall
        1\leqslant i\leqslant n
        \right)
    \]\begin{proof}
        首先证明存在性,因为$\bm{\varphi}$在基向量上的取值已确定,所以可以扩张到整个空间$V$上.

        $\forall \bm{\alpha}\in V$,设$\bm{\alpha}=\lambda_1\bm{e}_1
            +\lambda_2\bm{e}_2+\cdots+\lambda_n\bm{e}_n.$定义
        $\bm{\varphi}\left(\bm{\alpha}\right)=
            \lambda_1\bm{\beta}_1
            +\lambda_2\bm{\beta}_2+\cdots+
            \lambda_n\bm{\beta}_n
        $.首先$\bm{\varphi}:V\longrightarrow U$
        是一个映射,下面验证这是线性映射.

        $\forall \bm{\gamma}\in V$,设$\bm{\gamma}=
            b_1\bm{e}_1+b_2\bm{e}_2+\cdots+b_n\bm{e}_n.$则($\bm{\alpha}$指标变一下)
        \[
            \bm{\alpha}+\bm{\gamma}=
            \left(a_1+b_1\right)
            \bm{e}_1+\left(a_2+b_2\right)
            \bm{e}_2+\cdots+\left(a_n+b_n
            \right)\bm{e}_n
        \]
        于是\begin{align*}
            \bm{\varphi}
            \left(\bm{\alpha}+\bm{\gamma}\right) & =
            \left(a_1+b_1\right)
            \bm{\beta}_1+\left(a_2+b_2\right)
            \bm{\beta}_2+\cdots+\left(a_n+b_n
            \right)\bm{\beta}_n                                                                                               \\
                                                 & =\bm{\varphi}\left(\bm{\alpha}\right)+\bm{\varphi}\left(\bm{\gamma}\right)
        \end{align*}
        又因为$\forall k\in \mathbb{K}$
        \begin{align*}
            \bm{\varphi}\left(k\bm{\alpha}\right)
             & =k_1a_1\bm{\beta}_1+k_2a_2\bm{\beta}_2+\cdots+
            k_na_n\bm{\beta}_n
            \\
             & =
            k\bm{\varphi}\left(\bm{\alpha}\right)
        \end{align*}
        故保持线性组合.因此是线性组合,存在性证毕.

        下面证明唯一性.

        设$\bm{\psi}$也满足$\bm{\psi}\left(\bm{e}_i\right)=\bm{\beta}_i$.
        因为$\forall \bm{\alpha}\in V$
        \begin{align*}
            \bm{\psi}\left(\bm{\alpha}\right) & =\lambda_1\bm{\psi}\left(\bm{e}_1\right)+\lambda_2\bm{\psi}\left(\bm{e}_2\right)+\cdots+
            \lambda_n\bm{\psi}\left(\bm{e}_n\right)                                                                                      \\
                                              & =\lambda_1\bm{\beta}_1+\lambda_2\bm{\beta}_2+\cdots+\lambda_n\bm{\beta}_n                \\
                                              & =\bm{\varphi}\left(\bm{\alpha}\right)
        \end{align*}
        故$\bm{\psi}=\bm{\varphi}.$
    \end{proof}
}
\dfn{表示矩阵}{表示矩阵}{
    设线性空间$V,U$线性映射$\bm{\varphi}:
        V\longrightarrow U$,取定$V$的一组基为
    $\left\{\bm{e}_1,\bm{e}_2,\cdots,
        \bm{e}_n\right\}$,取定$U$的一组基为
    $\left\{\bm{f}_1,\bm{f}_2,\cdots,
        \bm{f}_m\right\}$,设
    \begin{align*}
                            & \begin{cases*}
                                  \bm{\varphi}\left(\bm{e}_1
                                  \right) = a_{11}\bm{f}_1+a_{21}\bm{f}_2
                                  +\cdots+a_{m1}\bm{f}_m \\
                                  \bm{\varphi}\left(\bm{e}_2
                                  \right) = a_{12}\bm{f}_1+a_{22}\bm{f}_2
                                  +\cdots+a_{m2}\bm{f}_m \\
                                  \qquad\qquad\cdots\cdots\cdots
                                  \cdots\cdots\cdots     \\
                                  \bm{\varphi}\left(\bm{e}_n
                                  \right) = a_{1n}\bm{f}_1+a_{2n}\bm{f}_2
                                  +\cdots+a_{mn}\bm{f}_m
                              \end{cases*} \\
        \Longleftrightarrow &
        \left(\bm{\varphi}\left(\bm{e}_1
            \right),\bm{\varphi}\left(\bm{e}_2\right),\cdots,\bm{\varphi}\left(
            \bm{e}_n
            \right)\right)=
        \left(
        \bm{f}_1,\bm{f}_2,\cdots,\bm{f}_m
        \right)\bm{A}^{m \times n}
    \end{align*}
    称
    \[
        \bm{A}=\begin{bmatrix}
            a_{11} & a_{12} & \cdots & a_{1n} \\
            a_{21} & a_{22} & \cdots & a_{2n} \\
            \vdots & \vdots &        & \vdots \\
            a_{m1} & a_{m2} & \cdots & a_{mn}
        \end{bmatrix}
    \]
    为$\bm{\varphi}$在给定基下的表示矩阵.
}
\cor{}{原像与像的坐标向量的关系}{
    在\cref{def:表示矩阵}的基础上,我们考虑两个空间中原像与像的坐标向量的关系取$\bm{\alpha}=\lambda_1\bm{e}_1+\lambda_2\bm{e}_2+
        \cdots+\lambda_n\bm{e}_n \in V$,$
        \left(\lambda_1,\lambda_2,\cdots,\lambda_n
        \right)^{\prime}\in \mathbb{K}^n$,故$\bm{\varphi}\left(
        \bm{\alpha}
        \right)=\mu_1\bm{f}_1+\mu_2\bm{f}_2+
        \cdots+
        \mu_m\bm{f}_m\in U$,$\left(
        \mu_1,\mu_2,\cdots,\mu_m
        \right)^{\prime}\in \mathbb{K}^m$,那么有
    \[
        \begin{pmatrix}
            \mu_1  \\
            \mu_2  \\
            \vdots \\
            \mu_m
        \end{pmatrix}
        =\bm{A}\begin{pmatrix}
            \lambda_1 \\
            \lambda_2 \\
            \vdots    \\
            \lambda_n
        \end{pmatrix}
    \]
}
\rem{过渡矩阵与表示矩阵}{}{
    本质上说,过渡矩阵描述的是同一线性空间选取的不同基之间的关系,表示矩阵描述的是两个线性空间之间的线性映射在给定基下的关系.

    形式上说,过渡矩阵一定是方阵,表示矩阵则不一定是方阵.
}
\subsection{矩阵与线性映射}
\dfn{}{线性映射到矩阵的映射}{
    经过上述定义,我们得到一个映射
    \begin{align*}
        \bm{T}:
        \mathcal{L}\left(
        V^n,U^m
        \right)      & \longrightarrow
        M_{m\times n}\left(\mathbb{K}
        \right)                        \\
        \bm{\varphi} & \longmapsto
        \bm{A}
    \end{align*}
    即该映射$\bm{T}$将两个给定线性空间$V$、$U$之间任一线性映射$\bm{\varphi}$映射为它在这两个给定线性空间在给定基下的表示矩阵.这是一个双射.\begin{proof}
        先证明满性,$\forall \bm{A}=\left(a_{ij}\right)_{m\times n}\in M_{m\times n}\left(\mathbb{K}
            \right)$.根据线性扩张定理,只需要考虑基向量上的取值即可
        \[
            \begin{cases*}
                \bm{\varphi}\left(\bm{e}_1
                \right) = a_{11}\bm{f}_1+a_{21}\bm{f}_2
                +\cdots+a_{m1}\bm{f}_m \\
                \bm{\varphi}\left(\bm{e}_2
                \right) = a_{12}\bm{f}_1+a_{22}\bm{f}_2
                +\cdots+a_{m2}\bm{f}_m \\
                \qquad\qquad\cdots\cdots\cdots
                \cdots\cdots\cdots     \\
                \bm{\varphi}\left(\bm{e}_n
                \right) = a_{1n}\bm{f}_1+a_{2n}\bm{f}_2
                +\cdots+a_{mn}\bm{f}_m
            \end{cases*}
        \]
        由线性扩张定理知,上述定义可扩张为$\bm{\varphi}:V\longrightarrow U$,且表示矩阵为$\bm{A}$.满性证毕.

        再来考虑单性.设$\bm{\varphi}$、$\bm{\psi}\in \mathcal{L}\left(V,U\right)$
        且二者表示矩阵均为$\bm{A}$,故二者在基向量上的取值相同, 由线性扩张定理知二者相同.单性证毕.
    \end{proof}
}
\rem{记号说明}{}{
    设$\mathbb{K}$上的线性空间$V^n$和$U^m$,分别取定$\left\{
        \bm{e}_1,\bm{e}_2,\cdots,\bm{e}_n
        \right\}$和$\left\{
        \bm{f}_1,\bm{f}_2,\cdots,\bm{f}_m
        \right\}$为基.定义坐标向量的线性同构为
    \begin{align*}
         & \bm{\eta}_V:V^n\longrightarrow \mathbb{K}^n \\
         & \bm{\eta}_U:U^m\longrightarrow \mathbb{K}^m
    \end{align*}
    设线性映射$\bm{\varphi}\in \mathcal{L}\left(
        V,U
        \right)$在给定基下的表示矩阵$\bm{A}\in M_{m \times n}\left(\mathbb{K}\right)$,记线性映射
    \begin{align*}
        \bm{\varphi}_{\bm{A}}:
        \mathbb{K}^n & \longrightarrow
        \mathbb{K}^m                   \\
        \bm{x}       & \longmapsto
        \bm{Ax}
    \end{align*}
}
\thm{}{矩阵与线性映射及交换图}{
对于上述定义的映射$\bm{T}:
    \mathcal{L}\left(V,U\right)
    \longrightarrow
    M_{m \times n}\left(\mathbb{K}\right)$

$(1)$ 该映射为线性同构

$(2)$ $\forall \bm{\varphi}\in \mathcal{L}\left(
    V,U
    \right),\bm{A} = \bm{T}
    \left(\bm{\varphi}\right)$

$(3)$ $\bm{\eta}_U \circ \bm{\varphi} =
    \bm{\varphi}_{\bm{A}}\circ \bm{\eta}_V$
即如下交换图所示:

\begin{center}
    \incfig[scale=0.9,float=H]{CommutativeDiagram}
\end{center}

\begin{proof}
    $(1)$ 设$\bm{\varphi}$、$\bm{\psi}\in \mathcal{L}\left(V,U\right)$且\[\bm{T}\left(\bm{\varphi}\right)=\bm{A}=\left(a_{ij}\right)_{m\times n}
        ,\bm{T}\left(\bm{\psi}\right)=\bm{B}=\left( b_{ij}
        \right)_{m\times n}\]即
    \[
        \bm{\varphi}\left(\bm{e}_j\right)=
        \sum_{i=1}^{m}a_{ij}\bm{f}_i,
        \bm{\psi}\left(\bm{e}_j\right)=
        \sum_{i=1}^{m}b_{ij}\bm{f}_i
    \]

    因为$\forall 1\leqslant j\leqslant n$:
    \[\left(\bm{\varphi}+\bm{\psi}\right)\left(\bm{e}_j\right)=
        \bm{\varphi}\left(
        \bm{e}_j
        \right)+\bm{\psi}\left(
        \bm{e}_j
        \right)\]故
    \[\bm{T}\left(
        \bm{\varphi}+\bm{\psi}
        \right)=\bm{A}+\bm{B}=
        \bm{T}\left(\bm{\varphi}\right)
        +\bm{T}\left(\bm{\psi}\right)\]

    因为$\forall k\in \mathbb{K}:\left(k\cdot \bm{\varphi}\right)\left(\bm{e}_j\right)=
        k\bm{\varphi}\left(\bm{e}_j\right)$,故\[\bm{T}\left(k\bm{\varphi}\right)=k\bm{A}=
        k\bm{T}\left(\bm{\varphi}\right)\]

    $(3)$

    \begin{center}
        \incfig[scale=0.9,float=H]{CommutativeDiagramSolution}
    \end{center}按图中得证.
\end{proof}
}
\subsection{线性变换}
\dfn{}{线性变换的表示矩阵}{
    特别地,我们考虑线性变换
    $\bm{\varphi}\in \mathcal{L}\left(
        V^n_{\mathbb{K}}
        \right)$,在给定的基$\left\{
        \bm{e}_1,\bm{e}_2,\cdots,\bm{e}_n
        \right\}$下有
    \begin{align*}
                            &
        \begin{cases*}
            \bm{\varphi}\left(\bm{e}_1
            \right) = a_{11}\bm{e}_1+
            a_{21}\bm{e}_2
            +\cdots+a_{n1}\bm{e}_n \\
            \bm{\varphi}\left(\bm{e}_2
            \right) = a_{12}\bm{e}_1+a_{22
            }\bm{e}_2
            +\cdots+a_{n2}\bm{e}_n \\
            \qquad\qquad\cdots\cdots\cdots
            \cdots\cdots\cdots     \\
            \bm{\varphi}\left(\bm{e}_n
            \right) = a_{1n}\bm{e}_1
            +a_{2n}\bm{e}_2
            +\cdots+a_{nn}\bm{e}_n
        \end{cases*} \\
        \Longleftrightarrow &
        \left(\bm{\varphi}\left(\bm{e}_1\right), \bm{\varphi}\left(\bm{e}_2\right),
        \cdots,\bm{\varphi}\left(
            \bm{e}_n
            \right)\right)
        =
        \left(\bm{e}_1,\bm{e}_2,\cdots,\bm{e}_n\right)
        \bm{A}
    \end{align*}
    那么
    \[
        \bm{A}=\begin{pmatrix}
            a_{11} & a_{12} & \cdots & a_{1n} \\
            a_{21} & a_{22} & \cdots & a_{2n} \\
            \vdots & \vdots &        & \vdots \\
            a_{n1} & a_{n2} & \cdots & a_{nn}
        \end{pmatrix}\in M_n\left(
        \mathbb{K}
        \right)
    \]
    为线性变换$\bm{\varphi}$在给定基下的表示矩阵.
}
\subsection{矩阵乘法的几何意义}
\thm{矩阵乘法的几何意义}{矩阵乘法的几何意义}{
    设$\mathbb{K}$上线性空间$V,U,W$间的线性映射
    $\bm{\varphi}:V^n\longrightarrow
        U^m \in \mathcal{L}\left(V,U
        \right),\bm{\psi}:U^m\longrightarrow
        W^p
        \in \mathcal{L}\left(U,W\right)$,则$\bm{\psi}\circ \bm{\varphi}
        :V^n\longrightarrow W^p \in
        \mathcal{L}\left(V,W\right)$
    也是线性映射且
    \[
        \bm{T}\left(\bm{\psi\circ\varphi}\right)
        =\bm{T}\left(\bm{\psi}\right)\cdot
        \bm{T}\left(\bm{\varphi}\right)
    \]
    也就是说,矩阵乘法的几何意义是线性映射的复合.\begin{proof}
        分别取$V,U,W$的一组基为$\left\{
            \bm{e}_1,\bm{e}_2,\cdots,\bm{e}_n
            \right\},\left\{\bm{f}_1,\bm{f}_2,\cdots,\bm{f}_m
            \right\},\left\{
            \bm{g}_1,\bm{g}_2,\cdots,\bm{g}_p
            \right\}$.$\bm{T}\left(\bm{\varphi}\right)=\bm{A}=
            \left(a_{ij}\right)_{m\times n}$,
        $\bm{T}\left(\bm{\psi}\right)=\bm{B}=\left(b_{ij}\right)_{n\times
                p}$.只要证:\[\bm{T}\left(\bm{\psi}\circ
            \bm{\varphi}\right)=\bm{BA}\]

        因为$\displaystyle
            \forall 1\leqslant j\leqslant n,
            \bm{\varphi}\left(\bm{e}_j\right)=
            \sum_{i=1}^{m}a_{ij}\bm{f}_i$,故
        \begin{align*}
            \bm{\psi}\left(\bm{\varphi}\left(\bm{e}_j\right)\right)
            =\bm{\psi}\left(\sum_{i=1}^{m}a_{ij}\bm{f}
            _i\right)=\sum_{i=1}^{m}a_{ij}\bm{\psi}\left(
            \bm{f}_i
            \right)
        \end{align*}
        即
        \[
            \left(\bm{\psi}\left(\bm{\varphi}\left(\bm{e}_1\right)\right),
            \bm{\psi}\left(\bm{\varphi}\left(\bm{e}_2\right)\right),
            \cdots,\bm{\psi}\left(\bm{\varphi}\left(
                    \bm{e}_n
                    \right)\right)\right)=
            \left(
            \bm{\psi}\left(\bm{f}_1\right),\bm{\psi}\left(\bm{f}_2\right),\cdots,
            \bm{\psi}\left(\bm{f}_m\right)
            \right)\bm{A}
        \]
        又因为
        \[
            \left(
            \bm{\psi}\left(\bm{f}_1\right),\bm{\psi}\left(\bm{f}_2\right),\cdots,
            \bm{\psi}\left(\bm{f}_m\right)
            \right)=\left(\bm{g}_1,\bm{g}_2,\cdots,\bm{g}_p\right)\bm{B}
        \]
        故有
        \[
            \left(\bm{\psi}\left(\bm{\varphi}\left(\bm{e}_1\right)\right),
            \bm{\psi}\left(\bm{\varphi}\left(\bm{e}_2\right)\right),
            \cdots,\bm{\psi}\left(\bm{\varphi}\left(
                    \bm{e}_n
                    \right)\right)\right)=
            \left(\bm{g}_1,\bm{g}_2,\cdots,\bm{g}_p\right)\bm{B}
            \bm{A}
        \]
        于是\[\bm{T}\left(\bm{\psi}\circ \bm{\varphi}\right)=\bm{BA}=\bm{T}\left(\bm{\psi}\right)\cdot\bm{T}\left(\bm{\varphi}\right)\qedhere\]
    \end{proof}
}
\cor{}{线性变换到矩阵的映射是代数同构}{
    $\bm{T}:\mathcal{L}\left(V\right)
        \longrightarrow M_n\left(\mathbb{K}
        \right)$是线性同构,且$\forall
        \bm{\varphi},\bm{\psi}\in \mathcal{L}\left(V
        \right)$
    \[
        \bm{T}\left(\bm{\varphi}\circ \bm{\psi}
        \right)=
        \bm{T}\left(\bm{\varphi}\right)
        \cdot \bm{T}\left(\bm{\psi}\right)
    \]
    即保持乘法,则$\bm{T}$是一个$\mathbb{K}$-代数同构.
}
\cor{}{对线性变换到矩阵的映射的代数同构的刻画}{
    对于\cref{cor:线性变换到矩阵的映射是代数同构}的$\bm{T}$,有
    \begin{enumerate}[label=\arabic*)]
        \item $\bm{T}\left(
                  \bm{I}_V
                  \right)=\bm{I}_n$
        \item $\bm{\varphi}$为$V$上的自同构
              $\Longleftrightarrow \bm{T}\left(
                  \bm{\varphi}
                  \right)$可逆且此时有
              \[
                  \bm{T}\left(\bm{\varphi}^{-1}
                  \right)=
                  \bm{T}\left(\bm{\varphi
                  }\right)^{-1}
              \]
    \end{enumerate}\begin{proof}
        \begin{enumerate}[label=\arabic*)]
            \item 显然的,考虑基向量即可.
            \item 先证必要性.设自同构$\bm{\varphi}$,$\bm{I}_V=\bm{\varphi}\bm{\varphi}^{-1}=
                      \bm{\varphi}^{-1}\bm{\varphi}.$同时作用$\bm{T}$即可.

                  考虑充分性,设$\bm{T}\left(\bm{\varphi}\right)$可逆即
                  $\bm{T}\left(\bm{\varphi}\right)^{-1}$存在.由于
                  $\bm{T}$为双射,故存在$\bm{\psi}$使得$\bm{T}\left(\bm{\psi}\right)=\bm{T}\left(\bm{\varphi}\right)^{-1}$,复合即得.
        \end{enumerate}
    \end{proof}
}
\subsection{不同基下的表示矩阵}
\thm{}{不同基下的表示矩阵的关系}{
    设$\bm{\varphi}\in \mathcal{L}\left(V
        \right)$,$V$是$\mathbb{K}$上的
    $n$维线性空间,取两组基$\left\{
        \bm{e}_1,\bm{e}_2,\cdots,\bm{e}_n
        \right\}$、$
        \left\{
        \bm{f}_1,\bm{f}_2,\cdots,\bm{f}_n
        \right\}$,
    两组基的过渡矩阵为
    $\bm{P}$,$\bm{\varphi}$在两组基下的表示矩阵分别为
    $\bm{A}$、$\bm{B}$,则
    \[
        \bm{B}=\bm{P}^{-1}\bm{AP}
    \]\begin{proof}
        因为
        \[
            \left(\bm{f}_1,\bm{f}_2,\cdots,\bm{f}_n\right)
            =\left(\bm{e}_1,\bm{e}_2,\cdots,\bm{e}_n\right)
            \bm{P}
        \]
        \[
            \left(
            \bm{\varphi}\left(\bm{e}_1\right),
            \bm{\varphi}\left(\bm{e}_2\right),\cdots,
            \bm{\varphi}\left(\bm{e}_n\right)
            \right)=
            \left(\bm{e}_1,\bm{e}_2,\cdots,\bm{e}_n\right)\bm{A}
        \]
        \[
            \left(
            \bm{\varphi}\left(\bm{f}_1\right),
            \bm{\varphi}\left(\bm{f}_2\right),\cdots,
            \bm{\varphi}\left(\bm{f}_n\right)
            \right)=
            \left(\bm{f}_1,\bm{f}_2,\cdots,\bm{f}_n\right)\bm{B}
        \]
        对第一个式子等式两边同时作用$\bm{\varphi}$得
        \begin{align*}
            \left(
            \bm{\varphi}\left(\bm{f}_1\right),
            \bm{\varphi}\left(\bm{f}_2\right),\cdots,
            \bm{\varphi}\left(\bm{f}_n\right)
            \right) & =
            \left(\bm{\varphi}\left(\bm{e}_1\right)
            ,\bm{\varphi}\left(\bm{e}_2\right)
            ,\cdots,\bm{\varphi}\left(\bm{e}_n\right)
            \right)\bm{P}                                             \\
                    & =\left(\bm{e}_1,\bm{e}_2,\cdots,\bm{e}_n\right)
            \bm{AP}
        \end{align*}
        同时
        \begin{align*}
            \left(
            \bm{\varphi}\left(\bm{f}_1\right),
            \bm{\varphi}\left(\bm{f}_2\right),\cdots,
            \bm{\varphi}\left(\bm{f}_n\right)
            \right) & =\left(\bm{f}_1,\bm{f}_2,\cdots,\bm{f}_n\right)
            \bm{B}                                                    \\
                    & =\left(\bm{e}_1,\bm{e}_2,\cdots,\bm{e}_n\right)
            \bm{PB}\qedhere
        \end{align*}
    \end{proof}
}
\dfn{相似}{相似}{
    设矩阵$\bm{A}$、$\bm{B}\in M_n\left(
        \mathbb{K}
        \right)$,若存在非异阵$\bm{P}\in M_n\left(
        \mathbb{K}
        \right)$使得
    \[
        \bm{B}=\bm{P}^{-1}\bm{AP}
    \]
    那么称$\bm{A}$与$\bm{B}$相似,记作$\bm{A}\thickapprox \bm{B}$
}