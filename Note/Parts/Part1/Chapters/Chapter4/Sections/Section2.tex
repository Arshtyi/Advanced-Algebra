\newpage
\section{线性映射的运算}
\subsection{运算}
\dfn{线性映射全体}{线性映射全体}{
    设$V_{\mathbb{K}}$、$U_{\mathbb{K}}$,记$\call   \left(V,U\right)$为$V \longrightarrow U$的线性映射全体.
}
\dfn{线性映射的加法与数乘}{线性映射的加法与数乘}{
    $\forall \bm{\varphi},\bm{\psi}
        \in \call   \left(
        V,U
        \right)$
    ,定义加法
    \[
        \left(\bm{\varphi}+\bm{\psi}\right)
        \left(\bm{\alpha}\right)\coloneqq \bm{\varphi}\left(\bm{\alpha}
        \right)+\bm{\psi}\left(\bm{\alpha}
        \right)
    \]
    $\forall k \in \mathbb{K}$,定义数乘
    \[
        \left(
        k\cdot \bm{\varphi}
        \right)\left(\bm{\alpha}
        \right) \coloneqq
        k \cdot \bm{\varphi}\left(
        \bm{\alpha}
        \right)
    \]
    容易验证二者均是线性映射且
    $\bm{\varphi}+\bm{\psi} \in \call  \left(V,U\right)$,$k\bm{\varphi}
        \in \call   \left(V,U\right)$.
}
\thm{}{线性映射全体是线性空间}{
    $\call   \left(V,U\right)$在\cref{def:线性映射的加法与数乘}的加法和数乘运算下是$\mathbb{K}$上的线性空间.

    特别地,$\call   \left(
        V,\mathbb{K}
        \right)$是$V$上的线性函数全体构成的线性空间,称为$V$的共轭空间,记作$V^*$,当$V$维数有限时,$V^*$也称为$V$的对偶空间.

    $\call
        \left(V,V\right) = \call
        \left(V\right)$为$V$上的线性变换
    全体构成的线性空间,它可以额外将映射的复合定义为乘法运算.
}
\subsection{代数}
\dfn{代数}{代数}{
    设$\mathbb{K}$上的线性空间$A$,如果定义一种乘法运算
    \begin{align*}
        \cdot :
        A \times A       & \longrightarrow A    \\
        \left(a,b\right) & \longmapsto a\cdot b
    \end{align*}
    若$\forall a,b,c\in A,\forall k \in \mathbb{K}$,满足\begin{enumerate}[label=\arabic*)]
        \item 结合律:$
                  \left(a \cdot b\right) \cdot c =
                  a \cdot \left(b \cdot c\right)$
        \item 存在单位元$e \in A $使得
              $e \cdot a = a \cdot e = a$
        \item 分配律:
              $a\cdot\left(b+c\right)
                  = a\cdot b +a\cdot c,
                  \left(b + c\right)\cdot a
                  =b \cdot a + c \cdot a$
        \item 与数乘的相容性:
              $k\left(a\cdot b\right)=
                  \left(ka\right)\cdot b=
                  a\cdot\left(kb\right)
              $
    \end{enumerate}
    那么称$A$为$\mathbb{K}$上的代数即$\bm{A}$是一个$\mathbb{K}$-代数,单位元$e$称为$A$的恒等元.
}
\exa{}{}{
    区间$\left[0,1\right]$上的连续函数全体构成的线性空间
    $V = C\left[0,1\right]/
        \bbr $是$\bbr $上的代数.乘法定义为
    \[
        \left(f\cdot g\right)\left(x
        \right)\coloneqq
        f\left(x\right)\cdot g\left(x
        \right)
    \]
}
\exa{}{}{
    若$\mathbb{K}_1 \subseteq\mathbb{K}_2$,定义$\mathbb{K}_2$中的乘法是数的乘法,则$\mathbb{K}_2$是$\mathbb{K}_1$-代数.
}
\exa{}{}{
    设线性空间$M_n\left(
        \mathbb{K}
        \right)$,定义乘法为矩阵的乘法,
    则$M_n\left(\mathbb{K}\right)$
    是$\mathbb{K}$-代数.
}
\exa{}{}{
    \cref{def:线性映射全体}提到$\call   \left(V_{\mathbb{K}}
        \right)$是定义有乘法的,那么它是$\mathbb{K}$-代数.
}
\subsection{幂}
\dfn{线性映射的幂}{线性映射的幂}{
    设$\bm{\varphi}\in \call   \left(
        V
        \right),\forall n \in \mathbb{Z}^+$,定义幂
    \[
        \bm{\varphi}^n = \bm{\varphi} \circ\bm{\varphi}\circ\cdots \circ \bm{\varphi}
    \]
    且\begin{enumerate}[label=\arabic*)]
        \item $\bm{\varphi}^n \circ
                  \bm{\varphi}^m = \bm{\varphi} ^{
                      n + m
                  } $
        \item $ \left(\bm{\varphi}^n\right) ^m
                  = \bm{\varphi}^{nm}$
        \item $\bm{\varphi}$是自同构时
              $\bm{\varphi}^{-n}\coloneqq
                  \left(\bm{\varphi}^{-1}\right) ^n
                  =\left(\bm{\varphi}^n\right)^{-1}$
        \item 约定$\bm{\varphi}^0=
                  \bm{1}_V$
    \end{enumerate}
}
\cor{}{线性映射的幂的不可交换性}{
    对于$\bm{\varphi},\bm{\psi}\in
        \call   \left(V\right)$,一般
    $\bm{\varphi}\circ \bm{\psi}
        \neq \bm{\psi} \circ \bm{\varphi}$,故
    \[
        \left(
        \bm{\varphi}\circ \bm{\psi}
        \right)^n =
        \bm{\varphi}\circ\bm{\psi}\circ\bm{\varphi}
        \circ\bm{\psi} \cdots
        \circ\bm{\varphi}\circ\bm{\psi}
        \neq
        \bm{\varphi}^n\circ\bm{\psi}^n
    \]
}
\cor{}{自同构的保持}{
    若$\bm{\varphi},\bm{\psi}
        \in \call   \left(
        V_{\mathbb{K}}
        \right)$为自同构,则$\bm{\varphi}\circ\bm{\psi}$
    是自同构,且$\left(\bm{\varphi \circ\psi}\right)
        ^{-1}= \bm{\psi}^{-1}\circ\bm{\varphi}^{-1}$.
    特别地,$\forall k\neq 0 \in \mathbb{K}$,则$k\cdot \bm{\varphi}$为自同构且
    $\left(k\bm{\varphi}\right)^{-1}=
        k^{-1}\bm{\varphi}^{-1}.$
}