\newpage
\section{不变子空间}
\subsection{定义}
\dfn{不变子空间}{不变子空间}{
    设线性映射$\bm{\varphi}\in \call  \left(V_{\mathbb{K}}\right)$,$U$是$V$的子空间.若
    \[
        \bm{\varphi}\left(U\right)\subseteq U
    \]
    则称$U$为线性变换$\bm{\varphi}$的不变子空间,或称为$\bm{\varphi}$-不变子空间.
}
\dfn{}{线性变换限制在自身的不变子空间上}{
    已知$U$为$\bm{\varphi}$-不变子空间,若进一步将$\bm{\varphi}$的定义域限制在$U$上,则这个$\bm{\varphi}$成为$U$上的一个线性变换,称为$\bm{\varphi}$诱导出的
    线性变换或者是$\bm{\varphi}$在$U$上的限制,记作
    \[
        \bm{\varphi}\Big |_U:U\longrightarrow U
    \]
}
\exa{}{}{
    设$\bm{\varphi}\in \call  \left(V
        \right)$,则一定有两个平凡的$\bm{\varphi}$-不变子空间
    即零子空间$0$和全子空间$V$.此外,核空间$\Ker \bm{\varphi}$和像空间$\Image\bm{\varphi}$存在时也一定是$\bm{\varphi}$-不变子空间.
}
\rem{}{}{
    事实上,像空间和核空间都是必定存在的,但是当映射是恒等变换时,像空间即全子空间,零空间即零子空间.因此上面应该说是像空间和核空间非平凡或者是非平凡的像空间和核空间存在时.
}
\clm{零空间与零子空间}{}{
    零子空间$0$指的是只含有零向量的线性子空间,零空间是核空间$\Ker \bm{\varphi}=\Null\left(\bm{\varphi}\right)$.
    显然\[
        0\subseteq \Ker \bm{\varphi}
    \]
}
\exa{}{}{
    考虑$V=\bbr ^2$,线性变换$\bm{\varphi}$为绕原点逆时针旋转$\theta$角,其在标准基下的表示矩阵为
    \[
        \bm{A}=\begin{bmatrix}
            \cos \theta & -\sin \theta \\
            \sin \theta & \cos \theta
        \end{bmatrix}
    \]
    设$L$为一维$\bm{\varphi}$-不变子空间即$\bm{\varphi}\left(
        L
        \right)\subseteq L$,容易知道
    $\theta \neq k\pi\left(k\in \mathbb{Z}\right)$时这样的$L$不存在即此时$\bm{\varphi}$没有非平凡的不变子空间.
}
\exa{}{}{
    考虑$\bm{\varphi}:\bm{\alpha}\longmapsto k\bm{\alpha}
        \left(k\in \mathbb{K}\text{固定}\right)$,则$V$的任一子空间均为$\bm{\varphi}$-不变子空间.
}
\lem{}{张成的子空间为不变子空间的判定}{
    设线性变换$\bm{\varphi}\in \call  \left(V
        \right)$,子空间$U=L\left(\bm{\alpha}_1,\bm{\alpha}_2,\cdots,\bm{\alpha}_m\right)$,其中$\bm{\alpha}_i\in V$,则$U$为$\bm{\varphi}$-不变子空间等价于
    \[
        \bm{\varphi}\left(\bm{\alpha}_i\right)
        \in U\left(\forall 1 \leqslant i \leqslant m \right)
    \]\begin{proof}
        必要性显然.下证充分性.

        设$\bm{\varphi}\left(\bm{\alpha}_i\right)\in U\left(1\leqslant i\leqslant m\right)$,任取$\bm{\alpha}
            \in U.$故有
        \[
            \bm{\alpha}=\lambda_1\bm{\alpha}_1+\lambda_2\bm{\alpha}_2+\cdots+\lambda_m\bm{\alpha}_m
        \]
        则两边作用$\bm{\varphi}$即证.
    \end{proof}
}
\subsection{表示矩阵}
\thm{}{不变子空间诱导的线性变换的表示矩阵}{
    设线性变换$\bm{\varphi}\in \call  \left(V^n\right)$,且$U$为$\bm{\varphi}$-不变子空间,取$U$的一组基
    为$\left\{\bm{e}_1,\bm{e}_2,
        \cdots,\bm{e}_r\right\}$,将其扩张为
    $V$的一组基$\left\{\bm{e}_1,\bm{e}_2,\cdots,\bm{e}_r,\bm{e}_{r+1},\cdots,
        \bm{e}_n\right\}$,则$\bm{\varphi}$在这一组基下的表示矩阵的形式为
    \[
        \begin{pmatrix}
            \bm{A}^r & \bm{B}\quad  \\
            \bm{O}   & \bm{D}^{n-r}
        \end{pmatrix}
    \]\begin{proof}
        因为$\bm{\varphi}\left(\bm{e}_i\right)\in U\left(1\leqslant i\leqslant r\right)$,即有
        \[
            \begin{cases*}
                \bm{\varphi}\left(\bm{e}_1\right)=a_{11}\bm{e}_1+a_{21}\bm{e}_2+\cdots+a_{r1}\bm{e}_r \\
                \bm{\varphi}\left(\bm{e}_2\right)=a_{12}\bm{e}_1+a_{22}\bm{e}_2+\cdots+a_{r2}\bm{e}_r \\
                \qquad\qquad\cdots\cdots\cdots\cdots\cdots                                            \\
                \bm{\varphi}\left(\bm{e}_r\right)=a_{1r}\bm{e}_1+a_{2r}\bm{e}_2+\cdots+a_{rr}\bm{e}_r
            \end{cases*}
        \]
        于是
        \begin{align*}
              & \left(\bm{\varphi}\left(\bm{e}_1\right),
            \bm{\varphi}\left(\bm{e}_2\right),
            \cdots,\bm{\varphi}\left(\bm{e}_r\right),
            \bm{\varphi}\left(\bm{e}_{r+1}\right),
            \cdots,\bm{\varphi}\left(\bm{e}_n\right)\right)
            \\
            = &
            \left(\bm{e}_1,\bm{e}_2,\cdots,\bm{e}_r,\bm{e}_{r+1},\cdots,\bm{e}_n\right)
            \begin{bmatrix}
                \bm{A}^r & \bm{B}\quad  \\
                \bm{O}   & \bm{D}^{n-r}
            \end{bmatrix}
            \qedhere
        \end{align*}
    \end{proof}
}
\cor{}{线性变换的表示矩阵诱导的不变子空间}{
    设线性变换$\bm{\varphi}\in \call  \left(
        V^n\right)$,且$\bm{\varphi}$在$V$的一组基
    $\left\{\bm{e}_1,\bm{e}_2,\cdots,
        \bm{e}_n\right\}$下的表示矩阵具有形式
    \[
        \begin{pmatrix}
            \bm{A}^r & \bm{B}\quad  \\
            \bm{O}   & \bm{D}^{n-r}
        \end{pmatrix}
    \]
    那么$U=L\left(\bm{e}_1,\bm{e}_2,\cdots,\bm{e}_r\right)$是$\bm{\varphi}$-不变子空间.
}
\cor{}{全空间分解为两个不变子空间的直和}{
    设线性变换$\bm{\varphi}\in \call  \left(V^n\right)$,且$V=V_1\oplus V_2$($V_1$、$V_2$均为$\bm{\varphi}$-不变子空间),那么一定可以分别取$V_1$、$V_2$的一组基来拼成$V$的一组基,使得
    $\bm{\varphi}$在这组基下的表示矩阵为
    \[
        \begin{pmatrix}
            \bm{A}^r & \bm{O}\quad  \\
            \bm{O}   & \bm{D}^{n-r}
        \end{pmatrix}
    \]\begin{proof}
        取$V_1$的基$\left\{\bm{e}_1,\bm{e}_2,\cdots,\bm{e}_r\right\}$,$V_2$的基$\left\{\bm{e}_{r+1},\cdots,\bm{e}_n\right\}$.因为
        $\bm{\varphi}\left(\bm{e}_i\right)\in V_1\left(\forall 1\leqslant i\leqslant r\right)$,$\bm{\varphi}\left(\bm{e}_i\right)\in V_2\left(\forall r+1\leqslant i\leqslant n\right)$.仿照\cref{thm:不变子空间诱导的线性变换的表示矩阵}即得.
    \end{proof}
}
\cor{}{全空间分解为多个不变子空间的直和}{
设线性变换$\bm{\varphi}\in \call  \left(V^n
    \right)$,且$V=V_1\oplus V_2\oplus \cdots \oplus V_m$($V_i$均为
$\bm{\varphi}$-不变子空间),分别取$V_i$的基,设$\bm{\varphi}\big |_{V_i}$在给定基下的表示矩阵为$\bm{A}_i\left(1\leqslant i \leqslant m\right)$,则将这些$V_i$的基拼成$V$的一组基后,$\bm{\varphi}$在这组基下的表示矩阵为\[\bm{A}=
    \diag \left\{
    \bm{A}_1,\bm{A}_2,\cdots,\bm{A}_m
    \right\}\]
}
\subsection{例子}
\exa{}{}{
    设$V^3$的一组基为$\left\{\bm{e}_1,\bm{e}_2,\bm{e}_3\right\}$,线性变换$\bm{\varphi}\in \call  \left(V\right)$在这组基下的表示矩阵为
    \[\bm{A}=
        \begin{pmatrix}
            3 & 1 & -1 \\
            2 & 2 & -1 \\
            2 & 2 & 0
        \end{pmatrix}
    \]
    求证:$U=L\left(\bm{e}_3,\bm{e}_1+\bm{e}_2+2\bm{e}_3\right)$为$\bm{\varphi}$-不变子空间.\begin{proof}
        只需证
        \[
            \bm{\varphi}\left(\bm{e}_3\right)\in U,
            \bm{\varphi}\left(\bm{e}_1+\bm{e}_2+2\bm{e}_3\right)\in U
        \]
        通过线性同构
        \[
            \bm{\eta}:V\longrightarrow \mathbb{K}^3
        \]
        我们仅需要验证坐标向量即可
        \[
            \bm{e}_3 \longmapsto \begin{pmatrix}
                0 \\
                0 \\
                1
            \end{pmatrix},\bm{e}_1+\bm{e}_2+2\bm{e}_3\longmapsto
            \begin{pmatrix}
                1 \\1\\2
            \end{pmatrix}
        \]
        作线性变换
        \[
            \bm{\varphi}\left(\bm{e}_3\right)=\bm{A}\begin{pmatrix}
                0 \\0\\1
            \end{pmatrix}=\begin{pmatrix}
                -1 \\-1\\0
            \end{pmatrix},\bm{\varphi}\left(\bm{e}_1+\bm{e}_2+2\bm{e}_3\right)=
            \bm{A}\begin{pmatrix}
                1 \\1\\2
            \end{pmatrix}=\begin{pmatrix}
                2 \\2\\4
            \end{pmatrix}
        \]
        只需要证明$\bm{\varphi}\left(\bm{e}_3\right)$和
        $\bm{\varphi}\left(\bm{e}_1+\bm{e}_2+2
            \bm{e}_3\right)$是$\bm{e}_3$、$\bm{e}_1
            +\bm{e}_2+2\bm{e}_3$的线性组合即可.一般地,这就是解线性方程组
        \[
            \begin{pmatrix}
                0 & 1 \\0&1\\1 & 2
            \end{pmatrix}\bm{x}=\bm{\varphi}\left(\bm{e}_3\right)
            ,
            \begin{pmatrix}
                0 & 1 \\0&1\\1 & 2
            \end{pmatrix}\bm{x}=\bm{\varphi}\left(\bm{e}_1+\bm{e}_2+2\bm{e}_3\right)
        \]
        进一步地,只需要证明线性方程组有解即系数矩阵与增广矩阵的秩相等即可.
        \[
            \rank \begin{pmatrix}
                0 & 1 \\0&1\\1 & 2
            \end{pmatrix}=\rank \begin{pmatrix}
                0 & 1 & -1 \\0&1&-1\\1 & 2 & 0
            \end{pmatrix}=2
        \]
        \[
            \rank \begin{pmatrix}
                0 & 1 \\0&1\\1 & 2
            \end{pmatrix}=\rank \begin{pmatrix}
                0 & 1 & 2 \\0&1&2\\1 & 2 & 4
            \end{pmatrix}=2
            \qedhere
        \]
    \end{proof}
}
\clm{}{}{
    关键在于代数与几何、具象与抽象的结合与转换.

    \begin{center}
        \incfig[scale=0.9,float=H]{CommutativeDiagram}
    \end{center}
}
\subsection{代数与几何}
\exa{}{}{
    设线性映射$\bm{\varphi}\in \call  \left(V^n,U^m\right)$,$\rmr \left(\bm{\varphi}\right)=r\geqslant 1$,证明:
    $\exists \bm{\varphi}_i\in \call  \left(V,U\right)$,满足
    $\rmr \left(\bm{\varphi}_i\right)=1$且$\bm{\varphi}=\bm{\varphi}_1+\bm{\varphi}_2+\cdots+\bm{\varphi}_r$.\begin{proof}
        代数角度看:设$\bm{A}\in M_{m\times n}\left(\mathbb{K}\right)$,且
        $\rmr \left(\bm{A}\right)=r\geqslant 1$,求证:
        $\exists \bm{A}_i\in M_{m\times n}\left(\mathbb{K}\right)$,满足
        $\rmr \left(\bm{A}\right)=1$且$\bm{A}=\bm{A}_1+\bm{A}_2+\cdots+\bm{A}_r$.

        事实上,根据相抵标准型,一定存在非异阵$\bm{P}$、$\bm{Q}$使得
        \[
            \bm{A}=\bm{P}\begin{pmatrix}
                \bm{I}_r & \bm{O} \\
                \bm{O}   & \bm{O}
            \end{pmatrix}
        \]
        则取
        \[
            \bm{A}_i=\bm{P}\begin{bmatrix}
                0 &        &   &        &   \\
                  & \ddots &   &        &   \\
                  &        & 1 &        &   \\
                  &        &   & \ddots &   \\
                  &        &   &        & 0
            \end{bmatrix}\bm{Q}\left(1\leqslant i \leqslant r\right)
        \]
        即可(其中$1$在主对角线第$i$位).
    \end{proof}
}
\exa{}{}{
    设$\bm{\varphi}\in \call  \left(V^n\right)$,且$\bm{\varphi}^m=\bm{0}$,$n=mq+1$,证明:
    \[\rmr \left(\bm{\varphi}\right)\leqslant n - q - 1\]\begin{proof}
        同样从代数角度看:设$\bm{A}\in M_{n}\left(\mathbb{K}\right)$,且$\bm{A}^m=\bm{O}$,$n=mq+1$,证明:
        \[\rmr \left(\bm{A}\right)\leqslant n-q-1\]

        采用反证法,设$\rmr \left(\bm{A}\right)\geqslant n-q$,由Sylvesler不等式知:
        \begin{align*}
            0 & =\rmr \left(\bm{A}^m\right)                    \\
              & =\rmr \left(\bm{AA}^{m-1}\right)               \\
              & \geqslant \rmr \left(\bm{A}\right)+\rmr
            \left(\bm{A}^{m-1}\right)-n                        \\
              & \geqslant \rmr \left(\bm{A}^{m-1}\right)-q     \\
              & \Longrightarrow \rmr \left(\bm{A}^{m-1}\right)
            \leqslant q
        \end{align*}
        不断做下去即有\[
            \left(m-1\right)q\geqslant \rmr \left(\bm{A}\right)\geqslant n-q = \left(m-1\right)q+1
        \]
        矛盾,故假设不成立,证毕.
    \end{proof}
}
\exa{像空间与核空间稳定}{像空间与核空间稳定}{
    设$\bm{A}\in M_n\left(\mathbb{K}\right)$,求证:$\bm{A}$的幂次超过$n$时其秩的值将稳定即
    \[
        \rmr \left(\bm{A}^n\right)=\rmr \left(
        \bm{A}^{n+1}
        \right)=\cdots
    \]\begin{proof}
        我们从几何角度给出一个更强的结论:设$\bm{\varphi}\in \call
            \left(V^n\right)$,则$\exists m \in \left[
                0,n
                \right]$,使得
        \begin{align*}
             & \Image\bm{\varphi}^m=\Image\bm{\varphi}^{m+1}=\cdots \\
             & \Ker \bm{\varphi}^m=\Ker \bm{\varphi}^{m+1}=\cdots
        \end{align*}
        且\[
            V=\Ker \bm{\varphi}^m\oplus \Image\bm{\varphi}^m
        \]

        首先因为
        \begin{align*}
             & V\supseteq \Image\bm{\varphi}\supseteq \Image\bm{\varphi}^2\supseteq \cdots \\
             & \Ker \bm{\varphi}\subseteq \Ker \bm{\varphi}^2\subseteq \cdots\subseteq V
        \end{align*}
        取定$n$,有
        \[
            V\supseteq \Image\bm{\varphi}\supseteq \Image\bm{\varphi}^2\supseteq \cdots \supseteq \Image\bm{\varphi}^n\supseteq \Image\bm{\varphi}^{n+1}
        \]
        它们都是$V$的子空间,故
        \[
            n = \dim V\geqslant \dim
            \Image\bm{\varphi}\geqslant\dim
            \Image\bm{\varphi}^2\geqslant
            \cdots \geqslant\dim
            \Image\bm{\varphi}^n\geqslant\dim
            \Image\bm{\varphi}^{n+1}
        \]
        $n+2$个空间的维数分布于$n+1$个抽屉中,由抽屉原理知,$\exists m \in \left[0,n\right]$,使得
        \[
            \dim \Image\bm{\varphi}^m=\dim \Image\bm{\varphi}^{m+1}
        \]
        又因为$\Image\bm{\varphi}^{m+1}$是$\Image\bm{\varphi}^m$的子空间,于是有$\Image\bm{\varphi}^m=\Image\bm{\varphi}^{m+1}.$

        再来证明:$\forall k \geqslant m,\Image\bm{\varphi}^k=\Image\bm{\varphi}^{k+1}$.因为
        \[
            \Image\bm{\varphi}^{k+1}\subseteq \Image\bm{\varphi}^k
        \]
        任取$\bm{\varphi}^k\left(\bm{v}\right)\in \Image
            \bm{\varphi}^k$.首先注意到
        $\bm{\varphi}^m\left(\bm{v}\right)\in \Image\bm{\varphi}^m = \Image\bm{\varphi}^{m+1}$,于是
        $\exists \bm{u}\in V$,使得\[
            \bm{\varphi}^m\left(\bm{v}\right)=\bm{\varphi}^{m+1}\left(\bm{u}\right)
        \]
        因为$k \geqslant m$,故
        \[
            \Image\bm{\varphi}^k\ni
            \bm{\varphi}^k\left(\bm{v}\right)=
            \bm{\varphi}^{k-m}\left(\bm{\varphi}^m\left(\bm{v}\right)\right)=
            \bm{\varphi}^{m+1}\left(\bm{v}\right)\in \Image\bm{\varphi}^{k+1}
        \]
        因为任意性,得证.

        考察核空间则简单了许多,由维数公式
        \[
            \dim \Ker \bm{\varphi}^k + \dim
            \Image\bm{\varphi}^k = \dim V =n
        \]
        立刻得证.

        维数的稳定证毕,下面证明直和.

        接下来先证明
        $\Image\bm{\varphi}^m
            \bigcap \Ker \bm{\varphi}^m = 0$(零子空间).任取
        $\bm{\alpha}\in \Image\bm{\varphi}^m\bigcap \Ker \bm{\varphi}^m$即
        \[
            \bm{\varphi}^m\left(\bm{\alpha}\right)=\bm{0},
            \bm{\alpha}=\bm{\varphi}^m\left(\bm{\beta}\right)
        \]
        于是$\bm{\varphi}^{2m}\left(\bm{\beta}\right)=
            \bm{\varphi}^m\left(\bm{\alpha}\right)=\bm{0}$即$\bm{\beta}\in
            \Ker \bm{\varphi}^{2m}=\Ker \bm{\varphi}^m$,故
        $\bm{\alpha}=\bm{\varphi}^m\left(\bm{\beta}\right)=\bm{0}$.

        然后证明$V=\Image\bm{\varphi}^m + \Ker \bm{\varphi}^m$(可以用维数公式证明).
        任取
        $\bm{\alpha}\in V$,因$\bm{\varphi}^m\left(
            \bm{\alpha}
            \right)\in \Image
            \bm{\varphi}^m=\Image
            \bm{\varphi}^{2m}$,故$\exists \bm{\beta}\in V$,使得$\bm{\varphi}^{m}\left(\bm{\alpha}
            \right)=\bm{\varphi}^{2m}\left(\bm{\beta}
            \right)$,则由线性映射的性质有
        $\bm{\varphi}^m\left(\bm{\alpha}-\bm{\varphi}^m\left(\bm{\beta}
                \right)\right)=\bm{0}$,即
        $\bm{\gamma}=\bm{\alpha}-\bm{\varphi}^m\left(\bm{\beta}\right)\in \Ker \bm{\varphi}^m$,从而有
        \[
            \bm{\alpha}=\bm{\gamma}+\bm{\varphi}^m\left(\bm{\beta}\right)
        \]
        由任意性得到
        \[
            V=\Ker \bm{\varphi}^m+\Image\bm{\varphi}^m
        \]
        于是有直和
        \[
            V=\Ker \bm{\varphi}^m\oplus\Image\bm{\varphi}^m
        \]
        综合以上证明过程,我们证明了一个更强的结论.
    \end{proof}
}
\exa{}{}{
    我们给出线性映射维数公式的一种不分别证明像空间和核空间维数的证明方法(实际上我们分别证明了\cref{thm:像空间与核空间的维数公式}像空间和核空间维数而将\cref{cor:线性映射维数公式}维数公式作为推论给出).
    \begin{proof}
        设$\dim \Ker \bm{\varphi}=k$,只需证$\dim \Image\bm{\varphi}=n-k$.

        首先任取$\Ker \bm{\varphi}$的一组基$\left\{
            \bm{e}_1,\bm{e}_2,\cdots,\bm{e}_k
            \right\}$,扩张为$V$的一组基\[\left\{
            \bm{e}_1,\bm{e}_2,\cdots,\bm{e}_k,\bm{e}_{k+1},\cdots,\bm{e}_n
            \right\}\]
        接下来$\forall \bm{\alpha}\in V$,设$\displaystyle
            \bm{\alpha}=\sum_{i=1}^n\lambda_i\bm{e}_i$,注意到\[
            \bm{\varphi}\left(\bm{e}_j\right)=\bm{0}\left(\forall 1\leqslant j\leqslant k\right)
        \]
        于是
        \begin{align*}
            \bm{\varphi}\left(\bm{\alpha}\right) & =\sum_{i=1}^{n}\lambda_i\bm{\varphi}\left(\bm{e}_i\right) \\
                                                 & =\lambda_{k+1}\bm{\varphi}\left(\bm{e}_{k+1}\right)+
            \lambda_{k+2}\bm{\varphi}\left(\bm{e}_{k+2}\right)+\cdots+
            \lambda_n\bm{\varphi}\left(\bm{e}_n\right)
        \end{align*}
        即\[
            \Image\bm{\varphi}=L\left(
            \bm{\varphi}\left(\bm{e}_{k+1}\right),
            \bm{\varphi}\left(\bm{e}_{k+2}\right),
            \cdots,
            \bm{\varphi}\left(\bm{e}_{n}\right)
            \right)
        \]
        于是只需要证明$\bm{\varphi}\left(\bm{e}_{k+1}\right),
            \bm{\varphi}\left(\bm{e}_{k+2}\right),
            \cdots,
            \bm{\varphi}\left(\bm{e}_{n}\right)$线性无关即可.

        设
        \[
            c_{k+1}\bm{\varphi}\left(\bm{e}_{k+1}\right)+c_{k+2}
            \bm{\varphi}\left(\bm{e}_{k+2}\right)+
            \cdots+c_n
            \bm{\varphi}\left(\bm{e}_{n}\right)=\bm{0}
        \]
        即
        \[
            \bm{\varphi}\left(
            c_{k+1}\bm{e}_{k+1}+c_{k+2}\bm{e}_{k+2}+\cdots
            +c_n\bm{e}_n
            \right)=\bm{0}
        \]
        于是
        \[
            c_{k+1}\bm{e}_{k+1}+c_{k+2}\bm{e}_{k+2}+\cdots
            +c_n\bm{e}_n\in \Ker \bm{\varphi}
        \]
        因为$\Ker \bm{\varphi}$的一组基为$\left\{
            \bm{e}_1,\bm{e}_2,\cdots,\bm{e}_k
            \right\}$,故得到(符号只是为了方便说明)
        \[
            c_{k+1}\bm{e}_{k+1}+c_{k+2}\bm{e}_{k+2}+\cdots
            +c_n\bm{e}_n=-c_1\bm{e}_1-c_2\bm{e}_2-\cdots-c_k\bm{e}_k
        \]
        即
        \[
            c_1\bm{e}_1+c_2\bm{e}_2+\cdots
            +c_k\bm{e}_k+c_{k+1}\bm{e}_{k+1}+c_{k+2}\bm{e}_{k+2}+\cdots
            +c_n\bm{e}_n=\bm{0}
        \]
        由于$\bm{e}_1,\bm{e}_2,\cdots,\bm{e}_k,\bm{e}_{k+1},\bm{e}_{k+2},\cdots,
            \bm{e}_n$是一组基,那么
        \[
            c_1=c_2=\cdots=c_{k}=c_{k+1}=c_{k+2}=\cdots=c_n=0\qedhere
        \]
    \end{proof}
}
\exa{}{}{
    设$\bm{A}$、$\bm{B}\in M_{m\times n}\left(\mathbb{K}\right)$,证明:$\bm{Ax}=\bm{0}$与$\bm{Bx}=\bm{0}$同解等价于存在$m$阶非异阵$\bm{P}$使得$\bm{B}=\bm{PA}$\begin{proof}
        即证:已知$\bm{\varphi}$、$\bm{\psi}\in \call  \left(V^n,U^m\right)$,那么$\Ker \bm{\varphi}=\Ker \bm{\psi}$等价于
        存在自同构$\bm{\xi}\in \call
            \left(U\right)$,\st
        $\bm{\psi}=\bm{\xi}\circ \bm{\varphi}$.
        下面考虑必要性,我们用下面的图说明这一思想过程,以三维欧式空间为例.

        \begin{center}
            \incfig[scale=1.0,float=H]{AlgebraAndGeometrySolution}
        \end{center}

        其中,自同构$\bm{\xi}$只需要保证基的变化,即将一组基映射到另一组基来保证自同构$\bm{\xi}$是线性同构.
        \begin{align*}
            \bm{\xi}:\bm{\varphi}\left(\bm{e}_1\right) & \longmapsto \bm{\psi}\left(\bm{e}_1\right) \\
            \bm{\varphi}\left(\bm{e}_2\right)          & \longmapsto \bm{\psi}\left(\bm{e}_2\right) \\
            \bm{f}_3                                   & \longmapsto \bm{g}_3
        \end{align*}
        即满足\[
            \bm{\psi}\left(\bm{e}_i\right)=\bm{\xi}\bm{\varphi}\left(\bm{e}_i\right)\left(1\leqslant i\leqslant 3\right)
        \]
        的$\bm{\xi}$为所求.
    \end{proof}
}