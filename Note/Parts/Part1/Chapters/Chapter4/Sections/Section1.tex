\section{线性映射}
\subsection{映射}
\exa{}{}{
    设\begin{align*}
        det:M_n\left(\bbr \right) & \longrightarrow \bbr            \\
        \bm{A}                    & \longmapsto \left|\bm{A}\right|
    \end{align*}
    是满射,$n\geqslant 2$是不是单射.
}
\dfn{恒等映射}{恒等映射}{
    如果一个变换是一个将任一元素变换为自身的双射,称为一个恒等映射,记作$\bm{1}_A$或$\bm{I}_A.$
}
\rem{恒等映射与变换}{}{
    恒等映射是特别的变换,后者只是变换到同一集合,前者在双射的基础上还要强调任一元素均变换到自身,这比变换的要求严格得多.

    事实上,在变换的基础上,如果任一元素均映射到自身,就可以确定这是双射了.因此上面的定义其实是有重合的.
}
\lem{逆映射的存在性}{逆映射的存在性}{
    $\bm{f} : A \longrightarrow B$的逆映射存在当且仅当$\bm{f}$为双射.\begin{proof}
        先证充分性,设双射$\bm{f}$,$\forall b\in B$,由满性知$\exists a\in A$使得$b=\bm{f}\left(a\right)$,由单性知这样的$a$唯一.构造
        \begin{align*}
            \bm{g}:B  & \longrightarrow A \\
            \forall b & \longmapsto a
        \end{align*}
        此处$a$是唯一满足$\bm{f}\left(a\right)=b$的$a$.
        这是一个映射且
        \[
            \begin{cases*}
                \bm{g}\circ\bm{f}=\bm{1}_A \\
                \bm{f}\circ\bm{g}=\bm{1}_B
            \end{cases*}
        \]
        于是这是符合要求的逆映射.

        再证必要性,即有$\bm{g}=\bm{f}^{-1}$.先证明单性,设$a_1,a_2\in A$使得$\bm{f}\left(a_1\right)=\bm{f}\left(a_2\right)$,两边复合$\bm{g}$即得$a_1=a_2.$

        考虑满性,$\forall b\in B$,因为
        \[
            \bm{f}\left(\bm{g}\left(b\right)\right)=b
        \]
        故$\bm{g}\left(b\right)$是$b$的原像.
    \end{proof}
}
\subsection{线性映射}
\dfn{线性映射}{线性映射}{
    设$\bbk $上的线性空间$V$、$U$,映射\[
        \bm{\varphi }:V \longrightarrow
        U
    \]

    若$\forall \bm{\alpha}$、$\bm{\beta} \in V$,$\forall k \in \bbk $
    \begin{enumerate}[label=\arabic*)]
        \item $\bm{\varphi}\left(
                  \bm{\alpha}+\bm{\beta}
                  \right)=\bm{\varphi}\left(
                  \bm{\alpha}
                  \right)   +
                  \bm{\varphi}\left(
                  \bm{\beta}
                  \right)$
        \item $\bm{\varphi}\left(k\bm{\alpha}
                  \right)=k\bm{\varphi}\left(
                  \bm{\alpha}
                  \right)$
    \end{enumerate}
    则称为一个线性映射.

    特别地,$\bm{\varphi}:V
        \longrightarrow V$
    称为线性变换.

    当$\bm{\varphi}$作为映射是单射或满射时,
    称为是单或满线性映射.

    当$\bm{\varphi}$
    作为映射是双射时,称为(线性)同构
    (与线性空间之间的同构的定义\cref{def:线性同构}相同),
    记作$V\cong U$.

    $V=U$时$V$自身上的同构称为自同构.
}
\rem{}{}{
    \cref{def:线性同构}定义线性空间间的同构时,我们先要求双射,再要求保持线性,\cref{def:线性映射}反过来先要求线性再要求双射.两种定义是相同的.
}
\exa{}{}{
    \[
        \bm{\varphi}:
        \forall \bm{\alpha}\in V
        \longmapsto \bm{0}_U
    \]
    这是一个零线性映射$\bm{0}:V\longrightarrow U.$
}
\exa{}{}{
    线性空间$V$上的线性变换
    \begin{align*}
        \bm{1}_V:
        V           & \longrightarrow V \\
        \bm{\alpha} & \longmapsto
        \bm{\alpha}
    \end{align*}
    称为一个恒等映射或者恒等变换,也是一个自同构.记作$\bm{1}_V$或$\bm{I}_V$或$\mathbf{Id}_V$或$\bm{I}$.
}
\exa{}{}{
    设线性空间$\displaystyle
        V = \bbk ^n$,$\displaystyle
        U = \bbk ^m$,$\bm{A}
        \in M_{m \times n}\left(
        \bbk
        \right)$,定义
    \begin{align*}
        \bm{\varphi}_{\bm{A}}
        : \bbk ^n   &
        \longrightarrow \bbk ^m   \\
        \bm{\alpha} & \longmapsto
        \bm{A\alpha}
    \end{align*}
    是一个线性映射.
}
\exa{}{}{
    \begin{align*}
        \bm{\varphi}:
        \bbk _n     &
        \longrightarrow
        \bbk ^n       \\
        \left(a_1,a_2,\cdots
        ,a_n\right) &
        \longmapsto
        \left(
        a_1,a_2,\cdots,a_n
        \right)'
    \end{align*}
    是线性同构.
}
\exa{}{}{
    设$V_{\bbk }$的一组基为
    $\left\{
        \bm{e}_1,\bm{e}_2,\cdots,\bm{e}_n
        \right\}$,则
    \begin{align*}
        \bm{\varphi}:
        V & \longrightarrow \bbk ^n \\
        \bm{\alpha} =
        \sum_{i = 1}^{n}a_i\bm{e}_i
          & \longmapsto\left(
        a_1,a_2,\cdots,a_n
        \right)'
    \end{align*}
    是一个线性同构.
}
\exa{}{}{
    设$\bbk $上的线性空间$V$,取定$k \in \bbk $,则\begin{align*}
        \bm{\varphi}:
        V           & \longrightarrow V \\
        \bm{\alpha} & \longmapsto
        k\bm{\alpha}
    \end{align*}
    这样一个线性变换称为数量/数乘变换.
}
\exa{}{}{
    设$\left[0,1\right]$上的实无穷次可微函数
    全体组成的实线性空间为$\displaystyle
        C^{\infty}\left[0,1\right]$,则有求导变换
    \begin{align*}
        \bm{\varphi}:
        C^{\infty}\left[0,1\right] &
        \longrightarrow
        C^{\infty}\left[0,1\right]               \\
        f\left(x\right)
                                   & \longmapsto
        \frac{\rmd }{\rmd x}
        f\left(x\right) = f'\left(x\right)
    \end{align*}
    是$\displaystyle
        C^{\infty}\left[0,1\right]$
    上的线性变换.
}
\pro{线性映射的若干基础性质}{线性映射的若干基础性质}{
    设线性映射$\bm{\varphi}:
        V \longrightarrow U$,则\begin{enumerate}[label=\arabic*)]
        \item $\bm{\varphi}\left(\bm{0}_V
                  \right)=\bm{0}_U$
        \item $\forall \bm{\alpha},\bm{\beta}
                  \in V,k,l \in \bbk ,$
              则\[\bm{\varphi}\left(
                  k\bm{\alpha} + l\bm{\beta}
                  \right)=k\bm{\varphi}\left(
                  \bm{\alpha}
                  \right)+l\bm{\varphi}\left(
                  \bm{\beta}
                  \right)\]即保持线性组合
        \item 若$\bm{\varphi}$为同构,则$\bm{\varphi}^{-1}$也是同构
        \item 若$\bm{\varphi}:V\longrightarrow U$和$\bm{\psi}:U \longrightarrow
                  W$均为线性映射或同构,则$\bm{\psi}
                  \circ \bm{\varphi}:
                  V \longrightarrow W$是一个线性映射或同构
    \end{enumerate}\begin{proof}
        \begin{enumerate}[label=\arabic*)]
            \item \[
                      \bm{\varphi}\left(\bm{0}_V\right)=
                      \bm{\varphi}\left(\bm{0}_V+\bm{0}_V\right)
                      =\bm{\varphi}\left(\bm{0}_V\right)+\bm{\varphi}\left(\bm{0}_V\right)
                      =\bm{0}_U
                  \]
            \item 拆分即可.
            \item 因为$\bm{\varphi}$为双射,则$\bm{\varphi}^{-1}$为双射,只要证
                  $\bm{\varphi}^{-1}$保持线性组合即可.任取$\bm{x},\bm{y}\in U,k,l\in\bbk $,只需证
                  \[
                      \bm{\varphi}^{-1}\left(k\bm{x}+l\bm{y}\right)=
                      k\bm{\varphi}^{-1}\left(\bm{x}\right)+l
                      \bm{\varphi}^{-1}\left(
                      \bm{y}\right)
                  \]
                  作
                  \begin{align*}
                      \bm{\varphi}\left(
                      \bm{\varphi}^{-1}\left(
                          k\bm{x}+l\bm{y}\right)-
                      k\bm{\varphi}^{-1}\left(\bm{x}\right)+l
                      \bm{\varphi}^{-1}\left(
                          \bm{y}\right)
                      \right)=k\bm{x}+l\bm{y}-k\bm{x}-l\bm{y}=\bm{0}_U=\bm{\varphi}\left(\bm{0}_
                      V\right)
                  \end{align*}
                  于是由单性即得.

            \item 首先$\bm{\psi}\circ \bm{\varphi}$显然为双射,那么
                  \begin{align*}
                      \bm{\psi}\circ \bm{\varphi}\left(k\bm{\alpha}+l\bm{\beta}\right) & =\bm{\psi}\left(k\bm{\varphi}\left(\bm{\alpha}\right)+l\bm{\varphi}\left(\bm{\beta}\right)\right)               \\
                                                                                       & =k\bm{\psi}\circ\bm{\varphi}\left(\bm{\alpha}\right)+l\bm{\psi}\circ\bm{\varphi}\left(\bm{\beta}\right)\qedhere
                  \end{align*}
        \end{enumerate}
    \end{proof}
}
\lem{线性同构的存在性}{线性同构的存在性}{
    事实上,应该说两个线性空间存在线性同构.
    \begin{enumerate}[label=\arabic*)]
        \item 线性同构是一种等价关系
        \item \label{temp:2}两个线性空间存在同构当且仅当
              它们的维数相同
    \end{enumerate}\begin{proof}
        证明\ref{temp:2},先证必要性,设线性同构$\bm{\varphi}:
            V\longrightarrow U$,任取$V$的基$\left\{\bm{e}_1,\bm{e}_2,\cdots,\bm{e}_n\right\}$,从而
        $\left\{\bm{\varphi}\left(\bm{e}_1\right),
            \bm{\varphi}\left(\bm{e}_2\right),\cdots,\bm{\varphi}\left(\bm{e}_n\right)\right\}$
        线性无关.

        任取$\bm{x}\in U,\exists \bm{\alpha}\in V$\st
        $\bm{\varphi}\left(\bm{\alpha}\right)=\bm{x}.$
        设$\bm{\alpha}=a_1\bm{e}_1+a_2\bm{e}_2+\cdots
            +a_n\bm{e}_n$,那么
        \[
            \bm{x}=a_1\bm{\varphi}\left(\bm{e}_1\right)+
            a_2\bm{\varphi}\left(\bm{e}_2\right)+\cdots+
            a_n\bm{\varphi}\left(\bm{e}_n\right)
        \]
        于是$\left\{\bm{\varphi}\left(\bm{e}_1\right),
            \bm{\varphi}\left(\bm{e}_2\right),\cdots,\bm{\varphi}\left(\bm{e}_n\right)\right\}$
        是$U$的一组基.即$\dim V=\dim U=n.$

        然后是充分性,设$\dim V=\dim U=n$,取$V$的一组基
        $\left\{\bm{e}_1,\bm{e}_2,\cdots,\bm{e}_n\right\}$和
        $U$的一组基$\left\{\bm{f}_1,\bm{f}_2,\cdots,\bm{f}_n\right\}$.
        构造线性同构:
        \begin{align*}
            \bm{\varphi}_V:
            V & \longrightarrow \bbk ^n    \\
            \bm{\alpha} =
            \sum_{i = 1}^{n}a_i\bm{e}_i
              & \longmapsto \begin{pmatrix}
                                a_1    \\
                                a_2    \\
                                \vdots \\
                                a_n
                            \end{pmatrix}
        \end{align*}
        和
        \begin{align*}
            \bm{\varphi}_U:
            U & \longrightarrow \bbk ^n    \\
            \bm{x} =
            \sum_{i = 1}^{n}b_i\bm{f}_i
              & \longmapsto \begin{pmatrix}
                                b_1    \\
                                b_2    \\
                                \vdots \\
                                b_n
                            \end{pmatrix}
        \end{align*}
        考虑$\bm{\varphi}_U^{-1}\circ \bm{\varphi}_V$即是线性同构.
    \end{proof}
}
\rem{}{}{
    两个线性空间存在同构时,将存在无数个线性同构,但这并不意味着这两个线性空间之间的任一线性映射均是线性同构.即存在无穷多个并不等于任取一个均满足.
}
\exa{}{}{
    考虑$\bm{I}:V\longrightarrow V$为线性同构,但是零映射$\bm{0}:V\longrightarrow V$不仅不是线性同构,甚至都不是单射.
}
\rem{}{}{
    在表达两个线性空间之间的线性同构时,实现这个同构的映射更加关键.

    不同数域上的线性空间之间的映射不是线性映射.
}