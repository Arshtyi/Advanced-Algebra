\newpage
\section{复正规算子}
\subsection{复正规算子}
设$V$为酉空间,$\bm{\varphi}\in\call \left(V\right)$,$\left\{
    \bm{e}_1,\bm{e}_2,\cdots,\bm{e}_n
    \right\}$使得$\bm{\varphi}$在这组基下的表示矩阵为复对角阵\[\bm{\varLambda}=\diag \,\left\{
    \lambda_1,\lambda_2,\cdots,\lambda_n
    \right\}\]来考虑$\bm{\varphi}$的性质.首先\[
    \left(
    \bm{\varphi}\left(
        \bm{e}_1
        \right),\bm{\varphi}\left(
        \bm{e}_2
        \right),\cdots,\bm{\varphi}\left(
        \bm{e}_n
        \right)\right)=\left(
    \bm{e}_1,\bm{e}_2,\cdots,\bm{e}_n
    \right)\bm{\varLambda}
\]即$\forall 1\leqslant i\leqslant n:$\[
    \bm{\varphi}\left(
    \bm{e}_i
    \right)=\lambda_i\bm{e}_i
\]同时$\bm{\varphi}^*$在这组基下的表示矩阵为$\overline{\bm{\varLambda}}'=\diag \,\left\{
    \overline{\lambda}_1,\overline{\lambda}_2,\cdots,\overline{\lambda}_n
    \right\}$且$\forall 1\leqslant i\leqslant n:$\[
    \bm{\varphi}^*\left(
    \bm{e}_i
    \right)=\overline{\lambda}_i\bm{e}_i
\]容易知道\[
    \bm{\varphi}\bm{\varphi}^*\left(
    \bm{e}_i
    \right)=\bm{\varphi}^*\bm{\varphi}\left(
    \bm{e}_i
    \right)\Longrightarrow\bm{\varphi\varphi}^*=\bm{\varphi}^*\bm{\varphi}
\]且\[
    \overline{\bm{A}}'\bm{A}=\bm{A}\overline{\bm{A}}'
\]
\dfn{正规算子}{正规算子}{
    设有限维内积空间$V$,线性算子$\bm{\varphi}\in\call \left(V\right)$.若$\bm{\varphi\varphi}^*=\bm{\varphi}^*\bm{\varphi}$,则称$\bm{\varphi}$是$V$上的正规算子.
}
\dfn{正规矩阵}{正规矩阵}{
    设$\bm{A}\in M_n\left(\bbr \right)$,若$\bm{A}'\bm{A}=\bm{AA}'$,则称$\bm{A}$是一个实正规矩阵;若$\bm{A}\in M_n\left(\bbc \right)$,若$\overline{\bm{A}}'\bm{A}=\bm{A}\overline{\bm{A}}'$,则称$\bm{A}$是一个复正规矩阵.
}
\exa{}{}{
    自伴随算子$\bm{\varphi}^*=\bm{\varphi}$一定是正规算子.
}
\exa{}{}{
    正交算子和酉算子$\bm{\varphi}^*=\bm{\varphi}^{-1}$一定是正规算子.
}
\exa{}{}{
    Hermite阵、酉阵都是复正规矩阵;实对称阵、正交阵都是实正规矩阵.
}
\lem{}{正规算子与正规阵}{
    设$V$为欧式空间(酉空间),$\bm{\varphi}\in\call \left(V\right)$,则$\bm{\varphi}$是实(复)正规算子当且仅当$\bm{\varphi}$在任一组(或者某一组)标准正交基下的表示矩阵是实(复)正规矩阵.\begin{proof}
        取一组标准正交基$\left\{
            \bm{e}_1,\bm{e}_2,\cdots,\bm{e}_n
            \right\}$,设$\bm{\varphi}$的表示矩阵为$\bm{A}$,根据\cref{thm:伴随算子的表示矩阵},$\bm{\varphi}^*$的表示矩阵为$\bm{A}'/\overline{\bm{A}}'$.那么$\bm{\varphi}$是实(复)正规算子当且仅当$\bm{\varphi\varphi}^*=\bm{\varphi}^*\bm{\varphi}$,即$\bm{\varphi\varphi}^*,\bm{\varphi}^*\bm{\varphi}$在任一组(或者某一组)标准正交基下的表示矩阵相等,即\[\bm{AA}'=\bm{A}'\bm{A}\left(\bm{A}\overline{\bm{A}}'=
            \overline{\bm{A}}'\bm{A}
            \right)\]于是$\bm{A}$是实(复)正规矩阵.
    \end{proof}
}
\lem{正规算子的范数、特征值、特征向量}{正规算子的范数、特征值、特征向量}{
    设$\bm{\varphi}$是酉空间$V$上的正规算子,则\begin{enumerate}[label=\arabic*)]
        \item $\forall\bm{\alpha}\in V:\left\lVert\bm{\varphi}\left(\bm{\alpha}\right)\right\rVert=\left\lVert\bm{\varphi}^*\left(\bm{\alpha}\right)\right\rVert$
        \item $\bm{\alpha}$是$\bm{\varphi}$关于特征值$\lambda$的特征向量当且仅当$\bm{\alpha}$是$\bm{\varphi}^*$关于特征值$\overline{\lambda}$的特征向量
        \item 属于$\bm{\varphi}$的不同特征值的特征向量相互正交
    \end{enumerate}\begin{proof}
        $(1)$考虑\begin{align*}
            \left(
            \bm{\varphi}\left(\bm{\alpha}\right),\bm{\varphi}\left(\bm{\alpha}\right)
            \right) & =\left(
            \bm{\alpha},\bm{\varphi}^*\bm{\varphi}\left(\bm{\alpha}\right)
            \right)           \\
                    & =\left(
            \bm{\alpha},\bm{\varphi}\bm{\varphi}^*\left(\bm{\alpha}\right)
            \right)           \\
                    & =\left(
            \bm{\varphi}^*\left(\bm{\alpha}\right),\bm{\varphi}^*\left(\bm{\alpha}\right)
            \right)
        \end{align*}

        $(2)$先证明$\lambda\bm{I}-\bm{\varphi}$也是正规算子.根据\cref{prop:伴随算子的运算},$\left(
            \lambda\bm{I}-\bm{\varphi}
            \right)^*=\overline{\lambda}\bm{I}-\bm{\varphi}^*.$由$(1)$知$\forall\bm{\alpha}\in V$\[
            \left\lVert\left(
            \lambda\bm{I}-\bm{\varphi}\right)\left(\bm{\alpha}\right)\right\rVert=\left\lVert\left(
            \overline{\lambda}\bm{I}-\bm{\varphi}^*\right)\left(\bm{\alpha}\right)\right\rVert
        \]设$\bm{0}\neq\bm{\alpha}\in V$是$\bm{\varphi}$关于特征值$\lambda$的特征向量\[\bm{\varphi}\left(\bm{\alpha}\right)=\lambda\bm{\alpha}\Longleftrightarrow\left(
            \lambda\bm{I}-\bm{\varphi}
            \right)\left(\bm{\alpha}\right)=\bm{0}=\left(
            \overline{\lambda}\bm{I}-\bm{\varphi}^*
            \right)\left(
            \bm{\alpha}
            \right)=\bm{0}\Longleftrightarrow\bm{\varphi}^*\left(\bm{\alpha}\right)=\overline{\lambda}\bm{\alpha}\]

        $(3)$设$\lambda\neq\mu:\bm{\varphi}\left(\bm{\alpha}\right)=\lambda\bm{\alpha},\bm{\varphi}\left(\bm{\beta}\right)=\mu\bm{\beta}\Longleftrightarrow\bm{\varphi}^*\left(\bm{\beta}\right)=\overline{\mu}\bm{\beta}$,作\begin{align*}
            \lambda\left(
            \bm{\alpha},\bm{\beta}
            \right) & =\left(
            \bm{\varphi}\left(\bm{\alpha}\right),\bm{\beta}
            \right)              \\
                    & =\left(
            \bm{\alpha},\bm{\varphi}^*\left(\bm{\beta}\right)
            \right)              \\
                    & =\left(
            \bm{\alpha},\overline{\mu}\bm{\beta}
            \right)              \\
                    & =\mu\left(
            \bm{\alpha},\bm{\beta}
            \right)
        \end{align*}于是$\left(
            \bm{\alpha},\bm{\beta}
            \right)=0.$
    \end{proof}
}
\subsection{Schur定理}
\lem{Schur定理}{Schur定理}{
    设$\bm{\varphi}$是酉空间$V$上的线性算子,则一定存在一组标准正交基,使得$\bm{\varphi}$在这组基下的表示矩阵是上三角阵.\begin{proof}
        对维数$n=\dim V$进行归纳,$n=1$显然,设$\dim V<n$成立,证明$\dim V=n$的情形.

        取$\bm{\varphi}^*$的一个特征值$\lambda$及对应的特征向量$\bm{\alpha}$,单位化$\displaystyle\bm{e}_n=\frac{\bm{\alpha}}{\left\lVert\bm{\alpha}\right\rVert}:\bm{\varphi}^*\left(\bm{e}_n\right)=\lambda\bm{e}_n$,令$U=L\left(\bm{e}_n\right)^{\perp}:\dim U=n-1.$因为$L\left(\bm{e}_n\right)$是$\bm{\varphi}^*$-不变子空间,故$U$是$\left(
            \bm{\varphi}^*
            \right)^*=\bm{\varphi}$-不变子空间.做限制$\bm{\varphi}\mid_U$,根据归纳假设,存在$U$的一组标准正交基$\left\{
            \bm{e}_1,\bm{e}_2,\cdots,\bm{e}_{n-1}
            \right\}$,使得$\bm{\varphi}\mid_U$在这组基下的表示矩阵是上三角阵,于是得到$\left\{
            \bm{e}_1,\bm{e}_2,\cdots,\bm{e}_{n-1},\bm{e}_n
            \right\}$是$V$的一组标准正交基,并且\[
            \left(
            \bm{\varphi}\left(
                \bm{e}_1
                \right),\bm{\varphi}\left(
                \bm{e}_2
                \right),\cdots,\bm{\varphi}\left(
                \bm{e}_n
                \right)
            \right)=\left(
            \bm{e}_1,\bm{e}_2,\cdots,\bm{e}_n
            \right)\begin{pmatrix}
                a_{11} & *      & *      & *           & *      \\
                       & a_{22} & *      & *           & *      \\
                       &        & \ddots & *           & *      \\
                       &        &        & a_{n-1,n-1} & *      \\
                       &        &        &             & a_{nn}
            \end{pmatrix}\qedhere
        \]
    \end{proof}
}
\cor{Schur定理}{Schur定理}{
    设$\bm{A}\in M_n\left(\bbc \right)$,则一定存在酉阵$\bm{P}$,使得$\overline{\bm{P}}'\bm{A}\bm{P}$是上三角阵.
}
\clm{}{}{
    \cref{cor:Schur定理}是\cref{thm:必定复相似于三角阵},\cref{lem:Schur定理}的推广.
}
\thm{}{酉空间算子正规判定}{
    设酉空间$V$上的线性算子$\bm{\varphi}$,则$\bm{\varphi}$正规当且仅当$\bm{\varphi}$在某一组标准正交基下的表示矩阵是复对角阵.\begin{proof}
        充分性由\cref{lem:正规算子与正规阵}给出.下证必要性.

        由\cref{lem:Schur定理},存在一组标准正交基$\left\{
            \bm{e}_1,\bm{e}_2,\cdots,\bm{e}_n
            \right\}$使得$\bm{\varphi}$在这组基下的表示矩阵是上三角阵$\bm{A}=\left(
            a_{ij}
            \right)_{
                n\times n
            }$,$\bm{\varphi}^*$在这组基下的表示矩阵为下三角阵$\overline{\bm{A}}'$.

        由\[
            \left(
            \bm{\varphi}\left(
                \bm{e}_1
                \right),\bm{\varphi}\left(
                \bm{e}_2
                \right),\cdots,\bm{\varphi}\left(
                \bm{e}_n
                \right)
            \right)=\left(
            \bm{e}_1,\bm{e}_2,\cdots,\bm{e}_n
            \right)\begin{pmatrix}
                a_{11} & a_{12} & \cdots & a_{1n} \\
                       & a_{22} & \cdots & a_{2n} \\
                       &        & \ddots & \vdots \\
                       &        &        & a_{nn}
            \end{pmatrix}
        \]\[
            \left(
            \bm{\varphi}^*\left(
                \bm{e}_1
                \right),\bm{\varphi}^*\left(
                \bm{e}_2
                \right),\cdots,\bm{\varphi}^*\left(
                \bm{e}_n
                \right)
            \right)=\left(
            \bm{e}_1,\bm{e}_2,\cdots,\bm{e}_n
            \right)\begin{pmatrix}
                \overline{a}_{11} &                   &        &                   \\
                \overline{a}_{12} & \overline{a}_{22} &        &                   \\
                \vdots            & \vdots            & \ddots &                   \\
                \overline{a}_{1n} & \overline{a}_{2n} & \cdots & \overline{a}_{nn}
            \end{pmatrix}
        \]注意到\begin{align*}
            \bm{\varphi}\left(
            \bm{e}_1
            \right)=a_{11}\bm{e}_1\xLongleftrightarrow{\cref{lem:正规算子的范数、特征值、特征向量}}\bm{\varphi}^*\left(
            \bm{e}_1
            \right) & =\overline{a}_{11}\bm{e}_1                                                            \\
                    & =\overline{a}_{11}\bm{e}_1+\overline{a}_{12}\bm{e}_2+\cdots+\overline{a}_{1n}\bm{e}_n
        \end{align*}即\[
            a_{12}=a_{13}=\cdots=a_{1n}=0
        \]如此继续,得到$\forall 1\leqslant i<j\leqslant n:a_{ij}=0\Longrightarrow\bm{A}$是复对角阵.
    \end{proof}
}
\cor{}{正规必定对角}{
    设复正规阵$\bm{A}$,则存在酉阵$\bm{P}$使得$\overline{\bm{P}}'\bm{A}\bm{P}$是复对角阵.
}
\cor{复正规阵在酉相似下的全系不变量}{复正规阵在酉相似下的全系不变量}{
    复正规阵在酉相似下的全系不变量是其全体特征值,且其酉相似标准型即为全体特征值构成的对角阵.
}
\cor{酉阵的酉相似标准型}{酉阵的酉相似标准型}{
    设$\bm{U}$为酉阵,则存在酉阵$\bm{P}$使得\[
        \overline{\bm{P}}'\bm{UP}= \diag \,\left\{
        c_1,c_2,\cdots,c_n
        \right\}
    \]其中$\forall 1\leqslant i\leqslant n:\left|c_i\right|=1.$\begin{proof}
        由\cref{cor:正规必定对角},必定存在酉阵$\bm{P}$使得\[
            \overline{\bm{P}}'\bm{UP}= \diag \,\left\{
            c_1,c_2,\cdots,c_n
            \right\}
        \]其中$c_i\in\bbc $.这是一个酉阵,于是与其共轭转置之积为单位阵,因此$\forall1\leqslant i\leqslant n:\left|c_i\right|=1.$
    \end{proof}
}
\thm{}{特征子空间确定的正规算子}{
    设$\bm{\varphi}$是酉空间$V$上的线性算子,其全体不同特征值为$\lambda_1,\lambda_2,\cdots,\lambda_k$,对应特征子空间为$V_1,V_2,\cdots,V_k$,则$\bm{\varphi}$正规当且仅当\[
        V=V_1\perp V_2\perp\cdots\perp V_k
    \]\begin{proof}
        先考虑必要性,设$\bm{\varphi}$正规,由\cref{thm:酉空间算子正规判定},其必可对角化,于是\[
            V=V_1\oplus V_2\oplus\cdots\oplus V_k
        \]根据\cref{lem:正规算子的范数、特征值、特征向量},$\forall i\neq j:V_i\perp V_j$,于是\[
            V=V_1\perp V_2\perp\cdots\perp V_k
        \]

        下面证明充分性.不妨取$V_i$的标准正交基,拼成$V$的一组标准正交基.注意到基向量都是特征向量,于是$\bm{\varphi}$在这组基下的表示矩阵是对角阵,故$\bm{\varphi}$正规.
    \end{proof}
}
\cor{}{自伴随算子的特征子空间}{
    设$\bm{\varphi}$为欧式空间上的自伴随算子,其全体不同特征值为$\lambda_1,\lambda_2,\cdots,\lambda_k$,对应特征子空间为$V_1,V_2,\cdots,V_k$,则\[
        V=V_1\perp V_2\perp\cdots\perp V_k
    \]
}
\subsection{实对称矩阵关于特征值的估计}
设$n$阶实对称阵$\bm{A}$,将其特征值升序排列(\cref{cor:实对称阵的特征值与特征向量}告诉我们一定是实数)为:$\lambda_1\leqslant\lambda_2\leqslant\cdots\leqslant\lambda_n$,于是存在正交阵$\bm{P}$使得\[\bm{P}'\bm{AP}=\diag \,\left\{
    \lambda_1,\lambda_2,\cdots,\lambda_n
    \right\}\]做列分块$\bm{P}=\left(
    \bm{e}_1,\bm{e}_2,\cdots,\bm{e}_n
    \right)$,其中$\left\{
    \bm{e}_1,\bm{e}_2,\cdots,\bm{e}_n
    \right\}$是$\bbr ^n$在标准内积下的一组标准正交基.因为\[
    \bm{A}\left(
    \bm{e}_1,\bm{e}_2,\cdots,\bm{e}_n
    \right)=\left(
    \bm{e}_1,\bm{e}_2,\cdots,\bm{e}_n
    \right)\begin{pmatrix}
        \lambda_1 &           &        &           \\
                  & \lambda_2 &        &           \\
                  &           & \ddots &           \\
                  &           &        & \lambda_n
    \end{pmatrix}
\]即$\bm{Ae}_i=\lambda_i\bm{e}_i$.

$\forall\bm{x}=x_1\bm{e}_1+x_2\bm{e}_2+\cdots+x_n\bm{e}_n\in\bbr ^n$,作\begin{align*}
    \bm{Ax} & =\bm{A}\left(
    x_1\bm{e}_1+x_2\bm{e}_2+\cdots+x_n\bm{e}_n
    \right)                                       \\
            & =\sum_{i=1}^{n}\lambda_ix_i\bm{e}_i
\end{align*}又\begin{align*}
    \bm{x}'\bm{Ax} & =\left(
    \sum_{i=1}^{n}x_i\bm{e}_i,\sum_{i=1}^{n}\lambda_ix_i\bm{e}_i
    \right)                                        \\
                   & =\sum_{i=1}^{n}\lambda_ix_i^2
\end{align*}于是\[
    \lambda_1\bm{x}'\bm{x}=\lambda_1\sum_{i=1}^{n}x_i^2\leqslant\bm{x}'\bm{Ax}\leqslant\lambda_n\sum_{i=1}^{n}x_i^2=\lambda_n\bm{x}'\bm{x}
\]等号分别必定在$\bm{x}=\bm{e}_1,\bm{e}_n$处取得.于是\[
    \lambda_n=\max_{\forall\bm{0}\neq\bm{x}\in\bbr ^n}\frac{\bm{x}'\bm{Ax}}{\bm{x}'\bm{x}},\lambda_1=\min_{\forall\bm{0}\neq\bm{x}\in\bbr ^n}\frac{\bm{x}'\bm{Ax}}{\bm{x}'\bm{x}}
\]即\[
    \min_{\forall\bm{x}\in\bbr ^n}\frac{\bm{x}'\bm{Ax}}{\bm{x}'\bm{x}}\leqslant\lambda\leqslant\max_{\forall\bm{x}\in\bbr ^n}\frac{\bm{x}'\bm{Ax}}{\bm{x}'\bm{x}}
\]
\thm{Courant-Fischer定理/极小极大定理}{Courant-Fischer定理/极小极大定理}{
    \begin{align*}
        \lambda_i & =\min_{V_i}\max_{\forall\bm{0}\neq\bm{x}\in V_i}\frac{\bm{x}'\bm{Ax}}{\bm{x}'\bm{x}}             \\
                  & =\max_{V_{n+1-i}}\min_{\forall\bm{0}\neq\bm{x}\in V_{n+1-i}}\frac{\bm{x}'\bm{Ax}}{\bm{x}'\bm{x}}
    \end{align*}其中$V_i$为$\bbr ^n$的$i$维子空间.\begin{proof}
        先取定一个$V_i\subseteq\bbr ^n$,取$U=L\left(
            \bm{e}_i,\bm{e}_{i+1},\cdots,\bm{e}_n
            \right)\Longrightarrow\dim U=n+1-i$,则$V_i\cap U\neq 0$(否则,二者为直和,即有$\bbr ^n\supseteq V_i\oplus U$,后者维数为$n+1$维,矛盾).取$\bm{0}\neq\bm{x}_0=x_i\bm{e}_i+x_{i+1}\bm{e}_{i+1}+\cdots+x_n\bm{e}_n\in V_i\cap U$.作\begin{align*}
            \bm{x}_0'\bm{Ax}_0=\sum_{j=i}^{n}\lambda_jx_j^2,\bm{x}'_0\bm{x}_0=\sum_{j=i}^{n}x_j^2
        \end{align*}于是\[
            \lambda_i\leqslant\frac{\bm{x}_0'\bm{Ax}_0}{\bm{x}_0'\bm{x}_0}\Longrightarrow\lambda_i\leqslant\max_{\forall\bm{0}\neq\bm{x}\in V_i}\frac{\bm{x}'\bm{Ax}}{\bm{x}'\bm{x}}\Longrightarrow\lambda_i\leqslant\min_{V_i}\max_{\forall\bm{0}\neq\bm{x}\in V_i}\frac{\bm{x}'\bm{Ax}}{\bm{x}'\bm{x}}
        \]任取$W=L\left(
            \bm{e}_1,\bm{e}_2,\cdots,\bm{e}_i
            \right)$,考虑\begin{align*}
            \max_{\forall\bm{0}\neq\bm{x}=\sum\limits_{j=1}^{i}x_j\bm{e}_j\in W}\frac{\bm{x}'\bm{Ax}}{\bm{x}'\bm{x}}\leqslant\lambda_i
        \end{align*}并且等号在$\bm{x}=\bm{e}_i$处取得.于是\[
            \max_{\forall\bm{0}\neq\bm{x}=\sum\limits_{j=1}^{i}x_j\bm{e}_j\in W}\frac{\bm{x}'\bm{Ax}}{\bm{x}'\bm{x}}=\lambda_i\]因此\[
            \lambda_i=\min_{V_i}\max_{\forall\bm{0}\neq\bm{x}\in V_i}\frac{\bm{x}'\bm{Ax}}{\bm{x}'\bm{x}}\qedhere
        \]
    \end{proof}
}
\thm{Cauchy交错定理}{Cauchy交错定理}{
    设$n$阶实对称阵$\bm{A}$,特征值$\lambda_1\leqslant\lambda_2\leqslant\cdots\leqslant\lambda_n$,$\bm{A}_m$是一个$m$阶主子阵,特征值$\mu_1\leqslant\mu_2\leqslant\cdots\leqslant\mu_m$,则$\forall i:$\[
        \lambda_i\leqslant\mu_i,\lambda_{n+1-i}\geqslant\mu_{m+1-i}
    \]
}
\thm{Weyl摄动定理}{Weyl摄动定理}{
    设$n$阶实对称阵$\bm{A},\bm{B}$,特征值分别为$\lambda_1\leqslant\lambda_2\leqslant\cdots\leqslant\lambda_n,\mu_1\leqslant\mu_2\leqslant\cdots\leqslant\mu_n$,他们的和$\bm{A}+\bm{B}$也是实对称阵,其特征值为$\gamma_1\leqslant\gamma_2\leqslant\cdots\leqslant\gamma_n$,则$\forall 1\leqslant i\leqslant n:$\[
        \lambda_i+\mu_1\leqslant\gamma_i\leqslant\lambda_i+\mu_n
    \]
}