\newpage
\section{谱分解与极分解}
\subsection{谱分解}
\thm{谱分解定理}{谱分解定理}{
    设$V$为有限维内积空间,$V$上的线性算子$\bm{\varphi}\in\mathcal{L}\left(V\right)$是\[\begin{cases*}
            \text{正规算子} & ,$V$是酉空间  \\
            \text{自伴算子} & ,$V$是欧式空间
        \end{cases*}\]$\lambda_1,\lambda_2,\cdots,\lambda_k$是$\bm{\varphi}$的全体不同特征值,对应的特征子空间分别为$V_1,V_2,\cdots,V_k$,则\[
        V=V_1\perp V_2\perp\cdots\perp V_k
    \]设$\bm{E}_i$是从$V$到$V_i$的正交投影算子,则\[
        \bm{\varphi}=\lambda_1\bm{E}_1+\lambda_2\bm{E}_2+\cdots+\lambda_k\bm{E}_k
    \]这称为谱分解.\begin{proof}
        前半部分由\cref{thm:特征子空间确定的正规算子}和\cref{cor:自伴随算子的特征子空间}给出.投影算子的性质由\cref{def:投影变换}给出,并且由\cref{lem:正交投影变换的对称性}知$\bm{E}_i$都是自伴随的.

        下证$\bm{\varphi}\circ\bm{E}_i=\lambda_i\bm{E}_i$.$\forall\bm{v}\in V:\bm{E}_i\left(\bm{v}\right)\in V_i=\mathrm{Ker}\,\left(\bm{\varphi}-\lambda_i\bm{I}_V\right)$即\[
            \left(
            \bm{\varphi}-\lambda_i\bm{I}_V
            \right)\bm{E}_i\left(\bm{v}\right)=\bm{0}\Longrightarrow\bm{\varphi}\bm{E}_i\left(\bm{v}\right)=\lambda_i\bm{E}_i\left(\bm{v}\right)
        \]考虑到$\bm{\varphi}=\bm{\varphi}\bm{I}_V$得证.
    \end{proof}
}
\lem{正交投影算子的多项式表达}{正交投影算子的多项式表达}{
    记号同\cref{thm:谱分解定理},设$\displaystyle f_j\left(x\right)=\prod_{i\neq j}\frac{x-\lambda_i}{\lambda_j-\lambda_i}:f_j\left(\lambda_i\right)=\delta_{ij}$,则$\bm{E}_j=f_j\left(\bm{\varphi}\right)$.\begin{proof}
        由\cref{thm:谱分解定理},考虑\begin{align*}
            \bm{\varphi}^2=\lambda_1^2\bm{E}_1+\lambda_2^2\bm{E}_2+\cdots+\lambda_k^2\bm{E}_k
        \end{align*}于是\[
            \bm{\varphi}^m=\lambda_1^m\bm{E}_1+\lambda_2^m\bm{E}_2+\cdots+\lambda_k^m\bm{E}_k
        \]设$f\left(x\right)=a_0+a_1x+\cdots+a_{m-1}x^{m-1}+a_mx^m$,则\begin{align*}
            f\left(\bm{\varphi}\right) & =a_0\bm{I}_V+a_1\bm{\varphi}+\cdots+a_{m-1}\bm{\varphi}^{m-1}+a_m\bm{\varphi}^m \\
                                       & =a_0\left(
            \sum_{i=1}^{k}\bm{E}_i
            \right)+a_1\left(
            \sum_{i=1}^{k}\lambda_i\bm{E}_i
            \right)+\cdots+a_{m-1}\left(
            \sum_{i=1}^{k}\lambda_i^{m-1}\bm{E}_i
            \right)+a_m\left(
            \sum_{i=1}^{k}\lambda_i^m\bm{E}_i
            \right)                                                                                                      \\
                                       & =\sum_{i=1}^{k}\left(
            a_0+a_1\lambda_i+\cdots+a_{m-1}\lambda_i^{m-1}+a_m\lambda_i^m
            \right)\bm{E}_i                                                                                              \\
                                       & =\sum_{i=1}^{k}f\left(\lambda_i\right)\bm{E}_i
        \end{align*}于是$\displaystyle f_j\left(\bm{\varphi}\right)=\sum_{i=1}^{k}f_j\left(\lambda_i\right)\bm{E}_i=\sum_{i=1}^{k}\delta_{ij}\bm{E}_i=\bm{E}_j$.
    \end{proof}
}
\thm{正规伴随的多项式表示}{正规伴随的多项式表示}{
    设酉空间$V$上的线性算子$\bm{\varphi}$,则$\bm{\varphi}$正规当且仅当$\exists f\left(x\right)\in\mathbb{C}\left[x\right]$使得$\bm{\varphi}^*=f\left(\bm{\varphi}\right)$.\begin{proof}
        考虑充分性,$\bm{\varphi\varphi}^*=\bm{\varphi}f\left(\bm{\varphi}\right)=f\left(\bm{\varphi}\right)\bm{\varphi}=\bm{\varphi}^*\bm{\varphi}.$

        考虑必要性,设正规算子$\bm{\varphi}$,考虑谱分解$\bm{\varphi}=\lambda_1\bm{E}_1+\lambda_2\bm{E}_2+\cdots+\lambda_k\bm{E}_k$,则\[
            \bm{\varphi}^*=\overline{\lambda_1}\bm{E}_1+\overline{\lambda_2}\bm{E}_2+\cdots+\overline{\lambda_k}\bm{E}_k
        \]作\begin{align*}
            f\left(x\right) & =\sum_{j=1}^{k}\overline{\lambda}_jf_j\left(\lambda\right)            \\
                            & =\sum_{j=1}^{k}\prod_{i\neq j}\frac{x-\lambda_i}{\lambda_j-\lambda_i}
        \end{align*}于是\begin{align*}
            f\left(\bm{\varphi}\right) & =\sum_{j=1}^{k}\overline{\lambda}_jf_j\left(\bm{\varphi}\right)                \\
                                       & \xlongequal{\cref{lem:正交投影算子的多项式表达}}\sum_{j=1}^{k}\overline{\lambda}_j\bm{E}_j \\
                                       & =\bm{\varphi}^*\qedhere
        \end{align*}
    \end{proof}
}
\dfn{正定自伴随算子}{正定自伴随算子}{
    设有限维内积空间$V$上的自伴随算子$\bm{\varphi}$,若$\forall\bm{0}\neq\bm{\alpha}\in V:$\begin{enumerate}[label=\arabic*)]
        \item $\left(\bm{\varphi}\left(\bm{\alpha}\right),\bm{\alpha}\right)>0$,则称$\bm{\varphi}$是正定自伴随算子
        \item $\left(\bm{\varphi}\left(\bm{\alpha}\right),\bm{\alpha}\right)\geqslant0$,则称$\bm{\varphi}$是半正定自伴随算子
    \end{enumerate}
}
\exa{}{}{
    考虑欧式空间$V$,任取标准正交基,自伴随算子$\bm{\varphi}$的表示矩阵为$\bm{A}$,于是$\bm{A}$实对称.设$\bm{0}\neq\bm{\alpha}\in V$的坐标向量$\bm{0}\neq\bm{x}\in\mathbb{R}^n$,则$\bm{\varphi}\left(\bm{\alpha}\right)$的坐标向量为$\bm{Ax}$.于是\[
        \left(
        \bm{\varphi}\left(\bm{\alpha}\right),\bm{\alpha}
        \right)=\left(\bm{Ax}\right)'\bm{x}=\bm{x}'\bm{A}\bm{x}
    \]于是这与实对称阵的正定性等价.
}
\thm{自伴随算子正定的等价结论}{自伴随算子正定的等价结论}{
    设$V$上的自伴随算子$\bm{\varphi}$,则$\bm{\varphi}$正定(半正定)当且仅当$\bm{\varphi}$在任一组(或者某一组)标准正交基下的表示矩阵是正定(半正定)的实对称阵(Hermite阵).
}
\thm{}{利用特征值判定}{
    设酉空间$V$上的正规算子$\bm{\varphi}$\begin{enumerate}[label=\arabic*)]
        \item 若$\bm{\varphi}$的特征值全为实数,则$\bm{\varphi}$是自伴随算子
        \item 若$\bm{\varphi}$的特征值全部大于等于$0$,则$\bm{\varphi}$是半正定自伴随算子
        \item 若$\bm{\varphi}$的特征值全部大于$0$,则$\bm{\varphi}$是正定自伴随算子
        \item 若$\bm{\varphi}$的特征值的模长全为$1$,则$\bm{\varphi}$是酉算子
    \end{enumerate}\begin{proof}
        首先有谱分解\[
            \bm{\varphi}=\lambda_1\bm{E}_1+\lambda_2\bm{E}_2+\cdots+\lambda_k\bm{E}_k
        \]

        $(1)$\[
            \bm{\varphi}^*=\sum_{i=1}^{k}\overline{\lambda}_i\bm{E}_i=\sum_{i=1}^{k}\lambda_i\bm{E}_i=\bm{\varphi}
        \]

        $(2)\forall\bm{\alpha}=\bm{E}_1\left(\alpha\right)+\bm{E}2\left(\bm{\alpha}\right)+\cdots+\bm{E}_k\left(\bm{\alpha}\right)\in V:$

        \[
            \bm{\varphi}\left(\bm{\alpha}\right)=\lambda_1\bm{E}_1\left(\bm{\alpha}\right)+\lambda_2\bm{E}_2\left(\bm{\alpha}\right)+\cdots+\lambda_k\bm{E}_k\left(\bm{\alpha}\right)
        \]于是\begin{align*}
            \left(
            \bm{\varphi}\left(\bm{\alpha}\right),\bm{\alpha}
            \right) & =\left(
            \sum_{i=1}^{k}\lambda_i\bm{E}_i\left(\bm{\alpha}\right),\sum_{j=1}^{k}\bm{E}_j\left(\bm{\alpha}\right)
            \right)                                  \\
                    & =\sum_{i=1}^{k}\lambda_i\left(
            \bm{E}_i\left(\bm{\alpha}\right),\bm{E}_i\left(\bm{\alpha}\right)
            \right)
        \end{align*}得证.

        $(3)$同理.

        $(4)$\[
            \bm{\varphi\varphi}^* =\sum_{i=1}^{k}\left\lVert\lambda_i\right\rVert^2\bm{E}_i=\sum_{i=1}^{k}\bm{E}_i=\bm{I}_V\qedhere
        \]
    \end{proof}
    以上均为等价结论.
}
\thm{算子的平方根}{算子的平方根}{
    设有限维内积空间$V$上的半正定自伴随算子$\bm{\varphi}$,则存在唯一的半正定自伴随算子$\bm{\psi}$使得$\bm{\varphi}=\bm{\psi}^2$,也称$\bm{\psi}$是$\bm{\varphi}$的算术平方根,记作$\bm{\psi}=\bm{\varphi}^{1/2}=\sqrt{\bm{\varphi}}$.\begin{proof}
        设谱分解\[
            \bm{\varphi}=\lambda_1\bm{E}_1+\lambda_2\bm{E}_2+\cdots+\lambda_k\bm{E}_k
        \]其中$\lambda_1,\lambda_2,\cdots,\lambda_k$是$\bm{\varphi}$的全体不同特征值,对应的特征子空间分别为$V_1,V_2,\cdots,V_k$,$\bm{E}_i$是从$V$到$V_i$的正交投影算子.根据\cref{thm:利用特征值判定},$\forall1\leqslant i\leqslant k:\lambda_i\geqslant0$,于是作$d_i=\sqrt{\lambda_i}:$\[
            \bm{\psi}=d_1\bm{E}_1+d_2\bm{E}_2+\cdots+d_k\bm{E}_k
        \]首先$d_1,d_2,\cdots,d_k$互不相同,是$\bm{\psi}$的全体不同特征值,$V_i=\bm{E}_i\left(V\right)$是$d_i$的特征子空间.于是这是$\bm{\psi}$的谱分解,则$\bm{\psi}$是半正定自伴随算子,且$\bm{\varphi}=\bm{\psi}^2$.于是存在性证毕.

        设另一个半正定自伴随算子$\bm{\theta}$使得$\bm{\varphi}=\bm{\theta}^2$,设其谱分解\[
            \bm{\psi}=b_1\bm{F}_1+b_2\bm{F}_2+\cdots+b_r\bm{F}_r
        \]其中$b_1,b_2,\cdots,b_r$且$\forall1\leqslant i\leqslant r:b_i\geqslant0$,$\bm{F}_i$是从$V$到$\bm{F}_i\left(V\right)$上的正交投影算子.于是\[
            b_1^2\bm{F}_1+b_2^2\bm{F}_2+\cdots+b_r^2\bm{F}_r=\lambda_1\bm{E}_1+\lambda_2\bm{E}_2+\cdots+\lambda_k\bm{E}_k
        \]首先$b_1^2,b_2^2,\cdots,b_r^2$是互不相同的非负实数,且\[
            \bm{\varphi}=b_1^2\bm{F}_1+b_2^2\bm{F}_2+\cdots+b_r^2\bm{F}_r
        \]于是\[
            V=\bm{F}_1\left(V\right)\perp\bm{F}_2\left(V\right)\perp\cdots\perp\bm{F}_r\left(V\right)
        \]于是$b_1^2,b_2^2,\cdots,b_r^2$是$\bm{\varphi}$的全体不同特征值,于是$k=r$且轮换一定次序后一定有$\forall1\leqslant i\leqslant k=r:b_i^2=\lambda_i\Longrightarrow b_i=\sqrt{\lambda_i}=d_i$.因为对于$\bm{\varphi}$,$\bm{F}_i\left(V\right)$是$b_i^2=\lambda_i$的特征子空间,而$\bm{E}_i$也是$d_i^2=\lambda_i$的特征子空间,于是$\forall1\leqslant i\leqslant k:\bm{F}_i\left(V\right)=\bm{E}_i\left(V\right)$.

        因为$\bm{E}_i:V\longrightarrow\bm{E}_i\left(V\right),\bm{F}_i:V\longrightarrow\bm{F}_i\left(V\right)$,于是$\forall1\leqslant i\leqslant k:\bm{E}_i=\bm{F}_i.$于是$\bm{\theta}=\bm{\psi}$唯一性证毕.
    \end{proof}
}
\exa{}{}{
    设$\bm{\varphi}=\begin{pmatrix}
            1 & 0 \\0 & -1
        \end{pmatrix},\bm{E}_1:V\longrightarrow L\left(\bm{e}_1\right),\bm{E}_2:V\longrightarrow L\left(\bm{e}_2\right)$,谱分解\[
        \bm{\varphi}=\bm{E}_1-\bm{E}_2
    \]但是\[
        \bm{\varphi}^2=1^2\cdot\bm{E}_1+\left(-1\right)^2\cdot\bm{E}_2
    \]这并不是谱分解而是\[
        \bm{\varphi}^2=\bm{I}_V
    \]
}
\cor{半正定实对称/Hermite阵的平方根}{半正定实对称/Hermite阵的平方根}{
    设$\bm{A}$为半正定实对称/Hermite阵,则存在唯一的半正定实对称/Hermite阵$\bm{B}$使得$\bm{A}=\bm{B}^2$,也称$\bm{B}$是$\bm{A}$的算术平方根,记作$\bm{B}=\bm{A}^{1/2}=\sqrt{\bm{A}}$.
}
\subsection{极分解}
\thm{极分解定理}{极分解定理}{
    设有限维内积空间$V$上的线性算子$\bm{\varphi}$,则\[
        \bm{\varphi}=\bm{\omega\psi}
    \]其中$\bm{\omega}$是保持内积的线性算子,$\bm{\psi}$为半正定自伴随算子.并且$\bm{\psi}$由$\bm{\varphi}$唯一确定,若$\bm{\varphi}$可逆,则$\bm{\omega}$也唯一确定.\begin{proof}
        考虑$\bm{\varphi}^*=\bm{\psi}^*\bm{\omega}^*=\bm{\psi\omega}^*$,由\cref{thm:正交算子、酉算子的判定}知\[
            \bm{\varphi}^*\bm{\varphi}=\bm{\psi}^2
        \]根据\cref{thm:算子的平方根},只要验证$\bm{\varphi}^*\bm{\varphi}$是半正定自伴随算子即可说明$\bm{\psi}$存在且唯一确定.

        考虑$\forall\bm{\alpha}\in V:$\[
            \left(
            \bm{\varphi}^*\bm{\varphi}\left(\bm{\alpha}\right),\bm{\alpha}
            \right)=\left(
            \bm{\varphi}^*\bm{\varphi}\left(\bm{\alpha}\right),\bm{\varphi}^*\bm{\varphi}\left(\bm{\alpha}\right)
            \right)\geqslant 0
        \]于是$\bm{\varphi}^*\bm{\varphi}$是半正定算子,同时明显自伴随,则$\bm{\varphi}^*\bm{\varphi}$是半正定自伴随算子,令$\bm{\psi}$是$\bm{\varphi}^*\bm{\varphi}$的算术平方根,是半正定自伴随算子且\[
            \bm{\psi}^2=\bm{\varphi}^*\bm{\varphi}
        \]
        由\cref{thm:算子的平方根}知$\bm{\psi}$存在且唯一确定.

        考虑$\forall\bm{\alpha}\in V:$\begin{align*}
            \left(
            \bm{\psi}\left(
                \bm{\alpha}
                \right),\bm{\psi}\left(
                \bm{\alpha}
                \right)
            \right) & =\left(
            \bm{\psi}^2\left(
                \bm{\alpha}
                \right),\bm{\alpha}
            \right)           \\
                    & =\left(
            \bm{\varphi}^*\bm{\varphi}\left(
                \bm{\alpha}
                \right),\bm{\alpha}
            \right)           \\
                    & =\left(
            \bm{\varphi}\left(
                \bm{\alpha}
                \right),\bm{\varphi}\left(
                \bm{\alpha}
                \right)
            \right)
        \end{align*}即$\forall\bm{\alpha}\in V:$\[
            \left\lVert\bm{\psi}\left(
            \bm{\alpha}
            \right)\right\rVert=\left\lVert\bm{\varphi}\left(
            \bm{\alpha}
            \right)\right\rVert
        \]

        一方面,设$\bm{\varphi}$可逆,此时$\mathrm{Ker}\,\bm{\varphi}=0\Longrightarrow\forall\bm{\alpha}\in\mathrm{Ker}\,\bm{\psi}:\bm{\psi}\left(\bm{\alpha}\right)=\bm{0}\Longrightarrow\bm{\varphi}\left(\bm{\alpha}\right)=\bm{0}\Longrightarrow\bm{\alpha}=\bm{0}\Longrightarrow\mathrm{Ker}\,\bm{\psi}=0\Longrightarrow\bm{\psi}$可逆,此时\[
            \bm{\omega}=\bm{\varphi\psi}^{-1}
        \]唯一确定,下证其保积.\begin{align*}
            \bm{\omega}^* & =\left(\bm{\psi}^{-1}\right)^*\bm{\varphi}^* \\
                          & =\bm{\psi}^{-1}\bm{\varphi}^*                \\
        \end{align*}那么\[
            \bm{\omega\omega}^*=\bm{\psi}^{-1}\bm{\varphi}^*\bm{\varphi}\bm{\psi}^{-1}=\bm{\psi}^{-1}\bm{\psi}^2\bm{\psi}^{-1}=\bm{I}_V
        \]于是$\bm{\omega}$是保积算子且由$\bm{\varphi}$唯一确定.

        另一方面,设$\bm{\varphi}$不可逆,由$\forall\bm{\alpha}\in V:$\[
            \left\lVert\bm{\psi}\left(
            \bm{\alpha}
            \right)\right\rVert=\left\lVert\bm{\varphi}\left(
            \bm{\alpha}
            \right)\right\rVert
        \]那么一定有$\mathrm{Ker}\,\bm{\psi}=\mathrm{Ker}\,\bm{\varphi}$且不为零空间,由\cref{thm:像空间与核空间的维数公式}维数公式的$\dim \mathrm{Im}\,\bm{\varphi}=\dim \mathrm{Im}\,\bm{\psi}$,此时参考\cref{ex:代数与几何}的构造,我们要考虑$\mathrm{Im}\,\bm{\psi}$到$\mathrm{Im}\,\bm{\varphi}$的保积同构,$\forall\bm{v}\in V:$构造\begin{align*}
            \bm{\eta}:\mathrm{Im}\,\bm{\psi} & \longrightarrow\mathrm{Im}\,\bm{\varphi}   \\
            \bm{\psi}\left(\bm{v}\right)     & \longmapsto\bm{\varphi}\left(\bm{v}\right)
        \end{align*}考虑$\exists\bm{u}\in V:\bm{\psi}\left(\bm{u}\right)=\bm{\varphi}\left(\bm{u}\right)$.若$\bm{\psi}\left(\bm{u}\right)=\bm{\psi}\left(\bm{v}\right)\Longrightarrow\bm{u}-\bm{v}\in\mathrm{Ker}\,\bm{\psi}=\mathrm{Ker}\,\bm{\varphi}\Longrightarrow\bm{\varphi}\left(\bm{u}\right)=\bm{\varphi}\left(\bm{v}\right)$即$\bm{\eta}$的构造与代表元$\bm{v}\in V$的选取是无关的.

        此时容易证明$\bm{\varphi}=\bm{\eta\psi}$.并且$\bm{\eta}:\mathrm{Im}\,\bm{\psi}\longrightarrow\mathrm{Im}\,\bm{\varphi}$是满射,考虑到$\dim \mathrm{Im}\,\bm{\varphi}=\dim \mathrm{Im}\,\bm{\psi}$,于是$\bm{\eta}$是单射,即$\bm{\eta}$是线性同构.再由上文,$\forall\bm{v}\in V:$\[
            \left\lVert\bm{\eta}\left(
            \bm{\psi}\left(
                \bm{v}
                \right)
            \right)\right\rVert=\left\lVert\bm{\varphi}\left(
            \bm{v}
            \right)\right\rVert=\left\lVert\bm{\psi}\left(
            \bm{v}
            \right)\right\rVert
        \]于是$\bm{\eta}$保范即保积.

        因为$V=\mathrm{Im}\,\bm{\psi}\perp\mathrm{Im}\,\bm{\psi}^{\perp}=\mathrm{Im}\,\bm{\varphi}\perp\mathrm{Im}\,\bm{\varphi}^{\perp}$,根据\cref{cor:内积空间保积同构的等价条件}知维数相同的$\mathrm{Im}\,\bm{\psi}^{\perp}$与$\mathrm{Im}\,\bm{\varphi}^{\perp}$一定存在保积同构$\bm{\xi}$,作\begin{align*}
            \bm{\omega}:V & \longrightarrow V \\
            \left(
            \bm{\psi}\left(\bm{v}\right),\bm{u}
            \right)       & \longmapsto\left(
            \bm{\varphi}\left(\bm{v}\right),\bm{\xi}\left(\bm{u}\right)
            \right)
        \end{align*}即$\bm{\omega}=\bm{\eta}\perp\bm{\xi}$.

        再来验证$\bm{\varphi}=\bm{\omega\psi},\forall\bm{v}\in V:\bm{\varphi}\left(\bm{v}\right)=\bm{\omega\psi}\left(\bm{v}\right)=\bm{\eta}\bm{\psi}\left(\bm{v}\right)$根据定义就是成立的.
    \end{proof}
}
我们知道复数的极分解为
\[
    z=\rho \rme^{\rmi\theta}
\]分别被推广为半正定自伴随算子和保积算子.
\clm{极分解的等价表述}{}{
    事实上,\cref{thm:极分解定理}的两种等价表达为\[
        \bm{\varphi}=\bm{\omega\psi}=\left(
        \bm{\omega\psi\omega}^*
        \right)\bm{\omega}
    \]
}
\cor{极分解定理}{极分解定理}{
    $\forall\bm{A}\in M_n\left(\mathbb{R}\right):$\[
        \bm{A}=\bm{OS}=\bm{S}_1\bm{O}
    \]其中$\bm{O}$是正交阵,$\bm{S},\bm{S}_1$是半正定实对称阵;$\forall\bm{A}\in M_n\left(\mathbb{C}\right):$\[
        \bm{A}=\bm{UH}=\bm{H}_1\bm{U}
    \]其中$\bm{U}$是酉阵,$\bm{H},\bm{H}_1$是半正定Hermite阵.特别地,$\bm{A}$非异时,分解唯一确定.
}
\exa{}{}{
    设$\bm{A}$为可逆实矩阵,求其$\bm{OS}$分解.\begin{solution}
        首先$\bm{A}'\bm{A}$正定,然后求正交阵$\bm{P}:$\[
            \bm{P}'\bm{A}'\bm{AP}=\begin{pmatrix}
                \lambda_1 &           &        &           \\
                          & \lambda_2 &        &           \\
                          &           & \ddots &           \\
                          &           &        & \lambda_n
            \end{pmatrix}
        \]其中$\lambda_i>0$,那么\[
            \bm{S}=\bm{P}\begin{pmatrix}
                \sqrt{\lambda_1} &                  &        &                  \\
                                 & \sqrt{\lambda_2} &        &                  \\
                                 &                  & \ddots &                  \\
                                 &                  &        & \sqrt{\lambda_n}
            \end{pmatrix}\bm{P}'
        \]满足$\bm{A}'\bm{A}=\bm{S}^2$,然后\[
            \bm{O}=\bm{AS}^{-1}
        \]
    \end{solution}
}
在$\bm{A}$不可逆时,作极分解相对繁琐,这时将考虑奇异值分解.