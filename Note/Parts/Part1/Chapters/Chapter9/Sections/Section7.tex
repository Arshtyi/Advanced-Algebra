\newpage
\section{实正规矩阵}
\subsection{极小多项式的不可约分解}
\lem{}{正规与多项式相容}{
    设$V$为$n$维欧式空间,$\bm{\varphi}$是$V$上的正规算子,则$\forall f\left(x\right)\in\bbr \left[x\right],f\left(\bm{\varphi}\right)$也是$V$上的正规算子.\begin{proof}
        设$f\left(x\right)=a_mx^m+a_{m-1}x^{m-1}+\cdots+a_1x+a_0$,于是\[
            f\left(
            \bm{\varphi}
            \right)=a_m\bm{\varphi}^m+a_{m-1}\bm{\varphi}^{m-1}+\cdots+a_1\bm{\varphi}+a_0\bm{I}
        \]容易有\[
            f\left(\bm{\varphi}\right)^*=a_m\bm{\varphi}^{*m}+a_{m-1}\bm{\varphi}^{*m-1}+\cdots+a_1\bm{\varphi}^*+a_0\bm{I}=f\left(
            \bm{\varphi}^*
            \right)
        \]因为$\bm{\varphi\varphi}^*=\bm{\varphi}^*\bm{\varphi}$,于是\[
            f\left(
            \bm{\varphi}
            \right)f\left(
            \bm{\varphi}
            \right)^*=f\left(
            \bm{\varphi}
            \right)^*f\left(
            \bm{\varphi}
            \right)\qedhere
        \]
    \end{proof}
}
\lem{}{互素的延拓}{
    设欧式空间$V$上的实正规算子$\bm{\varphi}$,$f\left(x\right),g\left(x\right)$是互素的实系数多项式,则$\forall\bm{\alpha}\in\Ker \,f\left(\bm{\varphi}\right),\bm{\beta}\in\Ker \,g\left(\bm{\varphi}\right):$\[
        \left(
        \bm{\alpha},\bm{\beta}
        \right)=0
    \]\begin{proof}
        因为$\left(
            f\left(x\right),g\left(x\right)
            \right)=1$,则$\exists u\left(x\right),v\left(x\right)\in\bbr \left[x\right]:$\[
            f\left(x\right)u\left(x\right)+g\left(x\right)v\left(x\right)=1
        \]即\[
            f\left(
            \bm{\varphi}
            \right)u\left(
            \bm{\varphi}
            \right)+g\left(
            \bm{\varphi}
            \right)v\left(
            \bm{\varphi}
            \right)=\bm{I}_V
        \]两边作用$\forall\bm{\alpha}\in\Ker \,f\left(\bm{\varphi}\right):$\[
            \bm{\alpha}=f\left(
            \bm{\varphi}
            \right)u\left(
            \bm{\varphi}
            \right)\left(
            \bm{\alpha}
            \right)+g\left(
            \bm{\varphi}
            \right)v\left(
            \bm{\varphi}
            \right)\left(
            \bm{\alpha}
            \right)=g\left(
            \bm{\varphi}
            \right)v\left(
            \bm{\varphi}
            \right)\left(
            \bm{\alpha}
            \right)
        \]作\begin{align*}
            \left(
            \bm{\alpha},\bm{\beta}
            \right) & =\left(
            g\left(
                \bm{\varphi}
                \right)v\left(
                \bm{\varphi}
                \right)\left(
                \bm{\alpha}
                \right),\bm{\beta}
            \right)           \\
                    & =\left(
            v\left(
                \bm{\varphi}
                \right)\left(
                \bm{\alpha}
                \right),g\left(
                \bm{\varphi}
                \right)^*\left(
                \bm{\beta}
                \right)
            \right)
        \end{align*}因为$g\left(\bm{\varphi}\right)\left(\bm{\beta}\right)=\bm{0}$,考虑到$\bm{\varphi}$是正规算子,即$\forall\bm{v}\in V:\left\lVert\bm{\varphi}\left(\bm{v}\right)\right\rVert=\left\lVert\bm{\varphi}^*\left(\bm{v}\right)\right\rVert$,于是容易导出$g\left(\bm{\varphi}\right)^*\left(\bm{\beta}\right)=\bm{0}$,于是\[
            \left(
            \bm{\alpha},\bm{\beta}
            \right)=0\qedhere
        \]
    \end{proof}
}
\thm{}{极小多项式不可约分解引出的正交直和分解}{
    设$n$维欧式空间$V$上的实正规算子$\bm{\varphi}$的极小多项式为$g\left(x\right)$,设$g_1\left(x\right),g_2\left(x\right),\cdots,g_k\left(x\right)$是$g\left(x\right)$所有互不相同的首一不可约多项式,$\forall1\leqslant i\leqslant k:W_i\coloneqq \Ker \,g_i\left(\bm{\varphi}\right)$,则\begin{enumerate}[label=\arabic*)]
        \item $g\left(x\right)=g_1\left(x\right)g_2\left(x\right)\cdots g_k\left(x\right)$,其中$\forall1\leqslant i\leqslant k:\deg g_i\left(x\right)\leqslant 2$
        \item $V=W_1\perp W_2\perp \cdots\perp W_k$
        \item $W_i$是$\bm{\varphi}$-不变子空间,$\displaystyle\bm{\varphi}\mid_{W_i}$是$W_i$上的实正规算子且$g_i\left(x\right)$是$\displaystyle\bm{\varphi}\mid_{W_i}$的极小多项式
    \end{enumerate}\begin{proof}
        $(1)$任取$V$的一组标准正交基,$\bm{\varphi}$在这组基下的表示矩阵为$\bm{A}$,则$\bm{A}$实正规($\bm{AA}'=\bm{A}'\bm{A}$)亦是复正规($\bm{A}\overline{\bm{A}}'=\overline{\bm{A}}'\bm{A}$),由\cref{cor:正规必定对角},$\bm{A}$作为复正规阵必定可对角化,根据\cref{cor:可对角化的等价条件},$\bm{A}$的极小多项式$g\left(x\right)$在复数域$\bbc $上无重根,那么$g\left(x\right)$在实数域$\bbr $上没有重因式(若不然,必定在复数域上有重根),于是$g\left(x\right)$等于其所有互异的首一不可约多项式的乘积即\[
            g\left(x\right)=g_1\left(x\right)g_2\left(x\right)\cdots g_k\left(x\right)
        \]并且根据实数域$\bbr $上的不可约多项式的性质,每个$g_i\left(x\right)$的次数$\deg g_i\left(x\right)\leqslant 2$.

        $(2)$作\[
            f_i\left(x\right)=\frac{g\left(x\right)}{g_i\left(x\right)}=g_1\left(x\right)g_2\left(x\right)\cdots\widehat{g_i\left(x\right)}\cdots g_k\left(x\right)\Longrightarrow\left(
            f_1\left(\bm{\varphi}\right),f_2\left(\bm{\varphi}\right),\cdots, f_k\left(\bm{\varphi}\right)
            \right)=1
        \]即$\exists u_i\left(x\right):$\[
            f_1\left(x\right)u_1\left(x\right)+f_2\left(x\right)u_2\left(x\right)+\cdots+f_k\left(x\right)u_k\left(x\right)=1
        \]即\[
            f_1\left(\bm{\varphi}\right)u_1\left(\bm{\varphi}\right)+f_2\left(\bm{\varphi}\right)u_2\left(\bm{\varphi}\right)+\cdots+f_k\left(\bm{\varphi}\right)u_k\left(\bm{\varphi}\right)=\bm{I}_V
        \]$\forall\bm{\alpha}\in V:$\[
            \bm{\alpha}=f_1\left(\bm{\varphi}\right)u_1\left(\bm{\varphi}\right)\left(\bm{\alpha}\right)+f_2\left(\bm{\varphi}\right)u_2\left(\bm{\varphi}\right)\left(\bm{\alpha}\right)+\cdots+f_k\left(\bm{\varphi}\right)u_k\left(\bm{\varphi}\right)\left(\bm{\alpha}\right)
        \]考虑\[
            g_i\left(
            \bm{\varphi}
            \right)f_i\left(
            \bm{\varphi}
            \right)u_i\left(
            \bm{\varphi}
            \right)\left(
            \bm{\alpha}
            \right)=g\left(
            \bm{\varphi}
            \right)u_i\left(
            \bm{\varphi}
            \right)\left(
            \bm{\alpha}
            \right)=\bm{0}u_i\left(
            \bm{\varphi}
            \right)\left(\bm{\alpha}\right)=\bm{0}
        \]于是\[
            f_i\left(
            \bm{\varphi}
            \right)u_i\left(
            \bm{\varphi}
            \right)\left(
            \bm{\alpha}
            \right)\in \Ker \,g_i\left(
            \bm{\varphi}
            \right)=W_i
        \]于是\[
            V=W_1+W_2+\cdots+W_k
        \]
        注意到$\forall1\leqslant i\neq j\leqslant k:\left(g_i\left(x\right),g_j\left(x\right)\right)=1$,由\cref{lem:互素的延拓},$W_i\perp W_j$,于是\[
            V=W_1\perp W_2\perp\cdots\perp W_k
        \]

        $(3)$先证明$W_i$是$\bm{\varphi}$-不变子空间且是$\bm{\varphi}^*$-不变子空间.任取$\bm{w}\in W_i=\Ker \,g_i\left(\bm{\varphi}\right)\Longrightarrow g_i\left(\bm{\varphi}\right)\left(\bm{w}\right)=\bm{0}$,那么\[
            g_i\left(
            \bm{\varphi}
            \right)\bm{\varphi}\left(\bm{w}\right)=\bm{\varphi}\left(\bm{0}\right)=\bm{0}\Longrightarrow\bm{\varphi}\left(\bm{w}\right)\in W_i=\Ker \,g_i\left(\bm{\varphi}\right)
        \]于是$W_i$是$\bm{\varphi}$-不变子空间,而因为正规\begin{align*}
            g_i\left(\bm{\varphi}\right)\left(
            \bm{\varphi}^*\left(\bm{w}\right)
            \right) & =\bm{\varphi}^*g\left(
            \bm{\varphi}
            \right)\left(\bm{w}\right)       \\
                    & =\bm{0}
        \end{align*}
        即$W_i$是$\bm{\varphi}^*$-不变子空间.
        做限制$\bm{\varphi}\mid_{W_i},\bm{\varphi}^*\mid_{W_i}$,下面证明二者就是限制关系,$\forall\bm{\alpha},\bm{\beta}\in W_i:$\[
            \left(
            \bm{\varphi}\left(
                \bm{\alpha}
                \right),\bm{\beta}
            \right)=\left(
            \bm{\alpha},\bm{\varphi}^*\left(
                \bm{\beta}
                \right)
            \right)
        \]自然地视作限制,于是\[
            \left(
            \bm{\varphi}\mid_{W_i}
            \right)^*=\bm{\varphi}^*\mid_{W_i}
        \]考虑是否正规:\begin{align*}
            \bm{\varphi}\mid_{W_i}\bm{\varphi}^*\mid_{W_i} & =\left(
            \bm{\varphi}\bm{\varphi}^*
            \right)_{W_i}                                                                                    \\
                                                           & =\left(
            \bm{\varphi}^*\bm{\varphi}
            \right)_{W_i}                                                                                    \\
                                                           & =\bm{\varphi}^*\mid_{W_i}\bm{\varphi}\mid_{W_i}
        \end{align*}因为$W_i=\Ker \,g_i\left(\bm{\varphi}\right)$,那么$\bm{\varphi}\mid_{W_i}$适合$g_i\left(x\right)$(即$g_i\left(\bm{\varphi}\mid_{W_i}\right)=\bm{0}\in\call \left(W_i\right)$),于是$\bm{\varphi}\mid_{W_i}$的极小多项式整除$g_i\left(x\right)$,但是$g_i\left(x\right)$是次数大于等于$1$的不可约多项式,于是$\bm{\varphi}\mid_{W_i}$的极小多项式为$g_i\left(x\right)$.
    \end{proof}
}
\subsection{极小多项式为二次不可约多项式的正规算子的结构}
一般地,二次不可约多项式\[
    g\left(x\right)=\left(x-a\right)^2+b^2
\]其中$a,b\neq0\in\bbr $.特别地,考虑$a=0,b=1:g\left(x\right)=x^2+1.$
\lem{}{适合简单多项式的正规算子}{
    设$n$维欧式空间$V$上的实正规算子$\bm{\varphi}$的极小多项式$g\left(x\right)=x^2+1$,则$\forall\bm{0}\neq\bm{v}\in V,\bm{u}=\bm{\varphi}\left(\bm{v}\right):$\begin{enumerate}[label=\arabic*)]
        \item $\left\lVert\bm{u}\right\rVert=\left\lVert\bm{v}\right\rVert,\left(
                  \bm{v},\bm{u}
                  \right)=0$
        \item $\bm{\varphi},\bm{\varphi}^*$的表示阵为\[
                  \left(
                  \bm{\varphi}\left(\bm{u}\right),\bm{\varphi}\left(\bm{v}\right)
                  \right)=\left(\bm{u},\bm{v}\right)\begin{pmatrix}
                      0  & 1 \\
                      -1 & 0
                  \end{pmatrix},\left(
                  \bm{\varphi}^*\left(\bm{u}\right),\bm{\varphi}^*\left(\bm{v}\right)
                  \right)=\left(\bm{u},\bm{v}\right)\begin{pmatrix}
                      0 & -1 \\
                      1 & 0
                  \end{pmatrix}
              \]即\[
                  \begin{cases*}
                      \bm{\varphi}\left(\bm{v}\right)=\bm{u}    \\
                      \bm{\varphi}\left(\bm{u}\right)=-\bm{v}   \\
                      \bm{\varphi}^*\left(\bm{v}\right)=-\bm{u} \\
                      \bm{\varphi}^*\left(\bm{u}\right)=\bm{v}
                  \end{cases*}
              \]
    \end{enumerate}\begin{proof}
        \[\bm{\varphi}
            \left(
            \bm{u}
            \right)=\bm{\varphi}^2\left(
            \bm{v}
            \right)=-\bm{v}
        \]左侧证毕.

        考虑\begin{align*}
            0 & =
            \left\lVert\bm{\varphi}\left(\bm{v}\right)-\bm{u}\right\rVert^2+\left\lVert\bm{\varphi}\left(\bm{u}\right)+\bm{v}\right\rVert^2      \\
              & =\left\lVert\bm{\varphi}\left(\bm{v}\right)\right\rVert^2-2\left(
            \bm{\varphi}\left(\bm{v}\right),\bm{u}
            \right)+\left\lVert\bm{u}\right\rVert^2+\left\lVert\bm{\varphi}\left(\bm{u}\right)\right\rVert^2+2\left(
            \bm{\varphi}\left(\bm{u}\right),\bm{v}
            \right)+\left\lVert\bm{v}\right\rVert^2                                                                                              \\
              & =\left\lVert\bm{\varphi}^*\left(\bm{v}\right)\right\rVert^2-2\left(
            \bm{\varphi}^*\left(\bm{u}\right),\bm{v}
            \right)+\left\lVert\bm{u}\right\rVert^2+\left\lVert\bm{\varphi}^*\left(\bm{u}\right)\right\rVert^2+2\left(
            \bm{\varphi}^*\left(\bm{v}\right),\bm{u}
            \right)+\left\lVert\bm{v}\right\rVert^2                                                                                              \\
              & =\left\lVert\bm{\varphi}\left(\bm{v}\right)+\bm{u}\right\rVert^2+\left\lVert\bm{\varphi}\left(\bm{u}\right)-\bm{v}\right\rVert^2
        \end{align*}于是右侧证毕.

        另一方面

        \[
            \left(
            \bm{u},\bm{u}
            \right)=\left(
            \bm{\varphi}\left(
                \bm{v}
                \right),\bm{u}
            \right)=\left(
            \bm{v},\bm{\varphi}^*\left(
                \bm{u}
                \right)
            \right)=\left(
            \bm{v},\bm{v}
            \right)
        \]\[
            \left(
            \bm{u},\bm{v}
            \right)=\left(
            \bm{v},\bm{\varphi}^*\left(
                \bm{v}
                \right)
            \right)=-\left(
            \bm{v},\bm{u}
            \right)\qedhere
        \]
    \end{proof}
}
\lem{}{适合一般多项式的正规算子}{
设$n$维欧式空间上的正规算子$\bm{\varphi}$极小多项式$g\left(x\right)=\left(x-a\right)^2+b^2,a,b\neq0\in\bbr $,则$\forall\bm{0}\neq\bm{v}\in V,\bm{u}=b^{-1}\left(\bm{\varphi}-a\bm{I}_V\right)\left(\bm{v}\right):$
\begin{enumerate}[label=\arabic*)]
    \item $\left\lVert\bm{u}\right\rVert=\left\lVert\bm{v}\right\rVert,\left(
              \bm{v},\bm{u}
              \right)=0$
    \item $\bm{\varphi},\bm{\varphi}^*$的表示阵为\[
              \left(
              \bm{\varphi}\left(\bm{u}\right),\bm{\varphi}\left(\bm{v}\right)
              \right)=\left(\bm{u},\bm{v}\right)\begin{pmatrix}
                  a  & b \\
                  -b & a
              \end{pmatrix},\left(
              \bm{\varphi}^*\left(\bm{u}\right),\bm{\varphi}^*\left(\bm{v}\right)
              \right)=\left(\bm{u},\bm{v}\right)\begin{pmatrix}
                  a & -b \\
                  b & a
              \end{pmatrix}
          \]即\[
              \begin{cases*}
                  \bm{\varphi}\left(\bm{v}\right)=a\bm{v}+b\bm{u}    \\
                  \bm{\varphi}\left(\bm{u}\right)=-b\bm{v}+a\bm{u}   \\
                  \bm{\varphi}^*\left(\bm{v}\right)=-b\bm{u}+a\bm{v} \\
                  \bm{\varphi}^*\left(\bm{u}\right)=b\bm{v}+a\bm{u}
              \end{cases*}
          \]
\end{enumerate}\begin{proof}
    因为$\bm{\varphi}$适合$g\left(x\right)=\left(x-a\right)^2+b^2$,那么\[
        \left(
        \bm{\varphi}-a\bm{I}_V
        \right)^2+b^2\bm{I}_V=\bm{0}\Longrightarrow\left(
        b^{-1}\left(
        \bm{\varphi}-a\bm{I}_V
        \right)
        \right)^2+\bm{I}_V=\bm{0}
    \]$\bm{\psi}\coloneqq b^{-1}\left(\bm{\varphi}-a\bm{I}_V\right)$,其极小多项式为$x^2+1$,由\cref{lem:适合简单多项式的正规算子}得证.
\end{proof}
}
\thm{}{一般极小多项式的空间的结构}{
    设$n$维欧式空间上的正规算子$\bm{\varphi}$极小多项式$g\left(x\right)=\left(x-a\right)^2+b^2,a,b\neq0\in\bbr $,则$\exists V_1,V_2,\cdots,V_s:\dim V_1=\dim V_2=\cdots=\dim V_s=2:$\[
        V=V_1\perp V_2\perp\cdots\perp V_s
    \]且$\forall1\leqslant i\leqslant s$,存在$V_i$的一组标准正交基$\left\{\bm{u}_i,\bm{v}_i\right\}$使得$\bm{\varphi}\mid_{V_i}$的表示矩阵为
    \[
        \begin{pmatrix}
            a  & b \\
            -b & a
        \end{pmatrix}
    \]\begin{proof}
        对$\dim V$归纳,$\dim V=0$表示证明结束,证明$\dim V=1,V=L\left(\bm{e}\right),\bm{\varphi}\left(\bm{e}\right)=\lambda_0\bm{e},\lambda_0\in\bbr $但$\lambda_0$不适合极小多项式$g\left(x\right)=\left(x-a\right)^2+b^2$,于是$\dim V=1$是不可能的,进一步地,所有$\dim V$为奇数的情形也是不可能的.

        设$\dim V<n$的情形成立,下证$\dim V=n$的情形,$\forall\bm{0}\neq\bm{v}_1\in V,\left\lVert\bm{v}_1\right\rVert=1,\bm{u}_1=b^{-1}\left(\bm{\varphi}-a\bm{I}_V\right)\left(\bm{v}_1\right)$,由\cref{lem:适合一般多项式的正规算子}知$\left\lVert\bm{u}_1\right\rVert=1,\left(
            \bm{v}_1,\bm{u}_1
            \right)=0$,令$V_1=L\left(\bm{u}_1,\bm{v}_1\right)$,则$V_1$的一组标准正交基就是$\left\{\bm{u}_1,\bm{v}_1\right\}$,且$\bm{\varphi}\mid_{V_1}:$\[
            \left(
            \bm{\varphi}\left(\bm{u}\right),\bm{\varphi}\left(\bm{v}\right)
            \right)=\left(\bm{u},\bm{v}\right)\begin{pmatrix}
                a  & b \\
                -b & a
            \end{pmatrix},\left(
            \bm{\varphi}^*\left(\bm{u}\right),\bm{\varphi}^*\left(\bm{v}\right)
            \right)=\left(\bm{u},\bm{v}\right)\begin{pmatrix}
                a & -b \\
                b & a
            \end{pmatrix}
        \]即$V_1$是$\bm{\varphi}$-不变子空间和$\bm{\varphi}^*$-不变子空间,于是$V_1^{\perp}$是$\bm{\varphi}^*$-不变子空间和$\left(\bm{\varphi}^*\right)^*= \bm{\varphi}$-不变子空间.作限制$\bm{\varphi}\mid_{V_1^{\perp}},\bm{\varphi}^*\mid_{V_1^{\perp}}$.以\cref{thm:极小多项式不可约分解引出的正交直和分解}同样方法可证$\bm{\varphi}\mid_{V_1^{\perp}}$正规.因为\begin{align*}
            \dim V_1^{\perp} & =\dim V-\dim V_1 \\
                             & =n-2             \\
                             & <n
        \end{align*}
        这已经证好,由归纳假设存在$s-1$个$2$维正规不变子空间$V_2,V_3,\cdots,V_s$,于是\[V_1^{\perp}=V_2\perp V_3\perp\cdots\perp V_s\]且$\forall2\leqslant i\leqslant s:$存在$V_i$的一组标准正交基$\left\{\bm{u}_i,\bm{v}_i\right\}$使得$\bm{\varphi}\mid_{V_i}$的表示矩阵为\[
            \begin{pmatrix}
                a  & b \\
                -b & a
            \end{pmatrix}
        \]于是$V=V_1\perp V_2\perp\cdots\perp V_s$且$\forall1\leqslant i\leqslant s:$存在$V_i$的一组标准正交基$\left\{\bm{u}_i,\bm{v}_i\right\}$使得$\bm{\varphi}\mid_{V_i}$的表示矩阵为\[
            \begin{pmatrix}
                a  & b \\
                -b & a
            \end{pmatrix}
        \]于是$\bm{\varphi}$在$\left\{
            \bm{u}_1,\bm{v}_1,\bm{u}_2,\bm{v}_2,\cdots,\bm{u}_s,\bm{v}_s
            \right\}$这组基下的表示矩阵为\[
            \diag \,\left\{
            \begin{pmatrix}
                a  & b \\
                -b & a
            \end{pmatrix},\begin{pmatrix}
                a  & b \\
                -b & a
            \end{pmatrix},\cdots,\begin{pmatrix}
                a  & b \\
                -b & a
            \end{pmatrix}
            \right\}\qedhere
        \]
    \end{proof}
}
\subsection{实正规算子的正交相似标准型}
\thm{实正规算子的正交相似标准型}{实正规算子的正交相似标准型}{
设$n$维欧式空间$V$上的实正规算子$\bm{\varphi}$,则存在$V$的一组标准正交基使得$\bm{\varphi}$的表示阵为\[
    \diag \,\left\{
    \begin{pmatrix}
        a_1  & b_1 \\
        -b_1 & a_1
    \end{pmatrix},\begin{pmatrix}
        a_2  & b_2 \\
        -b_2 & a_2
    \end{pmatrix},\cdots,\begin{pmatrix}
        a_r  & b_r \\
        -b_r & a_r
    \end{pmatrix},c_{2r+1},\cdots,c_n
    \right\}
\]其中$\forall1\leqslant i\leqslant r,2r+1\leqslant j\leqslant n:a_i,b_i\neq0,c_j\in\bbr $.这一矩阵称为实正规阵的正交相似标准型.\begin{proof}
    根据\cref{thm:极小多项式不可约分解引出的正交直和分解}对$\bm{\varphi}$的极小多项式进行不可约分解并有$V=W_1\perp W_2\perp\cdots\perp W_k$,考虑$\bm{\varphi}\mid_{W_i}$的极小多项式$g_i\left(x\right)$.

    一方面,如果$g_i\left(x\right)=x-c_i$,那么$\bm{\varphi}\mid_{W_i}=c_i\bm{I}_{W_i}$.

    另一方面,如果$g_i\left(x\right)=\left(x-a_i\right)^2+b_i^2$,根据\cref{thm:一般极小多项式的空间的结构},存在$W_i$的一组标准正交基使得$\bm{\varphi}\mid_{W_i}$的表示矩阵为\[
        \diag \,\left\{
        \begin{pmatrix}
            a_i  & b_i \\
            -b_i & a_i
        \end{pmatrix},\begin{pmatrix}
            a_i  & b_i \\
            -b_i & a_i
        \end{pmatrix},\cdots,\begin{pmatrix}
            a_i  & b_i \\
            -b_i & a_i
        \end{pmatrix}
        \right\}\]

    两种情形合并即证.
\end{proof}
}
\thm{正交相似标准型的特征值}{正交相似标准型的特征值}{
\cref{thm:实正规算子的正交相似标准型}正交相似标准型的特征值为\[
    a_1\pm b\rmi,a_2\pm b\rmi,\cdots,a_r\pm b\rmi,c_{2r+1},\cdots,c_n
\]
}
\clm{}{}{
    \cref{thm:实正规算子的正交相似标准型}和\cref{thm:正交相似标准型的特征值}说明,实正规矩阵的全体特征值与其正交相似标准型相互唯一确定.
}
\thm{实正规阵在正交相似下的全系不变量}{实正规阵在正交相似下的全系不变量}{
    实正规阵在正交相似下的全系不变量是全体特征值,也是其正交相似标准型.
}
\cor{}{旋转对角阵}{
    设$n$阶正交阵$\bm{A}$,则存在正交阵$\bm{P}$使得\[
        \bm{P}'\bm{AP}=\diag \,\left\{
        \begin{pmatrix}
            \cos\theta_1  & \sin\theta_1 \\
            -\sin\theta_1 & \cos\theta_1
        \end{pmatrix},\cdots,\begin{pmatrix}
            \cos\theta_r  & \sin\theta_r \\
            -\sin\theta_r & \cos\theta_r
        \end{pmatrix};1,\cdots,1;-1,\cdots,-1
        \right\}
    \]\begin{proof}
        首先一定有正交阵$\bm{P}$使得$\bm{P}'\bm{AP}$为正交相似标准型\[
            \diag \,\left\{
            \begin{pmatrix}
                a_1  & b_1 \\
                -b_1 & a_1
            \end{pmatrix},\begin{pmatrix}
                a_2  & b_2 \\
                -b_2 & a_2
            \end{pmatrix},\cdots,\begin{pmatrix}
                a_r  & b_r \\
                -b_r & a_r
            \end{pmatrix},c_{2r+1},\cdots,c_n
            \right\}
        \]是正交的,于是其中分块均正交,根据正交的定义$\bm{Q}'\bm{Q}=\bm{I}_2$即证.
    \end{proof}
}
\cor{实反对称阵的正交相似标准型}{实反对称阵的正交相似标准型}{
    设$n$阶实反对称阵$\bm{A}$,则存在正交阵$\bm{P}$使得\[
        \bm{P}'\bm{AP}=\diag \,\left\{
        \begin{pmatrix}
            0    & b_1 \\
            -b_1 & 0
        \end{pmatrix},
        \begin{pmatrix}
            0    & b_2 \\
            -b_2 & 0
        \end{pmatrix},
        \cdots,
        \begin{pmatrix}
            0    & b_r \\
            -b_r & 0
        \end{pmatrix},
        0,\cdots,0
        \right\}\]\begin{proof}
        首先$\bm{AA}'=-\bm{A}^2=\bm{A}'\bm{A}$即实反对称阵是实正规阵,根据\cref{thm:实正规算子的正交相似标准型}得到存在正交阵$\bm{P}$使得$\bm{P}'\bm{AP}$为\[
            \diag \,\left\{
            \begin{pmatrix}
                a_1  & b_1 \\
                -b_1 & a_1
            \end{pmatrix},\begin{pmatrix}
                a_2  & b_2 \\
                -b_2 & a_2
            \end{pmatrix},\cdots,\begin{pmatrix}
                a_r  & b_r \\
                -b_r & a_r
            \end{pmatrix},c_{2r+1},\cdots,c_n
            \right\}
        \]注意到\begin{align*}
            \left(
            \bm{P}'\bm{AP}
            \right)'=\bm{P}'\bm{A}'\bm{P}=-\bm{P}'\bm{AP}\qedhere
        \end{align*}
    \end{proof}
}
\cor{实反对称阵的秩}{实反对称阵的秩}{
    实反对称阵的秩为偶数.
}
\cor{实反对称阵的特征值}{实反对称阵的特征值}{
    实反对称阵的特征值均为$0$或者纯虚数.
}