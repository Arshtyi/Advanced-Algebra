\newpage
\section{伴随}
从泛函分析的角度,变换也称为算子.
\subsection{伴随算子}
设$n$维酉空间(欧式空间)$V$,取标准正交基$\left\{
    \bm{e}_1,\bm{e}_2,\cdots,\bm{e}_n
    \right\}$,线性算子$\bm{\varphi}\in \mathcal{L}\left(V\right)$,表示阵$\displaystyle\bm{A}=\left(a_{ij}\right)_{n\times n}\in M_n\left(\mathbb{C}\right)$.任取$\displaystyle\bm{\alpha}=\sum_{i=1}^{n}a_i\bm{e}_i$,其坐标向量为$\bm{x}=\begin{pmatrix}
        a_1 \\a_2\\\vdots\\a_n
    \end{pmatrix},
    \bm{\beta}=\sum_{i=1}^{n}b_i\bm{e}_i
$,其坐标向量为$\bm{y}=\begin{pmatrix}
        b_1 \\b_2\\\vdots\\b_n
    \end{pmatrix}$.容易知道\[
    \bm{\varphi}\left(\bm{\alpha}\right)=\bm{Ax},\bm{\varphi}\left(\bm{\beta}\right)=\bm{Ay}
\]

下面考虑\begin{align*}
    \left(
    \bm{\varphi}\left(\bm{\alpha}\right),\bm{\beta}
    \right) & =\left(\bm{Ax}\right)'\overline{\bm{y}} \\
            & =\bm{x}'\bm{A}'\overline{\bm{y}}        \\
            & =\bm{x}'\overline{\left(
        \overline{\bm{A}}'\bm{y}
        \right)}
\end{align*}
定义$\bm{\psi}\in\mathcal{L}\left(V\right)$,由\cref{cor:线性变换到矩阵的映射是代数同构}知一定存在表示矩阵为$\overline{\bm{A}}'$的$\bm{\psi}$.考虑\begin{align*}
    \left(
    \bm{\alpha},\bm{\psi}\left(\bm{\beta}\right)
    \right)=\bm{x}'\overline{\left(
        \overline{\bm{A}}'\bm{y}
        \right)}
\end{align*}即\[
    \left(
    \bm{\varphi}\left(\bm{\alpha}\right),\bm{\beta}
    \right)=\left(
    \bm{\alpha},\bm{\psi}\left(\bm{\beta}\right)
    \right)
\]其中$\bm{\psi}$只依赖于$\bm{\varphi}$,于是该式对$\forall\bm{\alpha},\bm{\beta}\in V$成立.
\dfn{伴随算子}{伴随算子}{
    设$\bm{\varphi}$是内积空间$V$上的一个线性算子,若存在线性算子$\bm{\varphi}^*,\forall\bm{\alpha},\bm{\beta}\in V:$\[
        \left(
        \bm{\varphi}\left(\bm{\alpha}\right),\bm{\beta}
        \right)=\left(
        \bm{\alpha},\bm{\varphi}^*\left(\bm{\beta}\right)
        \right)
    \]则称$\bm{\varphi}^*$是$\bm{\varphi}$的伴随算子,简称伴随.
}
\thm{伴随算子的存在性}{伴随算子的存在性}{
    \begin{enumerate}[label=\arabic*)]
        \item 伴随算子如果存在,则一定是唯一的
        \item 有限维内积空间上的任一线性算子一定有伴随算子
    \end{enumerate}\begin{proof}
        \begin{enumerate}[label=\arabic*)]
            \item 设$\bm{\varphi}^*,\bm{\varphi}^{\#}$都是$\bm{\varphi}$的伴随算子,即\[
                      \left(
                      \bm{\varphi}\left(\bm{\alpha}\right),\bm{\beta}
                      \right)=\left(
                      \bm{\alpha},\bm{\varphi}^*\left(\bm{\beta}\right)
                      \right)=\left(
                      \bm{\alpha},\bm{\varphi}^{\#}\left(\bm{\beta}\right)
                      \right)
                  \]于是$\forall\bm{\alpha}\in V:$\[
                      \left(\bm{\alpha},\bm{\varphi}^*\left(\bm{\beta}\right)-\bm{\varphi}^{\#}\left(\bm{\beta}\right)\right)=0
                  \]特别地\[
                      \left(
                      \bm{\varphi}^*\left(\bm{\beta}\right)-\bm{\varphi}^{\#}\left(\bm{\beta}\right),\bm{\varphi}^*\left(\bm{\beta}\right)-\bm{\varphi}^{\#}\left(\bm{\beta}\right)
                      \right)=0\Longrightarrow\bm{\varphi}^*\left(\bm{\beta}\right)=\bm{\varphi}^{\#}\left(\bm{\beta}\right)
                  \]于是$\bm{\varphi}^*=\bm{\varphi}^{\#}.$

            \item 由伴随的构造和\cref{cor:线性变换到矩阵的映射是代数同构}即得\qedhere
        \end{enumerate}
    \end{proof}
}
\thm{伴随算子的表示矩阵}{伴随算子的表示矩阵}{
    设$V$是$n$维内积空间,一组标准正交基$\left\{
        \bm{e}_1,\bm{e}_2,\cdots,\bm{e}_n
        \right\}$,线性算子$\bm{\varphi}\in\mathcal{L}\left(V\right)$在这组基下的表示矩阵为$\bm{A}$,则$\bm{\varphi}$的伴随$\bm{\varphi}^*$在这组基下的表示矩阵为\[
        \begin{cases*}
            \bm{A}'            & ,$V$是欧式空间 \\
            \overline{\bm{A}}' & ,$V$是酉空间
        \end{cases*}
    \]
}
\pro{伴随算子的运算}{伴随算子的运算}{
    设$V$为内积空间,$\bm{\varphi},\bm{\psi}\in\mathcal{L}\left(V\right)$且$\bm{\varphi}^*,\bm{\psi}^*$存在,$c$为常数,则有\begin{enumerate}[label=\arabic*)]
        \item $\left(\bm{\varphi}+\bm{\psi}\right)^*=\bm{\varphi}^*+\bm{\psi}^*$
        \item $\left(c\bm{\varphi}\right)^*=\overline{c}\bm{\varphi}^*$
        \item $\left(\bm{\varphi\psi}\right)^*=\bm{\psi}^*\bm{\varphi}^*$
        \item $\left(\bm{\varphi}^*\right)^*=\bm{\varphi}$
        \item 若$\bm{\varphi}$是可逆的,则$\bm{\varphi}^*$也是可逆的,且$\left(\bm{\varphi}^{-1}\right)^*=\left(\bm{\varphi}^*\right)^{-1}$
    \end{enumerate}\begin{proof}
        若$V$是有限维内积空间,设一组标准正交基$\left\{
            \bm{e}_1,\bm{e}_2,\cdots,\bm{e}_n
            \right\}$,且$\bm{\varphi},\bm{\psi}$在这组基下的表示矩阵分别为$\bm{A},\bm{B}$,于是容易证明.

        若$V$是无限维内积空间,考虑\begin{align*}
            \left(
            \bm{\varphi}\bm{\psi}\left(\bm{\alpha}\right),\bm{\beta}
            \right) & =\left(
            \bm{\psi}\left(\bm{\alpha}\right),\bm{\varphi}^*\left(\bm{\beta}\right)
            \right)           \\
                    & =\left(
            \bm{\alpha},\bm{\psi}^*\bm{\varphi}^*\left(\bm{\beta}\right)
            \right)
        \end{align*}由\cref{def:伴随算子},\cref{thm:伴随算子的存在性}知\[
            \left(
            \bm{\varphi}\bm{\psi}
            \right)^*=\bm{\psi}^*\bm{\varphi}^*
        \]其余证明较为简单.\qedhere
    \end{proof}
}
\pro{}{伴随的不变子空间与特征值}{
    设$\bm{\varphi}$是$n$维内积空间$V$上的一个线性算子,则\begin{enumerate}[label=\arabic*)]
        \item 若$U$是$\bm{\varphi}$的不变子空间,则$U^{\perp}$是$\bm{\varphi}^*$的不变子空间
        \item 若$\bm{\varphi}$的全体特征值为$\lambda_1,\lambda_2,\cdots,\lambda_n$,则$\bm{\varphi}^*$的全体特征值为$\overline{\lambda}_1,\overline{\lambda}_2,\cdots,\overline{\lambda}_n$
    \end{enumerate}\begin{proof}
        \begin{enumerate}[label=\arabic*)]
            \item $\forall\bm{u}\in U,\bm{w}\in U^{\perp}$,考虑\begin{align*}
                      \left(
                      0=\bm{\varphi}\left(
                          \bm{u}
                          \right),\bm{w}
                      \right)=\left(
                      \bm{u},\bm{\varphi}^*\left(
                          \bm{w}
                          \right)
                      \right)
                  \end{align*}于是$\forall\bm{w}\in U^{\perp}:$\[\bm{\varphi}^*\left(\bm{w}\right)\in U^{\perp}\]即$\bm{U}^{\perp}$是$\bm{\varphi}^*$-不变子空间
            \item 取标准正交基$\left\{
                      \bm{e}_1,\bm{e}_2,\cdots,\bm{e}_n
                      \right\}$,$\bm{\varphi}$的表示矩阵为$\bm{A}$,则$\bm{\varphi}^*$的表示矩阵为$\overline{\bm{A}}'$,设\[
                      \left|
                      \lambda\bm{I}-\bm{A}
                      \right|=\left(
                      \lambda-\lambda_1
                      \right)\left(
                      \lambda-\lambda_2
                      \right)\cdots\left(
                      \lambda-\lambda_n
                      \right)
                  \]考虑\begin{align*}
                      \left|
                      \lambda\bm{I}-\overline{\bm{A}}'
                      \right| & =\left|
                      \lambda\bm{I}-\overline{\bm{A}}
                      \right|                                             \\
                              & \xlongequal{\lambda=\overline{\mu}}\left|
                      \overline{\mu}\bm{I}-\overline{\bm{A}}
                      \right|                                             \\
                              & =\overline{
                          \left|
                          \mu\bm{I}-\bm{A}
                          \right|
                      }                                                   \\
                              & =\left(
                      \lambda-\overline{\lambda}_1
                      \right)\left(
                      \lambda-\overline{\lambda}_2
                      \right)\cdots\left(
                      \lambda-\overline{\lambda}_n
                      \right)
                  \end{align*}于是$\bm{\varphi}^*$的全体特征值为$\overline{\lambda}_1,\overline{\lambda}_2,\cdots,\overline{\lambda}_n$.\qedhere
        \end{enumerate}
    \end{proof}
}
\exa{}{}{
    设$V=U\perp U^{\perp}$,$\bm{E}$是从$V$到$U$的正交投影:$\left(\bm{E}\left(\bm{\alpha}\right),\bm{\beta}\right)=\left(\bm{\alpha},\bm{E}\left(\bm{\beta}\right)\right)$即$\bm{E}^*=\bm{E}$,这样的算子称为自伴随算子.
}
\exa{}{}{
    设$V=M_n\left(\mathbb{R}\right)$,取Forbenius内积,线性算子$\bm{\varphi}\in\mathcal{L}\left(V\right),\bm{\varphi}\left(\bm{A}\right)=\bm{PAQ},\bm{P},\bm{Q}\in M_n\left(\mathbb{R}\right)$,因为\begin{align*}
        \left(
        \bm{\varphi}\left(\bm{A}\right),\bm{B}
        \right) & =\mathrm{Tr}\left(
        \bm{\varphi}\left(\bm{A}\right)\bm{B}'
        \right)                                                                       \\
                & =\mathrm{Tr}\left(
        \bm{PAQ}\bm{B}'
        \right)                                                                       \\
                & \xlongequal{\cref{thm:迹的交换性}}\mathrm{Tr}\,\left(
        \bm{AQB}'\bm{P}
        \right)                                                                       \\
                & =\mathrm{Tr}\,\left(\bm{A}\left(\bm{P}'\bm{B}\bm{Q}'\right)'\right) \\
                & =\left(
        \bm{A},\left(
            \bm{P}'\bm{B}\bm{Q}'
            \right)'
        \right)
    \end{align*}于是$\bm{\varphi}^*\left(\bm{B}\right)=\bm{P}'\bm{BQ}'$.
}