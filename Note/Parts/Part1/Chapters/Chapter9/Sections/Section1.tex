\section{内积空间的概念}
\subsection{\texorpdfstring{$\bbr ^3$}{R3}}
\dfn{$\bbr^3$中向量的点积}{R^3中向量的点积}{
设$\bm{u}=\left(x_1,x_2,x_3\right),\bm{v}=\left(y_1,y_2,y_3\right)\in\bbr ^3$,定义点积/内积\[
    \bm{u}\cdot\bm{v}=x_1y_1+x_2y_2+x_3y_3
\]长度\[
    \left\lVert \bm{u}\right\rVert =\left(\bm{u}\cdot\bm{u}\right)^{\frac{1}{2}}=\sqrt{x_1^2+x_2^2+x_3^2}
\]     距离\[
    d\left(
    \bm{u},\bm{v}
    \right)=\left\lVert \bm{u}-\bm{v}\right\rVert =\left(
    \left(
    \bm{u}-\bm{v}
    \right)\cdot\left(
    \bm{u}-\bm{v}
    \right)
    \right)^{\frac{1}{2}}
\]
}
\pro{$\bbr^3$中向量点积的性质}{R^3中向量点积的性质}{
    设$\bm{u},\bm{v},\bm{w}\in\bbr ^3,c\in\bbr $,则\begin{enumerate}[label=\arabic*)]
        \item $\bm{u}\cdot\bm{v}=\bm{v}\cdot\bm{u}$
        \item $\left(\bm{u}+\bm{w}\right)\cdot\bm{v}=\bm{u}\cdot\bm{v}+\bm{w}\cdot\bm{v}$
        \item $\left(c\bm{u}\right)\cdot\bm{v}=c\left(\bm{u}\cdot\bm{v}\right)$
        \item $\bm{u}\cdot\bm{u}\geqslant 0$且等号成立当且仅当$\bm{u}=\bm{0}$
    \end{enumerate}
}
\subsection{一般线性空间}
\dfn{实内积空间}{实内积空间}{
    设实线性空间$V$,若存在二元运算\begin{align*}
        \left(\dashline,\dashline\right):V\times V & \longrightarrow\bbr                            \\
        \bm{\alpha}\times\bm{\beta}                & \longmapsto\left(\bm{\alpha},\bm{\beta}\right)
    \end{align*}且满足($\bm{\alpha},\bm{\beta},\bm{\gamma}\in V,c\in\bbr $)\begin{enumerate}[label=\arabic*)]
        \item 对称性:$\left(\bm{\alpha},\bm{\beta}\right)=\left(\bm{\beta},\bm{\alpha}\right)$
        \item 第一变元的线性:$\left(\bm{\alpha}+\bm{\beta},\bm{\gamma}\right)=\left(\bm{\alpha},\bm{\gamma}\right)+\left(\bm{\beta},\bm{\gamma}\right),\left(c\bm{\alpha},\bm{\beta}\right)=c\left(\bm{\alpha},\bm{\beta}\right)$
        \item 正定性:$\left(\bm{\alpha},\bm{\alpha}\right)\geqslant 0$且等号成立当且仅当$\bm{\alpha}=\bm{0}$
    \end{enumerate}则称该二元运算为$V$上的一个内积,实数$\left(\bm{\alpha},\bm{\beta}\right)$称为向量$\bm{\alpha},\bm{\beta}$的内积.这样给定一个内积结构的实线性空间称为实内积空间,有限维的实内积空间称为Euclid空间,简称欧式空间.
}
\dfn{复内积空间}{复内积空间}{
    设复线性空间$V$,若存在二元运算\begin{align*}
        \left(\dashline,\dashline\right):V\times V & \longrightarrow\bbc                            \\
        \bm{\alpha}\times\bm{\beta}                & \longmapsto\left(\bm{\alpha},\bm{\beta}\right)
    \end{align*}且满足($\bm{\alpha},\bm{\beta},\bm{\gamma}\in V,c\in\bbc $)\begin{enumerate}[label={\textup{(\arabic*)}}]
        \item 共轭对称性:$\left(\bm{\alpha},\bm{\beta}\right)=\overline{\left(\bm{\beta},\bm{\alpha}\right)}$
        \item 第一变元的线性:$\left(\bm{\alpha}+\bm{\beta},\bm{\gamma}\right)=\left(\bm{\alpha},\bm{\gamma}\right)+\left(\bm{\beta},\bm{\gamma}\right),\left(c\bm{\alpha},\bm{\beta}\right)=c\left(\bm{\alpha},\bm{\beta}\right)$
        \item 正定性:$\left(\bm{\alpha},\bm{\alpha}\right)\geqslant 0$且等号成立当且仅当$\bm{\alpha}=\bm{0}$
    \end{enumerate}则称该二元运算为$V$上的一个内积,复数$\left(\bm{\alpha},\bm{\beta}\right)$称为向量$\bm{\alpha},\bm{\beta}$的内积.这样给定一个内积结构的复线性空间称为复内积空间,有限维的复内积空间称为酉空间.
}
\clm{}{}{
    内积不一定是单射也不一定是满射,更不会只能构造出一种.
}
\rem{}{}{
    \cref{def:复内积空间}中由共轭对称性可得\[
        \left(
        \bm{\alpha},\bm{\alpha}
        \right)=\overline{\left(
            \bm{\alpha},\bm{\alpha}
            \right)}\Longrightarrow\left(
        \bm{\alpha},\bm{\alpha}
        \right)\in\bbr
    \]于是正定性是有意义的.
}
\rem{}{}{
    \cref{def:实内积空间}中第二变元也是线性的即\[\left(\bm{\alpha},c\bm{\beta}+d\bm{\gamma}\right)=c\left(\bm{\alpha},\bm{\beta}\right)+d\left(\bm{\alpha},\bm{\gamma}\right)\]但\cref{def:复内积空间}中第二变元是共轭线性的即\[\left(\bm{\alpha},c\bm{\beta}+d\bm{\gamma}\right)=\overline{c}\left(\bm{\alpha},\bm{\beta}\right)+\overline{d}\left(\bm{\alpha},\bm{\gamma}\right)\]
}
\clm{}{}{
    \cref{def:实内积空间}相容于\cref{def:复内积空间}.
}
\exa{}{}{
    设$V=\bbr ^n,\bm{\alpha}=\begin{pmatrix}
            x_1 \\x_2\\\vdots\\x_n
        \end{pmatrix},\bm{\beta}=\begin{pmatrix}
            y_1 \\y_2\\\vdots\\y_n
        \end{pmatrix}$,定义内积\[
        \left(
        \bm{\alpha},\bm{\beta}
        \right)\coloneqq \bm{\alpha}'\bm{\beta}=x_1y_1+x_2y_2+\cdots+x_ny_n
    \]称为$\bbr ^n$上的标准内积.

    设$V=\bbr _n,\bm{\alpha}=\left(
        x_1,x_2,\cdots,x_n
        \right),\bm{\beta}=\left(
        y_1,y_2,\cdots,y_n
        \right)$,定义内积\[\left(
        \bm{\alpha},\bm{\beta}
        \right)\coloneqq \bm{\alpha}\bm{\beta}'=x_1y_1+x_2y_2+\cdots+x_ny_n\]称为$\bbr _n$上的标准内积.

    设$V=\bbc ^n,\bm{\alpha}=\begin{pmatrix}
            x_1 \\x_2\\\vdots\\x_n
        \end{pmatrix},\bm{\beta}=\begin{pmatrix}
            y_1 \\y_2\\\vdots\\y_n
        \end{pmatrix}$,定义内积\[
        \left(
        \bm{\alpha},\bm{\beta}
        \right)\coloneqq \bm{\alpha}'\overline{\bm{\beta}}=x_1\overline{y_1}+x_2\overline{y_2}+\cdots+x_n\overline{y_n}
    \]称为$\bbc ^n$上的标准内积.

    设$V=\bbc _n,\bm{\alpha}=\left(
        x_1,x_2,\cdots,x_n
        \right),\bm{\beta}=\left(
        y_1,y_2,\cdots,y_n
        \right)$,定义内积\[\left(
        \bm{\alpha},\bm{\beta}
        \right)\coloneqq \bm{\alpha}\overline{\bm{\beta}}'=x_1\overline{y_1}+x_2\overline{y_2}+\cdots+x_n\overline{y_n}\]称为$\bbc _n$上的标准内积.
}
\exa{}{}{
    设$V=\bbr ^2,\bm{\alpha}=\begin{pmatrix}
            x_1 \\x_2
        \end{pmatrix},\bm{\beta}=\begin{pmatrix}
            y_1 \\y_2
        \end{pmatrix}$,则可定义内积\[
        \left(
        \bm{\alpha},\bm{\beta}
        \right)\coloneqq x_1y_1-x_2y_1-x_1y_2+4x_2y_2
    \]
}
\exa{}{}{
    设$V=\bbr ^n,\bm{\alpha},\bm{\beta}\in V,\bm{G}$是$n$阶正定实对称阵,定义\[
        \left(
        \bm{\alpha},\bm{\beta}
        \right)\coloneqq \bm{\alpha}'\bm{G}\bm{\beta}
    \]这称为$\bbr ^n$上相伴于正定阵$\bm{G}$的内积.

    设$V=\bbr _n,\bm{\alpha},\bm{\beta}\in V,\bm{G}$是$n$阶正定实对称阵,定义\[
        \left(
        \bm{\alpha},\bm{\beta}
        \right)\coloneqq \bm{\alpha}\bm{G}\bm{\beta}'
    \]这称为$\bbr _n$上相伴于正定阵$\bm{G}$的内积.

    设$V=\bbc ^n,\bm{\alpha},\bm{\beta}\in V,\bm{H}$是$n$阶正定Hermite阵,定义\[
        \left(
        \bm{\alpha},\bm{\beta}
        \right)\coloneqq \bm{\alpha}'\bm{G}\overline{\bm{\beta}}
    \]这称为$\bbc ^n$上相伴于正定Hermite阵$\bm{H}$的内积.

    设$V=\bbc _n,\bm{\alpha},\bm{\beta}\in V,\bm{H}$是$n$阶正定Hermite阵,定义\[
        \left(
        \bm{\alpha},\bm{\beta}
        \right)\coloneqq \bm{\alpha}\bm{G}\overline{\bm{\beta}}'
    \]这称为$\bbc _n$上相伴于正定Hermite阵$\bm{H}$的内积.
}
\exa{}{}{
    设$V$是$\left[a,b\right]$的连续函数全体构成的实线性空间,$f\left(t\right),g\left(t\right)\in V$,定义\[
        \left(f,g\right)=\int_a^bf\left(t\right)g\left(t\right)\,\rmd t
    \]为$V$上的内积,$V$是一个无限维实内积空间.
}
\exa{}{}{
    设$V=\bbr\left[x\right]$\begin{align*}
         & f\left(x\right)=a_0+a_1x+\cdots+a_nx^n \\
         & g\left(x\right)=b_0+b_1x+\cdots+b_mx^m
    \end{align*}且$n\geqslant m.$定义\[
        \left(
        f\left(x\right),g\left(x\right)
        \right)\coloneqq a_0b_0+a_1b_1+\cdots+a_mb_m
    \]
}
\exa{}{}{
    设$V=M_n\left(\bbr\right),\bm{A},\bm{B}\in V$.定义\[
        \left(
        \bm{A},\bm{B}
        \right)\coloneqq \Tr\left(
        \bm{A}\bm{B}'
        \right)
    \]为Forbenius内积.
}
\dfn{范数}{范数}{
设$V$为一个内积空间,定义$\forall \bm{\alpha}\in V$的范数(长度)为\[
    \left\lVert \bm{\alpha}\right\rVert =\left(\bm{\alpha},\bm{\alpha}\right)^{\frac{1}{2}}
\]及$\bm{\alpha},\bm{\beta}\in V$的距离为\begin{align*}
    d\left(
    \bm{\alpha},\bm{\beta}
    \right) & =\left\lVert \bm{\alpha}-\bm{\beta}\right\rVert \\
            & =\left(
    \left(
    \bm{\alpha}-\bm{\beta}
    \right),\left(
    \bm{\alpha}-\bm{\beta}
    \right)
    \right)^{\frac{1}{2}}
\end{align*}
}
\exa{}{}{
    设$V=\bbr^n,$取标准内积,则$\bm{\alpha}=\begin{pmatrix}
            x_1 \\x_2\\\vdots\\x_n
        \end{pmatrix}$的范数为\[
        \left\lVert \bm{\alpha}\right\rVert =\sqrt{
            x_1^2+x_2^2+\cdots+x_n^2
        }
    \]

    设$V=\bbc^n,$取标准内积,则$\bm{\alpha}=\begin{pmatrix}
            x_1 \\x_2\\\vdots\\x_n
        \end{pmatrix}$的范数为\[
        \left\lVert \bm{\alpha}\right\rVert =\sqrt{
            x_1\overline{x_1}+x_2\overline{x_2}+\cdots+x_n\overline{x_n}
        }
    \]
}
\pro{范数的基本性质}{范数的基本性质}{
    \begin{enumerate}[label=\arabic*)]
        \item 范数非负性:$
                  \left\lVert \bm{\alpha}\right\rVert \geqslant 0
              $且等号成立当且仅当$\bm{\alpha}=\bm{0}$
        \item 距离的对称性:$d\left(
                  \bm{\alpha},\bm{\beta}
                  \right)=d\left(
                  \bm{\beta},\bm{\alpha}
                  \right)$
    \end{enumerate}
}
\thm{范数的性质}{范数的性质}{
    设 $V$为内积空间,$\bm{\alpha},\bm{\beta}\in V,c$为一常数,则\begin{enumerate}[label=\arabic*)]
        \item 范数的齐次性:$\left\lVert c\bm{\alpha}\right\rVert = \left| c\right|\left\lVert \bm{\alpha}\right\rVert$
        \item Cauchy-Schwarz不等式:$\left|\left(
                  \bm{\alpha},\bm{\beta}
                  \right)\right|\leqslant \left\lVert \bm{\alpha}\right\rVert \left\lVert \bm{\beta}\right\rVert$且等号成立当且仅当$\bm{\alpha},\bm{\beta}$线性相关
        \item 三角不等式:$\left\lVert \bm{\alpha}+\bm{\beta}\right\rVert \leqslant \left\lVert \bm{\alpha}\right\rVert +\left\lVert \bm{\beta}\right\rVert$
    \end{enumerate}\begin{proof}
        \begin{enumerate}[label=\arabic*)]
            \item 是显然的.
            \item 一方面,$\bm{\alpha}=\bm{0}$时显然成立.另一方面,设$\bm{\alpha}\neq\bm{0}$,作\[
                      \bm{\gamma}=\bm{\beta}-\frac{
                          \left(
                          \bm{\beta},\bm{\alpha}
                          \right)
                      }{
                          \left\lVert \bm{\alpha}\right\rVert ^2
                      }\bm{\alpha}
                  \]容易知道$\left(
                      \bm{\gamma},\bm{\alpha}
                      \right)=0$且\begin{align*}
                      0\leqslant \left\lVert \bm{\gamma}\right\rVert ^2 & =\left(
                      \bm{\gamma},\bm{\gamma}
                      \right)                                                                                           \\
                                                                        & =\left(
                      \bm{\beta}-\frac{
                          \left(
                          \bm{\beta},\bm{\alpha}
                          \right)
                      }{
                          \left\lVert \bm{\alpha}\right\rVert ^2
                      }\bm{\alpha},\bm{\beta}-\frac{
                          \left(
                          \bm{\beta},\bm{\alpha}
                          \right)
                      }{
                          \left\lVert \bm{\alpha}\right\rVert ^2
                      }\bm{\alpha}
                      \right)                                                                                           \\
                                                                        & =\left(
                      \bm{\beta},\bm{\beta}
                      \right)-\frac{
                          \left(
                          \bm{\beta},\bm{\alpha}
                          \right)
                      }{
                          \left\lVert \bm{\alpha}\right\rVert ^2
                      }\left(
                      \bm{\alpha},\bm{\beta}
                      \right)                                                                                           \\
                                                                        & =\left\lVert \bm{\beta}\right\rVert ^2-\frac{
                          \left| \left(
                          \bm{\beta},\bm{\alpha}
                          \right)\right|^2
                      }{
                          \left\lVert \bm{\alpha}\right\rVert ^2
                      }
                  \end{align*}即\[
                      \left|\left(
                      \bm{\alpha},\bm{\beta}
                      \right)\right|\leqslant \left\lVert \bm{\alpha}\right\rVert \left\lVert \bm{\beta}\right\rVert\]且等号成立当且仅当$\bm{\gamma}=\bm{0}$即$\bm{\beta}$与$\bm{\alpha}$线性相关.
            \item 因为\begin{align*}
                      \left\lVert \bm{\alpha}+\bm{\beta}\right\rVert ^2 & =\left(
                      \bm{\alpha}+\bm{\beta},\bm{\alpha}+\bm{\beta}
                      \right)                                                                                                                                                                                                             \\
                                                                        & =\left(
                      \bm{\alpha},\bm{\alpha}
                      \right)+\left(
                      \bm{\alpha},\bm{\beta}
                      \right)+\left(
                      \bm{\beta},\bm{\alpha}
                      \right)+\left(
                      \bm{\beta},\bm{\beta}
                      \right)                                                                                                                                                                                                             \\
                                                                        & =\left\lVert \bm{\alpha}\right\rVert ^2+\left\lVert \bm{\beta}\right\rVert ^2+\left(
                      \bm{\alpha},\bm{\beta}
                      \right)+\overline{
                          \left(
                          \bm{\alpha},\bm{\beta}
                          \right)
                      }                                                                                                                                                                                                                   \\
                                                                        & =\left\lVert \bm{\alpha}\right\rVert ^2+\left\lVert \bm{\beta}\right\rVert ^2+2\Re\left(
                      \left(
                          \bm{\alpha},\bm{\beta}
                          \right)
                      \right)                                                                                                                                                                                                             \\
                                                                        & \leqslant \left\lVert \bm{\alpha}\right\rVert ^2+\left\lVert \bm{\beta}\right\rVert ^2+2\left|\left(
                      \bm{\alpha},\bm{\beta}
                      \right)\right|                                                                                                                                                                                                      \\
                                                                        & \leqslant \left\lVert \bm{\alpha}\right\rVert ^2+2\left\lVert \bm{\alpha}\right\rVert \left\lVert \bm{\beta}\right\rVert +\left\lVert \bm{\beta}\right\rVert ^2 \\
                                                                        & =\left(
                      \left\lVert \bm{\alpha}\right\rVert +\left\lVert \bm{\beta}\right\rVert
                      \right)^2\qedhere
                  \end{align*}
        \end{enumerate}
    \end{proof}
}
\exa{}{}{
    设$V=\bbr ^n$,取标准内积,于是Cauchy-Schwarz不等式推得Cauchy不等式\[
        \left(
        x_1y_1+x_2y_2+\cdots+x_ny_n
        \right)^2\leqslant \left(
        x_1^2+x_2^2+\cdots+x_n^2
        \right)\left(
        y_1^2+y_2^2+\cdots+y_n^2
        \right)
    \]

    设$V=C\left[a,b\right]$,取积分内积,于是Cauchy-Schwarz不等式推得Schwarz不等式\[
        \left(
        \int_a^bf\left(t\right)g\left(t\right)\,\rmd  t
        \right)^2\leqslant \int_a^bf^2\left(t\right)\,\rmd  t\int_a^bg^2\left(t\right)\,\rmd  t
    \]
}
\dfn{向量夹角}{向量夹角}{
    设$V$为内积空间,$\bm{0}\neq\bm{\alpha},\bm{0}\neq\bm{\beta}\in V,\bm{\alpha},\bm{\beta}$的夹角$\theta$的余弦值$\cos\theta$为\[
        \cos\theta=\begin{cases*}
            \cfrac{\left(
                \bm{\alpha},\bm{\beta}
                \right)}{
                \left\lVert \bm{\alpha}\right\rVert \left\lVert \bm{\beta}\right\rVert
            }\Longrightarrow\theta\in\left[0,\pi\right]            & ,$V$为实内积空间 \\
            \cfrac{
                \left|
                \left(
                \bm{\alpha},\bm{\beta}
                \right)
                \right|
            }{
                \left\lVert \bm{\alpha}\right\rVert \left\lVert \bm{\beta}\right\rVert
            }\Longrightarrow\theta\in\left[0,\cfrac{\pi}{2}\right] & ,$V$为复内积空间
        \end{cases*}
    \]
}
\dfn{向量的正交}{向量的正交}{
    若$\left(\bm{\alpha},\bm{\beta}\right)=0$,则称向量$\bm{\alpha},\bm{\beta}$正交或垂直,记作$\bm{\alpha}\perp\bm{\beta}$.
}
\clm{}{}{
    零向量与任意向量正交.
}
\cor{非零向量的正交与夹角}{非零向量的正交与夹角}{
    设$\bm{\alpha}\neq\bm{0},\bm{\beta}\neq\bm{0}$,则$\bm{\alpha}\perp\bm{\beta}$当且仅当$\cos\theta=0$即$\displaystyle\theta=\frac{\pi}{2}$.
}
\rem{}{}{
    在\cref{def:向量夹角}中,复内积空间定义下不取$\Re\left(\left(\bm{\alpha},\bm{\beta}\right)\right)$而取$\left|\left(\bm{\alpha},\bm{\beta}\right)\right|$正是因为尽管能够使得$\theta$取到$\left[0,\pi\right]$,却失去了\cref{cor:非零向量的正交与夹角}的几何意义.
}
\thm{勾股定理}{勾股定理}{
    若$\bm{\alpha}\perp\bm{\beta}$,则\[
        \left\lVert \bm{\alpha}+\bm{\beta}\right\rVert ^2=\left\lVert \bm{\alpha}\right\rVert ^2+\left\lVert \bm{\beta}\right\rVert ^2
    \]
}