\newpage
\section{自伴随算子}
\subsection{正交相似与酉相似}
\lem{标准正交基之间的过渡矩阵}{标准正交基之间的过渡矩阵}{
    内积空间$V$中两组标准正交基之间的过渡矩阵是正交阵或酉阵.\begin{proof}
        设$V$是欧式空间.取两组标准正交基$\left\{
            \bm{e}_1,\bm{e}_2,\cdots,\bm{e}_n
            \right\}$和$\left\{
            \bm{f}_1,\bm{f}_2,\cdots,\bm{f}_n
            \right\}$,过渡矩阵为$\bm{P}=\left(a_{ij}\right)_{n\times n}$,于是\[
            \left(
            \bm{f}_1,\bm{f}_2,\cdots,\bm{f}_n
            \right)=\left(
            \bm{e}_1,\bm{e}_2,\cdots,\bm{e}_n
            \right)\bm{P}\Longrightarrow \bm{f}_j = \sum_{i=1}^{n}a_{ij}\bm{e}_i
        \]于是\begin{align*}
            \delta_{jk} & =\left(
            \bm{f}_j,\bm{f}_k
            \right)                                   \\
                        & =\left(
            \sum_{i=1}^{n}a_{ij}\bm{e}_i,\sum_{i=1}^{n}a_{ik}\bm{e}_i
            \right)                                   \\
                        & =\sum_{i=1}^{n}a_{ij}a_{ik}
        \end{align*}于是$\bm{P}$的$n$个列向量是$\mathbb{R}^n$(标准内积)的一组标准正交基,根据\cref{thm:正交阵的判定},$\bm{P}$是正交阵.

        同理,若$V$是酉空间,则$\bm{P}$是酉阵.
    \end{proof}
}
事实上,若设$\bm{\varphi}\in\mathcal{L}\left(V\right)$在基$\left\{
    \bm{e}_1,\bm{e}_2,\cdots,\bm{e}_n
    \right\}$和$\left\{
    \bm{f}_1,\bm{f}_2,\cdots,\bm{f}_n
    \right\}$下的表示矩阵分别为$\bm{A}$和$\bm{B}$,根据\cref{thm:不同基下的表示矩阵的关系}即得\[
    \bm{B}=\bm{P}^{-1}\bm{A}\bm{P}\xlongequal{\cref{def:正交矩阵、酉矩阵}}\begin{cases*}
        \bm{P}'\bm{A}\bm{P}            & ,$V$是欧式空间 \\
        \overline{\bm{P}}'\bm{A}\bm{P} & ,$V$是酉空间
    \end{cases*}
\]
\dfn{正交相似与酉相似}{正交相似与酉相似}{
    正交相似与酉相似是特殊的相似关系.
    \begin{enumerate}[label=\arabic*)]
        \item 设$\bm{A},\bm{B}\in M_n\left(\mathbb{R}\right)$,若存在正交阵$\bm{P}$使得$\bm{B}=\bm{P}'\bm{AP}$,则称$\bm{A},\bm{B}$正交相似.
        \item 设$\bm{A},\bm{B}\in M_n\left(\mathbb{C}\right)$,若存在酉阵$\bm{P}$使得$\bm{B}=\overline{\bm{P}}'\bm{AP}$,则称$\bm{A},\bm{B}$酉相似.
    \end{enumerate}
}
剩余几节内容,我们主要考虑这样一个问题:是否可以找到一组标准正交基,使得某个线性算子在这组基下的表示矩阵尽可能地简单.一般来说这个问题是比较困难的,所以我们主要考虑三类相对简单的线性算子:自伴随算子、复正规算子、实正规算子.
\subsection{自伴随算子}
\dfn{自伴随算子}{自伴随算子}{
    设$\bm{\varphi}$是内积空间$V$上的线性算子,若$\bm{\varphi}^*=\bm{\varphi}$,则称$\bm{\varphi}$为自伴随算子.特别地,若$V$是欧式空间,则称$\bm{\varphi}$为对称算子;若$V$是酉空间,则称$\bm{\varphi}$为Hermite算子.
}
\exa{}{}{
    设$n$维内积空间$V$的子空间为$U$,则$V=U\oplus U^{\perp}$.设$\bm{E}$为$V$到$U$的正交投影算子,则$\forall\bm{\alpha},\bm{\beta}\in V:$\[
        \left(
        \bm{E}\left(
            \bm{\alpha}
            \right),\bm{\beta}
        \right)=\left(
        \bm{\alpha},\bm{E}\left(
            \bm{\beta}
            \right)
        \right)
    \]即$\bm{E}^*=\bm{E}$,故$\bm{E}$是自伴随算子.
}
\lem{自伴随算子的判定}{自伴随算子的判定}{
    设$\bm{\varphi}$是$n$维内积空间上的线性算子,则$\bm{\varphi}$是自伴随算子当且仅当$\bm{\varphi}$在任一组(或者某一组)基下的表示矩阵是\[\begin{cases*}
            \text{实对称阵}              & ,$V$是欧式空间 \\
            \textup{Hermite}\text{阵} & ,$V$是酉空间
        \end{cases*}\]\begin{proof}
        取一组标准正交基$\left\{
            \bm{e}_1,\bm{e}_2,\cdots,\bm{e}_n
            \right\}$,设$\bm{\varphi}$的表示矩阵为$\bm{A}$,根据\cref{thm:伴随算子的表示矩阵},$\bm{\varphi}^*$在这组基下的表示矩阵为$\bm{A}'/\overline{\bm{A}}'$.那么因为$\bm{\varphi}^*=\bm{\varphi}$,于是\[
            \bm{A}=\begin{cases*}
                \bm{A}'            & ,$V$是欧式空间 \\
                \overline{\bm{A}}' & ,$V$是酉空间
            \end{cases*}\qedhere
        \]
    \end{proof}
}
\lem{自伴随算子的特征值与特征向量}{自伴随算子的特征值与特征向量}{
    设$\bm{\varphi}$为酉空间$V$上的自伴随算子,则$\bm{\varphi}$的特征值都是实数,且对应不同特征值的特征向量相互正交.\begin{proof}
        任取$\bm{\varphi}$的特征值$\lambda$和对应的特征向量$\bm{\alpha}$即$\bm{\varphi}\left(\bm{\alpha}\right)=\lambda\bm{\alpha}$.容易有\[
            \lambda\left(
            \bm{\alpha},\bm{\alpha}
            \right)=\left(
            \bm{\varphi}\left(\bm{\alpha}\right),\bm{\alpha}
            \right)=\left(
            \bm{\alpha},\bm{\varphi}\left(\bm{\alpha}\right)
            \right)=\overline{\lambda}\left(
            \bm{\alpha},\bm{\alpha}
            \right)
        \]于是$\lambda=\overline{\lambda}\Longrightarrow\lambda\in\mathbb{R}$.

        另取不同特征值$\mu$,对应的特征向量$\bm{\beta}$,于是\begin{align*}
            \lambda\left(
            \bm{\alpha},\bm{\beta}
            \right) & =\left(
            \bm{\alpha},\bm{\varphi}\left(\bm{\beta}\right)
            \right)              \\
                    & =\mu\left(
            \bm{\alpha},\bm{\beta}
            \right)
        \end{align*}考虑到$\lambda\neq\mu\Longrightarrow\bm{\alpha}\perp\bm{\beta}.$
    \end{proof}
}
\cor{Hermite阵的特征值与特征向量}{Hermite阵的特征值与特征向量}{
    将\cref{lem:自伴随算子的特征值与特征向量}推到代数角度即得:Hermite阵的特征值都是实数,且对应不同特征值的特征向量相互正交.
}
\cor{实对称阵的特征值与特征向量}{实对称阵的特征值与特征向量}{
    因为实对称阵都是特殊的Hermite阵,故根据\cref{cor:Hermite阵的特征值与特征向量}得:实对称阵的特征值都是实数,且对应不同特征值的特征向量相互正交.
}
\cor{欧氏空间的自伴随算子的特征值与特征向量}{欧氏空间的自伴随算子的特征值与特征向量}{
    根据\cref{cor:实对称阵的特征值与特征向量},设$\bm{\varphi}$是欧式空间$V$上的自伴随算子,则$\bm{\varphi}$的特征值都是实数,且对应不同特征值的特征向量相互正交.
}
\thm{}{实对角阵表示矩阵的存在性}{
    设$n$维内积空间$V$上的自伴随算子$\bm{\varphi}$,则必存在一组标准正交基$\left\{
        \bm{e}_1,\bm{e}_2,\cdots,\bm{e}_n
        \right\}$,使得$\bm{\varphi}$在这组基下的表示矩阵是实对角阵.\begin{proof}
        取$\bm{\varphi}$的一个特征值$\lambda_1$和对应的特征向量$\bm{\alpha}\in V$,令$\displaystyle\bm{e}_1=\frac{\bm{\alpha}}{\left\lVert\bm{\alpha}\right\rVert}$即单位特征向量.

        下面对维数进行归纳,$n=1$显然.

        设维数$\dim V<n$成立,证明$\dim V=n$的情形.令$U=L\left(\bm{e}_1\right)^{\perp},\dim U=n-1$.因为$\bm{e}_1$是特征向量,由\cref{lem:特征子空间是不变子空间}知$L\left(\bm{e}_1\right)$是$\bm{\varphi}$-不变子空间,由$\cref{prop:伴随的不变子空间与特征值}$知$U$是$\bm{\varphi}^*=\bm{\varphi}$-不变子空间.做限制$\bm{\varphi}\mid_{U},\forall\bm{u},\bm{v}\in U\subseteq V:$\[
            \left(
            \bm{\varphi}\left(\bm{u}\right),\bm{v}
            \right)=\left(
            \bm{u},\bm{\varphi}\left(\bm{v}\right)
            \right)
        \]即\[
            \left(
            \bm{\varphi}\mid_U\left(
                \bm{u}
                \right),\bm{v}
            \right)=\left(
            \bm{u},\bm{\varphi}\mid_U\left(
                \bm{v}
                \right)
            \right)
        \]于是$\bm{\varphi}\mid_U$为自伴随,根据归纳假设,存在$U$的一组标准正交基$\left\{
            \bm{e}_2,\bm{e}_3,\cdots,\bm{e}_n
            \right\}$,使得$\bm{\varphi}\mid_U$在这组基下的表示矩阵是实对角阵$\mathrm{diag}\,\left\{\lambda_2,\lambda_3,\cdots,\lambda_n\right\}$.因为\[
            V=L\left(
            \bm{e}_1
            \right)\oplus U
        \]故$\left\{
            \bm{e}_1,\bm{e}_2,\bm{e}_3,\cdots,\bm{e}_n
            \right\}$是$V$的一组标准正交基,且$\bm{\varphi}$在这组基下的表示矩阵是\[
            \begin{pmatrix}
                \lambda_1 &           &        &           \\
                          & \lambda_2 &        &           \\
                          &           & \ddots &           \\
                          &           &        & \lambda_n
            \end{pmatrix}\qedhere
        \]
    \end{proof}
}
\thm{化归实对角阵}{化归实对角阵}{
    设$\bm{A}$为$n$阶实对称阵(Hermite阵),则存在正交阵(酉阵)$\bm{P}$,使得$\bm{P}'\bm{AP}\left(\overline{\bm{P}}'\bm{AP}\right)$是实对角阵.\begin{proof}
        构造\[\bm{\varphi}\in\mathcal{L}\left(\mathbb{C}^n\right):\bm{\alpha}\longmapsto\bm{A\alpha}\]取标准内积,这是一个线性算子.注意到$\bm{\varphi}$在标准单位列向量组$\left\{
            \bm{e}_1,\bm{e}_2,\cdots,\bm{e}_n
            \right\}$下的表示矩阵为$\bm{A}$,是一个\textup{Hermite}阵,于是$\bm{\varphi}$自伴随.根据\cref{thm:实对角阵表示矩阵的存在性},存在$\mathbb{C}^n$的另外一组标准正交基$\left\{
            \bm{\alpha}_1,\bm{\alpha}_2,\cdots,\bm{\alpha}_n
            \right\}$使得$\bm{\varphi}$在这组基下的表示矩阵为实对角阵$\bm{\varLambda}$即\[
            \left(
            \bm{\varphi}\left(
                \bm{\alpha}_1
                \right),\bm{\varphi}\left(
                \bm{\alpha}_2
                \right),\cdots,\bm{\varphi}\left(
                \bm{\alpha}_n
                \right)
            \right)=\left(
            \bm{\alpha}_1,\bm{\alpha}_2,\cdots,\bm{\alpha}_n
            \right)\bm{\varLambda}
        \]即\[
            \bm{A}\left(
            \bm{\alpha}_1,\bm{\alpha}_2,\cdots,\bm{\alpha}_n
            \right)=\left(
            \bm{\alpha}_1,\bm{\alpha}_2,\cdots,\bm{\alpha}_n
            \right)\bm{\varLambda}
        \]令$\bm{P}=\left(
            \bm{\alpha}_1,\bm{\alpha}_2,\cdots,\bm{\alpha}_n
            \right)$,根据\cref{thm:酉矩阵的判定},$\bm{P}$为酉阵,于是\[
            \overline{\bm{P}}'\bm{AP}=\bm{\varLambda}\qedhere
        \]
    \end{proof}
}
\cor{正交相似和酉相似的全系不变量}{正交相似和酉相似的全系不变量}{
    实对称阵(Hermite阵)在正交相似(酉相似)下的全系不变量是全体特征值,正交相似(酉相似)标准型就是全体特征值构成的对角阵.\begin{proof}
        证明实对称阵.首先一定是不变量,下证是全系不变量.

        设实对称阵$\bm{A},\bm{B}$的特征值均为$\lambda_1,\lambda_2,\cdots,\lambda_n$.根据\cref{thm:化归实对角阵},存在正交阵$\bm{P},\bm{Q}$使得\[
            \bm{P}'\bm{AP}=\begin{pmatrix}
                \lambda_1 &           &        &           \\
                          & \lambda_2 &        &           \\
                          &           & \ddots &           \\
                          &           &        & \lambda_n
            \end{pmatrix},\bm{Q}'\bm{BQ}=\begin{pmatrix}
                \lambda_{i_1} &               &        &               \\
                              & \lambda_{i_2} &        &               \\
                              &               & \ddots &               \\
                              &               &        & \lambda_{i_n}
            \end{pmatrix}
        \]只要证明这两个对角阵正交相似即可($\left(i_1,i_2,\cdots,i_n\right)$是$\left(1,2,\cdots,n\right)$的全排列).注意到合同变换\[
            \bm{P}_{ij}\mathrm{diag}\,\left\{
            \lambda_1,\lambda_2,\cdots,\lambda_i,\cdots,\lambda_j,\cdots,\lambda_n
            \right\}\bm{P}_{ij}= \mathrm{diag}\,\left\{
            \lambda_1,\lambda_2,\cdots,\lambda_j,\cdots,\lambda_i,\cdots,\lambda_n
            \right\}
        \]也是正交相似变换$\left(\bm{P}_{ij}'\bm{P}_{ij}=\bm{I}_n\right)$.于是若干个这样正交相似变换的复合即将\[\mathrm{diag}\,\left\{
            \lambda_{i_1},\lambda_{i_2},\cdots,\lambda_{i_n}
            \right\}\]变为\[\mathrm{diag}\,\left\{
            \lambda_1,\lambda_2,\cdots,\lambda_n
            \right\}\]于是$\bm{A},\bm{B}$正交相似.进一步地,正交相似标准型即\[
            \begin{pmatrix}
                \lambda_1 &           &        &           \\
                          & \lambda_2 &        &           \\
                          &           & \ddots &           \\
                          &           &        & \lambda_n
            \end{pmatrix}
        \]

        酉相似同理.
    \end{proof}
}
\rem{}{}{
    因为正交相似本身\[
        \bm{B}= \bm{P}'\bm{AP}= \bm{P}^{-1}\bm{AP}
    \]所以它既是合同变换,也是相似变换.

    几何地看,$f=\bm{x}'\bm{Ax}$作$\bm{x}=\bm{Cy}$时要保持度量,注意到$\bm{C}$其实是基向量之间的过渡矩阵:\[
        \left(
        \bm{f}_1,\bm{f}_2,\cdots,\bm{f}_n
        \right)=\left(
        \bm{e}_1,\bm{e}_2,\cdots,\bm{e}_n
        \right)\bm{C}
    \]这要求$\left\{
        \bm{f}_1,\bm{f}_2,\cdots,\bm{f}_n
        \right\}$应当也是一组标准正交基,根据\cref{lem:标准正交基之间的过渡矩阵},标准正交基之间的过渡矩阵是正交阵,所以$\bm{C}$是正交阵,这就是一种正交相似变换.
}
\cor{}{实二次型的正交相似变换}{
    设实二次型$f=\bm{x}'\bm{Ax},$实对称阵$\bm{A}$的特征值为$\lambda_1,\lambda_2,\cdots,\lambda_n$,则存在正交相似变换$\bm{x}=\bm{Py}$使得$f$化为\[
        \lambda_1y_1^2+\lambda_2y_2^2+\cdots+\lambda_ny_n^2
    \]它的正惯性指数即正特征值的个数,负惯性指数即负特征值的个数,秩等于非零特征值的个数.
}
\cor{}{半正定与半负定的确定}{
    设$n$阶实对称阵$\bm{A}$,则\begin{enumerate}[label=\arabic*)]
        \item $\bm{A}$正定当且仅当$\bm{A}$的全体特征值都大于$0$
        \item $\bm{A}$半正定当且仅当$\bm{A}$的全体特征值都大于等于$0$
        \item $\bm{A}$负定当且仅当$\bm{A}$的全体特征值都小于$0$
        \item $\bm{A}$半负定当且仅当$\bm{A}$的全体特征值都小于等于$0$
    \end{enumerate}
}
\exa{}{}{
    \[\bm{A}=
        \begin{pmatrix}
            4 & 2 & 2 \\
            2 & 4 & 2 \\
            2 & 2 & 4
        \end{pmatrix}
    \]求$\bm{P}$使得$\bm{P}'\bm{AP}$是实对角阵.\begin{solution}
        \begin{align*}
            \left|
            \lambda\bm{I}-\bm{A}
            \right| & =\left|
            \left(\lambda-2\right)\bm{I}_3-\begin{pmatrix}
                                               2 \\2\\2
                                           \end{pmatrix}\left(1,1,1\right)
            \right|                                                                                                                    \\
                    & \xlongequal{\cref{thm:特征多项式的降阶公式}}\left(\lambda-2\right)^2\left|\lambda-2-\left(1,1,1\right)\begin{pmatrix}
                                                                                                                      2 \\2\\2
                                                                                                                  \end{pmatrix}\right| \\
                    & =\left(
            \lambda-2
            \right)^2\left(
            \lambda-8
            \right)
        \end{align*}于是$\lambda_1=\lambda_2=2,\lambda_3=8$,$\lambda_3=8$对应特征向量$\bm{\alpha}_1=\begin{pmatrix}
                1 \\1\\1
            \end{pmatrix}$,而$\lambda_1=\lambda_2=2$代入得到线性无关但相互不正交的\[
            \bm{\alpha}_2=\begin{pmatrix}
                -1 \\1\\0
            \end{pmatrix},\bm{\alpha}_3=\begin{pmatrix}
                -1 \\0\\1
            \end{pmatrix}
        \]那么对这两个特征向量作正交化得到\begin{align*}
            \bm{\alpha}_3' & =\bm{\alpha}_3-\frac{\left(
                \bm{\alpha}_3,\bm{\alpha}_2
            \right)}{\left\lVert\bm{\alpha}_2\right\rVert^2}\bm{\alpha}_2 \\
                           & =\begin{pmatrix}
                                  -\cfrac{1}{2} \\-\cfrac{1}{2}\\1
                              \end{pmatrix}
        \end{align*}对所得向量均作单位化\newpage\begin{align*}
            \bm{\alpha}_1=\begin{pmatrix}
                              1 \\1\\1
                          \end{pmatrix}    & \longrightarrow\bm{e}_1=\begin{pmatrix}
                                                                         \cfrac{1}{\sqrt{3}} \\\cfrac{1}{\sqrt{3}}\\\cfrac{1}{\sqrt{3}}
                                                                     \end{pmatrix}    \\
            \bm{\alpha}_2=\begin{pmatrix}
                              -1 \\1\\0
                          \end{pmatrix}    & \longrightarrow\bm{e}_2=\begin{pmatrix}
                                                                         -\cfrac{1}{\sqrt{2}} \\\cfrac{1}{\sqrt{2}}\\0
                                                                     \end{pmatrix}                                 \\
            \bm{\alpha}_3'=\begin{pmatrix}
                               -\cfrac{1}{2} \\-\cfrac{1}{2}\\1
                           \end{pmatrix} & \longrightarrow\bm{e}_3=\begin{pmatrix}
                                                                       -\cfrac{1}{\sqrt{6}} \\-\cfrac{1}{\sqrt{6}}\\\cfrac{2}{\sqrt{6}}
                                                                   \end{pmatrix}
        \end{align*}所求$\bm{P}=\left(\bm{e}_1,\bm{e}_2,\bm{e}_3\right)$使得\[
            \bm{P}'\bm{AP}=\begin{pmatrix}
                8 &   &   \\
                  & 2 &   \\
                  &   & 2
            \end{pmatrix}
        \]
    \end{solution}
}
\exa{}{}{
    $\bm{A}$的特征值为$0,3,3$,$0$对应的特征向量为$\bm{v}_1=\left(1,1,1\right)',3$对应的其中一个特征向量为$\bm{v}_2=\left(-1,1,0\right)'$,求$\bm{A}$.\begin{proof}
        设$3$对应的另一个特征向量为$\bm{v}_3=\begin{pmatrix}
                x_1 \\x_2\\x_3
            \end{pmatrix}$且$\left\{
            \bm{v}_1,\bm{v}_2,\bm{v}_3
            \right\}$两两正交:\[
            \begin{cases*}
                x_1+x_2+x_3=0 \\
                -x_1+x_2=0
            \end{cases*}\Longrightarrow\bm{v}_3=\begin{pmatrix}
                1 \\1\\-2
            \end{pmatrix}
        \]将他们单位化得到

        \begin{align*}
            \bm{e}_1=\begin{pmatrix}
                         \cfrac{1}{\sqrt{3}} \\\cfrac{1}{\sqrt{3}}\\\cfrac{1}{\sqrt{3}}
                     \end{pmatrix},\bm{e}_2=\begin{pmatrix}
                                                -\cfrac{1}{\sqrt{2}} \\\cfrac{1}{\sqrt{2}}\\0
                                            \end{pmatrix},\bm{e}_3=\begin{pmatrix}
                                                                       \cfrac{1}{\sqrt{6}} \\\cfrac{1}{\sqrt{6}}\\-\cfrac{2}{\sqrt{6}}
                                                                   \end{pmatrix}
        \end{align*}于是\[
            \bm{P}=\begin{pmatrix}
                \cfrac{1}{\sqrt{3}} & -\cfrac{1}{\sqrt{2}} & \cfrac{1}{\sqrt{6}}  \\
                \cfrac{1}{\sqrt{3}} & \cfrac{1}{\sqrt{2}}  & \cfrac{1}{\sqrt{6}}  \\
                \cfrac{1}{\sqrt{3}} & 0                    & -\cfrac{2}{\sqrt{6}}
            \end{pmatrix}
        \]使得\[
            \bm{P}'\bm{AP}=\begin{pmatrix}
                0 &   &   \\
                  & 3 &   \\
                  &   & 3
            \end{pmatrix}
        \]然后反求出\[
            \bm{A}=\begin{pmatrix}
                -2 & -1 & 1  \\
                1  & -2 & 1  \\
                1  & 1  & -2
            \end{pmatrix}
        \]
    \end{proof}
}
\exa{}{}{
    设$n$阶实对称阵$\bm{A}:\bm{A}^3=\bm{I}_n$,求证:$\bm{A}=\bm{I}_n$.\begin{proof}
        任取$\bm{A}$的特征值$\lambda\in\mathbb{R}:\lambda^3=1\Longrightarrow\lambda=1$于是存在正交阵$\bm{P}$使得$\bm{P}'\bm{AP}=\bm{I}_n\Longrightarrow\bm{A}=\bm{I}_n$.
    \end{proof}事实上,只要给出特征值均为实数即可证明:\[
        \left(
        \bm{A}-a\bm{I}_n
        \right)\left(
        \bm{A}^2+\bm{A}+\bm{I}_n
        \right)=\bm{O}
    \]而$\bm{A}^2+\bm{A}+\bm{I}_n$非异于是$\bm{A}=\bm{I}_n$.
}