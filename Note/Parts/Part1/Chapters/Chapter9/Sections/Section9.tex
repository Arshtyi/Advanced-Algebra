\newpage
\section{奇异值分解}
\subsection{正交相抵标准型}
\dfn{奇异值和奇异向量}{奇异值和奇异向量}{
    设$\bm{A}\in M_{m\times n}\left(\bbr \right)$,若$\exists0\leqslant\sigma\in\bbr ,\bm{0}\neq\bm{v}\in\bbr ^n,\bm{0}\neq\bm{u}\in\bbr ^m:$\[
        \bm{Av}=\sigma\bm{u},\bm{A}'\bm{u}=\sigma\bm{v}
    \]则$\sigma$称为$\bm{A}$的奇异值,$\bm{v},\bm{u}$分别称为$\bm{A}$关于$\sigma$的右、左奇异向量.
}
\dfn{线性映射的伴随}{线性映射的伴随}{
    设欧式空间$V^n,U^m$,线性映射$\bm{\varphi}:V\longrightarrow U$,若存在线性映射$\bm{\varphi}^*:U\longrightarrow V,\forall\bm{v}\in V,\bm{u}\in U:$\[
        \left(
        \bm{\varphi}\left(\bm{v}\right),\bm{u}
        \right)=\left(
        \bm{v},\bm{\varphi}^*\left(\bm{u}\right)
        \right)
    \]则称$\bm{\varphi}^*$是$\bm{\varphi}$的伴随.
}
\pro{}{线性映射的伴随的存在性}{
    \cref{def:线性映射的伴随}中的伴随$\bm{\varphi}^*$存在且唯一.\begin{proof}
        取$V,U$的标准正交基分别为$\left\{
            \bm{e}_1,\bm{e}_2,\cdots,\bm{e}_n
            \right\},\left\{
            \bm{f}_1,\bm{f}_2,\cdots,\bm{f}_m
            \right\}$,此时$\bm{\varphi}$的表示阵为$\bm{A},\bm{v}\in V$的坐标向量是$\bm{x}\in\bbr ^n,\bm{u}\in U$的坐标向量是$\bm{y}\in\bbr ^m$.\begin{align*}
            \left(
            \bm{\varphi}\left(\bm{v}\right),\bm{u}
            \right)_U=\left(\bm{Ax}\right)'\bm{y}=\bm{x}'\bm{A}'\bm{y}
        \end{align*}令其中$\bm{A}'$是$\bm{\varphi}^*$在给定基下的表示阵,存在性证毕.

        唯一性参考\cref{thm:伴随算子的存在性}.
    \end{proof}
}
\dfn{奇异值和奇异向量}{奇异值和奇异向量的定义2}{
    设欧式空间$V^n,U^m,$线性映射$\bm{\varphi}\in\call \left(V,U\right)$,伴随$\bm{\varphi}^*$,若$\exists0\leqslant\sigma\in\bbr ,\bm{0}\neq\bm{v}\in V,\bm{0}\neq\bm{u}\in U$使得\[
        \bm{\varphi}\left(\bm{v}\right)=\sigma\bm{u},\bm{\varphi}^*\left(\bm{u}\right)=\sigma\bm{v}
    \]则称$\sigma$是$\bm{\varphi}$的奇异值,$\bm{v},\bm{u}$分别是$\bm{\varphi}$关于$\sigma$的右、左奇异向量.
}
\rem{}{}{
    \cref{def:奇异值和奇异向量的定义2}下$\bm{\varphi}^*\bm{\varphi}\in\call \left(V\right)$是半正定自伴随算子且\[
        \bm{\varphi}^*\bm{\varphi}\left(\bm{v}\right)=\bm{\varphi}^*\left(
        \sigma\bm{u}
        \right)=\sigma^2\bm{v}
    \]同样地,$\bm{\varphi\varphi}^*\in\call \left(U\right)$是半正定自伴随算子且\[
        \bm{\varphi\varphi}^*\left(\bm{u}\right)=\bm{\varphi}\left(
        \sigma\bm{v}
        \right)=\sigma^2\bm{u}
    \]

    也就是说,$\sigma$是$\bm{\varphi}^*\bm{\varphi}$的特征值,$\bm{v}$是$\bm{\varphi}^*\bm{\varphi}$对应于$\sigma$的特征向量;$\sigma$是$\bm{\varphi\varphi}^*$的特征值,$\bm{u}$是$\bm{\varphi\varphi}^*$对应于$\sigma$的特征向量.
}
\thm{}{}{
    设欧式空间$V^n,U^m,$线性映射$\bm{\varphi}\in\call \left(V,U\right)$,则一定有$V$的一组标准正交基$\left\{
        \bm{e}_1,\bm{e}_2,\cdots,\bm{e}_n
        \right\}$和$U$的一组标准正交基$\left\{
        \bm{f}_1,\bm{f}_2,\cdots,\bm{f}_m
        \right\}$使得$\bm{\varphi}$的表示矩阵为\[
        \begin{pmatrix}
            \bm{S} & \bm{O} \\
            \bm{O} & \bm{O}
        \end{pmatrix},
        \bm{S}=\begin{pmatrix}
            \sigma_1 &          &        &          \\
                     & \sigma_2 &        &          \\
                     &          & \ddots &          \\
                     &          &        & \sigma_r
        \end{pmatrix}
    \]其中$\sigma_1\geqslant\sigma_2\geqslant\cdots\geqslant\sigma_r>0$是$\bm{\varphi}$的所有非零(或者是正)奇异值(这由\cref{thm:利用特征值判定}保证一致).\begin{proof}
        $\bm{\varphi}^*\bm{\varphi}$是半正定自伴随算子,则存在$V$的一组标准正交基$\left\{
            \bm{e}_1,\bm{e}_2,\cdots,\bm{e}_n
            \right\}$使得$\bm{\varphi}^*\bm{\varphi}$的表示矩阵为\[
            \diag \,\left\{
            \lambda_1,\lambda_2,\cdots,\lambda_r;0,0,\cdots,0
            \right\}
        \]其中$r=\rmr \,\left(\bm{\varphi}^*\bm{\varphi}\right)=\rmr \,\left(\bm{\varphi}\right)$且$\lambda_1\geqslant\lambda_2\geqslant\cdots\lambda_r>0$,也就是\[
            \bm{\varphi}^*\bm{\varphi}\left(
            \bm{e}_i
            \right)=\begin{cases*}
                \lambda_i\bm{e}_i, & $1\leqslant i\leqslant r$   \\
                \bm{0},            & $r+1\leqslant i\leqslant n$
            \end{cases*}
        \]

        $\forall1\leqslant i,j\leqslant r:$\begin{align*}
            \left(
            \bm{\varphi}\left(
                \bm{e}_i
                \right),\bm{\varphi}\left(
                \bm{e}_j
                \right)
            \right) & =\left(
            \bm{e}_i,\bm{\varphi}^*\bm{\varphi}\left(
                \bm{e}_j
                \right)
            \right)                         \\
                    & =\left(
            \bm{e}_i,\lambda_j\bm{e}_j
            \right)                         \\
                    & =\lambda_j\delta_{ij}
        \end{align*}

        $\forall r<i\leqslant n:$
        \begin{align*}
            \left(
            \bm{\varphi}\left(
                \bm{e}_i
                \right),\bm{\varphi}\left(
                \bm{e}_i
                \right)
            \right) & =\left(
            \bm{e}_i,\bm{\varphi}^*\bm{\varphi}\left(
                \bm{e}_i
                \right)
            \right)                                     \\
                    & =0                                \\
                    & \Longrightarrow\bm{\varphi}\left(
            \bm{e}_i
            \right)=\bm{0}
        \end{align*}令$\displaystyle\forall1\leqslant i\leqslant r:\sigma_i=\sqrt{\lambda_i},\bm{f}_i=\frac{1}{\sigma_i}\bm{\varphi}\left(\bm{e}_i\right)$,即$\left\{
            \bm{f}_1,\bm{f}_2,\cdots,\bm{f}_r
            \right\}$是$U$中两两正交的单位向量,扩张为$U$的标准正交基$\left\{
            \bm{f}_1,\bm{f}_2,\cdots,\bm{f}_m
            \right\}$,则\[
            \bm{\varphi}\left(\bm{e}_i\right)=\begin{cases*}
                \sigma_i\bm{f}_i, & $1\leqslant i\leqslant r$   \\
                \bm{0},           & $r+1\leqslant i\leqslant n$
            \end{cases*}
        \]即\begin{align*}
              & \left(
            \bm{\varphi}\left(
                \bm{e}_1
                \right),\bm{\varphi}\left(
                \bm{e}_2
                \right)
            ,\cdots,\bm{\varphi}\left(
                \bm{e}_r
                \right),\bm{\varphi}\left(
                \bm{e}_{r+1}
                \right),\cdots,\bm{\varphi}\left(
                \bm{e}_n
                \right)
            \right)    \\
            = & \left(
            \bm{f}_1,\bm{f}_2,\cdots,\bm{f}_r,\bm{f}_{r+1},\cdots,\bm{f}_m
            \right)\begin{pmatrix}
                       \sigma_1 &          &        &          & \bm{O} \\
                                & \sigma_2 &        &          & \bm{O} \\
                                &          & \ddots &          & \bm{O} \\
                                &          &        & \sigma_r & \bm{O} \\
                       \bm{O}   & \bm{O}   & \bm{O} & \bm{O}   & \bm{O}
                   \end{pmatrix}_{m\times n}
        \end{align*}即\begin{align*}
              & \left(
            \bm{\varphi}^*\left(
                \bm{f}_1
                \right),\bm{\varphi}^*\left(
                \bm{f}_2
                \right)
            ,\cdots,\bm{\varphi}^*\left(
                \bm{f}_r
                \right),\bm{\varphi}^*\left(
                \bm{f}_{r+1}
                \right),\cdots,\bm{\varphi}^*\left(
                \bm{f}_m
                \right)
            \right)    \\
            = & \left(
            \bm{e}_1,\bm{e}_2,\cdots,\bm{e}_r,\bm{e}_{r+1},\cdots,\bm{e}_n
            \right)\begin{pmatrix}
                       \sigma_1 &          &        &          & \bm{O} \\
                                & \sigma_2 &        &          & \bm{O} \\
                                &          & \ddots &          & \bm{O} \\
                                &          &        & \sigma_r & \bm{O} \\
                       \bm{O}   & \bm{O}   & \bm{O} & \bm{O}   & \bm{O}
                   \end{pmatrix}_{n\times m}
        \end{align*}即\[
            \bm{\varphi}^*\left(
            \bm{f}_i
            \right)=
            \begin{cases*}
                \sigma_i\bm{e}_i, & $1\leqslant i\leqslant r$   \\
                \bm{0},           & $r+1\leqslant i\leqslant m$
            \end{cases*}\qedhere
        \]
    \end{proof}
    转化为代数语言即得矩阵的正交相抵标准型.
}
\thm{正交相抵标准型}{正交相抵标准型}{
    设$\bm{A}\in M_{m\times n}\left(\bbr \right)$,则存在$m$阶正交阵$\bm{P}$和$n$阶正交阵$\bm{Q}$使得\[
        \bm{P}'\bm{AQ}=\begin{pmatrix}
            \bm{S} & \bm{O} \\
            \bm{O} & \bm{O}
        \end{pmatrix},\bm{S}=\begin{pmatrix}
            \sigma_1 &          &        &          \\
                     & \sigma_2 &        &          \\
                     &          & \ddots &          \\
                     &          &        & \sigma_r
        \end{pmatrix}
    \]其中$\sigma_1\geqslant\sigma_2\geqslant\cdots\geqslant\sigma_r>0$是$\bm{A}$的全体非零(或者说正)奇异值.
}
\subsection{奇异值分解}
\dfn{奇异值分解}{奇异值分解}{
    $\forall\bm{A}\in M_{m\times n}\left(\bbr \right)$,一定存在$m$阶正交阵$\bm{P}$和$n$阶正交阵$\bm{Q}$使得\[
        \bm{A}=\bm{P}\begin{pmatrix}
            \bm{S} & \bm{O} \\
            \bm{O} & \bm{O}
        \end{pmatrix}\bm{Q}'
    \]称为$\bm{A}$的奇异值分解(SVD).
}
\clm{}{}{
    求$\bm{A}\in M_{m\times n}\left(\bbr \right)$的奇异值分解的求法:\begin{enumerate}[label=\arabic*)]
        \item 求$\bm{A}'\bm{A}$的正交相似标准型即$n$阶正交阵$\bm{Q}$使得\[
                  \bm{Q}'\bm{A}'\bm{AQ}=\diag \,\left\{
                  \lambda_1,\lambda_2,\cdots,\lambda_r;0,0,\cdots,0
                  \right\}
              \]其中$r=\rmr \,\left(\bm{A}'\bm{A}\right)=\rmr \,\left(\bm{A}\right),\lambda_1\geqslant\lambda_2\geqslant\cdots\geqslant\lambda_r>0$
        \item 做分块$\bm{Q}=\left(
                  \bm{\alpha}_1,\bm{\alpha}_2,\cdots,\bm{\alpha}_n
                  \right),\sigma_i=\sqrt{\lambda_i}$,作$\forall1\leqslant i\leqslant r:$\[
                  \bm{\beta}_i=\frac{1}{\sigma_i}\bm{A}\bm{\alpha}_i
              \]即$\left\{
                  \bm{\beta}_1,\bm{\beta}_2,\cdots,\bm{\beta}_r
                  \right\}$是$\bbr ^m$中的两两正交的单位向量,扩张为$\bbr ^m$的标准正交基\[\left\{
                  \bm{\beta}_1,\bm{\beta}_2,\cdots,\bm{\beta}_r,\bm{\beta}_{r+1},\cdots,\bm{\beta}_m
                  \right\}\]
        \item 令正交阵$\bm{P}=\left(
                  \bm{\beta}_1,\bm{\beta}_2,\cdots,\bm{\beta}_r,
                  \bm{\beta}_{r+1},\cdots,\bm{\beta}_m
                  \right)$,于是\[
                  \bm{A}=\bm{P}\begin{pmatrix}
                      \bm{S} & \bm{O} \\
                      \bm{O} & \bm{O}
                  \end{pmatrix}\bm{Q}'
              \]
    \end{enumerate}
}
\subsection{SVD的应用}
$(1)$对$\bm{A}\in M_n\left(\bbr \right)$,奇异值分解等价于极分解.设极分解\[
    \bm{A}=\bm{OS}=\bm{O}\left(
    \bm{P}'\bm{\varLambda}\bm{P}
    \right)
\]其中\[
    \bm{\varLambda}=\begin{pmatrix}
        \lambda_1 &           &        &           &   &        &   \\
                  & \lambda_2 &        &           &   &        &   \\
                  &           & \ddots &           &   &        &   \\
                  &           &        & \lambda_r &   &        &   \\
                  &           &        &           & 0 &        &   \\
                  &           &        &           &   & \ddots &   \\
                  &           &        &           &   &        & 0
    \end{pmatrix}
\]于是\[
    \bm{A}=\left(
    \bm{OP}'
    \right)\bm{AP}
\]
另一方面,设SVD\[
    \bm{A}=\bm{P}\begin{pmatrix}
        \bm{S} & \bm{O} \\
        \bm{O} & \bm{O}
    \end{pmatrix}\bm{Q}'=\left(
    \bm{PQ}'
    \right)\left(
    \bm{Q}'\begin{pmatrix}
        \bm{S} & \bm{O} \\
        \bm{O} & \bm{O}
    \end{pmatrix}\bm{Q}'
    \right)
\]

$(2)$主成分分析(PCA):\[
    \bm{A}=\bm{P}\begin{pmatrix}
        \bm{S} & \bm{O} \\
        \bm{O} & \bm{O}
    \end{pmatrix}\bm{Q}'
\]正交阵没有对数据产生根本的影响,他的所有信息均在奇异值矩阵$\bm{S}$中.例如\begin{enumerate}[label=\arabic*)]
    \item 图片去噪音:$\bm{A}\in M_{m\times n}\left(\bbr \right)$,取其$\begin{pmatrix}
                  \sigma_1 &          &        &          \\
                           & \sigma_2 &        &          \\
                           &          & \ddots &          \\
                           &          &        & \sigma_r
              \end{pmatrix}$,实际中,再把充分小的奇异值视作$0$
    \item 图片存储:对于$\bm{A}\in M_{m\times n}\left(\bbr \right)$,至少利用$m\cdot n$来存储,做分块\begin{align*}
              \bm{A} & =\left(
              \bm{\beta}_1,\bm{\beta}_2,\cdots,\bm{\beta}_r,\bm{\beta}_{r+1},\cdots,\bm{\beta}_m
              \right)\begin{pmatrix}
                         \bm{S} & \bm{O} \\
                         \bm{O} & \bm{O}
                     \end{pmatrix}\begin{pmatrix}
                                      \bm{\alpha}_1' \\\bm{\alpha}_2'\\\vdots\\\bm{\alpha}_n'
                                  \end{pmatrix}                                                \\
                     & =\sigma_1\bm{\beta}_1\bm{\alpha}_1'+\sigma_2\bm{\beta}_2\bm{\alpha}_2'+\cdots+\sigma_r\bm{\beta}_r\bm{\alpha}_r'
          \end{align*}存储量变为$r\left(
              m+n+1
              \right)$
\end{enumerate}

$(3)$潜在语言索引(LSI):考虑$m=100000\times 10000=n$的矩阵$\bm{A}$,作SVD只要考虑$r\times r$的那个矩阵就好了.
\exa{}{}{
    设$\bm{\varphi},\bm{\psi}\in\call \left(V^n,U^m\right)$使得$\bm{\varphi}^*\bm{\varphi}=\bm{\psi}^*\bm{\psi}$,证明:存在正交算子$\bm{\omega}\in\call \left(U\right)$使得$\bm{\varphi}=\bm{\omega\psi}$.\begin{proof}
        取$V$的标准正交基$\left\{
            \bm{e}_1,\bm{e}_2,\cdots,\bm{e}_n
            \right\}$使得$\bm{\varphi}^*\bm{\varphi}$和$\bm{\psi}^*\bm{\psi}$的表示阵为\[
            \diag \,\left\{
            \lambda_1,\lambda_2,\cdots,\lambda_r;0,0,\cdots,0
            \right\}
        \]其中$\lambda_1\geqslant\lambda_2\geqslant\cdots\geqslant\lambda_r>0$.作$\displaystyle\forall1\leqslant i\leqslant r:\sigma_i=\sqrt{\lambda_i},\bm{f}_i=\frac{1}{\sigma_i}\bm{\varphi}\left(\bm{e}_i\right)$,扩张为$U$的标准正交基$\left\{
            \bm{f}_1,\bm{f}_2,\cdots,\bm{f}_r,\bm{f}_{r+1},\cdots,\bm{f}_m
            \right\}$,另做$\displaystyle\forall1\leqslant i\leqslant r:\bm{g}_i=\frac{1}{\sigma_i}\bm{\psi}\left(\bm{e}_i\right)$并扩张为另一组$U$的标准正交基$
            \left\{
            \bm{g}_1,\bm{g}_2,\cdots,\bm{g}_r,\bm{g}_{r+1},\cdots,\bm{g}_m
            \right\}
        $于是\[
            \bm{\varphi}\left(
            \bm{e}_i
            \right)=\begin{cases*}
                \sigma_i\bm{f}_i, & $1\leqslant i\leqslant r$   \\
                \bm{0},           & $r+1\leqslant i\leqslant n$
            \end{cases*},\bm{\psi}\left(
            \bm{e}_i
            \right)=\begin{cases*}
                \sigma_i\bm{g}_i, & $1\leqslant i\leqslant r$   \\
                \bm{0},           & $r+1\leqslant i\leqslant n$
            \end{cases*}
        \]作正交算子\begin{align*}
            \bm{\omega}:U & \longrightarrow U   \\
            \bm{g}_i      & \longmapsto\bm{f}_i
        \end{align*}最后来验证.

        $\forall1\leqslant i\leqslant r:$\begin{align*}
            \bm{\omega\psi}\left(
            \bm{e}_i
            \right) & =\bm{\omega}\left(
            \sigma_i\bm{g}_i
            \right)                              \\
                    & =\sigma_i\bm{\omega}\left(
            \bm{g}_i
            \right)                              \\
                    & =\sigma_i\bm{f}_i          \\
        \end{align*}
        $\forall r<i\leqslant n:$\[
            \bm{\omega\psi}\left(
            \bm{e}_i
            \right)=\bm{0}
        \]于是\[
            \bm{\varphi}=\bm{\omega\psi}\qedhere
        \]
    \end{proof}
}
$(4)$Moore-Penrose广义逆.
\dfn{Moore-Penrose广义逆}{Moore-Penrose广义逆}{
    设欧式空间$V^n,U^m$,线性映射$\bm{\varphi}\in\call \left(V,U\right)$,若存在线性映射$\bm{\varphi}^{\dagger}\in\call \left(U,V\right)$,满足\begin{enumerate}[label=\arabic*)]
        \item $\bm{\varphi}\bm{\varphi}^{\dagger}\bm{\varphi}=\bm{\varphi}$
        \item $\bm{\varphi}^{\dagger}\bm{\varphi}\bm{\varphi}^{\dagger}=\bm{\varphi}^{\dagger}$
        \item $\bm{\varphi}\bm{\varphi}^{\dagger}$和$\bm{\varphi}^{\dagger}\bm{\varphi}$分别是是$V$和$U$上的自伴随算子
    \end{enumerate}则上述$\bm{\varphi}^{\dagger}$称为$\bm{\varphi}$的Moore-Penrose广义逆.
}
\thm{Moore-Penrose广义逆的存在性}{Moore-Penrose广义逆的存在性}{
    \cref{def:Moore-Penrose广义逆}Moore-Penrose广义逆$\bm{\varphi}^{\dagger}$存在且唯一.\begin{proof}
        做限制$\bm{\xi}=\bm{\varphi}\mid_{\left(\Ker \,\bm{\varphi}\right)^{\perp}}:\left(\Ker \,\bm{\varphi}\right)^{\perp}\longrightarrow\Image \,\bm{\varphi}$,于是$\Ker \,\bm{\xi}=\Ker \,\bm{\varphi}\cap\left(\Ker \,\bm{\varphi}\right)^{\perp}=0$,又\[
            \dim \left(
            \Ker \,\bm{\varphi}
            \right)^{\perp}=\dim V-\dim\Ker \,\bm{\varphi}=\dim\Image \,\bm{\varphi}
        \]即$\bm{\xi}:\left(
            \Ker \,\bm{\varphi}
            \right)^{\perp}\longrightarrow\Image \,\bm{\varphi}$是线性同构.

        定义\[
            \bm{\varphi}^{\dagger}\left(\bm{u}\right)=\begin{cases*}
                \bm{\xi}^{-1}\left(\bm{u}\right), & $\bm{u}\in\Image \,\bm{\varphi}$                      \\
                \bm{0},                           & $\bm{u}\in\left(\Image \,\bm{\varphi}\right)^{\perp}$
            \end{cases*}
        \]得到$\bm{\varphi}^{\dagger}\in \call \left(
            U,V
            \right)$.

        考虑$\bm{\varphi}$的奇异值分解.也就是取$V$的一组标准正交基$\left\{
            \bm{e}_1,\bm{e}_2,\cdots,\bm{e}_n
            \right\}$和$U$的一组标准正交基$\left\{
            \bm{f}_1,\bm{f}_2,\cdots,\bm{f}_m
            \right\}$使得\[
            \bm{\varphi}\left(
            \bm{e}_i
            \right)=
            \begin{cases*}
                \sigma_i\bm{f}_i, & $1\leqslant i\leqslant r$   \\
                \bm{0},           & $r+1\leqslant i\leqslant n$
            \end{cases*}
        \]于是\[
            \Ker \,\bm{\varphi}=L\left(
            \bm{e}_{r+1},\bm{e}_{r+2},\cdots,\bm{e}_n
            \right),\left(
            \Ker \,\bm{\varphi}
            \right)^{\perp}=L\left(
            \bm{e}_1,\bm{e}_2,\cdots,\bm{e}_r
            \right)
        \]\[
            \Image \,\bm{\varphi}=L\left(
            \bm{f}_1,\bm{f}_2,\cdots,\bm{f}_r
            \right),\left(
            \Image \,\bm{\varphi}
            \right)^{\perp}=L\left(
            \bm{f}_{r+1},\bm{f}_{r+2},\cdots,\bm{f}_m
            \right)
        \]于是$\forall1\leqslant i\leqslant r:$\[
            \bm{\xi}\left(
            \bm{e}_i
            \right)=\bm{\varphi}\left(
            \bm{e}_i
            \right)=\sigma_i\bm{f}_i
        \]从而\[
            \bm{\varphi}^{\dagger}\left(
            \bm{f}_i
            \right)=\begin{cases*}
                \frac{1}{\sigma_i}\bm{e}_i, & $1\leqslant i\leqslant r$   \\
                \bm{0},                     & $r+1\leqslant i\leqslant m$
            \end{cases*}
        \]于是$\bm{\varphi}^{\dagger}$的表示阵为\[
            \begin{pmatrix}
                \cfrac{1}{\sigma_1} &                     &        &                     &   &        &   \\
                                    & \cfrac{1}{\sigma_2} &        &                     &   &        &   \\
                                    &                     & \ddots &                     &   &        &   \\
                                    &                     &        & \cfrac{1}{\sigma_r} &   &        &   \\
                                    &                     &        &                     & 0 &        &   \\
                                    &                     &        &                     &   & \ddots &   \\
                                    &                     &        &                     &   &        & 0
            \end{pmatrix}
        \]那么可以验证$(1),(2),(3)$成立.

        存在性证毕,唯一性见\cite{torsor3}.
    \end{proof}
}
\thm{Moore-Penrose广义逆}{Moore-Penrose广义逆}{
    设$\bm{A}\in M_{m\times n}\left(\bbr \right)$,则存在唯一的$\bm{A}^{\dagger}\in M_{n\times m}\left(\bbr \right)$满足\begin{enumerate}[label=\arabic*)]
        \item $\bm{AA}^{\dagger}\bm{A}=\bm{A}$
        \item $\bm{A}^{\dagger}\bm{AA}^{\dagger}=\bm{A}^{\dagger}$
        \item $\bm{AA}^{\dagger}$和$\bm{A}^{\dagger}\bm{A}$分别是$M_n\left(\bbr \right)$和$M_m\left(\bbr \right)$上的实对称阵
    \end{enumerate}上述$\bm{A}^{\dagger}$称为$\bm{A}$的Moore-Penrose广义逆.
}
容易知道,若$\bm{A}$的奇异值分解为\[
    \bm{A}=\bm{P}\begin{pmatrix}
        \bm{S} & \bm{O} \\
        \bm{O} & \bm{O}
    \end{pmatrix}\bm{Q}'
\]则$\bm{A}^{\dagger}$的奇异值分解为\[
    \bm{A}^{\dagger}=\bm{Q}\begin{pmatrix}
        \bm{S}^{-1} & \bm{O} \\
        \bm{O}      & \bm{O}
    \end{pmatrix}\bm{P}'
\]
\clm{}{}{
    也就是说,知道了$\bm{A}$的奇异值分解,就知道了$\bm{A}^{\dagger}$.
}
\clm{}{}{
    在$\bm{\varphi},\bm{A}$可逆时,$\bm{\varphi}^{\dagger}=\bm{\varphi}^{-1},\bm{A}^{\dagger}=\bm{A}^{-1}$.
}
\clm{}{}{
    若$\bm{\varphi}=\bm{0},\bm{A}=\bm{O},\bm{\varphi}^{\dagger}=\bm{0},\bm{A}^{\dagger}=\bm{O}.$
}
\lem{}{正交投影算子}{
    设欧式空间$V^n,U^m$,线性映射$\bm{\varphi}\in\call \left(V,U\right)$,$\bm{\varphi}^{\dagger}$是$\bm{\varphi}$的广义逆,则$\bm{\varphi}^{\dagger}\bm{\varphi}$是从$V$到$\left(
        \Ker \,\bm{\varphi}
        \right)^{\perp}$的正交投影算子,且$\bm{\varphi}\bm{\varphi}^{\dagger}$是从$U$到$\Image \,\bm{\varphi}$的正交投影算子.\begin{proof}
        $\forall1\leqslant i\leqslant r:$\[
            \bm{\varphi}^{\dagger}\bm{\varphi}\left(
            \bm{e}_i
            \right)=\bm{\varphi}^{\dagger}\left(
            \sigma_i\bm{f}_i
            \right)=\frac{1}{\sigma_i}\sigma_i\bm{e}_i=\bm{e}_i
        \]而$\forall r<i\leqslant n:$\[
            \bm{\varphi}^{\dagger}\bm{\varphi}\left(
            \bm{e}_i
            \right)=\bm{0}
        \]
    \end{proof}
}
求解线性方程组是高等代数最早的问题,我们已经圆满解决了有解从而求解的问题,但是如何对无解的情形进行一个逼近是未曾考虑的.
\thm{求解线性方程组}{求解线性方程组}{
    设$\bm{A}\in M_{m\times n}\left(\bbr \right),\bm{\beta}\in\bbr ^m,\bm{x}\in\bbr ^n$,考虑\[
        \bm{Ax}=\bm{\beta}\tag{$\ast$}
    \]
    \begin{enumerate}[label=\arabic*)]
        \item 若$\ast$有解,则$\bm{z}=\bm{A}^{\dagger}\bm{\beta}$是$\ast$的范数最小的解
        \item 若$\ast$无解,则$\bm{z}=\bm{A}^{\dagger}\bm{\beta}$是$\ast$的最佳逼近即$\forall\bm{x}\in\bbr ^n:$\[
                  \left\lVert\bm{A}\bm{z}-\bm{\beta}\right\rVert\leqslant\left\lVert\bm{A}\bm{x}-\bm{\beta}\right\rVert
              \]
    \end{enumerate}\begin{proof}
        $(1)$设$\bm{x}_0$是解:$\bm{\beta}=\bm{Ax}_0\Longrightarrow\bm{z}=\bm{A}^{\dagger}\bm{Ax}_0$,于是
        \begin{align*}
            \bm{Az}=\bm{AA}^{\dagger}\bm{Ax}_0=\bm{Ax}_0=\bm{\beta}
        \end{align*}即$\bm{z}$是解.

        \incfig[scale=0.9]{problem-1}

        如图,这样的分解$\bm{z}=\bm{A}^{\dagger}\bm{Ax}_0$一定是范数最小的.

        $(2)$
        \incfig[scale=0.9]{problem-2}

        \[
            \bm{Az}=\bm{AA}^{\dagger}\bm{\beta}\perp\left(
            \Image \,\bm{A}
            \right)^{\perp}
        \]考虑\begin{align*}
            \bm{Ax}-\bm{\beta} & =\bm{A}\left(
            \bm{x}-\bm{z}
            \right)+\bm{Az}-\bm{\beta}
        \end{align*}分别属于$\Image \,\bm{A}$和$\left(
            \Image \,\bm{A}
            \right)^{\perp}$,于是\[
            \left\lVert\bm{Ax}-\bm{\beta}\right\rVert^2\geqslant\left\lVert\bm{Az}-\bm{\beta}\right\rVert^2
        \]即$\bm{z}$是最佳逼近.
    \end{proof}
}
利用\cref{thm:求解线性方程组},实际中求解\[
    \bm{A}^{m\times n}\bm{x}^{n\times 1}=\bm{\beta}^{m\times 1}
\](其中$m\gg 0,\rmr \,\left(\bm{A}\right)=n$)这样的线性方程组无解时,就可以利用\[
    \bm{x}=\bm{A}^{\dagger}\bm{\beta}=\left(
    \bm{A}'\bm{A}
    \right)^{-1}\bm{A}'\bm{\beta}
\]作逼近.