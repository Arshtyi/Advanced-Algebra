\newpage
\section{Cauchy-Binet公式}
\rem{}{}{
    Cauchy-Binet公式关注于矩阵积的行列式.
}
\subsection{方阵积的行列式}
\clm{}{}{
    如\cref{thm:方阵积的行列式},设$\bm{A}$、$\bm{B}$均为$n$阶方阵,则
    \[
        \left|\bm{A}\bm{B}\right| =
        \left|\bm{A}\right|\left|\bm{B}\right|
    \]
}
\subsection{Cauchy-Binet公式}
\thm{Cauchy-Binet公式}{Cauchy-Binet公式}{
    对于矩阵$\bm{A}_{m \times n}$、$\bm{B}_{n \times m}$
    \begin{enumerate}[label=\arabic*)]
        \item $m > n,\left|\bm{AB}\right| = 0;$
        \item $\displaystyle
                  m \leqslant n,\left|\bm{AB}\right|
                  = \sum_{1\leqslant j_1 < j_2<\cdots
                      < j_m \leqslant n}
                  \bm{A}\begin{pmatrix}
                      1   & 2   & \cdots & m   \\
                      j_1 & j_2 & \cdots & j_m
                  \end{pmatrix}
                  \bm{B}\begin{pmatrix}
                      j_1 & j_2 & \cdots & j_m \\
                      1   & 2   & \cdots & m
                  \end{pmatrix}$
    \end{enumerate}
    \begin{proof}
        构造如下$m+n$阶方阵
        \[
            \begin{bmatrix}
                \bm{A}    & \bm{O} \\
                -\bm{I}_n & \bm{B}
            \end{bmatrix}
        \]

        主要思路仍然是第三类分块初等变换
        \begin{align*}
            \begin{vmatrix}
                \bm{A}    & \bm{O} \\
                -\bm{I}_n & \bm{B}
            \end{vmatrix}=\begin{vmatrix}
                              \bm{O}    & \bm{AB} \\
                              -\bm{I}_n & \bm{B}
                          \end{vmatrix}
        \end{align*}
        均对前$m$行作\cref{thm:Laplace定理}Laplace展开
        \begin{align*}
            RHS & =\left|\bm{AB}\right|\left(-1\right)^{
                1+2+\cdots+m+\left(n+1\right)+\cdots+\left(n+m\right)
            }\left|-\bm{I}_n\right|                                        \\
                & =\left(-1\right)^{n\left(m+1\right)}\left|\bm{AB}\right|
        \end{align*}

        当$m>n$时,包含在前$m$行的任一$m$阶子式至少一列为$0$即行列式为$0$.由\cref{thm:Laplace定理}知
        \[
            LHS = 0
        \]
        故$\left|\bm{AB}\right|=0$.

        当$m\leqslant n$时,由\cref{thm:Laplace定理}Laplace定理(代数余子式记作$\bm{C}$)
        \[
            LHS =\sum_{1\leqslant j_1<j_2<\cdots<j_m\leqslant n}\bm{A}\begin{pmatrix}
                1   & 2   & \cdots & m   \\
                j_1 & j_2 & \cdots & j_m
            \end{pmatrix}\bm{C}\begin{pmatrix}
                1   & 2   & \cdots & m   \\
                j_1 & j_2 & \cdots & j_m
            \end{pmatrix}
        \]
        又代数余子式
        \begin{align*}
            \bm{C}\begin{pmatrix}
                      1   & 2   & \cdots & m   \\
                      j_1 & j_2 & \cdots & j_m
                  \end{pmatrix}=\left(-1
            \right)^{1+2+\cdots+m+j_1
                +j_2+\cdots+j_m}\left|-\bm{e}_{i_1},-\bm{e}_{i_2},\cdots,-\bm{e}_{i_{n-m}},\bm{B}\right|
        \end{align*}
        其中,$1\leqslant i_1<i_2<\cdots<i_{n-m}\leqslant n$为
        $1\leqslant j_1<j_2<\cdots<j_m
            \leqslant n$的余指标.$
            \bm{e}_i$为对应的标准单位
        列向量.

        记\[
            \bm{N}=\left(-\bm{e}_{i_1},-\bm{e}_{i_2},\cdots,-\bm{e}_{i_{n-m}},\bm{B}\right)
        \]
        对前$n-m$列作\cref{thm:Laplace定理}Laplace展开
        \begin{align*}
            \left|\bm{N}\right|=\left|-\bm{I}_{n-m}\right|
            \left(-1\right)^{i_1+i_2+\cdots+i_{n-m}+1+2+\cdots+\left(n-m\right)}\bm{B}\begin{pmatrix}
                                                                                          j_1 & j_2 & \cdots & j_m \\
                                                                                          1   & 2   & \cdots & m
                                                                                      \end{pmatrix}
        \end{align*}
        于是有
        \[
            LHS = \sum_{1\leqslant j_1<j_2<\cdots<j_m\leqslant n}
            \left(-1\right)^l\bm{A}\begin{pmatrix}
                1   & 2   & \cdots & m   \\
                j_1 & j_2 & \cdots & j_m
            \end{pmatrix}\bm{B}\begin{pmatrix}
                j_1 & j_2 & \cdots & j_m \\
                1   & 2   & \cdots & m
            \end{pmatrix}
        \]
        其中$l$为前面一系列指数和.只需证明$n\left(m+1\right)$和$l$模2同余或者二者之和为偶数.因为
        \begin{align*}
            l+n\left(m+1\right) & =
            1+2+\cdots+m                                                        \\
                                & +i_1+i_2+\cdots+i_{n-m}+j_1+j_2+\cdots+j_m    \\
                                & +1+2+\cdots+\left(n-m\right)+\left(n-m\right)
            \\
                                & +n\left(m+1\right)
        \end{align*}
        因为$1\leqslant i_1
            <i_2<\cdots<i_{n-m}
            \leqslant n$为
        $1\leqslant j_1
            <j_2<\cdots<j_m
            \leqslant n$的余指标,故
        \begin{align*}
            l+n\left(m+1\right) & =
            1+2+\cdots+m                                                                                                   \\
                                & +1+2+\cdots+n                                                                            \\
                                & +1+2+\cdots+\left(n-m\right)+\left(n-m\right)
            \\
                                & +n\left(m+1\right)                                                                       \\
                                & \equiv \left(m+1\right)+\cdots+n+1+2+\cdots+\left(n-m-1\right)+n\left(m+1\right)\pmod{2} \\
                                & \equiv n\left(m+1\right)+
            n\left(n-m\right)
            \pmod{2}                                                                                                       \\
                                & \equiv n\left(n+1\right)\pmod{2}                                                         \\
                                & \equiv 0\pmod{2}\qedhere
        \end{align*}
    \end{proof}
}
\cor{子式的Cauchy-Binet公式}{子式的Cauchy-Binet公式}{
    对于矩阵$\bm{A}_{m \times n}$、$
        \bm{B}_{n \times m}$,它们的积的$r\left(
        1 \leqslant
        r \leqslant m\right)$阶子式
    \begin{enumerate}[label=\arabic*)]
        \item $r > n$,则$\bm{AB}$
              任一
              $r$阶子式为零
        \item $ r \leqslant n$,则$\bm{AB}$
              的
              $r$阶子式
              \begin{align*}
                    & \bm{AB}\begin{pmatrix}
                                 i_1 & i_2 & \cdots & i_r \\
                                 j_1 & j_2 & \cdots & j_r
                             \end{pmatrix} \\
                  = &
                  \sum_{1 \leqslant  k_1 <k_2
                      < \cdots < k_r \leqslant
                      n}\bm{A}
                  \begin{pmatrix}
                      i_1 & i_2 & \cdots & i_r \\
                      k_1 & k_2 & \cdots & k_r
                  \end{pmatrix}
                  \bm{B}\begin{pmatrix}
                            k_1 & k_2 & \cdots & k_r \\
                            j_1 & j_2 & \cdots & j_r
                        \end{pmatrix}
              \end{align*}
    \end{enumerate}\begin{proof}
        设$\bm{C}_{m}=\bm{AB}$,那么
        \begin{align*}
            \bm{C}\begin{pmatrix}
                      i_1 & i_2 & \cdots & i_r \\
                      j_1 & j_2 & \cdots & j_r
                  \end{pmatrix}=\begin{vmatrix}
                                    \begin{bmatrix}
                    a_{i_11} & a_{i_12} & \cdots & a_{i_1n} \\
                    a_{i_21} & a_{i_22} & \cdots & a_{i_2n} \\
                    \vdots   & \vdots   &        & \vdots   \\
                    a_{i_r1} & a_{i_r2} & \cdots & a_{i_rn}
                \end{bmatrix}
                                    \begin{bmatrix}
                    b_{1j_1} & b_{1j_2} & \cdots & b_{1j_r} \\
                    b_{2j_1} & b_{2j_2} & \cdots & b_{2j_r} \\
                    \vdots   & \vdots   &        & \vdots   \\
                    b_{nj_1} & b_{nj_2} & \cdots & b_{nj_r}
                \end{bmatrix}
                                \end{vmatrix}
        \end{align*}
        由\cref{thm:Cauchy-Binet公式}Cauchy-Binet公式知:

        $(1)r > n$时
        \[
            \bm{C}\begin{pmatrix}
                i_1 & i_2 & \cdots & i_r \\
                j_1 & j_2 & \cdots & j_r
            \end{pmatrix}=0
        \]

        $(2)r\leqslant n$时
        \[
            \bm{C}\begin{pmatrix}
                i_1 & i_2 & \cdots & i_r \\
                j_1 & j_2 & \cdots & j_r
            \end{pmatrix}=
            \sum_{1\leqslant k_1<k_2<\cdots<k_r\leqslant n}\bm{A}\begin{pmatrix}
                i_1 & i_2 & \cdots & i_r \\
                k_1 & k_2 & \cdots & k_r
            \end{pmatrix}\bm{B}\begin{pmatrix}
                k_1 & k_2 & \cdots & k_r \\
                j_1 & j_2 & \cdots & j_r
            \end{pmatrix}\qedhere
        \]
    \end{proof}
}
\subsection{主子式}
\dfn{主子式}{主子式}{
    如果
    \[
        \bm{A}\begin{pmatrix}
            i_1 & i_2 & \cdots & i_r \\
            j_1 & j_2 & \cdots & j_r
        \end{pmatrix}
    \]
    有
    \[
        i_k = j_k \qquad
        \left(k = 1,2,\cdots ,r\right)
    \]
    称为$\bm{A}$的$r$阶主子式.
}
\lem{}{实矩阵的主子式非负}{
    $\bm{A}$是一个$m \times n$实矩阵,则$\bm{A}\bm{A}^{\prime}$的任一主子式均非负.
    \begin{proof}
        由\cref{thm:Cauchy-Binet公式}Cauchy-Binet公式:$r \leqslant n$时
        \[
            \bm{AA}^{\prime}\begin{pmatrix}
                i_1 & i_2 & \cdots & i_r \\
                i_1 & i_2 & \cdots & i_r
            \end{pmatrix}=
            \sum_{1\leqslant k_1<k_2<\cdots<k_r\leqslant n}\bm{A}\begin{pmatrix}
                i_1 & i_2 & \cdots & i_r \\
                k_1 & k_2 & \cdots & k_r
            \end{pmatrix}^2\geqslant 0
        \]

        $r > n$时任一$r$阶子式均为0亦满足.
    \end{proof}
}
\cor{}{复矩阵的主子式非负}{
    事实上,对于$m \times n$复矩阵$\bm{A}$,$\bm{A}\overline{\bm{A}}^{\prime} $任一主子式均非负.
}
\subsection{Cauchy-Binet公式的应用}
\lem{方阵的积的伴随阵}{方阵的积的伴随阵}{
    对于$n$阶方阵$\bm{A}$、
    $\bm{B}$
    \[
        \left(\bm{AB}\right)^* = \bm{B}^*\bm{A}^*
    \]\begin{proof}
        只需比较$\left(i,j\right)$元,因为$\left(\bm{AB}\right)^*$的$
            \left(i,j\right)$元是$\bm{AB}$的$\left(j,i\right)$
        元的代数余子式
        \[
            \left(-1\right)^{i+j}\bm{AB}\begin{pmatrix}
                1 & \cdots & \hat{j} & \cdots & n \\
                1 & \cdots & \hat{i} & \cdots & n
            \end{pmatrix}
        \]
        由\cref{thm:Cauchy-Binet公式}Cauchy-Binet公式知上式为
        \[
            \left(-1\right)^{i+j}\sum_{k=1}^{n}
            \bm{A}\begin{pmatrix}
                1 & \cdots & \hat{j} & \cdots & n \\
                1 & \cdots & \hat{k} & \cdots & n
            \end{pmatrix}\bm{B}\begin{pmatrix}
                1 & \cdots & \hat{k} & \cdots & n \\
                1 & \cdots & \hat{i} & \cdots & n
            \end{pmatrix}=\sum_{k=1}^{n}\bm{A}_{jk}\bm{B}_{ki}
        \]
        这个式子即是$\bm{A}^*$的第$j$列乘上$\bm{B}^*$的第$i$行即$\bm{B}^*\bm{A}^*$的$\left(i,j\right)$元.
    \end{proof}
}
\exa{Lagrange恒等式}{Lagrange恒等式}{
    $n \geqslant  2$时
    \[
        \left(\sum_{i = 1}^{n}a_i^2\right)
        \left(\sum_{i = 1}^{n}b_i^2\right)
        -
        \left(\sum_{i = 1}^{n}a_ib_i\right)^2
        =
        \sum_{1 \leqslant  i < j \leqslant n}
        \left(a_ib_j - a_jb_i\right)^2
    \]\begin{proof}
        \begin{align*}
            LHS & = \begin{vmatrix}
                        \sum\limits_{i=1}^{n}a_i^2  & \sum\limits_{i=1}^{n}a_ib_i \\
                        \sum\limits_{i=1}^{n}a_ib_i & \sum\limits_{i=1}^{n}b^2_i
                    \end{vmatrix} \\
                & =\left|\begin{pmatrix}
                             a_1 & a_2 & \cdots & a_n \\
                             b_1 & b_2 & \cdots & b_n
                         \end{pmatrix}\begin{pmatrix}
                                          a_1    & b_1    \\
                                          a_2    & b_2    \\
                                          \vdots & \vdots \\
                                          a_n    & b_n
                                      \end{pmatrix}\right|
            \\
                & \xlongequal{\text{Cauchy-Binet公式}}
            \sum_{1\leqslant i < j \leqslant n}
            \begin{vmatrix}
                a_i & a_j \\
                b_i & b_j
            \end{vmatrix}\begin{vmatrix}
                             a_i & b_i \\
                             a_j & b_j
                         \end{vmatrix}                                       \\
                & =\sum_{1\leqslant i < j\leqslant n}
            \left(a_ib_j-a_jb_i\right)^2
            \qedhere
        \end{align*}
    \end{proof}
}
同时\cref{ex:Lagrange恒等式}立刻得到下面的不等式
\exa{Cauchy-Schwarz不等式}{Cauchy-Schwarz不等式}{
    有实数形式与复数形式
    \paragraph{实形式} $ a_i,b_i\in\bb{R},n\in\bb{N}^+$
    \[
        \left(\sum_{i = 1}^{n}a_i^2\right)
        \left(\sum_{i = 1}^{n}b_i^2\right)
        \geqslant
        \left(\sum_{i = 1}^{n}a_ib_i\right)^2
    \]
    \paragraph{复形式}$a_i,b_i\in \bb{C},n \geqslant 2$
    \[
        \left(\sum_{i = 1}^{n}\left|a_i\right|^2\right)
        \left(\sum_{i = 1}^{n}\left|b_i\right|^2\right)
        \geqslant
        \left|\sum_{i = 1}^{n}a_i\overline{b_i}\right|^2
    \]
}