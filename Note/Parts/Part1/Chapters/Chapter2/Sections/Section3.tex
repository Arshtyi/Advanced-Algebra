\newpage
\section{矩阵的运算}
\subsection{矩阵的线性运算}
\dfn{矩阵的加减法}{矩阵的加减法}{
    对应位置相加减即可:
    \[
        \left(a_{ij}\right)_{m \times n} \pm \left(
        b_{ij}
        \right)_{m \times n} = \left(
        a_{ij} \pm b_{ij}
        \right)_{m \times n}
    \]
    \begin{align*}
          & \begin{pmatrix}
                a_{11} & a_{12} & \cdots & a_{1n} \\
                a_{21} & a_{22} & \cdots & a_{2n} \\
                \vdots & \vdots &        & \vdots \\
                a_{n1} & a_{n2} & \cdots & a_{nn}
            \end{pmatrix}_{m \times n}
        \pm
        \quad
        \begin{pmatrix}
            b_{11} & b_{12} & \cdots & b_{1n} \\
            b_{21} & b_{22} & \cdots & b_{2n} \\
            \vdots & \vdots &        & \vdots \\
            b_{n1} & b_{n2} & \cdots & b_{nn}
        \end{pmatrix}_{m \times n}
        \\
        = &
        \begin{pmatrix}
            a_{11} \pm b_{11} & a_{12}\pm b_{12} & \cdots & a_{1n}\pm b_{1n} \\
            a_{21} \pm b_{21} & a_{22}\pm b_{22} & \cdots & a_{2n}\pm b_{2n} \\
            \vdots            & \vdots           &        & \vdots           \\
            a_{n1}\pm b_{n1}  & a_{n2}\pm b_{n2} & \cdots & a_{nn}\pm b_{nn}
        \end{pmatrix}_{m \times n}
    \end{align*}
}
\pro{}{矩阵加减法的性质}{
    \begin{enumerate}[label=(\arabic*)]
        \item 交换律:$\bm{A} + \bm{B} = \bm{B} + \bm{A} $
        \item 结合律:$\left( \bm{A} + \bm{B} \right) + \bm{C} = \bm{A}+ \left( \bm{B} + \bm{C}\right)$
        \item 零元存在:$\bm{A} + \bm{O} = \bm{A}$
        \item 存在负矩阵:$\bm{A}+\left(-\bm{A}\right) = \bm{O}$
        \item 移项:$
                  \bm{A} + \bm{B} = \bm{C} \Longleftrightarrow \bm{A}= \bm{C} - \bm{B} $
    \end{enumerate}
}
\subsection{矩阵的数乘}
\dfn{矩阵的数乘}{矩阵的数乘}{
一个常数$k$乘一个矩阵$\bm{A}_{mn}$称为数乘运算,所有元素都乘这个常数即可:
\[
    k\bm{A}_{mn} = k\left(a_{ij}\right)_{m \times n}=\left(k a_{ij}\right)_{m \times n}
\]

\[
    k \begin{pmatrix}
        a_{11} & a_{12} & \cdots & a_{1n} \\
        a_{21} & a_{22} & \cdots & a_{2n} \\
        \vdots & \vdots &        & \vdots \\
        a_{n1} & a_{n2} & \cdots & a_{nn}
    \end{pmatrix}_{m \times n}
    = \quad
    \begin{pmatrix}
        ka_{11} & ka_{12} & \cdots & ka_{1n} \\
        ka_{21} & ka_{22} & \cdots & ka_{2n} \\
        \vdots  & \vdots  &        & \vdots  \\
        ka_{n1} & ka_{n2} & \cdots & ka_{nn}
    \end{pmatrix}_{m \times n}
\]
}
\pro{矩阵数乘的性质}{矩阵数乘的定义}{
    \begin{enumerate}[label=(\arabic*)]
        \item 数的分配律:$k \left(\bm{A} + \bm{B}\right) = k\bm{A} + k\bm{B}$
        \item 矩阵的分配律:$\left(k + l\right)\bm{A} = k\bm{A} + l \bm{B}$
        \item 结合律:$k\left(l\bm{A}\right) = \left(kl\right) \bm{A} = l \left(k\bm{A}\right)$
        \item 单位元存在:$1\cdot\bm{A} = \bm{A}$
        \item 零元存在:$0\cdot\bm{A} = \bm{O}$
    \end{enumerate}
}
\subsection{矩阵的乘法}
\dfn{矩阵的乘法}{矩阵的乘法}{
    对于矩阵乘法$\bm{A}_{m\times n}\bm{B}_{n\times k}=\bm{C}_{m\times k}$,结果矩阵中的元素$c_{ij}$的算法是用矩阵$\bm{A}$中的第$i$行$\left(a_{i1},a_{i2},\cdots ,a_{in}\right)$各元素分别与矩阵$\bm{B}$中的第$j$列$\left(b_{1j},b_{2j},\cdots,b_{nj}\right)$各元素分别相乘,再把结果相加得到$c_{ij}$即这两个向量做内积:
    \begin{align*}
        c_{ij}=
        \left(
        a_{i1},a_{i2},\cdots ,a_{in}
        \right)
        \cdot
        \begin{pmatrix}
            b_{1j} \\
            b_{2j} \\
            \vdots \\
            b_{nj}
        \end{pmatrix}= \sum _{k=1} ^{n} a_{ik}b_{kj}
    \end{align*}
}
\rem{}{}{
    矩阵乘法最大的特点就是不能随意交换即
    \[
        \bm{A}\bm{B}=\bm{B}\bm{A}
    \]
    多数情况下不成立.这一点通常称为不对易.
}
\pro{矩阵乘法的性质}{矩阵乘法的性质}{
    设$\bm{A}_{m \times n} =\left(a_{ij}\right)_{m \times n}$,$\bm{B}_{n \times p} = \left(
        b_{ij}
        \right)_{n \times p}$,$\bm{C}
        _{p \times q} = \left(
        c_{ij}
        \right)_{p \times q}$,则
    \begin{enumerate}[label=(\arabic*)]
        \item 结合律:$\left(\bm{A}\bm{B}\right)\bm{C} = \bm{A}\left(\bm{B}\bm{C}\right) $
        \item 分配律:$\bm{A}\left(\bm{B} + \bm{C}\right) = \bm{A}\bm{B} + \bm{A}\bm{C} $
        \item 分配律:$\left(\bm{B}+\bm{C}\right)\bm{A}=\bm{B}\bm{A}+\bm{C}\bm{A} $
        \item 与数乘相容:$k\left(\bm{A}\bm{B}\right) =\left(k\bm{A}\right)\bm{B}= \bm{A}\left(k\bm{B}\right)$
        \item 单位元:$
                  \bm{E}_m \bm{A}_{m \times n} =
                  \bm{A}_{m \times n} =
                  \bm{A}_{m \times n}\bm{E}_{n} $
        \item 注意:$\left(\bm{A}+\bm{B}\right)^2=\bm{A}^2+\bm{A}\bm{B}+\bm{B}\bm{A}+\bm{B}^2.$
        \item 对于方阵$\bm{A}\text{、}\bm{B}, \det\left(\bm{AB}\right) =  \det\left(\bm{A}\right)\cdot  \det\left(\bm{B}\right)$
    \end{enumerate}
    \begin{proof}
        \item 因为$\bm{AB}$的$\left(i,j\right)$
        元为
        \[
            \sum_{k = 1}^{n}a_{ik}b_{kj}
        \]
        $\left(\bm{AB}\right)\bm{C}$的$
            \left(i,j\right)$元为
        \[
            \sum_{r = 1}^{p}\left(\sum_{k = 1}
            ^{n}a_{ik}b_{kr}\right)c_{kj}
            =
            \sum_{r = 1}^{p}\sum_{k = 1}
            ^{n}a_{ik}b_{kr}c_{kj}
        \]
        $\bm{A}\left(\bm{BC}\right)$的
        $\left(i,j\right)$元为
        \[
            \sum_{k = 1}^{n}\sum_{r = 1}
            ^{p}a_{ik}b_{kr}c_{kj}
        \]
        其余性质易证.\qedhere
    \end{proof}
}
\clm{}{}{
    一般而言,因为整性不成立,矩阵的乘法消去律亦不成立,从线性映射的角度来看,两个非零映射复合得到零映射也并非不可能.
}
\subsection{方阵的非负整数次幂}
\dfn{方阵的幂}{方阵的幂}{
    对于$n$阶方阵$\bm{A}_{n}$,定义它的
    非负整数$k$次幂为:
    \begin{equation*}
        \left(\bm{A}_n\right)^k =
        \begin{cases}
            \bm{A}\bm{A} \cdots \bm{A} , & k \in \bb{N} ^* \\
            \bm{E}_n,                    & k = 0
        \end{cases}
    \end{equation*}
}
\pro{方针的幂的性质}{方阵的幂的性质}{
    \begin{enumerate}[label=(\arabic*)]
        \item$\bm{A}^{k + l} = \bm{A}^k\bm{A}^l$
        \item$\bm{A}^{kl} = \left(\bm{A}^{k}\right)^{l}$
        \item 由于矩阵乘法的不可交换,一般有:$\left(\bm{A}\bm{B}\right)^k\neq \bm{A}^k \bm{B}^k$
    \end{enumerate}
}
\dfn{方针的多项式}{方阵的多项式}{
    类似实数的多项式:
    \[
        f\left(x\right) = a_nx^n + \cdots a_1x+a_0
    \]
    定义方阵$\bm{A}_n$的多项式:
    \[
        f\left(\bm{A}_n\right) = a_n\left(\bm{A}_n\right)^n + \cdots a_1\bm{A}_n+a_0\bm{E}_n
    \]
}