\section{从线性映射看矩阵与行列式}
前面我们并未纠结矩阵的引入问题,这里略引线性映射来作解释.二者的深刻关系需要到后期才会被解释.下文只考虑实矩阵.

我们考虑全体$n$阶方阵的集合为$
    \bm{M}_n\left(\bb{R}
    \right)$,然后构造这样一个映射$\det$:
\begin{align*}
    \det : \bm{M}_n\left(\bb{R}
    \right) & \longrightarrow
    \bb{R}                    \\
    \bm{A}  & \longmapsto
    \left|\bm{A}\right|
\end{align*}
即函数:
\[
    \left|\bm{A}\right| =  \det\left(\bm{A}\right)
\]
这样的线性映射就是矩阵和行列式的内在关系.

说白了,我们的研究,就是为了回答下面这两个问题:
\begin{enumerate}[label =\textup{(\arabic*)}]
    \item $\det\left(\bm{A}\right)$能够反映出$\bm{A}$的哪些性质?
    \item 映射$\det$本身的性质如何?
\end{enumerate}

进一步地,我们将要探讨代数与几何二者之间的关系.