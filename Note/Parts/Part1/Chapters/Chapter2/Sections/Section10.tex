\newpage
\section{分块矩阵}
\subsection{分块矩阵}
\dfn{分块矩阵}{分块矩阵}{
    将矩阵分割为几块,得到一个分块矩阵
    \[
        \bm{A} = \left(\bm{A}_{ij}\right)_
        {r \times s} =
        \begin{pmatrix}
            \bm{A}_{11} & \bm{A}_{12} & \cdots & \bm{A}_{1s} \\
            \bm{A}_{21} & \bm{A}_{22} & \cdots & \bm{A}_{2s} \\
            \vdots      & \vdots      &        & \vdots      \\
            \bm{A}_{r1} & \bm{A}_{r2} & \cdots & \bm{A}_{rs}
        \end{pmatrix}
    \]
    其中\[
        \bm{A}_{i1},\bm{A}_{i2},\cdots,\bm{A}_{is}
        \quad \left(i = 1,2,\cdots,r\right)
    \]
    为第$i$分块行.分块$\bm{A}_{ij}=\left(a_{lk}\right)_{m_i\times n_j},\forall 1\leqslant i\leqslant r,1\leqslant j\leqslant n$,其中
    \[
        \left(m_1,m_2,\cdots,m_r\right),\left(
        n_1,n_2,\cdots,n_s
        \right)
    \]
    分别是$\bm{A}$的行、列分块方式.
}
\rem{}{}{
    分块矩阵只是一种表示方式,本质并无变化.
}
\subsection{运算}
\dfn{分块矩阵的运算}{分块矩阵的运算}{
    定义分块矩阵相关运算为:
    \begin{enumerate}[label=(\arabic*)]
        \item 加法:若$\bm{A} = \left(\bm{A}_{ij}\right)_{
                      r \times s
                  }$、$\bm{B} =
                  \left(\bm{B}_{ij}\right)_{
                      r \times s
                  }$行、列分块方式相同,定义加法
              \[
                  \bm{A}+\bm{B} = \left(
                  \bm{A}_{ij} + \bm{B}_{ij}
                  \right)_{r \times s}
              \]
        \item 数乘:$\bm{A} = \left(\bm{A}_{ij}\right)_{
                      r \times s
                  }$,$c \in \bb{R} $,数乘
              \[
                  c\bm{A} = \left(c\bm{A}_{ij}\right)
                  _{r \times s}
              \]
        \item 乘法:设$\bm{A} = \left(\bm{A}_{ij}\right)_{
                      r \times s
                  }$、$\bm{B} =
                  \left(\bm{B}_{ij}\right)_{
                      s \times k
                  }$,且$\bm{A}$的列分块方式与
              $\bm{B}$的行分块方式相同,那么
              \[
                  \bm{AB} = \left(\bm{C}_{ij}
                  \right)_{r \times k}
                  = \left(
                  \sum_{l = 1}^{s}\bm{A}_{il}
                  \bm{B}_{lj}
                  \right)_{
                      r \times k
                  }
              \]
              特别地,如果有形式行向量一样的形式记号即具有行分块
              $\bm{A} = \left(a_{ij}\right)_{
                      m \times n
                  } = \left(
                  \bm{\alpha }_1,\cdots
                  ,\bm{\alpha }_m
                  \right)^{\prime}$,列分块
              $\bm{B} = \left(b_{ij}\right)_{
                      n \times p
                  } = \left(
                  \bm{\beta  }_1,\cdots
                  ,\bm{\beta  }_p
                  \right)$,那么
              \[
                  \bm{AB} = \left(\bm{\alpha }_1
                  \bm{B},\cdots,
                  \bm{\alpha }_m\bm{B}\right)^{\prime} =
                  \left(
                  \bm{A}\bm{\beta }_1,\cdots
                  ,\bm{A}\bm{\beta }_p
                  \right)
              \]
        \item 转置:设$\bm{A} = \left(\bm{A}_{ij}\right)_{
                      r \times s
                  }$,转置为
              \[
                  \bm{A}^{\prime} = \left(
                  \bm{A}_{ij}
                  \right)_{s \times r}
              \]
        \item 共轭:设复矩阵$\bm{A} = \left(\bm{A}_{ij}\right)_{
                      r \times s
                  }$,共轭为
              \[
                  \overline{\bm{A}} =
                  \left(\overline{\bm{A}_{ij}} \right)_{r \times s}
              \]
    \end{enumerate}
}
\exa{}{}{
    若$\bm{A} = \mathrm{diag}\left\{
        \bm{A}_1,\cdots,\bm{A}_k
        \right\}$,$
        \bm{B}= \mathrm{diag}\left\{
        \bm{B}_1,\cdots,\bm{B}_k
        \right\}$,则
    \[
        \bm{AB} = \mathrm{diag}\left\{
        \bm{A}_1\bm{B}_1,\cdots
        ,\bm{A}_k\bm{B}_k
        \right\}
    \]
}
\cor{分块对角阵的逆阵}{分块对角阵的逆阵}{
若$\bm{A} = \mathrm{diag}\left\{\bm{A}_1,\cdots,\bm{A}_k\right\}$可逆则
\[
    \bm{A}^{-1} =
    \mathrm{diag}\left\{
    \bm{A}_1^{-1},\cdots,\bm{A}_k^{-1}
    \right\}
\]
}
\subsection{分块初等矩阵}
\dfn{分块单位阵}{分块单位阵}{
    分块单位阵为
    \[
        \bm{I} = \mathrm{diag}\left\{
        \bm{I}_{m_1},\cdots,
        \bm{I}_{m_n}
        \right\}
    \]
}
\dfn{分块初等矩阵}{分块初等矩阵}{
    类似于\cref{def:初等矩阵}
    \paragraph{第一类分块初等矩阵}对换$\bm{I}$的第$i$分块行(列)与第$j$分块行(列)得到.
    \[
        \bm{P}_{ij} =
        \begin{pmatrix}
            \bm{I}_{m_{1}} &        &              &        &              &        &              \\
                           & \ddots &              &        &              &        &              \\
                           &        & 0            & \cdots & \bm{I}_{m_j} &        &              \\
                           &        & \vdots       &        & \vdots       &        &              \\
                           &        & \bm{I}_{m_i} & \cdots & 0            &        &              \\
                           &        &              &        &              & \ddots &              \\
                           &        &              &        &              &        & \bm{I}_{m_n}
        \end{pmatrix}
    \]
    \paragraph{第二类分块初等矩阵}
    $ \det\left(\bm{M}\right) \neq 0$下以$\bm{M}$左(右)乘
    $\bm{I}$的第$i$分块行(列)得到.
    \[
        \bm{P}_i\left(\bm{M}\right) =
        \begin{pmatrix}
            \bm{I}_{m_1} &        &        &        &              \\
                         & \ddots &        &        &              \\
                         &        & \bm{M} &        &              \\
                         &        &        & \ddots &              \\
                         &        &        &        & \bm{I}_{m_n}
        \end{pmatrix}
    \]
    % \newpage
    \paragraph{第三类分块初等矩阵}
    以矩阵$\bm{M}$左(右)乘$\bm{I}$的第$i$分块行(列)后加到第
    $j$分块行(列)得到.
    \[
        \bm{T}_{ij} \left(\bm{M}\right)=
        \begin{pmatrix}
            \bm{I}_{m_1} &        &              &        &              &        &              \\
                         & \ddots &              &        &              &        &              \\
                         &        & \bm{I}_{m_i} & \cdots & 0            &        &              \\
                         &        & \vdots       &        & \vdots       &        &              \\
                         &        & \bm{M}       & \cdots & \bm{I}_{m_j} &        &              \\
                         &        &              &        &              & \ddots &              \\
                         &        &              &        &              &        & \bm{I}_{m_n}
        \end{pmatrix}
    \]
}
\thm{分块初等矩阵的逆阵}{分块初等矩阵的逆阵}{
    分块初等矩阵均可逆且逆阵为同类分块初等矩阵
    \[
        \bm{P}_{ij}^{-1} = \bm{P}_{ij}^{\prime} \quad
        \bm{P}_i\left(\bm{M}\right)^{-1}
        =\bm{P}_i\left(\bm{M}^{-1}\right) \quad
        \bm{T}_{ij}\left(\bm{M}\right)^{-1}
        = \bm{T}_{ij}\left(-\bm{M}\right)
    \]
}
\thm{分块初等矩阵的行列式}{分块初等矩阵的行列式}{
    \[
        \det\left(\bm{P}_{ij}\right) = \left(-1
        \right)^l, \det\left(\bm{P}_i\left(\bm{M}\right)\right)
        =  \det\left(\bm{M}\right),
        \det\left(\bm{T}_{ij}\left(\bm{M}
        \right)\right) = 1
    \]
    其中
    \[
        l = m_i m_j + \left(m_i + m_j\right)
        \sum_{i < r <j}m_r
    \]
}
\subsection{分块初等变换}
\thm{分块初等变换与分块初等矩阵}{分块初等变换与分块初等矩阵}{
    类似于\cref{thm:初等变换与初等矩阵},每一种分块初等行(列)变换等价于左(右)乘这种变换对应的分块初等矩阵.
}
\cor{}{第三类分块初等变换不改变行列式的值}{
    第三类分块初等变换不会改变矩阵行列式的值.
}
\subsection{行列式降阶公式}
\lem{}{分块三角行列式}{
    对分块三角行列式有
    \[
        \left|\bm{G}\right| =
        \begin{vmatrix}
            \bm{A} & \bm{B} \\
            \bm{O} & \bm{C}
        \end{vmatrix}
        = \left|\bm{A}\right|
        \left|\bm{C}\right|
    \]
    \[
        \left|\bm{H}\right| =
        \begin{vmatrix}
            \bm{A} & \bm{O} \\
            \bm{B} & \bm{C}
        \end{vmatrix}
        = \left|\bm{A}\right|
        \left|\bm{C}\right|
    \]\begin{proof}
        利用\cref{thm:Laplace定理}易证.
    \end{proof}
}
\thm{行列式降阶公式}{行列式降阶公式}{
    对于矩阵
    \[
        \bm{M} = \begin{pmatrix}
            \bm{A} & \bm{B} \\
            \bm{C} & \bm{D}
        \end{pmatrix}
    \]
    \begin{enumerate}[label=\arabic*)]
        \item 若$\bm{A}$可逆,则
              \[
                  \left|\bm{M}\right| = \left|\bm{A}
                  \right|\left|\bm{D} - \bm{C}\bm{A}
                  ^{-1}\bm{B}\right|
              \]
        \item 若$\bm{D}$可逆,则\[
                  \left|\bm{M}\right| = \left|
                  \bm{D}
                  \right|\left|
                  \bm{A} - \bm{B}\bm{D}^{-1}
                  \bm{C}
                  \right|
              \]
        \item 若$\bm{A}$、$\bm{D}$均可逆,则
              \[
                  \left|\bm{A}
                  \right|\left|\bm{D} - \bm{C}\bm{A}
                  ^{-1}\bm{B}\right|
                  =
                  \left|
                  \bm{D}
                  \right|\left|
                  \bm{A} - \bm{B}\bm{D}^{-1}
                  \bm{C}
                  \right|
              \]
    \end{enumerate}
}
\exa{}{}{
    计算下面矩阵的行列式
    \[
        \bm{M}=\begin{pmatrix}
            a_1^2    & a_1a_2+1 & \cdots & a_1a_n+1 \\
            a_2a_1+1 & a_2^2    & \cdots & a_2a_n+1 \\
            \vdots   & \vdots   &        & \vdots   \\
            a_na_1+1 & a_na_2+1 & \cdots & a_n^2
        \end{pmatrix}
    \]\begin{solution}
        注意到
        \begin{align*}
            \bm{M} & =\begin{pmatrix}
                          a_1^2+1  & a_1a_2+1 & \cdots & a_1a_n+1 \\
                          a_2a_1+1 & a_2^2+1  & \cdots & a_2a_n+1 \\
                          \vdots   & \vdots   &        & \vdots   \\
                          a_na_1+1 & a_na_2+1 & \cdots & a_n^2+1
                      \end{pmatrix}-\bm{I}_n         \\
                   & =-\bm{I}_n+\begin{pmatrix}
                                    a_1    & 1      \\
                                    a_2    & 1      \\
                                    \vdots & \vdots \\
                                    a_n    & 1
                                \end{pmatrix}\begin{pmatrix}
                                                 a_1 & a_2 & \cdots & a_n \\
                                                 1   & 1   & \cdots & 1
                                             \end{pmatrix} \\
                   & =-\bm{I}_n+\begin{pmatrix}
                                    a_1    & 1      \\
                                    a_2    & 1      \\
                                    \vdots & \vdots \\
                                    a_n    & 1
                                \end{pmatrix}
            \bm{I}_2^{-1}\begin{pmatrix}
                             a_1 & a_2 & \cdots & a_n \\
                             1   & 1   & \cdots & 1
                         \end{pmatrix}
        \end{align*}
        由\cref{thm:行列式降阶公式}降阶公式有

        \begin{align*}
            \left|\bm{M}\right| & =
            \left|\bm{I}_2\right|^{-1}\cdot
            \left|-\bm{I}_n\right|\cdot
            \left|
            \bm{I}_2+\begin{pmatrix}
                         a_1 & a_2 & \cdots & a_n \\
                         1   & 1   & \cdots & 1
                     \end{pmatrix}\left(-\bm{I}_n\right)^{-1}
            \begin{pmatrix}
                a_1    & 1      \\
                a_2    & 1      \\
                \vdots & \vdots \\
                a_n    & 1
            \end{pmatrix}
            \right|                                        \\
                                & =\left(-1\right)^n\left|
            \bm{I}_2-\begin{pmatrix}
                         \sum\limits_{i=1}^{n}a_i^2 & \sum\limits_{i=1}^{n}a_i \\
                         \sum\limits_{i=1}^{n}a_i   & n
                     \end{pmatrix}
            \right|                                        \\
                                & =\left(-1\right)^n\left(
            \left(1-n\right)
            \left(1-\sum\limits_{i=1}^{n}a_i^2\right)-\left(
                \sum\limits_{i=1}^{n}a_i
                \right)^2
            \right)
        \end{align*}
    \end{solution}
}
\exa{}{}{
    $\bm{A}$、$\bm{D}$均可逆,求分块矩阵的逆阵
    \[
        \begin{pmatrix}
            \bm{A} & \bm{B} \\
            \bm{O} & \bm{D}
        \end{pmatrix}
    \] \begin{solution}
        设$\bm{A}$、$\bm{D}$的阶数分别为$m$、$n$,那么
        \begin{align*}
            \begin{pmatrix}
                \bm{A} & \bm{B} & \bm{I}_m & \bm{O}   \\
                \bm{O} & \bm{D} & \bm{O}   & \bm{I}_n
            \end{pmatrix}
             & \longrightarrow
            \begin{pmatrix}
                \bm{A} & \bm{O} & \bm{I}_m & -\bm{B}\bm{D}^{-1} \\
                \bm{O} & \bm{D} & \bm{O}   & \bm{I}_n
            \end{pmatrix} \\
             & \longrightarrow
            \begin{pmatrix}
                \bm{I}_m & \bm{O}   & \bm{A}^{-1} & -\bm{A}^{-1}\bm{BD}^{-1} \\
                \bm{O}   & \bm{I}_n & \bm{O}      & \bm{D}^{-1}
            \end{pmatrix}
        \end{align*}
    \end{solution}
}
\exa{}{}{
    设$n$阶方阵$\bm{A}$、$\bm{B}$,求证:
    \[
        \begin{vmatrix}
            \bm{A} & \bm{B} \\
            \bm{B} & \bm{A}
        \end{vmatrix}
        =
        \left|\bm{A}+\bm{B}\right|
        \left|\bm{A}-\bm{B}\right|
    \]\begin{proof}
        作第三类分块初等变换
        \begin{align*}
            \begin{vmatrix}
                \bm{A} & \bm{B} \\
                \bm{B} & \bm{A}
            \end{vmatrix}
             & =
            \begin{vmatrix}
                \bm{A}+\bm{B} & \bm{A}+\bm{B} \\
                \bm{B}        & \bm{A}
            \end{vmatrix}     \\
             & =\begin{vmatrix}
                    \bm{A}+\bm{B} & \bm{O}        \\
                    \bm{B}        & \bm{A}-\bm{B}
                \end{vmatrix} \\
             & =
            \left|\bm{A}+\bm{B}\right|
            \left|\bm{A}-\bm{B}\right|\qedhere
        \end{align*}
    \end{proof}
}
\exa{}{temp1}{
    设$n$阶复矩阵$\bm{A}$、$\bm{B}$,求证:
    \[
        \begin{vmatrix}
            \bm{A} & -\bm{B} \\
            \bm{B} & \bm{A}
        \end{vmatrix}=\left|
        \bm{A}+\mathrm{i}\bm{B}
        \right|\left|\bm{A}-\mathrm{i}\bm{B}\right|
    \]
    \begin{proof}
        \begin{align*}
            \begin{vmatrix}
                \bm{A} & -\bm{B} \\
                \bm{B} & \bm{A}
            \end{vmatrix}
             & =
            \begin{vmatrix}
                \bm{A}+\mathrm{i}\bm{B} & \mathrm{i}\bm{A}-\bm{B} \\
                \bm{B}                  & \bm{A}
            \end{vmatrix}     \\
             & =\begin{vmatrix}
                    \bm{A}+\mathrm{i}\bm{B} & \bm{O}                  \\
                    \bm{B}                  & \bm{A}-\mathrm{i}\bm{B}
                \end{vmatrix} \\
             & =
            \left|\bm{A}+\mathrm{i}\bm{B}\right|
            \left|\bm{A}-\mathrm{i}\bm{B}\right|\qedhere
        \end{align*}
    \end{proof}
}
\exa{}{}{
    设$n$阶方阵$\bm{A}$、$\bm{B}$且$\bm{AB}=\bm{BA}$,求证:
    \[
        \begin{vmatrix}
            \bm{A} & -\bm{B} \\
            \bm{B} & \bm{A}
        \end{vmatrix}=\left|\bm{A}^2+\bm{B}^ 2\right|
    \]\begin{proof}
        由\cref{ex:temp1}立得.
    \end{proof}
}