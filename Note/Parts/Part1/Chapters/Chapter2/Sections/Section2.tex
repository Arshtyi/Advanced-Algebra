\newpage
\section{矩阵的基本形式}
\subsection{矩阵的表示}
\dfn{矩阵的形式}{矩阵的形式}{
矩阵是一张\emph{数表},通常用大写英文字母表示,如:
\[
    \bm{A} = \begin{pmatrix}
        a_{11} & a_{12} & \cdots & a_{1n} \\
        a_{21} & a_{22} & \cdots & a_{2n} \\
        \vdots & \vdots &        & \vdots \\
        a_{n1} & a_{n2} & \cdots & a_{nn}
    \end{pmatrix}_{m \times n}
\]
记作$a_{ij}$或$\left(
    a_{ij}
    \right)_{m \times n}$或$\bm{A}_{mn}$或$
    \bm{A}_{m \times n}$,一些情况下也会记作
$\bm{A}^{m\times n}$等形式.
}
\subsection{若干基础矩阵}
\pro{}{若干基础矩阵}{
    \begin{enumerate}[label=(\arabic*)]
        \item 方阵:一个行数和列数相等的矩阵,比如
              \[
                  \bm{A}_{n} = \begin{pmatrix}
                      a_{11} & a_{12} & \cdots & a_{1n} \\
                      a_{21} & a_{22} & \cdots & a_{2n} \\
                      \vdots & \vdots &        & \vdots \\
                      a_{n1} & a_{n2} & \cdots & a_{nn}
                  \end{pmatrix}_{n \times n}
              \]
        \item 单位阵:一个主对角线上全是$1$,其他元素都是$0$的方阵
              \[
                  \bm{I}_{n}  = \bm{E}_n = \begin{pmatrix}
                      1 &        &   \\
                        & \ddots &   \\
                        &        & 1
                  \end{pmatrix}
                  _{n}
              \]
        \item 常量矩阵(常数矩阵):一个主对角线上全是同一个常数$k$,其他元素都是零的方阵
              \[
                  \bm{K}_n = \begin{pmatrix}
                      k &        &   \\
                        & \ddots &   \\
                        &        & k
                  \end{pmatrix}
                  _n
              \]
        \item 对角矩阵:主对角线之外的所有元素都是$0$的方阵 \[
                  \bm{\varLambda} _n = \begin{pmatrix}
                      a_{11} &        &        &        \\
                             & a_{22} &        &        \\
                             &        & \ddots &        \\
                             &        &        & a_{nn}
                  \end{pmatrix}= \diag\left\{
                  a_{11},a_{22},\cdots,a_{nn}
                  \right\}
              \]
        \item 三角阵:一个方阵主对角线以上(下)全是$ 0$,称为下(上)三角阵\[
                  \bm{A}_n = \begin{pmatrix}
                      a_{11} & a_{12} & \cdots & a_{1n} \\
                             & a_{22} & \cdots & a_{2n} \\
                             &        & \ddots & \vdots \\
                             &        &        & a_{nn}
                  \end{pmatrix}
                  ,
                  \bm{B}_n = \begin{pmatrix}
                      b_{11} &        &        &        \\
                      b_{21} & b_{22} &        &        \\
                      \vdots & \vdots & \ddots &        \\
                      b_{n1} & b_{n2} & \cdots & b_{nn}
                  \end{pmatrix}
              \]
        \item 旋转矩阵:这里给出二阶形式,注意,逆时针为正方向\[
                  \bm{A} = \begin{pmatrix}
                      \cos \theta & -\sin\theta \\
                      \sin\theta  & \cos\theta
                  \end{pmatrix}
                  = \mathrm{exp}\left(
                  \theta \begin{bmatrix}
                          0 & -1 \\
                          1 & 0
                      \end{bmatrix}
                  \right)
              \]
        \item 设$\bm{A}_{mn}$为非零矩阵,若所有非零行(即至少有一个非零元素的行)全在零行的上面,$\bm{A}_{mn}$中各非零行中第一个(最后一个)非零元素(称为阶梯点)前(后)面零元素的个数随行数增大而严格增多(减少),则称为上(下)梯形矩阵,简称为上(下)梯形阵,上下梯形阵统称为梯形阵.
    \end{enumerate}
}
\rem{}{}{
    \emph{ 一个矩阵有可能同时是上、下梯形阵,也可能都不是,也可能只是其中之一!}
}
\subsection{矩阵的迹}
\dfn{迹}{迹}{
方阵$\bm{A}=\left(a_{ij}\right)_{m\times n}$的主对角线上元素之和称为这个方阵的迹,记作$ \tr\bm{A}=\Tr\bm{A}=a_{11} + a_{22} + \cdots + a_{nn}$.
}
\thm{}{迹的交换性}{
    设$\bm{A}\in M_{m\times n}\left(\bb{R}\right),\bm{B}\in M_{n\times m}\left(\bb{R}\right)$,则\[
        \Tr \bm{AB}=\Tr\bm{BA}
    \]
    \begin{proof}
        注意到\[
            \Tr\bm{AB}-\bm{BA}=0\qedhere
        \]
    \end{proof}
}