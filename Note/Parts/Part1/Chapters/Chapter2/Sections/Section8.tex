\newpage
\section{初等变换与初等矩阵}
\subsection{行列式运算的性质}
\pro{行列式的若干性质}{行列式的若干性质}{
    设$\bm{A},\bm{B}\in M_{n}\left(\bb{R}\right),c\in \bb{R}$,则\begin{enumerate}[label=\arabic*)]
        \item 一般$ \det\left(\bm{A}+ \bm{B}\right) \neq  \det\left(\bm{A}\right) +  \det\left(\bm{B}\right)$而是如\cref{cor:矩阵和的行列式}
        \item $ \det\left(c\bm{A}\right) =c^n  \det\left(\bm{A}\right)$
        \item $ \det\left(\bm{AB}\right) =\det\left(\bm{A}\right) \cdot\det\left(\bm{B}\right)$
        \item $ \det\left(\bm{A}^{\prime}\right) =\det\left(\bm{A}\right)$
        \item $ \det\left(\overline{\bm{A}}\right) = \overline{\det\left(\bm{A}\right)}$
        \item $ \det\left(\bm{A}^{-1}\right)= \left[ \det\left(\bm{A}\right)\right]^{-1}$
        \item $ \det\left(\bm{A}^*\right) = \det^{n - 1}\left(\bm{A}\right)$
    \end{enumerate}
}
\subsection{矩阵的初等变换}
\dfn{矩阵的初等变换}{矩阵的初等变换}{
    对矩阵的第一类、第二类、第三类初等
    行(列)变换为
    \begin{enumerate}[label=(\arabic*)]
        \item 对换矩阵中两行(列)的位置
        \item 用一非零常数$c$乘以某一行(列)
        \item 将矩阵某一行(列)乘以一常数$c$加到另一行(列)上
    \end{enumerate}
}
\subsection{相抵}
\dfn{相抵}{相抵}{
    若$\bm{A}$经过有限次的初等变换(行列均可)化为$\bm{B}$,则称$\bm{A}$与$\bm{B}$是等价的,或者说$\bm{A}$相抵于$\bm{B}$,记作$\bm{A} \sim \bm{B}$.
}
\clm{}{}{
    相抵关系是一种等价关系.
}
\thm{相抵标准型}{相抵标准型}{
    任一矩阵$\bm{A}_{m \times n}$一定相抵于它的相抵标准型$\bm{B}$
    \[
        \bm{B} =
        \begin{pmatrix}
            \bm{I}_r & \bm{O} \\
            \bm{O}   & \bm{O}
        \end{pmatrix}
        \quad
        \left(0 \leqslant r \leqslant \min \left\{m,n\right\},r \in \bb{N} \right)
    \]并且其中的$r$不依赖于初等变换的选取而仅由$\bm{A}$唯一确定.\begin{proof}
        考虑对$\min\left\{m,n\right\}$进行归纳,若$\min\left\{m,n\right\}=0$则归纳结束.设$\min\left\{m,n\right\}<k$时成立,下面证明$\min\left\{m,n\right\}=k$的情形.

        显然$\bm{A}=\bm{O}$时成立.下设$\bm{A}\neq\bm{O} $,不妨设$a_{ij}\neq 0.$那么将第$i$行与第$1$行对换,第$j$列与第$1$列对换,那么以下设$a_{11}\neq 0$
        \[
            \bm{A}=\begin{pmatrix}
                a_{11} & a_{12} & \cdots & a_{1n} \\
                a_{21} & a_{22} & \cdots & a_{2n} \\
                \vdots & \vdots &        & \vdots \\
                a_{m1} & a_{m2} & \cdots & a_{mn}
            \end{pmatrix}\longrightarrow
            \begin{pmatrix}
                1      & 0      & \cdots & 0      \\
                0      & b_{22} & \cdots & b_{2n} \\
                \vdots & \vdots &        & \vdots \\
                0      & b_{m2} & \cdots & b_{mn}
            \end{pmatrix}
        \]于是根据归纳假设即证.
    \end{proof}
}
\thm{}{相抵于阶梯形}{
    任一矩阵$\bm{A}_{m \times n}$在初等行变换下一定相抵于一个阶梯形矩阵.
    \begin{proof}
        对列数$n$进行归纳,$n=0$表示归纳结束,设列数小于$n$时成立,下面证明$n$时成立.

        一方面,如果$\bm{A}$第一列全为零,由归纳假设得证.另一方面,如果$\bm{A}$第一列不全为零,用行变换将非零元换到$\left(1,1\right)$位置,消去同列其他元素得到一个$\left(m-1\right)\times\left(n-1\right)$阶矩阵,由归纳假设得证.于是证毕.
    \end{proof}
}
\subsection{初等矩阵}
\dfn{初等矩阵}{初等矩阵}{
    对单位阵$\bm{I}_n$分别施加第一类、第二类、第三类初等变换后得到的矩阵分别称为第一类、第二类、第三类初等矩阵.
    \paragraph{第一类初等矩阵}
    第一类初等矩阵$P_{ij}$表示将单位阵$\bm{I}_n$
    的$i$行与第$j$列(或者是第$i$列与第$j$列)对换得到的
    矩阵
    \[
        \bm{P}_{ij} =
        \begin{pmatrix}
            1 &        &        &        &        &        &   \\
              & \ddots &        &        &        &        &   \\
              &        & 0      & \cdots & 1      &        &   \\
              &        & \vdots &        & \vdots &        &   \\
              &        & 1      & \cdots & 0      &        &   \\
              &        &        &        &        & \ddots &   \\
              &        &        &        &        &        & 1
        \end{pmatrix}
    \]
    \paragraph{第二类初等矩阵}
    第二类初等矩阵$P_i\left(c\right)$表示
    用常数$c \in \bb{R}\left(c \neq 0\right)$
    单位阵第$i$行(或者是第$i$列)得到的矩阵
    \[
        \bm{P}_i\left(c\right) =
        \begin{pmatrix}
            1 &        &   &        &   \\
              & \ddots &   &        &   \\
              &        & c &        &   \\
              &        &   & \ddots &   \\
              &        &   &        & 1
        \end{pmatrix}
    \]
    \paragraph{第三类初等矩阵}
    第三类初等矩阵$\bm{T}_{ij}\left(c\right)$
    表示将单位阵第$i$行(或者是第$i$列)乘以一常数$c$
    后加到第$j$行(或者是第$j$列)得到的矩阵
    \[
        \bm{T}_{ij} \left(c\right)=
        \begin{pmatrix}
            1 &        &        &        &        &        &   \\
              & \ddots &        &        &        &        &   \\
              &        & 1      & \cdots & 0      &        &   \\
              &        & \vdots &        & \vdots &        &   \\
              &        & c      & \cdots & 1      &        &   \\
              &        &        &        &        & \ddots &   \\
              &        &        &        &        &        & 1
        \end{pmatrix}
    \]
}
\cor{初等矩阵的逆阵}{初等矩阵的逆阵}{
    初等矩阵均为可逆阵且其逆阵仍为同类初等矩阵
    \[
        \bm{P}_{ij}^{-1} = \bm{P}_{ij} \quad
        \bm{P}_i\left(c\right)^{-1}
        = \bm{P}_i\left(\frac{1}{c}\right)
        \quad
        \bm{T}_{ij}\left(c\right)^{-1}
        = \bm{T}_{ij}\left(-c\right)
    \]
}
\thm{初等变换与初等矩阵}{初等变换与初等矩阵}{
    对于$m \times n$矩阵$\bm{A}$,对$\bm{A}$作一次初等行变换得到的矩阵等于用一个$m$阶对应的初等矩阵左乘$\bm{A}$的积;对$\bm{A}$作一次初等列变换得到的矩阵等于用一个$m$阶对应的初等矩阵右乘$\bm{A}$的积.
}
\clm{}{}{
    初等行(列)变换等价于左(右)乘对应的初等矩阵.
}
\cor{初等矩阵的行列式}{初等矩阵的行列式}{
    \[
        \left|\bm{P}_{ij}\right|=-1,
        \left|\bm{P}_i\left(c\right)\right|=c,
        \left|\bm{T}_{ij}\left(c\right)\right|=1
    \]
}
\cor{}{第三类初等变换不改变行列式的值}{
    由\cref{thm:初等变换与初等矩阵},\cref{cor:初等矩阵的行列式}和\cref{thm:方阵积的行列式}可知对矩阵施加第三类初等变换不改变其行列式的值.
}
\cor{}{第三类初等变换不改变矩阵的奇异性}{
    \cref{cor:第三类初等变换不改变行列式的值}进一步表明了第三类初等变换不改变矩阵的奇异性.
}
\subsection{相抵}
\pro{方阵的若干等价结论}{方阵的若干等价结论}{
    设$n$阶方阵$\bm{A}$,则下列命题等价:
    \begin{enumerate}[label=(\arabic*)]
        \item $\bm{A}$是非异阵
        \item $\bm{A}$ 的相抵标准型为$\bm{I}_n$
        \item $\bm{A}$仅通过初等行(列)变换就可以化为$\bm{I}_n$
        \item $\bm{A}$等于若干个初等矩阵的积
    \end{enumerate}\begin{proof}
        $(1)\Longrightarrow (2)$设$\bm{A}$的相抵标准型为$\begin{pmatrix}
                \bm{I}_r & \bm{O} \\
                \bm{O}   & \bm{O}
            \end{pmatrix}$,由$\bm{A}$非异及\cref{cor:第三类初等变换不改变矩阵的奇异性}知该相抵标准型非奇异,故$r=n$.

        $(2)\Longrightarrow (3)$设存在初等阵$\bm{P}_1,\bm{P}_2,\cdots,\bm{P}_r,\bm{Q}_1,\bm{Q}_2,\cdots,\bm{Q}_s\st$
        \[
            \bm{P}_1\bm{P}_2\cdots\bm{P}_r\bm{A}\bm{Q}_1\bm{Q}_2\cdots\bm{Q}_s=\bm{I}_n
        \]作逆阵即证.

        $(3)\Longrightarrow (4)$依据上一证明,显然.

        $(4)\Longrightarrow (1)$由\cref{cor:初等矩阵的逆阵}知初等矩阵均非异,故$\bm{A}$非异.
    \end{proof}
}
\cor{相抵的等价表述}{相抵的等价表述}{
    对于$\bm{A}=\left(a_{ij}\right)_{m\times n}\in M_{m\times n}\left(\bb{R}\right)$,若存在初等阵$\bm{P}_1,\bm{P}_2,\cdots,\bm{P}_r,\bm{Q}_1,\bm{Q}_2,\cdots,\bm{Q}_s\st$
    \[
        \bm{P}_1\bm{P}_2\cdots\bm{P}_r\bm{A}\bm{Q}_1\bm{Q}_2\cdots\bm{Q}_s=\bm{B}
    \]则称$\bm{A}$与$\bm{B}$相抵.
}