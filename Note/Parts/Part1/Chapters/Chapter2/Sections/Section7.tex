\newpage
\section{方阵的逆阵}
\subsection{定义}
\dfn{逆阵}{逆阵}{
    如果对于一个$n$阶方阵$\bm{A}$,存在一个
    $n$阶方阵$\bm{B}$使得
    \[
        \bm{AB} = \bm{BA} = \bm{I}_n
    \]
    那么$\bm{B}$称为$\bm{A}$的逆矩阵,记作
    \[
        \bm{B} = \bm{A}^{-1}
    \]
}
\rem{}{}{
    只有方阵才具有逆阵.同时并不是所有非零方阵都具有逆阵.
}
\dfn{奇异阵}{奇异阵}{
    称有逆阵的方阵为可逆阵或非奇异阵或非异阵,而称无逆阵的方阵为奇异阵.
}
\subsection{性质}
\pro{逆阵的性质}{逆阵的性质}{
    \begin{enumerate}[label=(\arabic*)]
        \item 逆阵存在则必定唯一
        \item $\left(\bm{A}^{-1}\right)^{-1} = \bm{A}$
        \item $\bm{A}$、$\bm{B}$均可逆则$\bm{AB}$可逆且$\left(\bm{AB}\right)^{-1} =\bm{B}^{-1}\bm{A}^{-1}$
        \item 若$0\neq c \in \bb{R}$,则$c\bm{A}$可逆且$\left(c\bm{A}\right)^{-1}=c^{-1}\bm{A}^{-1}$
        \item $\bm{A}$可逆则$\bm{A}^{\prime}$可逆且$\left(\bm{A}^{\prime}\right)^{-1}= \left(\bm{A}^{-1}\right)^{\prime}$
        \item 一般$\bm{AB}\neq\bm{BA}$时$\left(\bm{AB}\right)^{-1} \neq\bm{A}^{-1} \bm{B}^{-1}$
        \item 整性对可逆阵成立
        \item 对可逆阵,乘法消去律成立
    \end{enumerate}
}
\cor{}{积的逆阵}{
设可逆阵$\bm{A}_1,\bm{A}_2,\cdots,\bm{A}_m\in M_n\left(\bb{R}\right)$,则$\bm{A}_1\bm{A}_2\cdots\bm{A}_m$也可逆且\[
    \left(
    \bm{A}_1\bm{A}_2\cdots\bm{A}_m
    \right)^{-1}= \bm{A}_m^{-1}\bm{A}_{m-1}^{-1}\cdots\bm{A}_1^{-1}
\]
}
\cor{}{逆阵的若干推论}{
    \begin{enumerate}[label=(\arabic*)]
        \item $\bm{A}_1,\cdots,\bm{A}_m$均为$n$阶方阵,且存在$\bm{A}_i$是奇异阵,那么$\bm{A}_1\cdots \bm{A}_m$是奇异阵
        \item $ \bm{A}$可逆$\Longleftrightarrow\left|\bm{A}\right|^{-1} = \left|\bm{A}^{-1}\right|$
        \item 对于$n$阶方阵$\bm{A}$、$\bm{B}$,若$\bm{AB}=\bm{I}_n$或$\bm{BA}=\bm{I}_n\Longleftrightarrow\bm{B} = \bm{A}^{-1}$
    \end{enumerate}
    \begin{proof}
        \begin{enumerate}[label=(\arabic*)]
            \item 计算下面矩阵的行列式
            \item \[
                      \big|\bm{A}\big|\left|\bm{A}^{-1}\right|=1
                      \Longrightarrow \left| \bm{A}^{-1}\right|=
                      \big | \bm{A}\big|^{-1}
                  \]
            \item 仅证$\bm{AB}=\bm{I}_n$,因为
                  \[
                      1=\left|\bm{I}_n\right|=\left|\bm{A}\right|\left|\bm{B}\right|
                      \Longrightarrow \left|\bm{A}\right|\neq 0 \Longrightarrow \bm{A}^{-1}\text{存在}
                  \]
                  故
                  \[
                      \bm{B}=\bm{I}_n\bm{B}=\left(\bm{A}^{-1}\bm{A}\right)\bm{B}=\bm{A}^{-1}\bm{I}_n=\bm{A}^{-1}\qedhere
                  \]
        \end{enumerate}
    \end{proof}
}
\subsection{伴随阵}
\dfn{伴随阵}{伴随阵}{
    $\bm{A}$为$n$阶方阵,$A_{ij}$为$ \det\left(\bm{A}\right)$的$\left(i,j\right)$元的代数余子式,则$\bm{A}$的伴随阵为
    \[
        \bm{A}^* = \begin{pmatrix}
            A_{11} & A_{21} & \cdots & A_{n1} \\
            A_{12} & A_{22} & \cdots & A_{n2} \\
            \vdots & \vdots &        & \vdots \\
            A_{1n} & A_{2n} & \cdots & A_{nn}
        \end{pmatrix}
    \]
}
\rem{}{}{
    关于\cref{def:伴随阵}的下标,应当以$ \det\left(\bm{A}\right)$剩下的元的位置看而不是划去的元的位置,或者也可以说多了一步转置.
}
\clm{}{}{
    矩阵的伴随矩阵一定存在.
}
\lem{}{与伴随阵的积}{
    设$n$阶方阵$\bm{A}$的伴随阵$\bm{A}^*$,则
    \[
        \bm{A}\bm{A}^* =\bm{A}^*\bm{A}
        = \left|\bm{A}\right|\bm{I}_n
        =  \det\left(\bm{A}\right)\bm{I}_n
    \]\begin{proof}
        考虑到\cref{thm:列形式的异乘变零定理}异乘变零定理,有
        \begin{align*}
            \begin{vmatrix}
                a_{11} & a_{12} & \cdots & a_{1n} \\
                a_{21} & a_{22} & \cdots & a_{2n} \\
                \vdots & \vdots &        & \vdots \\
                a_{n1} & a_{n2} & \cdots & a_{nn}
            \end{vmatrix}
            \begin{vmatrix}
                A_{11} & A_{21} & \cdots & A_{n1} \\
                A_{12} & A_{22} & \cdots & A_{n2} \\
                \vdots & \vdots &        & \vdots \\
                A_{1n} & A_{2n} & \cdots & A_{nn}
            \end{vmatrix} & =
            \begin{vmatrix}
                \left|\bm{A}\right| & 0                   & \cdots & 0                   \\
                0                   & \left|\bm{A}\right| & \cdots & 0                   \\
                \vdots              & \vdots              &        & \vdots              \\
                0                   & 0                   & \cdots & \left|\bm{A}\right|
            \end{vmatrix} \qedhere
        \end{align*}
    \end{proof}
}
\thm{}{逆阵的存在性}{
    对于矩阵$\bm{A}$
    \[\det\left(\bm{A}\right)
        \begin{cases*}
            =0\Longrightarrow \text{不可逆/奇异} \\
            \neq 0\Longrightarrow \text{可逆/非奇异}
        \end{cases*}
    \]
}
\thm{}{逆阵与伴随}{
    若$\bm{A}$可逆,即$ \det\left(\bm{A}\right) \neq 0$,那么
    \[
        \bm{A}^{-1} = \frac{1}{\left|\bm{A} \right|} \bm{A}^*
    \]\begin{proof}
        $\left|\bm{A}\right| \neq 0$时
        \[
            \bm{A}\left(\frac{1}{\left|\bm{A}\right|}\bm{A}^*\right)=
            \frac{\bm{AA}^*}{\left|\bm{A}\right|}=
            \bm{I}_n \qedhere
        \]
    \end{proof}
}
\clm{}{}{
    有了这样一套矩阵的逆阵理论,对于\cref{thm:Cramer法则}中的线性方程组
    \[
        \begin{cases*}
            a_{11}x_1+a_{12}x_2+\cdots+a_{1n}x_n=b_1 \\
            a_{21}x_1+a_{22}x_2+\cdots+a_{2n}x_n=b_2 \\
            \qquad\qquad\cdots\cdots\cdots\cdots     \\
            a_{n1}x_1+a_{n2}x_2+\cdots+a_{nn}x_n=b_n
        \end{cases*}\Longleftrightarrow \bm{Ax}=\bm{\beta}
    \]在$\left|\bm{A}\right|\neq 0$时就有解为
    \[
        \bm{x}=\bm{A}^{-1}\bm{\beta}
    \]
}
\thm{方阵积的行列式}{方阵积的行列式}{
    $\bm{A}$、$\bm{B}$为$n$阶方阵则
    \[
        \left|\bm{AB}\right|=\left|\bm{A}\right|\left|\bm{B}\right|
    \]\begin{proof}
        设$\bm{A}=\left(a_{ij}\right)_{n \times n}$,$\bm{B} =\left(b_{ij}\right)_{n \times n}$,构造
        \[
            \bm{C}_{2n \times 2n}=\begin{bmatrix}
                \bm{A}    & \bm{O} \\
                -\bm{I}_n & \bm{B}
            \end{bmatrix}
        \]
        则
        \begin{align*}
            \left|\bm{C}\right| & =
            \begin{vmatrix}
                a_{11} & a_{12} & \cdots & a_{1n} & 0      & 0      & \cdots & 0      \\
                a_{21} & a_{22} & \cdots & a_{2n} & 0      & 0      & \cdots & 0      \\
                \vdots & \vdots & \ddots & \vdots & \vdots & \vdots & \ddots & \vdots \\
                a_{n1} & a_{n2} & \cdots & a_{nn} & 0      & 0      & \cdots & 0      \\
                -1     & 0      & \cdots & 0      & b_{11} & b_{12} & \cdots & b_{1n} \\
                0      & -1     & \cdots & 0      & b_{21} & b_{22} & \cdots & b_{2n} \\
                \vdots & \vdots & \ddots & \vdots & \vdots & \vdots & \ddots & \vdots \\
                0      & 0      & \cdots & -1     & b_{n1} & b_{n2} & \cdots & b_{nn}
            \end{vmatrix}                                                                                  \\
                                & =
            \begin{vmatrix}
                0      & 0      & \cdots & 0      & \displaystyle\sum_{r}a_{1r}b_{r1} & \displaystyle\sum_{r}a_{1r}b_{r2} & \cdots & \displaystyle\sum_{r}a_{1r}b_{rn} \\
                0      & 0      & \cdots & 0      & \displaystyle\sum_{r}a_{2r}b_{r1} & \displaystyle\sum_{r}a_{2r}b_{r2} & \cdots & \displaystyle\sum_{r}a_{2r}b_{rn} \\
                \vdots & \vdots & \ddots & \vdots & \vdots                            & \vdots                            & \ddots & \vdots                            \\
                0      & 0      & \cdots & 0      & \displaystyle\sum_{r}a_{nr}b_{r1} & \displaystyle\sum_{r}a_{nr}b_{r2} & \cdots & \displaystyle\sum_{r}a_{nr}b_{rn} \\
                -1     & 0      & \cdots & 0      & b_{11}                            & b_{12}                            & \cdots & b_{1n}                            \\
                0      & -1     & \cdots & 0      & b_{21}                            & b_{22}                            & \cdots & b_{2n}                            \\
                \vdots & \vdots & \ddots & \vdots & \vdots                            & \vdots                            & \ddots & \vdots                            \\
                0      & 0      & \cdots & -1     & b_{n1}                            & b_{n2}                            & \cdots & b_{nn}
            \end{vmatrix} \\
                                & =
            \begin{vmatrix}
                \bm{O}    & \bm{AB} \\
                -\bm{I}_n & \bm{B}
            \end{vmatrix}
        \end{align*}
        接下来用Laplace定理以前$n$行展开,化简前
        \begin{align*}
            \left|\bm{C}\right| & =
            \begin{vmatrix}
                \bm{A}    & \bm{O} \\
                -\bm{I}_n & \bm{B}
            \end{vmatrix}                           \\
                                & =\left| \bm{A} \right|
            \left| \bm{B} \right|
        \end{align*}
        化简后
        \begin{align*}
            \left|
            \bm{C}
            \right|
             & =
            \left( -1 \right) ^{1+2+\cdots+2n}
            \left|\bm{AB}\right|
            \left|-\bm{I}_n\right|                                      \\
             & =\left(-1\right)^{n\left(n+1\right)}\left|\bm{AB}\right| \\
             & =\left|\bm{AB}\right|
        \end{align*}
        于是\[
            \left|\bm{AB}\right|=\left|\bm{A}\right|\left|\bm{B}\right|\qedhere
        \]
    \end{proof}
}