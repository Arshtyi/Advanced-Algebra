\newpage
\section{对角化}
\subsection{前提}
\dfn{几何意义的对角化}{几何意义的对角化}{
    设$\bm{\varphi}\in \mathcal{L}\left(
        V_{\mathbb{K}}^n
        \right)$,若$\bm{\varphi}$在某组基下的表示矩阵为对角阵,则称$\bm{\varphi}$可对角化.
}
\dfn{代数意义的对角化}{代数意义的对角化}{
    设$\bm{A}\in M_n\left(
        \mathbb{K}
        \right)$,若$\bm{A}\approx \bm{\varLambda}$则称$\bm{A}$可对角化.
}
\thm{几何意义的可对角化条件}{几何意义的可对角化条件}{
    设$V$是$\mathbb{K}$上的$n$维线性空间,$\bm{\varphi}\in\mathcal{L}\left(V\right)$.则$\bm{\varphi}$可对角化当且仅当$\bm{\varphi}$有$n$个线性无关的特征向量.\begin{proof}
        先考虑必要性,设$\bm{\varphi}$可对角化,即存在$V$的一组基$\left\{
            \bm{e}_1,\bm{e}_2,\cdots,\bm{e}_n
            \right\}$使得$\bm{\varphi}$在这组基下的表示矩阵为$\mathrm{diag}\left\{
            \lambda_1,\lambda_2,\cdots,\lambda_n
            \right\}$,则有
        \[
            \left(
            \bm{\varphi}\left(\bm{e}_1\right),
            \bm{\varphi}\left(\bm{e}_2\right),
            \cdots,
            \bm{\varphi}\left(\bm{e}_n\right)
            \right)=
            \left(
            \bm{e}_1, \bm{e}_2,\cdots,\bm{e}_n
            \right)\begin{bmatrix}
                \lambda_1 &           &        &           \\
                          & \lambda_2 &        &           \\
                          &           & \ddots &           \\
                          &           &        & \lambda_n
            \end{bmatrix}
        \]于是$\bm{\varphi}\left(\bm{e}_i\right)=\lambda_i\bm{e}_i,\forall 1\leqslant i\leqslant n.$于是$\lambda_1,\lambda_2,\cdots,\lambda_n$正是
        $\bm{\varphi}$的特征值,而$\bm{e}_1,\bm{e}_2,\cdots,\bm{e}_n$正是对应于这$n$个特征值的特征向量,并且它们作为基向量一定是线性无关的.必要性证毕.

        再考虑充分性,设$\bm{\varphi}$有$n$个线性无关的特征向量$\bm{e}_1,\bm{e}_2,\cdots,\bm{e}_n$,则它们构成$V$的一组基,且$\bm{\varphi}\left(\bm{e}_i\right)=\lambda_i\bm{e}_i,\forall 1\leqslant i\leqslant n$.于是
        \[
            \left(
            \bm{\varphi}\left(\bm{e}_1\right),
            \bm{\varphi}\left(\bm{e}_2\right),
            \cdots,
            \bm{\varphi}\left(\bm{e}_n\right)
            \right)=
            \left(
            \bm{e}_1, \bm{e}_2,\cdots,\bm{e}_n
            \right)\begin{bmatrix}
                \lambda_1 &           &        &           \\
                          & \lambda_2 &        &           \\
                          &           & \ddots &           \\
                          &           &        & \lambda_n
            \end{bmatrix}
        \]
        即$\bm{\varphi}$在这组基下的表示矩阵为对角阵,故$\bm{\varphi}$可对角化.
    \end{proof}
}
\thm{代数意义的可对角化条件}{代数意义的可对角化条件}{
    设$\bm{A}\in M_n\left(
        \mathbb{K}
        \right)$则$\bm{A}$可复对角化当且仅当$\bm{A}$有$n$个线性无关的特征向量.
}
\exa{}{}{
    考虑矩阵
    \[
        \begin{bmatrix}
            1 & 1 \\
            0 & 1
        \end{bmatrix}
    \]其特征值为$1,1$,但只有一个线性无关的特征向量$k\begin{pmatrix}
            1 \\0
        \end{pmatrix}\left(
        k\in\mathbb{K}\backslash \left\{0\right\}
        \right)$,故不可对角化.

    或者考虑反证法,若$\bm{A}$可对角化,则存在非异阵$\bm{P}\in M_2\left(
        \mathbb{K}
        \right)$使得
    $\bm{P}^{-1}\bm{AP}=\bm{I}_2$,于是$\bm{A}=\bm{P}\bm{I}_2\bm{P}^{-1}=\bm{I}_2$,矛盾.
}
\thm{}{特征子空间的和是直和}{
    设$\bm{\varphi}\in
        \mathcal{L}\left(
        V_{\mathbb{K}}^n
        \right)
    $,$\lambda_1,\lambda_2,\cdots,\lambda_k$是$\bm{\varphi}$的不同特征值,$V_1,V_2,\cdots,V_k$是对应的特征子空间,则
    \[
        V_1+V_2+\cdots+V_k=V_1\oplus
        V_2\oplus\cdots\oplus V_k
    \]\begin{proof}
        对$k$进行归纳,$k=1$显然.设不同特征值的个数$<k$时成立,下面考虑其等于$k$时的情况.此时有
        \[
            V_1+V_2+\cdots+V_{k-1}=V_1\oplus
            V_2\oplus\cdots\oplus V_{k-1}
        \]即$V_i\cap
            \left(
            V_1+V_2+\cdots+V_{i-1}
            \right)=0,\forall 2\leqslant i\leqslant k-1$.要证
        \[
            V_1+V_2+\cdots+V_k=V_1\oplus
            V_2\oplus\cdots\oplus V_k
        \]只需证$V_k\cap
            \left(
            V_1+V_2+\cdots+V_{k-1}
            \right)=0$.

        任取$\bm{v}\in V_k\cap\left(
            V_1+V_2+\cdots+V_{k-1}
            \right)$则
        \[
            V_k \ni \bm{v} =
            \bm{v}_1+\bm{v}_2+\cdots+\bm{v}_{k-1}
        \]其中$\bm{v}_i\in V_i,\forall 1\leqslant i\leqslant k -1.$两边同时作用$\bm{\varphi}$,则
        \[
            \lambda_k\bm{v} =
            \lambda_1\bm{v}_1+\lambda_2\bm{v}_2+\cdots+\lambda_{k-1}\bm{v}_{k-1}
        \]又因为
        \[
            \lambda_k\bm{v} =
            \lambda_k\bm{v}_1+\lambda_k\bm{v}_2+\cdots+\lambda_{k}\bm{v}_{k-1}
        \]即有
        \[
            \bm{0} =
            \left(
            \lambda_1-\lambda_k
            \right)\bm{v}_1+\left(
            \lambda_2-\lambda_k
            \right)\bm{v}_2+\cdots+\left(
            \lambda_{k-1}-\lambda_k
            \right)\bm{v}_{k-1}
        \]
        由直和性质\cref{thm:直和的判定}知零向量分块表示唯一,又因为$\lambda_i\neq \lambda_j,\forall i\neq j$,故$\bm{v}_1=\bm{v}_2=\cdots=\bm{v}_{k-1}=\bm{0}$,即$\bm{v}=\bm{0}$,从而
        \[
            V_1+V_2+\cdots+V_k=V_1\oplus
            V_2\oplus\cdots\oplus V_k\qedhere
        \]
    \end{proof}
}
\cor{}{不同特征值的特征向量相互正交}{
    不同特征值的特征向量必定线性无关.\begin{proof}
        设$\lambda_1,\lambda_2,\cdots,\lambda_k$是不同的特征值,$\bm{v}_1,\bm{v}_2,\cdots,\bm{v}_k$是对应的特征向量.设
        \[
            c_1\bm{v}_1+c_2\bm{v}_2+\cdots+c_k\bm{v}_k=\bm{0}
        \]
        由\cref{thm:特征子空间的和是直和}及\cref{thm:直和的判定}且$\bm{v}_i\neq \bm{0}$知$c_1=c_2=\cdots=c_k$.
    \end{proof}
}
\thm{}{不同特征值个数与可对角化}{
    设$\bm{\varphi}\in \mathcal{L}\left(
        V_{\mathbb{K}}^n
        \right)$有$n$个不同的特征值,则$\bm{\varphi}$可对角化.\begin{proof}
        设$\bm{\varphi}$的特征值为$\lambda_1,\lambda_2,\cdots,\lambda_n$,由于它们不同,故对应的特征向量$\bm{v}_1,\bm{v}_2,\cdots,\bm{v}_n$线性无关,于是$\bm{\varphi}$有$n$个线性无关的特征向量,故可对角化.
    \end{proof}
    这是一个充分条件,但不是必要条件.
}
\exa{}{}{
    设$\bm{\varphi}=c\bm{I}_V,\bm{A}=c\bm{I}_n$是可对角化的但特征值全为$c$.
}
实际上,任给一个矩阵,其可对角化的概率要远大于不可对角化的概率,并且可以用可对角化的矩阵取去逼近不可对角化的矩阵.
\thm{可对角化的充分必要条件}{可对角化的充分必要条件}{
    设$V$是$\mathbb{C}$上的$n$维线性空间,$
        \bm{\varphi}\in \mathcal{L}\left(
        V
        \right)
    $,设全体不同特征值$\lambda_1,\lambda_2,\cdots,\lambda_k$,$V_1,V_2,\cdots,V_k$为对应的特征子空间,则$\bm{\varphi}$可对角化当且仅当
    \[
        V=V_1\oplus V_2\oplus\cdots\oplus V_k
    \]\begin{proof}
        考虑充分性,设
        \[
            V=V_1\oplus V_2\oplus\cdots\oplus V_k
        \]取$V_1$的基$\left\{
            \bm{v}_{11},\bm{v}_{12},\cdots,\bm{v}_{1n_1}
            \right\}$,$V_2$的基$\left\{
            \bm{v}_{21},\bm{v}_{22},\cdots,\bm{v}_{2n_2}
            \right\}$,$\cdots$,$V_k$的基$\left\{
            \bm{v}_{k1},\bm{v}_{k2},\cdots,\bm{v}_{kn_k}
            \right\}$,则$
            \left\{
            \bm{v}_{11},\bm{v}_{12},\cdots,\bm{v}_{1n_1},\bm{v}_{21},\bm{v}_{22},\cdots,\bm{v}_{2n_2},\cdots,\bm{v}_{k1},\bm{v}_{k2},\cdots,\bm{v}_{kn_k}
            \right\}
        $是$V$的一组基.于是$\bm{\varphi}$有$n$个线性无关的特征向量,故可对角化.

        再考虑必要性,设$\bm{\varphi}$有$n$个线性无关的特征向量$\bm{e}_1,\bm{e}_2,\cdots,\bm{e}_n$,这也是$V$的一组基,于是$\forall \bm{\alpha}\in V$
        \[
            \bm{\alpha}=a_1\bm{e}_1 + a_2\bm{e}_2+\cdots+a_n\bm{e}_n
        \]
        进行适当调序可以保证$a_{i_1}\bm{e}_{i_1},a_{i_1}\bm{e}_{i_r}\cdots,a_{i_r}\bm{e}_{i_r}\in V_r\left(
            1   \leqslant r\leqslant k
            \right)$,于是
        \[
            V=V_1+ V_2+\cdots+V_k
        \]又考虑到\cref{thm:特征子空间的和是直和},故
        \[
            V=V_1\oplus V_2\oplus\cdots\oplus V_k
            \qedhere
        \]
    \end{proof}
}
下面利用几何重数与代数重数的关系来讨论对角化的条件.
\subsection{重数}
\dfn{代数重数与几何重数}{代数重数与几何重数}{
    设$\lambda_0$是$\bm{\varphi}$的特征值,$V_0$是其特征子空间,则称$\dim V_0$为$\lambda_0$的度数或几何重数.

    $\lambda_0$作为$\bm{\varphi}$的特征多项式的根的重数称为$\lambda_0$的重数或代数重数.
}
\clm{几何重数与代数重数的区别}{}{
    几何重数与代数重数不见得相等.由线性映射维数公式知,几何重数
    \[
        \dim V _0=\dim\mathrm{Ker}\left(
        \lambda_0\bm{I}_V-\bm{\varphi}
        \right)=n-\mathrm{r}\left(
        \lambda_0\bm{I}_V-\bm{\varphi}
        \right)
    \]
    代数重数取决于特征多项式$\left|
        \lambda\bm{I}_V-\bm{\varphi}
        \right|$的因式分解.
}
\lem{代数重数与几何重数的关系}{代数重数与几何重数的关系}{
    设$
        \bm{\varphi}\in \mathcal{L}\left(
        V_{\mathbb{K}}^n
        \right)
    $,$\lambda_0$是$\bm{\varphi}$的一个特征值,则$\lambda_0$的几何重数一定小于等于其代数重数.\begin{proof}
        设几何重数$t=\dim V_0$,代数重数为$m$.

        取$V_0$的一组基$\left\{
            \bm{e}_1    ,\bm{e}_2,\cdots,\bm{e}_t
            \right\}$扩张为$V$的一组基$\left\{
            \bm{e}_1,\bm{e}_2,\cdots,\bm{e}_t,\bm{e}_{t+1},\cdots,\bm{e}_n
            \right\}$,则有
        \begin{align*}
              & \left(
            \bm{\varphi}\left(
                \bm{e}_1
                \right),
            \bm{\varphi}\left(
                \bm{e}_2
                \right),
            \cdots,
            \bm{\varphi}\left(
                \bm{e}_t
                \right),
            \bm{\varphi}\left(
                \bm{e}_{t+1}
                \right),
            \cdots,
            \bm{\varphi}\left(
                \bm{e}_n
                \right)
            \right)    \\
            = & \left(
            \bm{e}_1,\bm{e}_2,\cdots,\bm{e}_t,\bm{e}_{t+1},\cdots,\bm{e}_n
            \right)\begin{bmatrix}
                       \lambda_0\bm{I}_t & \bm{C} \\
                       \bm{O}            & \bm{B}
                   \end{bmatrix}
        \end{align*}
        不妨设$\bm{A}=\begin{bmatrix}
                \lambda_0\bm{I}_t & \bm{C} \\
                \bm{O}            & \bm{B}
            \end{bmatrix}$,于是特征多项式
        \begin{align*}
            \left|\lambda\bm{I}_V-\bm{\varphi}\right| & =
            \left|
            \lambda \bm{I}_n-\bm{A}
            \right|                                       \\
                                                      & =
            \begin{vmatrix}
                \left(
                \lambda-\lambda_0
                \right)\bm{I}_t & -\bm{C}                    \\
                \bm{O}          & \lambda\bm{I}_{n-t}-\bm{B}
            \end{vmatrix}
        \end{align*}
        由Laplace定理知上式等于$\left(
            \lambda-\lambda_0
            \right)^{\prime}\left|
            \lambda\bm{I}_{n-t}-\bm{B}
            \right|$.于是$m\geqslant t.$
    \end{proof}
}
\dfn{完全特征向量系}{完全特征向量系}{
    设
    $\bm{\varphi}\in \mathcal{L}\left(
        V_{\mathbb{K}}^n
        \right)
    $,若对$\bm{\varphi}$的任一特征值$\lambda_0$,其几何重数等于代数重数,则称$\bm{\varphi}$有完全的特征向量系.
}
\thm{可对角化的条件}{可对角化的条件}{
    设
    $\bm{\varphi}\in \mathcal{L}\left(
        V_{\mathbb{K}}^n
        \right)
    $,则$\bm{\varphi}$可对角化当且仅当$\bm{\varphi}$有完全的特征向量系.\begin{proof}
        只要证明$\bm{\varphi}$有完全的特征向量系当且仅当
        \[
            V= V_1\oplus V_2\oplus\cdots\oplus V_k
        \]
        其中,$\lambda_1,\lambda_2,\cdots,\lambda_k$是$\bm{\varphi}$的全体不同特征值,$V_1,V_2,\cdots,V_k$是对应的特征子空间.

        先考虑充分性,设$\lambda_i$的几何重数
        $t_i=\dim V_i\left(
            \forall 1\leqslant i\leqslant k
            \right)$和代数重数$m_i\left(
            \forall 1\leqslant i\leqslant k
            \right)$.于是
        $t_i\leqslant m_i\left(
            \forall 1\leqslant i\leqslant k
            \right)$且\[m_1+m_2+\cdots+m_k=n\]因为
        \begin{align*}
            n & =\dim V=\dim \left(
            V_1\oplus   V_2\oplus\cdots\oplus V_k
            \right)                                \\
              & =\dim V_1+\dim V_2+\cdots+\dim V_k \\
              & =t_1+t_2+\cdots+t_k                \\
              & \leqslant m_1+m_2+\cdots+m_k       \\
              & =n                                 \\
              & \Longrightarrow
            m_i=t_i\left(
            \forall 1\leqslant i\leqslant k
            \right)
        \end{align*}
        于是$\bm{\varphi}$有完全的特征向量系.

        在考虑必要性,即$\bm{\varphi}$有完全的特征向量系即$t_i=m_i\left(
            \forall 1\leqslant i\leqslant k
            \right)$,则
        \begin{align*}
            \dim\left(
            V_1\oplus V_2\oplus\cdots\oplus V_k
            \right) & =\dim V_1+\dim V_2+\cdots+\dim V_k
            \\
                    & =t_1+t_2+\cdots+t_k                                           \\
                    & =m_1+m_2+\cdots+m_k                                           \\
                    & =n                                                            \\
                    & =\dim V                                                       \\
                    & \Longrightarrow V=V_1\oplus V_2\oplus\cdots\oplus V_k\qedhere
        \end{align*}
    \end{proof}
}
\exa{}{}{
    考虑矩阵
    \[
        \bm{A}=\begin{bmatrix}
            1 & 1 \\
            0 & 1
        \end{bmatrix}
    \]对于其特征值$1$,代数重数为$2$,几何重数为$1$,故没有完全的特征向量系,因此不可对角化.
}
\exa{}{}{
    若已知$\bm{A}$可对角化,如何求得$\bm{P}$使得$\bm{P}^{-1}\bm{AP}$为对角阵.\begin{solution}
        令$\bm{P}=\left(
            \bm{\alpha}_1,\bm{\alpha}_2,\cdots,\bm{\alpha}_n
            \right)$,其为非异阵即$\bm{\alpha}_1,
            \bm{\alpha}_2,\cdots,\bm{\alpha}_n$线性无关.若
        \[
            \bm{P}^{-1}\bm{AP}=\begin{pmatrix}
                \lambda_1 &           &        &           \\
                          & \lambda_2 &        &           \\
                          &           & \ddots &           \\
                          &           &        & \lambda_n
            \end{pmatrix}
            \Longrightarrow
            \bm{AP}=
            \bm{P}\begin{pmatrix}
                \lambda_1 &           &        &           \\
                          & \lambda_2 &        &           \\
                          &           & \ddots &           \\
                          &           &        & \lambda_n
            \end{pmatrix}
        \]即$\bm{A\alpha}_1=\lambda_1\bm{\alpha}_1,
            \bm{A\alpha}_2=\lambda_2\bm{\alpha}_2,\cdots,
            \bm{A\alpha}_n=\lambda_n\bm{\alpha}_n
        $.即$\bm{P}$的$n$个列向量就是$\bm{A}$的$n$个线性无关的特征向量.
    \end{solution}

}
\rem{}{}{
    由于特征向量的线性组合仍为特征向量,故$\bm{P}$的选取一定不是唯一的,但是$\bm{P}$的第$i$个列向量一定对应于$\bm{A}$的第$i$个特征值.
}
\exa{}{}{
    考虑矩阵
    \[
        \bm{A}=\begin{bmatrix}
            1 & 0 & 0 \\-2&5&-2\\-2 & 4 & -1
        \end{bmatrix}
    \]\begin{solution}
        先计算各代数重数与几何重数
        \begin{align*}
            \left|
            \lambda\bm{I}_3-\bm{A}
            \right| & =\begin{vmatrix}
                           \lambda-1 & 0 & 0 \\2&\lambda-5&2\\2 & -4 & \lambda+1
                       \end{vmatrix} \\
                    & =\left(
            \lambda-1
            \right)^2\left(
            \lambda-3
            \right)
        \end{align*}于是特征值$\lambda_1=1$,代数重数为$2$,解方程组得
        \begin{align*}
            \begin{bmatrix}
                0 & 0 & 0 \\2&-4&2\\2 & -4 & 2
            \end{bmatrix}\longrightarrow
            \begin{bmatrix}
                1 & -2 & 1 \\0&0&0\\0 & 0 & 0
            \end{bmatrix}
        \end{align*}基础解系为$\left(
            2,1,0
            \right)',\left(
            -1,0,1
            \right)'$,故几何重数为$2$.

        特征值$\lambda_2=3$,代数重数为$1$,解方程组得
        \begin{align*}
            \begin{bmatrix}
                2 & 0 & 0 \\2&-2&2\\2 & -4 & 4
            \end{bmatrix}\longrightarrow
            \begin{bmatrix}
                1 & 0 & 0 \\0&1&-1\\0 & 0 & 0
            \end{bmatrix}
        \end{align*}基础解系为$\left(
            0,1,1
            \right)'$,故几何重数为$1$(事实上,可以证明,代数重数为$1$时几何重数必定为$1$).

        于是$\bm{A}$一定可对角化且
        \[
            \bm{P}=\begin{bmatrix}
                2 & -1 & 0 \\1&0&1\\0 & 1 & 1
            \end{bmatrix}
            ,
            \bm{P}^{-1}\bm{AP}=\begin{bmatrix}
                1 & 0 & 0 \\0&1&0\\0 & 0 & 3
            \end{bmatrix}
        \]
    \end{solution}
}
\exa{}{}{
    求可对角化矩阵$\bm{A}$的幂$\bm{A}^m$.\begin{solution}
        显然
        \[
            \bm{A}^m=
            \bm{P}\begin{bmatrix}
                \lambda_1^m &             &        &             \\
                            & \lambda_2^m &        &             \\
                            &             & \ddots &             \\
                            &             &        & \lambda_n^m
            \end{bmatrix}\bm{P}^{-1}
        \]
    \end{solution}
}
\exa{}{}{
    设$\bm{A}=\begin{bmatrix}
            1 & 0 \\1 & -2
        \end{bmatrix}$,求$\bm{A}^{m}.$\begin{solution}
        注意到这是一个下三角阵,特征值为$1$和$-2$,于是一定有两个线性无关的特征向量,故一定可对角化.求出\[
            \bm{P}=\begin{bmatrix}
                3 & 0 \\ 1 & 1
            \end{bmatrix}
        \]使得
        \[
            \bm{P}^{-1}\bm{AP}=\begin{bmatrix}
                1 & 0 \\0 & -2
            \end{bmatrix}
        \]于是
        \begin{align*}
            \bm{A}^m & =\bm{P}\begin{bmatrix}
                                  1^m & 0 \\0 & (-2)^m
                              \end{bmatrix}\bm{P}^{-1} \\
                     & =\begin{bmatrix}
                            1       & 0      \\
                            \frac{1}{3}\left(
                            1   -(-2)^m
                            \right) & (-2)^m
                        \end{bmatrix}
        \end{align*}
    \end{solution}
}
\clm{}{}{
    事实上,考虑几何重数还可以有另外的方法,这在前文已述.因为
    \[
        V_{\lambda_0}=\left\{
        \bm{v}\in \mathbb{C}^n\mid
        \bm{A}\bm{v}=\lambda_0\bm{v}
        \right\}=\left(
        \lambda_0\bm{I}_n-\bm{A}
        \right)\bm{x}=\bm{0}\textup{的解空间}
    \]故
    几何重数$\dim V_{\lambda_0}=n-\mathrm{r}\left(
        \lambda_0\bm{I}_n-\bm{A}
        \right)$.
}