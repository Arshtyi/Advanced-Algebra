\section{特征值与特征向量}
\subsection{引入}
考虑一个曾提出的问题,如何找到一个线性空间的一组基,使得该线性空间上所有线性变换的表示矩阵尽量简单(比如对角阵).或者说,能否找到一些尽量简单的矩阵使得其他矩阵均与它们中的某一个相似(比如对角阵).

设线性空间$V_{\bbk}^n$,$\bm{\varphi}\in\call  \left(V\right)$,且$\bm{\varphi}$在$\left\{
    \bm{e}_1,\bm{e}_2,\cdots,\bm{e}_n
    \right\}$这组基下的表示矩阵为$\bm{A}=\diag \,\left\{
    \lambda_1,\lambda_2,\cdots,\lambda_n
    \right\}$,从而
\[
    \left(
    \bm{\varphi}\left(\bm{e}_1\right),\bm{\varphi}\left(\bm{e}_2\right),\cdots,\bm{\varphi}\left(\bm{e}_n\right)
    \right)=
    \left(
    \bm{e}_1,\bm{e}_2,\cdots,\bm{e}_n
    \right)\begin{pmatrix}
        \lambda_1 & 0         & \cdots & 0         \\
        0         & \lambda_2 & \cdots & 0         \\
        \vdots    & \vdots    & \ddots & \vdots    \\
        0         & 0         & \cdots & \lambda_n
    \end{pmatrix}
\]即$\bm{\varphi}\left(
    \bm{e}_i
    \right)=\lambda_i\bm{e}_i,\forall 1\leqslant i\leqslant n.$那么
\[
    \bm{\alpha}= \sum_{i=1}^{n} a_i\bm{e}_i\Longrightarrow
    \bm{\varphi}\left(
    \bm{\alpha}
    \right)=\sum_{i=1}^{n}a_i\lambda_i
    \bm{e}_i
\]

设$\lambda_1,\cdots,\lambda_r$非零,$
    \lambda_{r+1}=\cdots=\lambda_n=0.$则$\rmr \left(
    \bm{\varphi}
    \right)=r$且$\Ker \bm{\varphi}=L\left(
    \bm{e}_{r+1},\cdots,\bm{e}_n
    \right),\Image \bm{\varphi}=L\left(
    \bm{e}_1,\cdots,\bm{e}_r
    \right)$.

这一切的关键在于$
    \bm{\varphi}\left(
    \bm{e}_i
    \right)=\lambda_i\bm{e}_i,\bm{e}_i\neq 0,\forall 1\leqslant i\leqslant n
$.这样的$\lambda_i$称为线性变换$\bm{\varphi}$的特征值,$\bm{e}_i$称为$\bm{\varphi}$的特征向量.
\subsection{特征值与特征向量}
\dfn{线性变换意义的特征值与特征向量}{线性变换意义的特征值与特征向量}{
    设$\bm{\varphi}\in\call  \left(
        V_{\bbk }^n
        \right)$,若存在$\lambda\in\bbk ,\bm{0}\neq \bm{e}\in V\st$
    \[
        \bm{\varphi}\left(\bm{e}\right)=\lambda\bm{e}
    \]则称$\lambda$为$\bm{\varphi}$的特征值,$\bm{e}$为$\bm{\varphi}$对应于特征值$\lambda$的特征向量.
}
\dfn{线性变换的特征子空间}{线性变换的特征子空间}{
    对于每一个特征值$\lambda$,其对应的特征向量加上一个零向量构成的集合称为$\bm{\varphi}$关于特征值$\lambda$的特征子空间,记作$V_{\lambda}\left(\bm{\varphi}\right)=
        \left\{
        \bm{v}\in V\mid \bm{\varphi}\left(\bm{v}\right)=\lambda\bm{v}
        \right\}$.
}
\thm{特征子空间的维数}{特征子空间的维数}{
根据\cref{def:线性变换的特征子空间}知特征子空间$V_{\lambda_0}$的维数$\mathrm{dim}V_{\lambda_0}$等于$\bm{\varphi}$的特征值$\lambda_0$的几何重数.
}
\lem{}{特征子空间是不变子空间}{
    特征子空间$V_{\lambda}\left(\bm{\varphi}\right)$是$V$的子空间,并且是不变子空间.
}
下面考虑代数的角度.一般地,设$\bm{\varphi}$在某一组基下的表示矩阵为$\bm{A}\in M_n\left(
    \bbk
    \right)$,$\bm{e}$的坐标向量为$\bm{\alpha}\in\bbk ^n$,则$\bm{\varphi}\left(
    \bm{e}
    \right)$的坐标向量为$\bm{A\alpha}$,于是
\[
    \bm{\varphi}\left(\bm{e}\right)=\lambda\bm{e}\Longleftrightarrow
    \bm{A\alpha}=\lambda\bm{\alpha}
\]
\dfn{矩阵的特征值与特征向量}{矩阵的特征值与特征向量}{
    设$\bm{A}\in M_n\left(
        \bbk
        \right)$,若存在$\lambda\in\bbk ,\bm{0}\neq\bm{\alpha}\in\bbk ^n\st$
    \[
        \bm{A\alpha}=\lambda\bm{\alpha}
    \]则称$\lambda$为$\bm{A}$的特征值,$\bm{\alpha}$为$\bm{A}$对应于特征值$\lambda$的特征向量.
}
\dfn{矩阵的特征子空间}{矩阵的特征子空间}{
    对于每一个特征值$\lambda$,定义$V_{\lambda}$为线性方程组
    \[
        \left(
        \lambda\bm{I}_n-\bm{A}
        \right)\bm{\alpha}=\bm{0}
    \]
    的解空间,称为$\bm{A}$关于特征值$\lambda$的特征子空间.
}
\clm{}{}{
    $\lambda_0$为$\bm{A}$的特征值当且仅当
    $\exists \bm{0}\neq \bm{\alpha}\in\bbk ^n \st\bm{A\alpha}=\lambda_0\bm{\alpha}$当且仅当
    线性方程组$\left(
        \lambda_0\bm{I}_n-\bm{A}
        \right)\bm{\alpha}=\bm{0}$有非零解当且仅当$\left(
        \lambda_0\bm{I}_n-\bm{A}
        \right)$非奇异当且仅当
    $\left|\lambda_0\bm{I}_n-\bm{A}\right|=0$.

    即$\lambda$是方程
    \[\left|\lambda\bm{I}_n-\bm{A}\right|=0\]的根.即考虑
    \[
        \left|
        \lambda\bm{I}_ n-\bm{A}
        \right|=\begin{vmatrix}
            \lambda-a_{11} & -a_{12}        & \cdots & -a_{1n}        \\
            -a_{21}        & \lambda-a_{22} & \cdots & -a_{2n}        \\
            \vdots         & \vdots         & \ddots & \vdots         \\
            -a_{n1}        & -a_{n2}        & \cdots & \lambda-a_{nn}
        \end{vmatrix}
    \]
    这是一个关于$\lambda$的$n$次首一多项式,称为$\bm{A}$的特征多项式.
}
\dfn{矩阵的特征多项式}{矩阵的特征多项式}{
    设$\bm{A}\in M_n\left(\bbk \right)$,称$\left|
        \lambda\bm{I}_n-\bm{A} \right|$为$\bm{A}$的特征多项式.
}
\lem{特征值在相似关系下不改变}{特征值在相似关系下不改变}{
    相似矩阵有相同的特征多项式,进一步地有相同的特征值(计重数).\begin{proof}
        设$\bm{B}=\bm{P}^{-1}\bm{AP}$,则
        \begin{align*}
            \left|
            \lambda\bm{I}_n-\bm{B}
            \right| & =
            \left|
            \lambda\bm{I}_n-\bm{P}^{-1}\bm{AP}
            \right|=\left|
            \bm{P}^{-1}\left(
            \lambda\bm{I}_n-\bm{A}
            \right)\bm{P}
            \right|           \\
                    & =\left|
            \bm{P}^{-1}
            \right|\left|
            \lambda\bm{I}_n-\bm{A}
            \right|\left|
            \bm{P}
            \right|=\left|
            \lambda\bm{I}_n-\bm{A}
            \right|\qedhere
        \end{align*}
    \end{proof}
}
\dfn{线性变换的特征多项式}{线性变换的特征多项式}{
    设$\bm{\varphi}\in\call  \left(
        V_{\bbk }^n
        \right)$,任取一组基下的表示矩阵$\bm{A}$,定义$\bm{\varphi}$的特征多项式是$\bm{A}$的特征多项式(由上知特征多项式的定义不依赖于基或表示矩阵的选取).记作$\left|
        \lambda\bm{I}_V-\bm{\varphi}\right|$.
}
\lem{}{特征值的和与积}{
    设$\bm{A}$的特征值为$\lambda_1,\lambda_2,\cdots,\lambda_n$,则
    \begin{align*}
        \lambda_1+\lambda_2+\cdots+\lambda_n & =\Tr \bm{A}          \\
        \lambda_1\lambda_2\cdots\lambda_n    & =\left|\bm{A}\right|
    \end{align*}\begin{proof}
        设\begin{align*}
            \left|\lambda\bm{I}_n-\bm{A}\right| & =
            \lambda^n+a_1\lambda^{n-1}+\cdots+a_{n-1}\lambda+a_n \\
                                                & =\left(
            \lambda-\lambda_1
            \right)\left(
            \lambda-\lambda_2
            \right)\cdots\left(
            \lambda-\lambda_n
            \right)
        \end{align*}
        其中$a_1=
            -\left(
            a_{11}+a_{22}+\cdots+a_{nn}
            \right)=-\Tr \bm{A}$,令$\lambda = 0$得$
            a_n=\left(-1\right)^n\left|\bm{A}\right|
        $.考虑\cref{thm:Vieta定理}Veita定理即可.
    \end{proof}
}
\cor{}{r阶主子式之和}{
    进一步,可以证明
    \begin{align*}
        \sum_{1\leqslant i_1<i_2<\cdots<i_r\leqslant n}\lambda_{i_1}\lambda_{i_2}\cdots\lambda_{i_r}=
        \sum_{1\leqslant i_1<i_2<\cdots<i_r\leqslant n} \bm{A}\begin{pmatrix}
                                                                  i_1 & i_2 & \cdots & i_r \\
                                                                  i_1 & i_2 & \cdots & i_r
                                                              \end{pmatrix}
    \end{align*}
}
\cor{}{非异阵判定}{
    设$\bm{A}\in M_n\left(\bbk \right)$,则$\bm{A}$非异当且仅当
    $\lambda_i\neq 0,\forall 1\leqslant i\leqslant n.$\begin{proof}
        \[
            \lambda_1\lambda_2\cdots \lambda_n=\left|\bm{A}\right|\qedhere
        \]
    \end{proof}
}
\pro{求解特征值和特征向量}{求解特征值和特征向量}{
    \begin{enumerate}[label=\arabic*)]
        \item 写出特征矩阵$\lambda \bm{I}_n-\bm{A}$,求出特征多项式$\left|\lambda \bm{I}_n-\bm{A}\right|=0$的根$\lambda_1,\lambda_2,\cdots,\lambda_n$即为特征值.
        \item $\forall 1\leqslant i\leqslant n$,解线性方程组$\left(
                  \lambda_i\bm{I}_n-\bm{A}
                  \right)\bm{\alpha}=\bm{0}$,得到的非零解即为对应于$\lambda_i$的特征向量.
    \end{enumerate}
}
\rem{特征值的范围}{}{
    对于$\bm{A}\in M_n\left(
        \bbk
        \right)$,其特征值可能不在$\bbk $中,例如矩阵
    $\begin{bmatrix}
            0 & -1 \\
            1 & 0
        \end{bmatrix}\in M_2\left(
        \bbr
        \right)$的特征值为$\pm \rmi  \notin \bbr $。因此更一般地,我们考虑特征值相关问题时通常在复数域$\bbc $上考虑.
}
\exa{}{}{
考虑上三角阵
\[
    \bm{A}=\begin{pmatrix}
        a_{11} & a_{12} & \cdots & a_{1n} \\
        0      & a_{22} & \cdots & a_{2n} \\
        \vdots & \vdots & \ddots & \vdots \\
        0      & 0      & \cdots & a_{nn}
    \end{pmatrix}
\]其特征多项式显然为
\begin{align*}
    \left|
    \lambda\bm{I}_n-\bm{A}
    \right| & =
    \begin{vmatrix}
        \lambda-a_{11} & -a_{12}        & \cdots & -a_{1n}        \\
        0              & \lambda-a_{22} & \cdots & -a_{2n}        \\
        \vdots         & \vdots         & \ddots & \vdots         \\
        0              & 0              & \cdots & \lambda-a_{nn}
    \end{vmatrix} \\
            & =\left(
    \lambda-a_{11}
    \right)\left(
    \lambda-a_{22}
    \right)\cdots\left(
    \lambda-a_{nn}
    \right)
\end{align*}特征值为
$\lambda_1=a_{11},\lambda_2=a_{22},\cdots,\lambda_n=a_{nn}$.
}
\exa{}{}{
    考虑矩阵
    \[
        \begin{bmatrix}
            3 & 1 & -1 \\
            2 & 2 & -1 \\
            2 & 2 & 0
        \end{bmatrix}
    \]
    则其特征根
    $\lambda_1=1,\lambda_2=\lambda_3=2$.求解
    \[
        \left(
        \lambda\bm{I}_3-\bm{A}
        \right)\bm{x}=\bm{0}
    \]得:$\lambda_1=1$时
    \[
        \begin{bmatrix}
            -2 & -1 & 1 \\
            -2 & -1 & 1 \\
            -2 & -2 & 1
        \end{bmatrix}\longrightarrow
        \begin{bmatrix}
            1 & 0 & -\frac{1}{2} \\
            0 & 1 & 0            \\
            0 & 0 & 0
        \end{bmatrix}\Longrightarrow
        \bm{\xi}_1=\begin{pmatrix}
            1 \\0\\2
        \end{pmatrix}
    \]
    对应特征向量为$c_1\bm{\xi}_1,c\neq 0.$
    $\lambda_2=\lambda_3=2$时
    \[
        \begin{bmatrix}
            -1 & -1 & 1 \\
            -2 & 0  & 1 \\
            -2 & -2 & 2
        \end{bmatrix}\longrightarrow
        \begin{bmatrix}
            1 & 0 & -\frac{1}{2} \\
            0 & 1 & -\frac{1}{2} \\
            0 & 0 & 0
        \end{bmatrix}\Longrightarrow
        \bm{\xi}_2=\begin{pmatrix}
            1 \\1\\2
        \end{pmatrix}
    \]对应特征向量为$c_2\bm{\xi}_2,c_2\neq 0$.
}
\thm{特征多项式的降阶公式}{特征多项式的降阶公式}{
    设$\bm{A}\in M_{m\times n}\left(\bbc \right),\bm{B}\in M_{n\times m}\left(\bbc \right)\left(m\geqslant n\right)$,则\[
        \left|
        \lambda\bm{I}_m-\bm{A}\bm{B}
        \right|=\lambda^{m-n}\left|
        \lambda\bm{I}_n-\bm{B}\bm{A}
        \right|
    \]
}
\subsection{相似}
\thm{}{必定复相似于三角阵}{
    任一复方阵必复相似于一个上(下)三角阵.\begin{proof}
        设$\bm{A}\in M_n\left(
            \bbc
            \right)$,对阶数$n$进行归纳,当$n=1$时结论显然成立.假设$n-1$时结论成立,考虍$n$时的情况.由\cref{thm:代数学基本定理}代数学基本定理,任取$\bm{A}$的特征值$\lambda_1$和特征向量$\bm{0}\neq \bm{\alpha}_1\in \bbc ^n$即
        \[
            \bm{A\alpha}_1=\lambda_1\bm{\alpha}_1
        \]

        根据\cref{thm:基扩张定理}基扩张定理,将$\bm{\alpha}_1$扩张为全空间$\bbc ^n$的一组基$\left\{
            \bm{\alpha}_1,\bm{\alpha}_2,\cdots,\bm{\alpha}_n
            \right\}$.令$\bm{P}=\left(
            \bm{\alpha}_1,\bm{\alpha}_2,\cdots,\bm{\alpha}_n
            \right)\in M_n\left(
            \bbc
            \right)$且$\bm{P}$非异.因为
        \begin{align*}
            \bm{AP} & =\left(
            \bm{A\alpha}_1,\bm{A\alpha}_2,\cdots,\bm{A\alpha}_n
            \right)           \\
                    & =\left(
            \bm{\alpha}_1,  \bm{A\alpha}_2,\cdots,\bm{A\alpha}_n
            \right)\begin{bmatrix}
                       \lambda_1 & *            \\
                       \bm{O}    & \bm{A}_{n-1}
                   \end{bmatrix}
        \end{align*}
        其中$\bm{A}_{n-1}\in M_{n-1}\left(
            \bbc
            \right)$.于是
        \[
            \bm{P}^{-1}\bm{AP}=\begin{pmatrix}
                \lambda_1 & *            \\
                \bm{O}    & \bm{A}_{n-1}
            \end{pmatrix}
        \]
        由归纳假设,$\exists\textup{非异阵}
            \bm{Q}\in M_{n-1}\left(
            \bbc
            \right)\st$
        \[
            \bm{Q}^{-1}\bm{A}_{n-1}\bm{Q}=\begin{pmatrix}

                \lambda_2 & \cdots & *           \\
                \vdots    & \ddots & \vdots      \\
                \bm{O}    & \cdots & \lambda_{n}
            \end{pmatrix}
        \]作非异阵$
            \bm{R}=\begin{pmatrix}
                1      & \bm{O} \\
                \bm{O} & \bm{Q}
            \end{pmatrix}\in M_n\left(
            \bbc
            \right)$,则
        \begin{align*}
            \bm{R}^{-1}\bm{P}^{-1}\bm{APR} & =
            \bm{R}^{-1}\left(
            \bm{P}^{-1}\bm{A}\bm{P}
            \right)\bm{R}                                                               \\
                                           & =\begin{bmatrix}
                                                  1      & \bm{O}      \\
                                                  \bm{O} & \bm{Q}^{-1}
                                              \end{bmatrix}\begin{bmatrix}
                                                               \lambda_1 & *            \\
                                                               \bm{O}    & \bm{A}_{n-1}
                                                           \end{bmatrix}\begin{bmatrix}
                                                                            1      & \bm{O} \\
                                                                            \bm{O} & \bm{Q}
                                                                        \end{bmatrix} \\
                                           & =
            \begin{bmatrix}
                \lambda_1 & \cdots & *         \\
                \vdots    & \ddots & \vdots    \\
                \bm{O}    & \cdots & \lambda_n
            \end{bmatrix}\qedhere
        \end{align*}
    \end{proof}
}
\cor{}{特征值在同一个域中变换所用非异阵}{
    设$\bm{A}\in M_n\left(
        \bbk
        \right)$的特征值均在$\bbk $中,则$\exists
        \text{非异阵}\bm{P}\in M_n\left(
        \bbk
        \right)\st$
    \[
        \bm{P}^{-1}\bm{AP}=\begin{pmatrix}
            \lambda_1 & *         & \cdots & *         \\
                      & \lambda_2 & \cdots & *         \\
                      &           & \ddots & \vdots    \\
                      &           &        & \lambda_n
        \end{pmatrix}
    \]
}
\rem{}{}{
一个问题是,如果知道了$\bm{A}\in M_n\left(
    \bbk
    \right)$的特征值,那么
\[
    f\left(
    \bm{A}
    \right)=a_m\bm{A}^m+a_{m-1}\bm{A}^{m-1}+\cdots+a_1\bm{A}+a_0\bm{I}_n\in M_n\left(
    \bbk
    \right)
\]的特征值是否也能够知道.
}
\lem{相似关系在幂运算下保持}{相似关系在幂运算下保持}{
    相似关系在幂运算下保持.
    \[
        \left(
        \bm{P}^{-1}\bm{AP}
        \right)^m=\bm{P}^{-1}\bm{A}^m\bm{P}
    \]
}
\cor{相似关系与多项式相容}{相似关系与多项式相容}{
    由\cref{lem:相似关系在幂运算下保持},有
    \[
        f\left(
        \bm{P}^{-1}\bm{AP}
        \right)= \bm{P}^{-1}f\left(
        \bm{A}
        \right)\bm{P}
    \]即相似与多项式相容.
}
\cor{相似关系在求逆运算下保持}{相似关系在求逆运算下保持}{
    相似关系在求逆运算下保持.
    \[
        \left(
        \bm{P}^{-1}\bm{AP}
        \right)^{-1}=\bm{P}^{-1}\bm{A}^{-1}\bm{P}
    \]
}
\cor{相似关系在伴随运算下保持}{相似关系在伴随运算下保持}{
    相似关系在伴随运算下保持. \[
        \left(
        \bm{P}^{-1}\bm{AP}
        \right)^*=\bm{P}^*\bm{A}^*\left(
        \bm{P}^*
        \right)^{-1}
    \]
}
\subsection{运算}
\pro{}{多项式的特征值}{
    设$\bm{A}\in M_n\left(
        \bbk
        \right)$的特征值为$\lambda_1,\lambda_2,\cdots,\lambda_n$,则$f\left(
        \bm{A}
        \right)$的特征值为$f\left(
        \lambda_1
        \right),f\left(
        \lambda_2
        \right),\cdots,f\left(
        \lambda_n
        \right)$.\begin{proof}
        存在非异阵$\bm{P}\in M_n\left(
            \bbk
            \right)$使得
        \[
            \bm{P}^{-1}\bm{AP}=\begin{pmatrix}
                \lambda_1 & \cdots & *         \\
                \vdots    & \ddots & \vdots    \\
                \bm{O}    & \cdots & \lambda_n
            \end{pmatrix}
        \]考虑
        $
            \bm{P}^{-1}f\left(
            \bm{A}
            \right)\bm{P}=f\left(
            \bm{P}^{-1}\bm{AP}
            \right)
        $,因为上三角阵的加减乘运算仍为上三角阵,且运算的主对角线元素等于主对角线元素的运算,则
        \[
            f\left(
            \bm{P}^{-1}\bm{AP}
            \right)=\begin{pmatrix}
                f\left(
                \lambda_1
                \right) & \cdots & *       \\
                \vdots  & \ddots & \vdots  \\
                \bm{O}  & \cdots & f\left(
                \lambda_n
                \right)
            \end{pmatrix}\qedhere
        \]
    \end{proof}
}
\pro{}{特征值适合矩阵适合的多项式}{
    设$\bm{A}\in M_n\left(
        \bbk
        \right)$适合$g\left(
        x
        \right)\in \bbk \left[x\right]$即$g\left(\bm{A}\right)=\bm{O}$,则$\bm{A}$的任意特征值也适合该多项式即$g\left(
        \lambda_i
        \right)=0,\forall 1\leqslant i\leqslant n$.\begin{proof}
        设$\bm{A\alpha}=\lambda\bm{\alpha},\bm{0}\neq \bm{\alpha}\in \bbc ^n$,则$\bm{A}^m\bm{\alpha}=\lambda^m\bm{\alpha}$.于是
        \[
            \bm{O}=g\left(\bm{A}\right)\bm{\alpha}=g\left(\lambda\right)\bm{\alpha}\Longrightarrow
            g\left(\lambda\right)=0
        \]

        亦可用\cref{prop:多项式的特征值}完成,因为$\bm{O}=g\left(\bm{A}\right)$的特征值为$g\left(\lambda_1\right),
            g\left(\lambda_2\right),\cdots,g\left(\lambda_n\right)$且零矩阵的特征值均为零,得证.
    \end{proof}
}
\lem{逆阵的特征值}{逆阵的特征值}{
    设 可逆阵$\bm{A}$的特征值为$\lambda_1,\lambda_2,\cdots,\lambda_n$,则$\bm{A}^{-1}$的特征值为$\lambda_1^{-1},\lambda_2^{-1},\cdots,\lambda_n^{-1}$.\begin{proof}
        因为存在非异阵$\bm{P}\in M_n\left(
            \bbk
            \right)$使得
        \[
            \bm{P}^{-1}\bm{AP}=\begin{pmatrix}
                \lambda_1 & \cdots & *         \\
                \vdots    & \ddots & \vdots    \\
                \bm{O}    & \cdots & \lambda_n
            \end{pmatrix}
        \]因为非异上三角阵的逆阵仍为非异上三角阵,且主对角线元素的逆等于原主对角线元素的逆,所以
        \[
            \bm{P}^{-1}\bm{A}^{-1}\bm{P}=\begin{pmatrix}
                \lambda_1^{-1} & \cdots & *              \\
                \vdots         & \ddots & \vdots         \\
                \bm{O}         & \cdots & \lambda_n^{-1}
            \end{pmatrix}\qedhere
        \]
    \end{proof}
}
\pro{伴随的特征值}{伴随的特征值}{
    设$\bm{A}\in M_n\left(
        \bbk
        \right)$的特征值为$\lambda_1,\lambda_2,\cdots,\lambda_n$,则$\bm{A}^*$的特征值为
    \[
        \prod_{i\neq 1}\lambda_i,\prod_{i\neq 2}\lambda_i,\cdots,\prod_{i\neq n}\lambda_i
    \]\begin{proof}
        注意到上三角阵的伴随阵仍为上三角阵,且主对角线元素的伴随等于其余元素的乘积即可.
    \end{proof}
}