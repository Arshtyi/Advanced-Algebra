\newpage
\section{总结}
\exa{}{}{
    设$\bm{A}^2-\bm{A}-3\bm{I}_n=\bm{O}$,求证:$\bm{A}-2\bm{I}_n$可逆.\begin{proof}
        \begin{enumerate}[label=\arabic*)]
            \item 凑因子:\[
                      \left(\bm{A}-2\bm{I}_n\right)\left(\bm{A}+\bm{I}_n\right)=\bm{I}_n
                  \]
            \item 求解方程组:只要证$\left(
                      \bm{A}-2\bm{I}_n
                      \right)\bm{x}=\bm{0}$只有零解
            \item 互素多项式:\[
                      \left(
                      x^2-x-3,x-2
                      \right)=1
                  \]
            \item 特征值:反证法,考虑设$\bm{A}-2\bm{I}_n$不可逆,即其行列式为零,则$2$是$\bm{A}$的一个特征值,但是该特征值却不适合多项式$x^2-x-3$,矛盾.\qedhere
        \end{enumerate}
    \end{proof}
}
\exa{}{}{
    设$4$阶方阵$\bm{A}$满足$\mathrm{tr}\bm{A}^i=i,i=1,2,3,4$.求$\left|\bm{A}\right|$.\begin{solution}
        设$\bm{A}$的特征值$\lambda_1,\lambda_2,\lambda_3,\lambda_4$,则
        $\bm{A}^i$的特征值为$\lambda_1^i,\lambda_2^i,\lambda_3^i,\lambda_4^i$,于是$\displaystyle
            \sum_{k=1}^{4}\lambda_k^i=i$即Fermal和$s_1=1,s_2=2,s_3=3,s_4=4.$

        因为$\left|\bm{A}\right|=\lambda_1\lambda_2\lambda_3\lambda_4=\sigma_4$.于是由牛顿公式,$k\leqslant n=4$时
        \[
            s_k-s_{k-1}\sigma_1+\cdots+\left(-1\right)^kk\sigma_k=0
        \]
        于是取$k=1,2,3,4$得到$\displaystyle
            \sigma_1=1,\sigma_2=-\frac{1}{2},\sigma_3=\frac{1}{6},\sigma_4=\frac{1}{24}.$
    \end{solution}
}
\exa{}{}{
    设矩阵$\bm{A}^m,\bm{B}^n$且无公共特征值,证明矩阵方程$\bm{AX}=\bm{XB}$只有零解.\begin{proof}
        利用\cref{thm:Cayley-Hamilton定理}Cayley-Hamilton定理,设$\bm{A}$的特征多项式为$f\left(\lambda\right)=\left|\lambda\bm{I}_m-\bm{A}\right|$,于是$f\left(\bm{A}\right)=\bm{O}$.同时容易有$\bm{A}^2\bm{X}=\bm{XB}^2,\cdots$于是有\[\bm{O}=f\left(\bm{A}\right)\bm{X}=\bm{X}f\left(\bm{B}\right)\]

        下面考虑证明$f\left(\bm{B}\right)$非异,不妨设$\bm{B}$的特征值$\mu_1,\mu_2,\cdots,\mu_n$则$f\left(\bm{B}\right)$的特征值为$f\left(\mu_1\right)\neq 0,f\left(\mu_2\right)\neq 0,\cdots,f\left(\mu_n\right)\neq 0$,那么$f\left(\bm{B}\right)$可逆.于是$\bm{X}=\bm{O}$得证.

        也可设$\bm{B}$特征多项式$g\left(\lambda\right)$且$\left(f\left(\lambda\right),g\left(\lambda\right)
            \right)=1$,考虑带余除法则
        \[
            f\left(\lambda\right)u\left(\lambda\right)+g\left(\lambda\right)v\left(\lambda\right)=1
        \]于是$f\left(\bm{B}\right)u\left(\bm{B}\right)=\bm{I}_n$得证$f\left(\bm{B}\right)$可逆.
    \end{proof}
}
\exa{}{}{
    设$n$阶方阵$\bm{A},\bm{B}$,特征值均大于零,且$\bm{A}^2=\bm{B}^2.$求证:$\bm{A}=\bm{B}$.\begin{proof}
        即上例中$\bm{X}=\bm{A}-\bm{B}$,因为
        \[
            \bm{A}\left(\bm{A}-\bm{B}\right)=
            \left(\bm{A}-\bm{B}\right)\left(-\bm{B}\right)
        \]考虑到$\bm{A}$和$-\bm{B}$没有公共特征值,由上例得证.
    \end{proof}
}
\exa{}{}{
    设$n$阶方阵$\bm{A}$的特征值均为偶数,证明:$\bm{X}+\bm{AX}=\bm{XA}^2$只有零解.\begin{proof}
        显然考虑
        \[
            \left(\bm{A}+\bm{I}\right)\bm{X}=\bm{X}\bm{A}^2
        \]由上例,考虑$\bm{A}+\bm{I}$与$\bm{A}^2$的特征值即可,显然成立.
    \end{proof}
}
\exa{}{}{
    设$n$阶方阵$\bm{A}$适合$a_mx^m+
        \cdots+a_1x+a_0
    $且$\displaystyle
        \left|a_m\right|\geqslant
        \sum_{i=0}^{m-1}\left|a_i\right|
    $.证明:$2\bm{X}+\bm{AX}=\bm{XA}$仅有零解.\begin{proof}
        任取一特征值$\lambda_0$,断言$\left|\lambda_0\right|<1$,反证法,设$\left|\lambda_0\right|\geqslant 1.$因为$\lambda_0$适合该多项式即
        \[
            a_m\lambda_0^m+\cdots+a_1\lambda_0+a_0=0
        \]
        于是
        \[
            a_m=-\frac{a_{m-1}}{\lambda_0}-\cdots-\frac{a_1}{\lambda_0^{m-1}}-\frac{a_0}{\lambda_0^m}
        \]
        两边取模长
        \[
            \left|a_m\right|\leqslant
            \sum_{i=0}^{m-1}\left|a_i\right|
        \]显然矛盾.

        于是来考虑$\bm{A}$和$\bm{A}+2\bm{I}_n$的特征值.因为$\bm{A}$的特征值均在复平面单位圆内部,而$\bm{A}+2\bm{I}_2$的特征值在向右平移两个单位的圆内且均不包含边界于是二者没有公共特征值.
    \end{proof}
}
\exa{}{}{
    设矩阵$\bm{A}^m\in M_m\left(
        \mathbb{K}
        \right),\bm{B}^n\in M_n\left(\mathbb{K}\right)$无公共特征值,则$\forall \bm{C}\in M_{m\times n}\left(\mathbb{K}\right)$,矩阵方程$\bm{AX}-\bm{XB}=\bm{C}$有唯一解.\begin{proof}
        考虑$\bm{\varphi}\in \mathcal{L}\left(
            M_{m\times n}\left(\mathbb{K}\right)
            \right):\bm{X}\longmapsto
            \bm{AX}-\bm{XB}$,由上例可知$\bm{\varphi}$是单射,那么也是满射,于是是线性同构.于是$\forall \bm{C}\in M_{m\times n}\left(\mathbb{K}\right)$,一定存在唯一一个$\bm{X}_0\in M_{m\times n}\left(\mathbb{K}\right)$使得\[\bm{\varphi}\left(\bm{X}_0\right)=\bm{AX}_0-\bm{X}_0\bm{B}=\bm{C}\qedhere\]
    \end{proof}
}
\exa{}{}{
    设矩阵$\bm{A}^m,\bm{B}^n$无公共特征值且均可对角化,则$\bm{M}=\begin{pmatrix}
            \bm{A} & \bm{C} \\\bm{O} & \bm{B}
        \end{pmatrix}$可对角化.\begin{proof}
        由上例知存在$\bm{X}_0$使得\[\bm{AX}_0-\bm{X}_0\bm{B}=\bm{C}\]于是
        \begin{align*}
            \begin{pmatrix}
                \bm{I}_m & \bm{X}_0 \\\bm{O} & \bm{I}_n
            \end{pmatrix}\begin{pmatrix}
                             \bm{A} & \bm{C} \\\bm{O} & \bm{B}
                         \end{pmatrix}\begin{pmatrix}
                                          \bm{I}_m & -\bm{X}_0 \\\bm{O} & \bm{I}_n
                                      \end{pmatrix}=\begin{pmatrix}
                                                        \bm{A} & \bm{O} \\\bm{O} & \bm{B}
                                                    \end{pmatrix}
        \end{align*}
        相似得证,于是$\bm{M}$可对角化.
    \end{proof}
}
\exa{}{}{
    设$\bm{A}$为非异循环矩阵,求证:$\bm{A}^{-1}$也是循环矩阵.\begin{proof}
        法一,设\[
            \bm{A}=\begin{pmatrix}
                a_1    & a_2    & \cdots & a_n     \\
                a_n    & a_1    & \cdots & a_{n-1} \\
                \vdots & \vdots &        & \vdots  \\
                a_2    & a_3    & \cdots & a_1
            \end{pmatrix}
        \]并有基础循环矩阵$\bm{J}=\begin{pmatrix}
                 & \bm{I}_{n-1} \\1 &
            \end{pmatrix}$则$\bm{A}=a_1\bm{I}_n+a_2\bm{J}+\cdots+a_n\bm{J}^{n-1}$.作$g\left(x\right)=a_1+a_2x+\cdots+a_nx^{n-1}$且$\bm{J}$的特征多项式$\displaystyle
            f\left(\lambda\right)=\lambda^n-1\Longrightarrow\omega_k=e^{\frac{2k\pi \mathrm{i}}{n}},0\leqslant k\leqslant n-1
        $.因此$\bm{A}$的特征值是$g\left(\omega_k\right),0\leqslant k\leqslant n-1$即特征值,因为$\bm{A}$非异即$g\left(\omega_k\right)$非零即推知$\left(
            f\left(x\right),g\left(x\right)
            \right)=1$即带余除法
        \[
            f\left(x\right)u\left(x\right)+g\left(x\right)v\left(x\right)=1
        \]代入$\bm{J}$即得$\bm{A}^{-1}=v\left(\bm{J}\right).$

        法二,因为$\bm{A}$可逆则$\bm{A}^{-1}=h\left(\bm{A}\right)=h\left(g\left(\bm{J}\right)\right)$得证.
    \end{proof}
}