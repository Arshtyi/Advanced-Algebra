\newpage
\section{极小多项式与Cayley-Hamilton定理}
\subsection{极小多项式}
设$\bm{A}\in M_n\left(
    \bbk
    \right)$,因为
\[
    \dim_{\bbk }M_n\left(
    \bbk
    \right)=n^2
\]故$\bm{A}^{n^2},\bm{A}^{n^2-1},\cdots,\bm{A},\bm{I}_n$这$n^2+1$个矩阵必定线性相关.

即存在不全为$0$的数$c_{n^2},c_{n^2-1},\cdots,c_1,c_0\in\bbk \st$
\[
    c_{n^2}\bm{A}^{n^2}+c_{n^2-1}\bm{A}^{n^2-1}+\cdots+c_1\bm{A}+c_0\bm{I}_n=\bm{O}
\]即$\bm{A}$适合多项式$g\left(x\right)=
    c_{n^2}x^{n^2}+c_{n^2-1}x^{n^2-1}+\cdots+c_1x+c_0\neq 0\in\bbk \left[x\right].$

定义$S=\left\{
    f\left(x\right)\in\bbk \left[x\right]\mid
    f\left(x\right)\neq 0 \textup{且} f\left(\bm{A}\right)=\bm{O}
    \right\}\neq \varnothing $,则$\displaystyle
    k=\min_{f\left(x\right)\in S}\deg f\left(x\right)$必定存在,即$\exists h\left(x\right)\in S\st\deg h\left(x\right)=k.$首一化得到$m\left(x\right)$,即有首一多项式$0\neq m\left(x\right)\in \bbk \left[x\right]\st  m\left(\bm{A}\right)=\bm{O}$且有次数最小性.
\dfn{极小多项式}{极小多项式}{
    设$\bm{A}\in M_n\left(
        \bbk
        \right),m\left(x\right)\in \bbk \left[x\right]$为首一多项式且$m\left(\bm{A}\right)=\bm{O}$.若$m\left(x\right)$是$\bm{A}$适合的所有非零多项式中次数最小者,则称为$\bm{A}$的一个极小多项式(最小多项式).
}
\clm{极小多项式存在性}{}{
    极小多项式一定存在.
}
\lem{}{极小多项式是矩阵适合的多项式的因子}{
    设$\bm{A}$适合$f\left(x\right)\neq 0$,$m\left(x\right)$是$\bm{A}$的极小多项式则$m\left(x\right)\mid f\left(x\right).$\begin{proof}
        考虑带余除法$f\left(x\right)=m\left(x\right)q\left(x\right)+r\left(x\right),\deg r\left(x\right)<\deg m\left(x\right)$,考虑反证法,即设$r\left(x\right)\neq 0$,上式代入$\bm{A}$,于是$r\left(\bm{A}\right)=\bm{O}$,这与$m\left(x\right)$的次数最小性矛盾.
    \end{proof}
}
\lem{极小多项式唯一}{极小多项式唯一}{
    设$\bm{A}\in M_n\left(\bbk \right)$,其极小多项式存在且唯一.\begin{proof}
        设$m\left(x\right),g\left(x\right)$均为$\bm{A}$的极小多项式,只需证$m\left(x\right)=g\left(x\right)$.事实上,$m\left(x\right)\mid g\left(x\right),g\left(x\right)\mid m\left(x\right)$即$\exists 0\neq c\in\bbk \st g\left(x\right)=cm\left(x\right)$但极小多项式均首一于是$c=1$.
    \end{proof}
}
\exa{}{}{
    纯量矩阵$\bm{A}=c\bm{I}_n$的极小多项式为$m\left(x\right)=x-c$.
}
\exa{}{}{
    考虑矩阵$\bm{A}=\begin{bmatrix}
            0 & 1 \\0 & 0
        \end{bmatrix}$其满足$\bm{A}^2=\bm{O}$即$\bm{A}$适合多项式$f\left(x\right)=x^2$,故其极小多项式为必定整除$f\left(x\right)=x^2$即$m\left(x\right)=x$或$m\left(x\right)=x^2$于是$m\left(x\right)=x^2.$
}
\lem{极小多项式在相似关系下不改变}{极小多项式在相似关系下不改变}{
    相似矩阵具有相同的极小多项式.\begin{proof}
        设$\bm{B}=\bm{P}^{-1}\bm{AP}$,$\bm{A}$的极小多项式为$m\left(x\right)$,$\bm{B}$的极小多项式为$g\left(x\right)$,则考虑到相似关系与多项式的相容有
        \[
            g\left(\bm{A}\right)=g\left(\bm{P}\bm{BP}^{-1}\right)=\bm{P}g\left(\bm{B}\right)\bm{P}^{-1}=\bm{O}
        \]于是$m\left(x\right)\mid g\left(x\right)$,同理得证.
    \end{proof}
}
\lem{分块对角矩阵的极小多项式}{分块对角矩阵的极小多项式}{
    设分块对角阵
    \[
        \bm{A}=\begin{bmatrix}
            \bm{A}_1 &          &        &          \\
                     & \bm{A}_2 &        &          \\
                     &          & \ddots &          \\
                     &          &        & \bm{A}_k
        \end{bmatrix}
    \]
    其中$\bm{A}_i\left(
        1\leqslant i\leqslant k
        \right)$为方阵,则$\bm{A}$的极小多项式为诸$\bm{A}_i$的极小多项式的最小公倍式即
    \[\left[
            m_1\left(x\right),m_2\left(x\right),\cdots,m_k\left(x\right)
            \right]\]其中$m_i\left(x\right)$为$\bm{A}_i$的极小多项式.\begin{proof}
        令$g\left(x\right)=\left[
                m_1\left(x\right),m_2\left(x\right),\cdots,m_k\left(x\right)
                \right]$只需证$m\left(x\right)=g\left(x\right)$.显然$m_i\left(x\right)\mid g\left(x\right)$,于是$g\left(\bm{A}_i\right)=\bm{O}$.因为\[
            g\left(\bm{A}\right)=\begin{bmatrix}
                g\left(\bm{A}_1\right) &                        &        &                        \\
                                       & g\left(\bm{A}_2\right) &        &                        \\
                                       &                        & \ddots &                        \\
                                       &                        &        & g\left(\bm{A}_k\right)
            \end{bmatrix}=\bm{O}
        \]于是$m\left(x\right)\mid g\left(x\right)$.

        另一方面
        \[
            \bm{O}=m\left(\bm{A}\right)=\begin{bmatrix}
                m\left(\bm{A}_1\right) &                        &        &                        \\
                                       & m\left(\bm{A}_2\right) &        &                        \\
                                       &                        & \ddots &                        \\
                                       &                        &        & m\left(\bm{A}_k\right)
            \end{bmatrix}
        \]于是$m\left(\bm{A}_i\right)=\bm{O}$故$m_i\left(x\right)\mid m\left(x\right),\forall
            1\leqslant i\leqslant k$.由于$g\left(x\right)$是最小公倍式则$g\left(x\right)\mid m\left(x\right)$.类似得证.
    \end{proof}
}
\exa{}{极小多项式无重根等价于可对角化}{
    设$\bm{A}$的全体不同特征值$\lambda_1,
        \lambda_2,\cdots,\lambda_k$,证明:若$\bm{A}$可对角化,则其极小多项式为
    $m\left(x\right)=\left(
        x-\lambda_1
        \right)\left(
        x-\lambda_2
        \right)\cdots\left(
        x-\lambda_k
        \right)$.\begin{proof}
        因为$\exists \textup{非异阵}\bm{P}\st$
        \[
            \bm{P}^{-1}\bm{AP}=\begin{pmatrix}
                \lambda_1\bm{I} &                 &        &                 \\
                                & \lambda_2\bm{I} &        &                 \\
                                &                 & \ddots &                 \\
                                &                 &        & \lambda_k\bm{I}
            \end{pmatrix}=\bm{B}
        \]于是$\bm{A}$的极小多项式也为$\bm{B}$的极小多项式,而由上命题知后者等于\[\left[
                x-\lambda_1,x-\lambda_2,\cdots,x-\lambda_k
                \right]=\left(
            x-\lambda_1
            \right)
            \left(
            x-\lambda_2
            \right)\cdots\left(
            x-\lambda_k
            \right)\qedhere
        \]
    \end{proof}
}
\clm{}{}{
    于是可对角化的矩阵的特征值一定是极小多项式的根.事实上,对于任一矩阵也是如此.

    后续会证明:若矩阵的极小多项式无重根,则矩阵一定可对角化. 即一个矩阵可对角化当且仅当其极小多项式无重根.
}
\lem{}{特征值是极小多项式的根}{
    设$\bm{A}\in M_n\left(
        \bbk
        \right)$的极小多项式为$m\left(x\right)$,$\lambda_0$是$\bm{A}$的一个特征值,则
    \[
        \left(
        x-\lambda_0
        \right)\mid m\left(x\right)
    \]\begin{proof}
        因为$m\left(\bm{A}\right)=\bm{O}$,因为一个矩阵适合一个多项式则其特征值均适合该多项式即$m\left(\lambda_0\right)=0$.由\cref{thm:余数定理}余数定理立得.
    \end{proof}
}
\subsection{Cayley-Hamilton定理}
\thm{简单的Cayley-Hamilton定理}{简单的Cayley-Hamilton定理}{
    设
    \[
        \bm{A}=\begin{pmatrix}
            \lambda_1 & a_{12}    & \cdots & a_{1n}    \\
                      & \lambda_2 & \cdots & a_{2n}    \\
                      &           & \ddots & \vdots    \\
                      &           &        & \lambda_n
        \end{pmatrix}
    \]则$\left(
        \bm{A}-\lambda_1\bm{I}_n
        \right)
        \left(
        \bm{A}-\lambda_2\bm{I}_n
        \right)\cdots\left(
        \bm{A}-\lambda_n\bm{I}_n
        \right)=\bm{O}.
    $即上三角阵$\bm{A}$适合自己的特征多项式$f\left(\lambda\right)=\left(\lambda-\lambda_1\right)
        \left(\lambda-\lambda_2\right)\cdots\left(\lambda-\lambda_n\right).
    $\begin{proof}
        取标准单位列向量,于是
        \[
            \bm{Ae}_1=\lambda_1\bm{e}_1,\bm{Ae}_2=a_{12}\bm{e}_1+\lambda_2\bm{e}_2,\cdots,\bm{Ae}_n=a_{1n}\bm{e}_1+a_{2n}\bm{e}_2+\cdots+\lambda_n\bm{e}_n
        \]要证明
        \[
            \left(
            \bm{A}-\lambda_1\bm{I}_n
            \right)
            \left(
            \bm{A}-\lambda_2\bm{I}_n
            \right)\cdots\left(
            \bm{A}-\lambda_n\bm{I}_n
            \right)=\bm{O}
        \]只要证
        \[
            \left(
            \bm{A}-\lambda_1\bm{I}_n
            \right)\left(
            \bm{A}-\lambda_2\bm{I}_n
            \right)
            \cdots\left(
            \bm{A}-\lambda_n\bm{I}_n
            \right)\bm{e}_i=\bm{0},\forall 1\leqslant i\leqslant n
        \]甚至,只要证明
        \[
            \left(
            \bm{A}-\lambda_1\bm{I}_n
            \right)\left(
            \bm{A}-\lambda_2\bm{I}_n
            \right)\cdots\left(
            \bm{A}-\lambda_i\bm{I}_i
            \right)\bm{e}_i=\bm{0},\forall 1\leqslant i\leqslant n
        \]即可(这是因为我们有$f\left(\bm{A}\right)g\left(\bm{A}\right)=g\left(\bm{A}\right)f\left(\bm{A}\right)$).

        于是考虑归纳法证明,对$i$进行归纳,一方面$i=1$时,显然$\left(\bm{A}-\lambda_1\bm{I}_n
            \right)\bm{e}_1=\bm{0}$.设$i$时结论成立,则证$\left(\bm{A}-\lambda_1\bm{I}_n
            \right)
            \left(
            \bm{A}-\lambda_2\bm{I}_n
            \right)\cdots\left(
            \bm{A}-\lambda_i\bm{I}_n
            \right)\bm{e}_i=\bm{0}
        $
        \begin{align*}
            LHS & = \left(
            \bm{A}-\lambda_1\bm{I}_n
            \right)
            \left(
            \bm{A}-\lambda_2\bm{I}_n
            \right)\cdots\left(
            \bm{A}-\lambda_{i-1}\bm{I}_n
            \right)\left(
            a_{1i}\bm{e}_1+a_{2i}\bm{e}_2+\cdots+a_{i-1,i}\bm{e}_{i-1}
            \right)
        \end{align*}
        展开后由归纳假设知上式等于$\bm{0}$,故得证.
    \end{proof}
}
\thm{Cayley-Hamilton定理}{Cayley-Hamilton定理}{
    设$\bm{A}\in M_n\left(
        \bbk
        \right)$,$f\left(\lambda\right)=\left|
        \lambda\bm{I}_n-\bm{A}
        \right|$为其特征多项式,则$f\left(\bm{A}\right)=\bm{O}$.\begin{proof}
        矩阵运算是否为零矩阵与域无关,于是扩张到$\bbc .$因为任一方阵均复相似于一个上三角阵,于是存在非异阵$\bm{P}\st$
        \[
            \bm{B}=\bm{P}^{-1}\bm{AP}=\begin{pmatrix}
                \lambda_1 & a_{12}    & \cdots & a_{1n}    \\
                          & \lambda_2 & \cdots & a_{2n}    \\
                          &           & \ddots & \vdots    \\
                          &           &        & \lambda_n
            \end{pmatrix}
        \]则\[f\left(\lambda\right)=
            \left(
            \lambda-\lambda_1
            \right)\left(
            \lambda-\lambda_2
            \right)\cdots\left(
            \lambda-\lambda_n
            \right)
        \]由上知$f\left(\bm{B}\right)=\bm{O}$,考虑相似关系与多项式相容.
    \end{proof}
}
\cor{}{特征多项式与极小多项式}{
    设$\bm{A}\in M_n\left(\bbk \right)$的特征多项式是$f\left(\lambda\right)$,极小多项式是$m\left(\lambda\right)$,则
    \begin{enumerate}[label = \arabic*)]
        \item $m\left(\lambda\right)\mid f\left(x\right)$,特别地,$\deg m\left(\lambda\right)\leqslant n$
        \item $f\left(\lambda\right),m\left(\lambda\right)$有相同的根(不计重数)
        \item $f\left(\lambda\right)\mid m\left(\lambda\right)^n$
    \end{enumerate}\begin{proof}
        \begin{enumerate}[label = \arabic*)]
            \item 由\cref{thm:Cayley-Hamilton定理}Cayley-Hamilton定理知矩阵适合自己的特征多项式,由极小多项式的定义得证.
            \item 因为特征值均为极小多项式的根,结合上一推论得证.
            \item 设特征多项式\[f\left(\lambda\right)=\left(\lambda-\lambda_1\right)^{m_1}
                      \left(\lambda-\lambda_2\right)^{m_2}\cdots\left(\lambda-\lambda_k\right)^{m_k}
                  \]其中$\lambda_1,\lambda_2,\cdots,\lambda_k$是$\bm{A}$的全体不同特征值,$m_1,m_2,\cdots,m_k$是对应的代数重数.于是极小多项式$m\left(\lambda\right)=\left(
                      \lambda-\lambda_1
                      \right)^{r_1}
                      \left(
                      \lambda-\lambda_2
                      \right)^{r_2}\cdots\left(
                      \lambda-\lambda_k
                      \right)^{r_k}
                  $其中$r_1,r_2,\cdots,r_k\in\mathbb{Z}^+.$注意到$m_1+m_2+\cdots+m_k=n\Longrightarrow
                      m_i\leqslant n\leqslant n\cdot r_i,\forall 1\leqslant i\leqslant k.$于是$f\left(\lambda\right)\mid m\left(\lambda\right)^n$.\qedhere
        \end{enumerate}
    \end{proof}
}
\exa{}{}{
    设矩阵$\bm{A}$有$n$个不同的特征值$\lambda_1,\lambda_2,\cdots,\lambda_n$.

    一方面,$f\left(\lambda\right)
        =\left(
        \lambda-\lambda_1
        \right)\left(
        \lambda-\lambda_2
        \right)\cdots\left(
        \lambda-\lambda_n
        \right)$.

    另一方面,$m\left(\lambda\right)=\left(
        \lambda-\lambda_1
        \right)\left(
        \lambda-\lambda_2
        \right)\cdots\left(
        \lambda-\lambda_n
        \right)$,故$f\left(\lambda\right)=m\left(\lambda\right)$.
}
\exa{}{}{
    设$\bm{A}=c\bm{I}_n$,显然$f\left(\lambda\right)=m\left(\lambda\right)^n.$
}
\cor{几何意义的Cayley-Hamilton定理}{几何意义的Cayley-Hamilton定理}{
    设$\bm{\varphi}\in \call  \left(V_{\bbk }^n\right)$,其特征多项式$f\left(\lambda\right)=\left|
        \lambda\bm{I}_V-\bm{\varphi}
        \right|$,则$f\left(\bm{\varphi}\right)=\bm{0}.$
}
我们曾提到过,不可对角化的矩阵远远没有可对角化的矩阵来得多,并且可以用可对角化的矩阵取去逼近不可对角化的矩阵.
\exa{}{}{
    设$\bm{A}=\begin{pmatrix}
            a_{11} & a_{12} \\
            a_{21} & a_{22}
        \end{pmatrix}\in M_2\left(\bbc \right)$,若$\bm{A}$不可对角化则$\lambda_1=\lambda_2.$其特征多项式
    \[
        \left|
        \lambda\bm{I}_2-\bm{A}
        \right|=\lambda^2-\left(
        a_{11}+a_{22}
        \right)\lambda+\left(
        a_{11}a_{22}-a_{12}a_{21}
        \right)
    \]于是$\varDelta
        = \left(
        a_{11}+a_{22}
        \right)^2-4\left(
        a_{11}a_{22}-a_{12}a_{21}
        \right)=0
    $.不妨记作$f\left(
        a_{11}, a_{12}, a_{21}, a_{22}
        \right).$

    考虑映射
    \begin{align*}
        \bm{\varphi}:M_2\left(\bbc \right) & \longrightarrow \bbc ^4 \\
        \begin{pmatrix}
            a_{11} & a_{12} \\
            a_{21} & a_{22}
        \end{pmatrix}                 & \longmapsto \left(
        a_{11},
        a_{12},
        a_{21},
        a_{22}
        \right)'
    \end{align*}显然线性同构.

    也就是说,若上述矩阵$\bm{A}$不可对角化,其对应的$\bbc ^4$中的点$\left(
        a_{11},a_{12},a_{21},a_{22}
        \right)$必定是$f$的零点即$f\left(
        a_{11},a_{12},a_{21},a_{22}
        \right)=0.$

    从拓扑的角度看,不可对角化的点落在某一个超平面内,我们可以从该平面之外的某一个点(其一定邻域内的点均是可对角化的)逼近该点.
}
基于上述分析,可以给出\cref{thm:Cayley-Hamilton定理}Cayley-Hamilton定理的另一种证明.
\begin{proof}
    先证可对角化的矩阵,设
    \[
        \bm{\varLambda}=
        \bm{P}^{-1}\bm{AP}=\begin{pmatrix}
            \lambda_1 &           &        &           \\
                      & \lambda_2 &        &           \\
                      &           & \ddots &           \\
                      &           &        & \lambda_n
        \end{pmatrix}
    \]$f\left(\lambda\right)=
        \left(
        \lambda-\lambda_1
        \right)\left(
        \lambda-\lambda_2
        \right)\cdots\left(
        \lambda-\lambda_n
        \right)$于是
    \begin{align*}
          & f\left(\bm{\varLambda}\right)                      \\
        = &
        \begin{pmatrix}
            0 &                     &        &                     \\
              & \lambda_2-\lambda_1 &        &                     \\
              &                     & \ddots &                     \\
              &                     &        & \lambda_n-\lambda_1
        \end{pmatrix}\begin{pmatrix}
                         \lambda_1-\lambda_2 &   &        &                     \\
                                             & 0 &        &                     \\
                                             &   & \ddots &                     \\
                                             &   &        & \lambda_n-\lambda_2
                     \end{pmatrix}
        \begin{pmatrix}
            \lambda_1-\lambda_n &                     &        &   \\
                                & \lambda_2-\lambda_n &        &   \\
                                &                     & \ddots &   \\
                                &                     &        & 0
        \end{pmatrix} \\
        = & \bm{O}
    \end{align*}于是$f\left(\bm{A}\right)=\bm{O}.$

    考虑任意性,设$\bm{A}\in M_n\left(\bbk \right)$,设非异阵$\bm{P}$使得
    \[
        \bm{P}^{-1}\bm{AP}=\begin{pmatrix}
            \lambda_1 & *         & \cdots & *         \\
                      & \lambda_2 & \cdots & *         \\
                      &           & \ddots & \vdots    \\
                      &           &        & \lambda_n
        \end{pmatrix}
    \]$\exists c_1,c_2,\cdots,c_n\in\bbk \st\forall 0<t\ll 1,\lambda_1+c_1t,
        \lambda_2+c_2t,\cdots,\lambda_n+c_nt$
    互不相同.然后构造
    \[
        \bm{A}_t=
        \bm{P}\begin{pmatrix}
            \lambda_1+c_1t & *              & \cdots & *              \\
                           & \lambda_2+c_2t & \cdots & *              \\
                           &                & \ddots & \vdots         \\
                           &                &        & \lambda_n+c_nt
        \end{pmatrix}\bm{P}^{-1}
    \]当$0<t\ll 1$时其特征值互不相同则必定可对角化,特别地,$t=0$时$\bm{A}_0=\bm{A}.$考虑其特征多项式
    \[
        \left(
        \bm{A}_t-\left(\lambda_1+c_1t\right)\bm{I}_n
        \right)\cdots\left(
        \bm{A}_t-\left(\lambda_n+c_nt\right)\bm{I}_n
        \right)=\bm{O},\forall 0<t\ll 1
    \]
    考虑到连续性,令$t\to 0$则
    \[
        \left(
        \bm{A}-\lambda_1\bm{I}_n
        \right)\cdots\left(
        \bm{A}-\lambda_n\bm{I}_n
        \right)=\bm{O}\qedhere
    \]
\end{proof}