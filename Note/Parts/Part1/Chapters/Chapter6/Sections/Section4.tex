\newpage
\section{特征值的估计}
\clm{记号说明}{}{
\[
    \bm{A}\coloneqq\left(
    a_{ij}
    \right)_{n\times n}\in M_n\left(\mathbb{C}\right),R_i=\sum_{j\neq i}\left|a_{ij}\right|=\left|
    a_{i1}
    \right|+
    \left|
    a_{i2}
    \right|+\cdots+\left|
    a_{i,i-1}
    \right|+\left|
    a_{i,i+1}
    \right|+
    \cdots+\left|
    a_{in}
    \right|
\]
}
\subsection{Gerschgorin圆盘第一定理}
下面给出Gerschgorin圆盘第一定理(戈尔斯哥利戈氏圆盘第一定理,戈氏第一圆盘定理,圆盘定理).
\thm{Gerschgorin圆盘第一定理}{Gerschgorin圆盘第一定理}{
    设$\bm{A}=\left(
        a_{ij}
        \right)_{n\times n}\in M_n\left(\mathbb{C}\right)$,则$\bm{A}$的所有特征值一定落在以下$n$个$\textup{Gerschgorin}$圆盘中
    \[
        \left|
        z-  a_{ii}
        \right|\leqslant R_i,\forall 1\leqslant i\leqslant n
    \]\begin{proof}
        任取$\bm{A}$的特征值$\lambda_0$,其特征向量$\bm{\xi}=\left(
            \xi_1,\xi_2,\cdots,\xi_n
            \right)'\neq \bm{0}$即$\bm{A\xi}=\lambda_0\bm{\xi}.$于是有
        \[
            \begin{cases*}
                a_{11}x_1+a_{12}x_2+\cdots+a_{1n}x_n=\lambda_0x_1 \\
                a_{21}x_1+a_{22}x_2+\cdots+a_{2n}x_n=\lambda_0x_2 \\
                \qquad\qquad\cdots\cdots\cdots\cdots              \\
                a_{n1}x_1+a_{n2}x_2+\cdots+a_{nn}x_n=\lambda_0x_n
            \end{cases*}
        \]因为$\bm{\xi}$的分量重至少有一个不为零,于是考虑设$0<\left|x_r\right|=\max\left\{
            \left|x_1\right|,\left|x_2\right|,\cdots,\left|x_n\right|
            \right\}.$考虑
        \[
            a_{r1}x_1+\cdots+a_{rr}x_r+\cdots+a_{rn}x_n=\lambda_0x_r
        \]得到
        \begin{align*}
            \left|
            \lambda_0-a_{rr}
            \right|
            \left|x_r\right| & =\left|
            a_{r1}x_1+\cdots+a_{r,r-1}x_{r-1}+a_{r,r+1}x_{r+1}+\cdots+a_{rn}x_n
            \right|                                          \\
                             & \leqslant \left|
            a_{r1}x_1
            \right|+\cdots+\left|
            a_{r,r-1}x_{r-1}
            \right|+\left|
            a_{r,r+1}x_{r+1}
            \right|+\cdots+\left|
            a_{rn}x_n
            \right|                                          \\
                             & \leqslant R_r\left|x_r\right|
        \end{align*}于是$\left|
            \lambda_0-a_{rr}
            \right|\leqslant R_r$.
    \end{proof}
}
\exa{}{}{
    估计矩阵的特征值
    \[
        \begin{pmatrix}
            1    & 0.5  & -0.2 & -1  \\
            0.3  & 2    & -0.2 & 1.1 \\
            -0.5 & 0.1  & -4   & 0.2 \\
            -1   & -0.1 & 0.2  & 0
        \end{pmatrix}
    \]\begin{proof}
        \begin{align*}
             & D_1: \left|
            z-1
            \right|\leqslant 0.5+0.2+1=1.7   \\
             & D_2: \left|
            z-2
            \right|\leqslant 0.3+0.2+1.1=1.6 \\
             & D_3: \left|
            z+4
            \right|\leqslant 0.5+0.1+0.2=0.8 \\
             & D_4: \left|
            z
            \right|\leqslant 1+0.1+0.2=1.3
        \end{align*}
    \end{proof}
}
\dfn{连通区域}{连通区域}{
    若几个戈氏圆盘相连在一起,则称相连区域(指它们覆盖的所有区域)为连通区域,这几个圆盘为连通圆盘.
}
\rem{}{}{
    对于单个非连通的圆盘,该圆盘内必定存在一个特征值.但对于连通区域中的某一圆盘,只能保证特征值在该连通区域内但不一定在这个圆盘内.这将在Gerschgorin圆盘第二定理中体现.
}
\subsection{Gerschgorin圆盘第二定理}
\lem{}{多项式的根关于系数连续}{
设$f\left(x\right)=a_nx^n+
    a_{n-1}x^{n-1}+\cdots+a_1x+a_0
$,则$f\left(x\right)$的根关于$a_n,\cdots,a_1,a_0$连续.
}
\rem{}{}{
我们曾说过,对于高于$4$次的多项式,没有能够用系数的初等运算得到的显式求根公式,也无从谈起像二次多项式那样$\displaystyle
    \frac{
        -b\pm\sqrt{b^2-4ac}
    }{2a}$的连续性,因此这里需要做一个解释.

设$f\left(x\right)$的根为$\lambda_1,\lambda_2,\cdots,\lambda_n$,则$\lambda_i=\lambda_i\left(
    a_n,\cdots,a_1,a_0
    \right),\forall
    1\leqslant i\leqslant n
$,考虑
\[
    \widetilde{f}\left(x\right)=\widetilde{a}_nx^n+\widetilde{a}_{n-1}x^{n-1}+\cdots+\widetilde{a}_1x+\widetilde{a}_0
\]的根为
$\widetilde{\lambda}_1,\widetilde{\lambda}_2,\cdots,\widetilde{\lambda}_n$,则$\widetilde{\lambda}_i=\widetilde{\lambda}_i\left(
    \widetilde{a}_n,\cdots,\widetilde{a}_1,\widetilde{a}_0
    \right),\forall
    1\leqslant i\leqslant n$.

于是$\forall \varepsilon >0,\exists \delta$使得当$\left|
    \widetilde{a}_n-a_n
    \right|<\delta
    ,\cdots,\left|
    \widetilde{a}_1-a_1
    \right|<\delta,\left|
    \widetilde{a}_0-a_0
    \right|<\delta$时,必定有\[\left|
    \widetilde{\lambda}_i\left(
    \widetilde{a}_n,\cdots,\widetilde{a}_1,\widetilde{a}_0
    \right)-\lambda_i\left(
    a_n,\cdots,a_1,a_0
    \right)
    \right|<\varepsilon,\forall 1\leqslant i\leqslant n\]

上定理证明需要利用复变函数论中的Rouch\'e定理,因此这里不再赘述.可参考\cite{蒋尔雄1978}中的证明.
}
\rem{}{}{
    上述定理并非断言多项式的某一个根是关于系数的连续函数,因此一般而言,矩阵的特征值也并不是关于矩阵元素的连续函数.

    例如$\bm{A}=\begin{pmatrix}
            0 & z \\1 & 0
        \end{pmatrix}$,当$z$在复平面上的单位圆内变化时,其特征值不是关于$z$的连续函数,但对于$z$在实轴上的某一区间来说,是成立的,这证明下面Gerschgorin圆盘第二定理利用摄动法证明所用到的.
}
下面给出Gerschgorin圆盘第二定理,该定理能够对连通区域内特征值作进一步的估计.
\rem{}{}{
    戈氏圆盘可能是相同的,也可能是不同的.相同时在下面的第二定理中需要计算重数.
}
\thm{Gerschgorin圆盘第二定理}{Gerschgorin圆盘第二定理}{
    设$k$个戈氏圆盘连成一个连通区域,则有且仅有$k$个特征值落在该连通区域内.

    特别地,重合的圆盘计重数,相同的特征值亦计重数.\begin{proof}
        考虑摄动法,设$\bm{A}=\left(a_{ij}\right)_{n\times n}$,考虑摄动
        \[
            \bm{A}\left(0\right)=\begin{pmatrix}
                a_{11} &        &        &        \\
                       & a_{22} &        &        \\
                       &        & \ddots &        \\
                       &        &        & a_{nn}
            \end{pmatrix}
        \]而
        \[
            \bm{A}\left(t\right)=\begin{pmatrix}
                a_{11}  & ta_{12} & \cdots & ta_{1n} \\
                ta_{21} & a_{22}  & \cdots & ta_{2n} \\
                \vdots  & \vdots  & \ddots & \vdots  \\
                ta_{n1} & ta_{n2} & \cdots & a_{nn}
            \end{pmatrix}
        \]$\bm{A}\left(1\right)=\bm{A}.$根据Gerschgorin圆盘第一定理,$\bm{A}\left(t\right)$的特征值落在
        \[
            \left|
            z-a_{ii}
            \right|\leqslant t\cdot R_i,\forall
            1\leqslant i\leqslant n
        \]内.因此$\forall 0\leqslant t\leqslant 1,\bm{A}\left(t\right)$的特征值均落在$\bm{A}$的对应圆盘中
        \[
            \left|
            z-a_{ii}
            \right|\leqslant
            R_i,\forall 1\leqslant i\leqslant n
        \]考虑$\bm{A}\left(t\right)$的特征多项式,其系数均是$t$的多项式,于是由\cref{lem:多项式的根关于系数连续}知其关于$t$连续.于是特征值关于系数连续,系数关于$t$连续,因此有$\bm{A}\left(t\right)$的特征值关于$t$连续.

        断言,特征值在连续变化的过程中,只能在同一个连通区域内变化而不能跳出该连通区域.考虑反证法,如果变化过程中跳出,意味着$\exists t_0\in \left(0,1\right)$使得$\lambda_i\left(t_0\right)$不落在任何一个戈氏圆盘中.这与上文$\forall 0\leqslant t\leqslant 1,\bm{A}\left(t\right)$的特征值均落在$\bm{A}$的对应圆盘中
        \[
            \left|
            z-a_{ii}
            \right|\leqslant
            R_i,\forall 1\leqslant i\leqslant n
        \]矛盾.

        于是从$t:0\to 1$,直到$t=1$,对于一个有$k$个圆盘连通的区域,其内至少有$t=0$时的$k$个特征值,另一方面,该断言对于所有连通区域均成立,于是也至多是$k$个,故有且仅有$k$个特征值落在该连通区域内.
    \end{proof}
}
\exa{}{}{
    考虑矩阵$\bm{A}=\left(a_{ij}\right)_{n\times n}$且$0\ll a_{11}\ll a_{22}\ll\cdots\ll a_{nn}$,则$\bm{A}$可对角化.
}