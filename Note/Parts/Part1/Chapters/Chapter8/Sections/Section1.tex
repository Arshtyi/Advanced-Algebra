\section{二次型的化简与矩阵的合同}
\subsection{引入}
\dfn{二次超曲面}{二次超曲面}{
    设二次多项式$f\left(
        x_{1},x_{2},\cdots,x_{n}
        \right)$,则称\[
        H=\left\{
        \left(a_1,
        a_2,\cdots,a_n
        \right)\mid f\left(
        a_1,a_2,\cdots,a_n
        \right)=0
        \right\}
    \]为一个二次超曲面.
}
\rem{}{}{
    一个问题是给出二次超曲面的分类.基本思路是对二次型进行化简,最后得到一个足够简单的二次型,并且他的不同形式可以代表所有的二次超曲面.
}
对于一般的二次曲线:\[
    ax^2+bxy+cy^2+dx+ey+f=0
\]首先尝试利用坐标变换消去交错项:\[
    \begin{cases*}
        x=x'\cos\theta-y'\sin\theta \\
        y=x'\sin\theta+y'\cos\theta
    \end{cases*}
\]然后平移坐标轴消去一次项
\clm{}{}{
    该坐标变换的给出为:首先考虑基向量的旋转\[
        \left(
        \bm{e}_1,\bm{e}_2
        \right)=\left(
        \bm{e}_1',\bm{e}_2'
        \right)\begin{pmatrix}
            \cos\theta & -\sin\theta \\
            \sin\theta & \cos\theta
        \end{pmatrix}
    \]于是坐标向量\[
        \begin{pmatrix}
            x \\y
        \end{pmatrix}=\begin{pmatrix}
            \cos \theta & -\sin\theta \\
            \sin\theta  & \cos\theta
        \end{pmatrix}\begin{pmatrix}
            x' \\y'
        \end{pmatrix}
    \]解得\[
        \begin{cases*}
            x=x'\cos\theta-y'\sin\theta \\
            y=x'\sin\theta+y'\cos\theta
        \end{cases*}
    \]
}
于是一般化简$f\left(
    x_1,x_2,\cdots,x_n
    \right)$的思路是\begin{enumerate}[label=\arabic*)]
    \item 通过坐标向量的非异线性变换消去方程$f\left(
              x_1,x_2,\cdots,x_n
              \right)$中的交错项$x_ix_j\left(\forall i\neq j\right)$使得只剩下平方项
    \item 通过坐标向量的平移消去方程中的低次项
\end{enumerate}

这样的思路中,第一步是最为重要的,于是我们只考虑如\cref{def:齐次多项式}的二次型,不考虑低次项.
\subsection{二次型}
\dfn{二次型}{二次型}{
数域$\bbk $上的$n$元二次齐次多项式\[
    f\left(
    x_1,x_2,\cdots,x_n
    \right)=\sum_{i=1}^{n}a_{ii}x_i^2+2\sum_{1\leqslant i<j\leqslant n}a_{ij}x_ix_j
\]称为一个$n$元二次型.
}
\rem{}{}{
    \cref{def:二次型}和\cref{def:齐次多项式}相同,注意绝对不会出现低次项.
}
下面构造一个$\bbk $上的对称矩阵$\bm{A}$使得\[
    f\left(
    x_1,x_2,\cdots,x_n
    \right)=\bm{x}'\bm{Ax}
\]其中\[
    \bm{A}=\begin{pmatrix}
        a_{11} & a_{12} & \cdots & a_{1n} \\
        a_{21} & a_{22} & \cdots & a_{2n} \\
        \vdots & \vdots & \ddots & \vdots \\
        a_{n1} & a_{n2} & \cdots & a_{nn}
    \end{pmatrix},\bm{x}=\begin{pmatrix}
        x_1 \\x_2\\\vdots\\x_n
    \end{pmatrix}
\]其中$a_{ij}=a_{ji},\forall 1\leqslant i,j\leqslant n$.

容易验证\begin{align*}
    \bm{x}'\bm{Ax} & =\left(
    x_1,x_2,\cdots,x_n
    \right)\begin{pmatrix}
               a_{11} & a_{12} & \cdots & a_{1n} \\
               a_{21} & a_{22} & \cdots & a_{2n} \\
               \vdots & \vdots & \ddots & \vdots \\
               a_{n1} & a_{n2} & \cdots & a_{nn}
           \end{pmatrix}\begin{pmatrix}
                            x_1 \\x_2\\\vdots\\x_n
                        \end{pmatrix}                                                 \\
                   & =\left(
    \sum_{i=1}^{n}a_{i1}x_i,\sum_{i=1}^{n}a_{i2}x_i,\cdots,\sum_{i=1}^{n}a_{in}x_i
    \right)\begin{pmatrix}
               x_1 \\x_2\\\vdots\\x_n
           \end{pmatrix}                                                              \\
                   & =\sum_{i=1}^{n}\sum_{j=1}^{n}a_{ij}x_ix_j                                \\
                   & =\sum_{i=1}^{n}a_{ii}x_i^2+\sum_{1\leqslant i<j\leqslant n}\left(
    a_{ij}+a_{ji}
    \right)x_ix_j                                                                             \\
                   & =\sum_{i=1}^{n}a_{ii}x_i^2+2\sum_{1\leqslant i<j\leqslant n}a_{ij}x_ix_j \\
                   & =f\left(
    x_1,x_2,\cdots,x_n
    \right)
\end{align*}

构造映射\begin{align*}
    \bm{\varphi}:\left\{
    \bbk \text{上所有}n\text{阶对称阵}
    \right\} & \longrightarrow\left\{
    n\text{元二次型}
    \right\}                             \\
    \bm{A}   & \longmapsto\bm{x}'\bm{Ax}
\end{align*}
显然这是一个满射,下面考虑单性.设对称阵$\bm{A},\bm{B}$满足\[
    \bm{x}'\bm{Ax}=\bm{x}'\bm{Bx}
\]即\[
    \sum_{i=1}^{n}a_{ii}x_i^2+2\sum_{1\leqslant i<j\leqslant n}a_{ij}x_ix_j=\sum_{i=1}^{n}b_{ii}x_i^2+2\sum_{1\leqslant i<j\leqslant n}b_{ij}x_ix_j
\]注意到是对称阵,明显这要求$a_{ij}=b_{ij},\forall 1\leqslant i,j\leqslant n\Longrightarrow \bm{A}=\bm{B}.$于是该映射是一个一一对应.
\rem{}{}{
    一般称矩阵$\bm{A}$为二次型$f\left(
        x_1,x_2,\cdots,x_n
        \right)$的表示矩阵或者相伴的矩阵,$f\left(
        x_1,x_2,\cdots,x_n
        \right)$称为$\bm{A}$相伴的二次型.
}
\exa{}{}{
    二次型\[
        f\left(x,y\right)=x^2+3xy+y^2
    \]的表示矩阵为\[
        \begin{pmatrix}
            1 &  & \cfrac{3}{2} \\\cfrac{3}{2} &  & 1
        \end{pmatrix}
    \]
}
\exa{}{}{
    矩阵\[
        \begin{pmatrix}
            1            & \cfrac{1}{2} & -\sqrt{2} \\
            \cfrac{1}{2} & -1           & 0         \\
            -\sqrt{2}    & 0            & 2
        \end{pmatrix}
    \]相伴的二次型为\[
        f\left(x_1,x_2,x_3\right)=x_1^2-x_2^2+2x_3^2+x_1x_2-2\sqrt{2}x_1x_3
    \]
}
\rem{}{}{
    容易看出,一个只含平方项的二次型的表示矩阵是对角阵,反之亦然.于是化简二次型就成了对角化矩阵这样一件我们熟知的事.
}
\subsection{合同关系}
设$V$是数域$\bbk $上的$n$维线性空间,一组基$\left\{
    \bm{e}_1,\bm{e}_2,\cdots,\bm{e}_n
    \right\},\forall\bm{\alpha}\in V,$其唯一对应于坐标向量$\bm{x}=\begin{pmatrix}
        x_1 \\x_2\\\vdots\\x_n
    \end{pmatrix}$.于是任何一个二次型$
    f\left(
    x_1,x_2,\cdots,x_n
    \right)
$自然看做$V$上的一个函数.另取一组基$\left\{
    \bm{f}_1,\bm{f}_2,\cdots,\bm{f}_n
    \right\}$并设基$\bm{e}$到$\bm{f}$的过渡矩阵为$\bm{C}$即\[
    \left(
    \bm{f}_1,\bm{f}_2,\cdots,\bm{f}_n
    \right)=\left(
    \bm{e}_1,\bm{e}_2,\cdots,\bm{e}_n
    \right)\bm{C}
\]新的坐标向量为$\bm{y}=\begin{pmatrix}
        y_1 \\y_2\\\vdots\\y_n
    \end{pmatrix}$于是$\bm{x}=\bm{Cy}$即\[
    f\left(
    x_1,x_2,\cdots,x_n
    \right)=\bm{x}'\bm{Ax}=\bm{y}'\bm{C}'\bm{ACy}=\bm{y}'\left(
    \bm{C}'\bm{AC}
    \right)\bm{y}=g\left(
    y_1,y_2,\cdots,y_n
    \right)
\]也就是说,坐标向量的非异变换等价于基变换,这样的变换称为合同变换.、
\dfn{合同关系}{合同关系}{
    设$\bm{A},\bm{B}\in M_n\left(\bbk \right)$,若存在非异阵$\bm{C}\in M_n\left(\bbk \right)$使得\[
        \bm{B}=\bm{C}'\bm{AC}
    \]则称$\bm{A}$与$\bm{B}$合同,或者具有合同关系.
}
\rem{}{}{
    于是,问题是是否对于任意一个二次型$f\left(
        x_1,x_2,\cdots,x_n
        \right)$都存在一个坐标向量的非异变换$\bm{x}=\bm{Cy}$使得得到的新的二次型$g\left(
        y_1,y_2,\cdots,y_n
        \right)$只含平方项.也就是说,是否任意一个对称矩阵都可以通过合同变换化为对角阵.
    \incfig[scale=0.9]{CongruentTransformation}
}
\subsection{对称初等变换}
\dfn{对称初等变换}{对称初等变换}{
    下面的变换都是合同变换,称为对称初等变换.\begin{enumerate}[label=\arabic*)]
        \item 对换$\bm{A}$的第$i$行与第$j$行,再对换第$i$列与第$j$列\[
                  \bm{P}_{ij}\bm{AP}_{ij},\bm{P}'_{ij}=\bm{P}_{ij}            \]
        \item 用非零常数$c$乘$\bm{A}$的第$i$行与第$i$列\[
                  \bm{P}_i\left(c\right)\bm{AP}_i\left(c\right),\bm{P}'_i\left(c\right)=\bm{P}_i\left(c\right)
              \]
        \item $\bm{A}$的第$i$行乘上常数$c$加到第$j$行,再将第$i$列乘上常数$c$加到第$j$列\[
                  \bm{T}_{ij}\left(c\right)\bm{AT}_{ji}\left(c\right),\bm{T}'_{ij}\left(c\right)=\bm{T}_{ji}\left(c\right)
              \]
    \end{enumerate}
}
\lem{}{对称阵变换为首元非零矩阵}{
    设$\bbk $上的非零对称阵$\bm{A}$,则必定有非异阵$\bm{C}$使得$\bm{C}'\bm{AC}$的$\left(1,1\right)$元非零.\begin{proof}
        设$a_{11}\neq 0$,若$\exists 2\leqslant i\leqslant n,a_{ii}\neq 0$,用第一类对称初等变换将第$i$行与第$1$行对换,再将第$i$列与第$1$列对换,则新的$\bm{A}$的$\left(1,1\right)$元非零.若$\forall 2\leqslant i\leqslant n,a_{ii}=0$,则必定存在一个$a_{ij}\neq 0$,考虑到对称阵,设$1\leqslant i<j\leqslant n$即块\[
            \begin{pmatrix}
                0      & \cdots & a_{ij} \\
                \vdots &        & \vdots \\
                a_{ji} & \cdots & 0
            \end{pmatrix}
        \]考虑第三类对称初等变换$\bm{T}_{ji}\left(1\right)\bm{AT}_{ij}\left(1\right)$得到块\[
            \begin{pmatrix}
                2a_{ij} & \cdots & a_{ij} \\
                \vdots  &        & \vdots \\
                a_{ij}  & \cdots & 0
            \end{pmatrix}
        \]即$\left(i,i\right)$元不为零,由上文得证.
    \end{proof}
}
\thm{}{任意矩阵必定合同于对角阵}{
    设$\bm{A}\in M_n\left(\bbk \right)$,则存在非异阵$\bm{C}\in M_n\left(\bbk \right)$使得$\bm{C}'\bm{AC}$为对角阵.\begin{proof}
        对阶数进行归纳,$n=1$显然.下设阶数$<n$时结论成立,下面证明$n$时结论成立.并设$\bm{A}\neq\bm{O}$.

        由\cref{lem:对称阵变换为首元非零矩阵},不妨设$a_{11}\neq 0.$假设存在$a_{1i}=a_{i1}\neq 0$,利用第三类对称初等变换$\displaystyle \bm{T}_{1i}\left(
            -\frac{a_{i1}}{a_{11}}
            \right)\bm{AT}_{i1}\left(
            -\frac{a_{1i}}{a_{11}}
            \right)$即可将二者化为零,若干次后,得到\[
            \bm{A}=\begin{pmatrix}
                a_{11} & \bm{O}       \\
                \bm{O} & \bm{B}^{n-1}
            \end{pmatrix}
        \]根据归纳假设,存在非异阵$\bm{D}\in M_{n-1}\left(\bbk \right)$使得$\bm{D}'\bm{BD}=\bm{\varLambda}$为对角阵.于是\[
            \begin{pmatrix}
                1      & \bm{O}  \\
                \bm{O} & \bm{D}'
            \end{pmatrix}\begin{pmatrix}
                a_{11} & \bm{O} \\
                \bm{O} & \bm{B}
            \end{pmatrix}\begin{pmatrix}
                1      & \bm{O} \\
                \bm{O} & \bm{D}
            \end{pmatrix}=\begin{pmatrix}
                a_{11} & \bm{O}          \\
                \bm{O} & \bm{\varLambda}
            \end{pmatrix}
        \]为对角阵,得证.
    \end{proof}
}
\cor{合同变换与对称初等变换}{合同变换与对称初等变换}{
    合同变换是若干次对称初等变换的复合.\begin{proof}
        考虑合同变换$\bm{A}\leadsto \bm{C}'\bm{AC}$,其中非异阵\[
            \bm{C}=\bm{P}_1\bm{P}_2\cdots\bm{P}_r
        \]其中$\bm{P}_i$为初等阵,于是\begin{align*}
            \bm{C}'\bm{AC} & =\left(
            \bm{P}_1\bm{P}_2\cdots\bm{P}_r
            \right)' \bm{A}\left(
            \bm{P}_1\bm{P}_2\cdots\bm{P}_r
            \right)                  \\
                           & =\left(
            \bm{P}_r'\cdots\left(
                \left(
                    \bm{P}_2'\left(
                        \bm{P}_1'\bm{A}\bm{P}_1
                        \right)\bm{P}_2
                    \right)
                \right)
            \cdots  \bm{P}_r        \right)\qedhere
        \end{align*}
    \end{proof}
}
\subsection{分块对称初等变换}
\dfn{分块对称初等变换}{分块对称初等变换}{
    下面的变换都是合同变换,称为分块对称初等变换.\begin{enumerate}[label=\arabic*)]
        \item 对换$\bm{A}$的第$i$分块行与第$j$分块行,再对换第$i$分块列与第$j$分块列\[
                  \bm{P}_{ij}\bm{AP}_{ij}'
              \]
        \item 将第$i$分块行左乘非异阵$\bm{M}$,第$i$分块行右乘$\bm{M}'$\[
                  \bm{P}_i\left(\bm{M}\right)\bm{AP}_i\left(\bm{M}'\right),\bm{P}_i'\left(\bm{M}\right)=\bm{P}_i\left(\bm{M}'\right)
              \]
        \item 第$i$分块行左乘非异阵$\bm{M}$加到第$j$分块行,第$i$分块列右乘$\bm{M}'$加到第$j$分块列\[
                  \bm{T}_{ij}\left(\bm{M}\right)\bm{AT}_{ji}\left(\bm{M}'\right),\bm{T}_{ij}'\left(\bm{M}\right)=\bm{T}_{ji}\left(\bm{M}'\right)
              \]
    \end{enumerate}
}