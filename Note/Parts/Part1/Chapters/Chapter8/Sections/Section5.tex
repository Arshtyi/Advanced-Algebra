\newpage
\section{Hermite型}
\subsection{Hermite型与Hermite阵}
考虑将这套理论推广到复数域,即
\[
    f\left(
    x_1,x_2,\cdots,x_n
    \right)=d_1x_1^2+d_2x_2^2+\cdots+d_rx_r^2
\]其中$d_1,d_2,\cdots,d_r\in\bbc ,x_1,x_2,\cdots,x_r$为复变元.
\clm{}{}{
    由于复数运算的特殊性,很难直接平行地做推广.因而必须做一些调整.
    \begin{enumerate}[label=\arabic*)]
        \item $d_1,d_2,\cdots,d_r\in\bbr $
        \item 考虑模长的平方而不是复数的平方
    \end{enumerate}
}
\dfn{Hermite型}{Hermite型}{
复数域上的$n$元二次齐次函数\[
    f\left(
    x_1,x_2,\cdots,x_n
    \right)=\sum_{j=1}^{n}\sum_{i=1}^{n}a_{ij}\overline{x_i}x_j
\]其中$\overline{a}_{ij}=a_{ji},\forall 1\leqslant i,j\leqslant n.$
}
\clm{}{}{
    \cref{def:Hermite型}Hermite型不是一个多项式,多项式的形式必须是平方,而是一个函数.并且取共轭容易发现,它是一个实值函数.
}
\dfn{Hermite矩阵}{Hermite矩阵2}{
取\cref{def:Hermite型}中系数组成复矩阵\[
    \bm{A}=\begin{pmatrix}
        a_{11} & a_{12} & \cdots & a_{1n} \\
        a_{21} & a_{22} & \cdots & a_{2n} \\
        \vdots & \vdots &        & \vdots \\
        a_{n1} & a_{n2} & \cdots & a_{nn}
    \end{pmatrix}
\]即\[
    f\left(
    x_1,x_2,\cdots,x_n
    \right)=\sum_{j=1}^{n}\sum_{i=1}^{n}a_{ij}\overline{x_i}x_j=\overline{\bm{x}}'\bm{A}\bm{x}
\]满足$\overline{\bm{A}}'=\bm{A}$(\cref{def:Hermite矩阵}),称为Hermite矩阵.
}
\cor{Hermite矩阵对角元为实数}{Hermite矩阵对角元为实数}{
    Hermite阵的主对角元必定是实数.
}
\thm{Hermite型与Hermite阵}{Hermite型与Hermite阵}{
因为\[
    f\left(
    x_1,x_2,\cdots,x_n
    \right)=\sum_{j=1}^{n}\sum_{i=1}^{n}a_{ij}\overline{x_i}x_j=\overline{\bm{x}}'\bm{A}\bm{x}
\]其中$\bm{x}=\begin{pmatrix}
        x_1 \\x_2\\\vdots\\x_n
    \end{pmatrix}$,$\bm{A}$是一个 Hermite 阵.

构造映射\begin{align*}
    \bm{\varphi}:\left\{
    n\textup{阶Hermite阵}\right\} & \longrightarrow \left\{n\textup{元Hermite型}\right\} \\
    \bm{A}                      & \longmapsto f\left(
    x_1,x_2,\cdots,x_n
    \right)=\overline{\bm{x}}'\bm{A}\bm{x}
\end{align*}这是一个一一对应.\begin{proof}
    首先这是满射,然后证明这是单射.设Hermite阵$\bm{A},\bm{B}$满足$\overline{\bm{x}}'\bm{Ax}=\overline{\bm{x}}'\bm{Bx}$于是\[\overline{\bm{x}}\left(\bm{A}-\bm{B}\right)\bm{x}=0\Longrightarrow\bm{A}=\bm{B}\]这通过赋值即可证.
\end{proof}
}
作非异线性变换$\bm{x}=\bm{Cy},\bm{C}$为非异复矩阵,则得到的\[
    f=\left(\overline{\bm{Cy}}\right)'\bm{A}\left(\bm{Cy}\right)=\overline{\bm{y}}'\left(
    \overline{\bm{C}}'\bm{AC}
    \right)\bm{y}
\]中的$\overline{\bm{C}}'\bm{AC}$仍然是Hermite阵.
\subsection{复相合}
\dfn{复相合}{复相合}{
    设$\bm{A},\bm{B}$是两个Hermite型,若存在非异复矩阵$\bm{C}$使得\[
        \bm{B}=\overline{\bm{C}}'\bm{AC}
    \]则称$\bm{A},\bm{B}$是复相合的.
}
于是接下来的问题与实二次型类似,即对于任意一个Hermite阵$\bm{A}$,是否存在非异复矩阵$\bm{C}$使得$\overline{\bm{C}}'\bm{AC}$为一个对角阵.
\dfn{共轭对称初等变换}{共轭对称初等变换}{
    共轭对阵初等变换是复相合变换\begin{enumerate}[label=\arabic*)]
        \item 对换第$i$行与第$j$行,再对换第$i$列与第$j$列\[
                  \bm{P}_{ij}\bm{AP}_{ij},\overline{\bm{P}}_{ij}'=\bm{P}_{ij}
              \]
        \item 第$i$行乘上$0\neq c\in\bbc $,再第$i$列乘上$\overline{c}$\[
                  \bm{P}_i\left(c\right)\bm{AP}_i\left(\overline{c}\right),\overline{\bm{P}}_i'\left(c\right)=\bm{P}_i\left(\overline{c}\right)
              \]
        \item 第$i$行乘上$0\neq c\in\bbc $加到第$j$行,再第$i$列乘上$\overline{c}$加到第$j$列\[
                  \bm{T}_{ij}\left(c\right)\bm{AT}_{ji}\left(\overline{c}\right),\overline{\bm{T}}'_{ij}\left(c\right)=\bm{T}_{ij}\left(\overline{c}\right)
              \]
    \end{enumerate}
}
\subsection{惯性定理}
\dfn{标准型}{标准型}{
    Hermite型\[
        f\left(
        x_1,x_2,\cdots,x_n
        \right)=d_1\overline{x}_1x_1+d_2\overline{x}_2x_2+\cdots+d_n\overline{x}_rx_r
    \]其中$d_1,d_2,\cdots,d_r\in\bbr $称为标准型.
}
\thm{Hermite阵对角化}{Hermite阵对角化}{
    设$\bm{A}$是一个Hermite阵,则存在非异阵$\bm{C}$使得$\overline{\bm{C}}\bm{AC}$为实对角阵.\begin{proof}
        类似于\cref{lem:对称阵变换为首元非零矩阵},\cref{thm:任意矩阵必定合同于对角阵}.
    \end{proof}
}
\dfn{Hermite阵的规范标准型}{Hermite阵的规范标准型}{
    Hermite型总可以化为规范标准型\[
        g\left(
        y_1,y_2,\cdots,y_n
        \right)=y_1\overline{y}_1+y_2\overline{y}_2+\cdots+y_p\overline{y}_p-
        y_{p+1}\overline{y}_{p+1}-\cdots-y_r\overline{y}_r
    \]
}
\thm{Hermite阵的惯性定理}{Hermite阵的惯性定理}{
    设Hermite型$f\left(x_1,x_2,\cdots,x_n\right)$的两个规范标准型为\begin{align*}
         & \overline{y}_1y_1+\overline{y}_2y_2+\cdots+\overline{y}_py_p-\overline{y}_{p+1}y_{p+1}-\cdots-\overline{y}_ry_r \\
         & \overline{z}_1z_1+\overline{z}_2z_2+\cdots+\overline{z}_kz_k-\overline{z}_{k+1}z_{k+1}-\cdots-\overline{z}_rz_r
    \end{align*}则$p=k.$
}
\dfn{Hermite型的惯性指数}{Hermite型的惯性指数}{
    设规范标准型\[
        g\left(
        y_1,y_2,\cdots,y_n
        \right)=y_1\overline{y}_1+y_2\overline{y}_2+\cdots+y_p\overline{y}_p-
        y_{p+1}\overline{y}_{p+1}-\cdots-y_r\overline{y}_r
    \]其中,$r$称为Hermite型的秩,而$p$称为Hermite型的正惯性指数,而$q=r-p$称为Hermite型的负惯性指数,$s=p-q$称为Hermite型的符号差.
}
\clm{}{}{
    \cref{def:Hermite型的惯性指数}对Hermite阵也适用.同样也是四者知其二即可全推知.
}
\thm{复相合关系的全系不变量}{复相合关系的全系不变量}{
    \cref{def:Hermite型的惯性指数}中秩与符号差(或正负惯性指数)是复相合关系下的全系不变量.
}
\thm{复相合标准型}{复相合标准型}{
    任意Hermite阵在复相合关系下的标准型为\[
        \begin{pmatrix}
            \bm{I}_p &           &        \\
                     & -\bm{I}_q &        \\
                     &           & \bm{O}
        \end{pmatrix}
    \]这与\cref{def:合同标准型}是相同的.
}
\subsection{正定Hermite型}
\dfn{正定Hermite型}{正定Hermite型}{
    设 Hermite 型$f\left(x_1,x_2,\cdots,x_n\right)=\overline{\bm{x}}'\bm{Ax}$\begin{enumerate}[label=\arabic*)]
        \item 若$\forall \bm{0}\neq\bm{\alpha}\in\bbc ^n,\overline{\bm{\alpha}}'\bm{A\alpha}>0$,则称$f$为正定 Hermite 型,$\bm{A}$为正定 Hermite 阵.
        \item 若$\forall \bm{0}\neq\bm{\alpha}\in\bbc ^n,\overline{\bm{\alpha}}'\bm{A\alpha}\geqslant 0$,则称$f$为半正定 Hermite 型,$\bm{A}$为半正定 Hermite 阵.
        \item 若$\forall \bm{0}\neq\bm{\alpha}\in\bbc ^n,\overline{\bm{\alpha}}'\bm{A\alpha}<0$,则称$f$为负定 Hermite 型,$\bm{A}$为负定 Hermite 阵.
        \item 若$\forall \bm{0}\neq\bm{\alpha}\in\bbc ^n,\overline{\bm{\alpha}}'\bm{A\alpha}\leqslant 0$,则称$f$为半负定 Hermite 型,$\bm{A}$为半负定 Hermite 阵.
        \item 若$\exists \bm{\alpha},\bm{\beta}\in\bbc ^n$使得$\overline{\bm{\alpha}}'\bm{A\alpha}>0,\overline{\bm{\beta}}'\bm{A\beta}<0$,则称$f$为不定 Hermite 型,$\bm{A}$为不定 Hermite 阵.
    \end{enumerate}
}
\exa{}{}{
    \[
        f=\overline{x}_1x_1+\overline{x}_2x_2+\cdots+\overline{x}_nx_n
    \]正定\[
        f=\overline{x}_1x_1+\overline{x}_2x_2+\cdots+\overline{x}_rx_r
    \]半正定
}
\thm{正定Hermite型的判定}{正定Hermite型的判定}{
    设 Hermite 型$f=\overline{\bm{x}}'\bm{Ax}$的秩为$r$,正负惯性指数$p,q$,则\begin{enumerate}[label=\arabic*)]
        \item $f$正定当且仅当$p=n$,当且仅当复相合标准型为$\bm{I}_n$
        \item $f$半正定当且仅当$p=r$,当且仅当复相合标准型为$\begin{pmatrix}
                      \bm{I}_r & \bm{O} \\
                      \bm{O}   & \bm{O}
                  \end{pmatrix}$
        \item $f$负定当且仅当$q=n$,当且仅当复相合标准型为$-\bm{I}_n$
        \item $f$半负定当且仅当$q=r$,当且仅当复相合标准型为$\begin{pmatrix}
                      -\bm{I}_r & \bm{O} \\
                      \bm{O}    & \bm{O}
                  \end{pmatrix}$
        \item $f$不定当且仅当$p,q>0$
    \end{enumerate}
}
\pro{正定Hermite型的等价结论}{正定Hermite型的等价结论}{
    设$n$阶 Hermite 阵$\bm{A}$,则下列结论等价\begin{enumerate}[label=\arabic*)]
        \item $\bm{A}$正定
        \item $\bm{A}$复相合于$\bm{I}_n$
        \item 存在非异复矩阵$\bm{C}$使得$\bm{A}=\overline{\bm{C}}'\bm{C}$
        \item $\bm{A}$的所有主子式均大于$0$
        \item $\bm{A}$的所有顺序主子式均大于$0$
        \item $\bm{A}$的特征值均大于$0$
    \end{enumerate}
}
\pro{半正定Hermite型的等价结论}{半正定Hermite型的等价结论}{
    设$n$阶 Hermite 阵$\bm{A}$,则下列结论等价\begin{enumerate}[label=\arabic*)]
        \item $\bm{A}$半正定
        \item $\bm{A}$复相合于$\begin{pmatrix}
                      \bm{I}_r & \bm{O} \\
                      \bm{O}   & \bm{O}
                  \end{pmatrix}$
        \item $\exists \bm{C}\in M_n\left(\bbc \right)$使得$\bm{A}=\overline{\bm{C}}'\bm{C}$
        \item $\bm{A}$的所有主子式均大于等于$0$
        \item $\bm{A}$的所有顺序主子式均大于等于$0$
        \item $\bm{A}$的特征值均大于等于$0$
    \end{enumerate}
}