\newpage
\section{二次型的化简}
\subsection{配方法}
配方法的理论基础即完全平方公式\[
    \left(
    \sum_{i=1}^{n}x_i
    \right)^2=\sum_{i=1}^{n}x_i^2+2\sum_{1\leqslant i<j\leqslant n}x_ix_j
\]\begin{enumerate}[label=\arabic*)]
    \item 先将含$x_1$的所有项凑成完全平方消去$x_1$
    \item 剩余项中将含$x_2$的所有项凑成完全平方消去$x_2$
    \item 重复上述过程直到所有的项都凑成完全平方
\end{enumerate}
\exa{}{}{
    \begin{align*}
        f\left(
        x_1,x_2,x_3
        \right) & =x_1^2+2x_1x_2-4x_1x_3-3x_2^2-6x_2x_3+x_3^2 \\
                & =\left(
        x_1^2+2x_1x_2+x_2^2-4x_1x_3+4x_3^2-4x_2x_3
        \right)-4x_2^2-3x_3^2-2x_2x_3                         \\
                & =\left(x_1+x_2-2x_3\right)^2-4\left(
        x_2^2+\frac{1}{2}x_2x_3+\frac{1}{16}x_3^2
        \right)-\frac{11}{4}x_3^2                             \\
                & =\left(x_1+x_2-2x_3\right)^2-4\left(
        x_2+\frac{1}{4}x_3
        \right)^2-\frac{11}{4}x_3^2
    \end{align*}令\[
        \begin{cases*}
            y_1=x_1+x_2-2x_3        \\
            y_2=x_2+\cfrac{1}{4}x_3 \\
            y_3=x_3
        \end{cases*}\Longrightarrow\begin{cases*}
            x_1=y_1-y_2+\cfrac{9}{4}y_3 \\
            x_2=y_2-\cfrac{1}{4}y_3     \\
            x_3=y_3
        \end{cases*}
    \]即矩阵\[
        \bm{C}=\begin{pmatrix}
            1 & -1 & \cfrac{9}{4}  \\
            0 & 1  & -\cfrac{1}{4} \\
            0 & 0  & 1
        \end{pmatrix}
    \]使得$\bm{x}=\bm{Cy}.$
}
\rem{}{}{
    配方法得到的过渡矩阵一定是一个非异的上三角矩阵.

    有时,那些一眼显然得到的配方结果得到的可能并不是非异上三角阵,这表明结果是错误的.
}
\exa{}{}{
    \begin{align*}
        f\left(
        x_1,x_2,x_3
        \right) & =2x_1^2+2x_2^2+2x_3^2-2x_1x_2+2x_1x_3+x_2x_3 \\
                & =\left(
        x_1-x_2
        \right)^2+\left(
        x_1+x_3
        \right)^2+\left(
        x_2+x_3
        \right)^2
    \end{align*}得到的过渡矩阵\[
        \begin{pmatrix}
            1 & -1 & 0 \\
            1 & 0  & 1 \\
            0 & 1  & 1
        \end{pmatrix}
    \]是奇异阵,这种变换是错误的.
}
\exa{}{}{
    \begin{align*}
        f\left(
        x_1,x_2,x_3,x_4
        \right)=2x_1x_2-x_1x_3+x_1x_4-x_2x_3+x_2x_4-2x_3x_4
    \end{align*}不含平方项,考虑\[
        \begin{cases*}
            x_1=y_1+y_2 \\
            x_2=y_1-y_2 \\
            x_3=y_3     \\
            x_4=y_4
        \end{cases*}
    \]即\begin{align*}
        f & =2y_1^2-2y_2^2-2y_1y_3+2y_1y_4-2y_3y_4                \\
          & =2\left(
        \left(
            y_1-\frac{1}{2}y_3+\frac{1}{2}y_4
            \right)^2-\frac{1}{4}y_3^2-\frac{1}{4}y_4^2+\frac{1}{2}y_3y_4
        \right)-2y_2^2-2y_3y_4                                    \\
          & =2\left(
        y_1-\frac{1}{2}y_3+\frac{1}{2}y_4
        \right)^2-2y_2^2-\frac{1}{2}y_3^2-y_3y_4-\frac{1}{2}y_4^2 \\
          & =2\left(
        y_1-\frac{1}{2}y_3+\frac{1}{2}y_4
        \right)^2-2y_2^2-\frac{1}{2}\left(
        y_3+y_4
        \right)^2
    \end{align*}解得\[
        \begin{cases*}
            z_1=y_1-\cfrac{1}{2}y_3+\cfrac{1}{2}y_4 \\
            z_2=y_2                                 \\
            z_3=y_3+y_4                             \\
            z_4=y_4
        \end{cases*}\Longrightarrow\begin{cases*}
            y_1=z_1+\cfrac{1}{2}z_3-z_4 \\
            y_2=z_2                     \\
            y_3=z_3-z_4                 \\
            y_4=z_4
        \end{cases*}
    \]从而\[
        \begin{cases*}
            x_1=z_1+z_2+\cfrac{1}{2}z_3-z_4 \\
            x_2=z_1-z_2+\cfrac{1}{2}z_3-z_4 \\
            x_3=z_3-z_4                     \\
            x_4=z_4
        \end{cases*}\Longrightarrow\bm{C}=\begin{pmatrix}
            1 & 1  & \cfrac{1}{2} & -1 \\
            1 & -1 & \cfrac{1}{2} & -1 \\
            0 & 0  & 1            & -1 \\
            0 & 0  & 0            & 1
        \end{pmatrix}
    \]
}
\subsection{初等变换法}
此法的理论基础即\cref{cor:合同变换与对称初等变换}.
\begin{enumerate}[label=\arabic*)]
    \item 对矩阵\[\left(\begin{array}{c:c}
                      \bm{A} & \bm{I}
                  \end{array}\right)\]
          实施初等行变换,再对左边矩阵实施对称的初等列变换,直到左边矩阵化为对角阵$\bm{\varLambda}$,此时右边矩阵即为所求的过渡矩阵$\bm{C}$的转置.
\end{enumerate}
\rem{}{}{
    特别地,若$\bm{A}$的$\left(1,1\right)$元为零,上述过程中首先利用同样方法将$\bm{A}$的$\left(1,1\right)$化为非零.
}
\exa{}{}{
    考虑二次型\[
        f\left(
        x_1,x_2,x_3
        \right)=2x_1x_2+4x_1x_3-4x_2x_3
    \]\begin{solution}
        \begin{align*}
            \left(
            \begin{array}{c:c}
                \bm{A} & \bm{I}
            \end{array}
            \right) & =\begin{pmatrix}
                           0 & 1  & 1  & 1 & 0 & 0 \\
                           1 & 0  & -2 & 0 & 1 & 0 \\
                           2 & -2 & 0  & 0 & 0 & 1
                       \end{pmatrix}                                                \\
                    & \longrightarrow\begin{pmatrix}
                                         2 & 1  & 0  & 1 & 1 & 0 \\
                                         1 & 0  & -2 & 0 & 1 & 0 \\
                                         0 & -2 & 0  & 0 & 0 & 1
                                     \end{pmatrix}                                  \\
                    & \longrightarrow\begin{pmatrix}
                                         2 & 0             & 0 & 1             & 1            & 0 \\
                                         0 & -\cfrac{1}{2} & 0 & -\cfrac{1}{2} & \cfrac{1}{2} & 0 \\
                                         0 & 0             & 8 & 2             & -2           & 1
                                     \end{pmatrix}
        \end{align*}于是二次型为\[
            2y_1^2-\cfrac{1}{2}y_2^2+8y_3^2
        \]过渡矩阵\[
            \bm{C}=\begin{pmatrix}
                1 & -\frac{1}{2} & 2  \\
                1 & \frac{1}{2}  & -2 \\
                0 & 0            & 1
            \end{pmatrix}
        \]
    \end{solution}
}