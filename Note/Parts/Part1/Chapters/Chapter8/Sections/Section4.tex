\newpage
\section{正定型与正定矩阵}
\subsection{定义}
\dfn{正定型与负定型}{正定型与负定型}{
    设$n$元实二次型$f\left(x_1,x_2,\cdots,x_n\right)=\bm{x}'\bm{Ax},\bm{A}$是相伴矩阵\begin{enumerate}[label=\arabic*)]
        \item 若$\forall \bm{0}\neq\bm{\alpha}\in\bbr ^n,\bm{\alpha}'\bm{A\alpha}>0$,则称$f$为正定二次型(正定型),$\bm{A}$为正定矩阵(正定阵)
        \item 若$\forall \bm{0}\neq\bm{\alpha}\in\bbr ^n,\bm{\alpha}'\bm{A\alpha}\geqslant 0$,则称$f$为半正定二次型(半正定型),$\bm{A}$为半正定矩阵(半正定阵)
        \item 若$\forall \bm{0}\neq\bm{\alpha}\in\bbr ^n,\bm{\alpha}'\bm{A\alpha}<0$,则称$f$为负定二次型(负定型),$\bm{A}$为负定矩阵(负定阵)
        \item 若$\forall \bm{0}\neq\bm{\alpha}\in\bbr ^n,\bm{\alpha}'\bm{A\alpha}\leqslant 0$,则称$f$为半负定二次型(半负定型),$\bm{A}$为半负定矩阵(半负定阵)
        \item 若$\exists \bm{\alpha},\bm{\beta}\in\bbr ^n$使得$\bm{\alpha}'\bm{A\alpha}>0,\bm{\beta}'\bm{A\beta}<0$,则称$f$为不定二次型(不定型),$\bm{A}$为不定矩阵(不定阵)
    \end{enumerate}
}
\exa{}{}{
\[
    f=x_1^2+x_2^2+\cdots+x_n^2
\]是正定型\[
    f=x_1^2+x_2^2+\cdots+x_r^2\left(r\leqslant n\right)
\]是半正定型\[
    f=-x_1^2-x_2^2-\cdots-x_n^2
\]是负定型\[
    f=-x_1^2-x_2^2-\cdots-x_r^2\left(r\leqslant n\right)
\]是半负定型\[
    f=x_1^2+x_2^2+\cdots+x_p^2-x_{p+1}^2-\cdots-x_r^2\left(1\leqslant p<r\leqslant n\right)
\]是不定型.
}
\thm{正定型与负定型的规范标准型}{正定型与负定型的规范标准型}{
    设实二次型$f\left(
        x_1,x_2,\cdots,x_n
        \right)$的秩为$r$,正惯性指数为$p$,负惯性指数$q$,则\begin{enumerate}[label=\arabic*)]
        \item $f$是正定型当且仅当$p=n$
        \item $f$是半正定型当且仅当$p=r$
        \item $f$是负定型当且仅当$q=n$
        \item $f$是半负定型当且仅当$q=r$
        \item $f$为不定型当且仅当$p,q>0$
    \end{enumerate}
}
\subsection{正定阵}
\thm{正定阵与负定阵}{正定阵与负定阵}{
    设$n$阶实对称阵$\bm{A},\rank \left(\bm{A}\right)=r$,,\begin{enumerate}[label=\arabic*)]
        \item $\bm{A}$是正定阵当且仅当$\bm{A}$合同于$\bm{I}_n$
        \item $\bm{A}$是半正定阵当且仅当$\bm{A}$合同于$\begin{pmatrix}
                      \bm{I}_r & \bm{O} \\
                      \bm{O}   & \bm{O}
                  \end{pmatrix}$
        \item $\bm{A}$是负定阵当且仅当$\bm{A}$合同于$-\bm{I}_n$
        \item $\bm{A}$是半负定阵当且仅当$\bm{A}$合同于$\begin{pmatrix}
                      -\bm{I}_r & \bm{O} \\
                      \bm{O}    & \bm{O}
                  \end{pmatrix}$
    \end{enumerate}
}
\dfn{顺序主子式}{顺序主子式}{
    设$n$阶矩阵$\bm{A}=\left(a_{ij}\right)_{n\times n}$,则$\bm{A}$的$n$个子式\[
        \begin{vmatrix}
            a_{11} & a_{12} & \cdots & a_{1k} \\
            a_{21} & a_{22} & \cdots & a_{2k} \\
            \cdots & \cdots &        & \cdots \\
            a_{k1} & a_{k2} & \cdots & a_{kk}
        \end{vmatrix}\left(k=1,2,\cdots,n\right)
    \]称为$\bm{A}$的顺序主子式.
}
\thm{正定阵的判定}{正定阵的判定}{
    设$n$阶实对称阵$\bm{A}$,则$\bm{A}$是正定阵当且仅当$\bm{A}$的所有顺序主子式均大于零.\begin{proof}
        先证充分性,设$\bm{A}=\left(a_{ij}\right)_{n\times n}$正定,则二次型\[
            f\left(
            x_1,x_2,\cdots,x_n
            \right)=\bm{x}'\bm{Ax}=\sum_{i,j=1}^{n}a_{ij}x_ix_j
        \]正定,令\[
            f_k\left(
            x_1,x_2,\cdots,x_k
            \right)=f\left(
            x_1,x_2,\cdots,x_k,0,0,\cdots,0
            \right)=\sum_{i,j=1}^{k}a_{ij}x_ix_j
        \]正定,其相伴的矩阵$\bm{A}_k$正定.$\forall\bm{0}\neq\left(c_1,c_2,\cdots,c_k\right)\in\bbr ^k$\[
            f_k\left(
            c_1,c_2,\cdots,c_k
            \right)=f\left(
            c_1,c_2,\cdots,c_k,0,0,\cdots,0
            \right)>0
        \]于是$\bm{A}_k\left(
            1\leqslant k\leqslant n
            \right)$正定,由\cref{thm:正定阵与负定阵}知存在非异阵$\bm{B}_k\in M_k\left(\bbr \right) $使得\[
            \bm{B}_k'\bm{A}_k\bm{B}_k=\bm{I}_k
        \]于是\[
            1=\left|
            \bm{B}_k'\bm{A}_k\bm{B}_k
            \right|=|\bm{A}_k|\left|\bm{B}_k\right|^2\Longrightarrow\left|\bm{A}_k\right|>0\left(
            1\leqslant k\leqslant n
            \right)
        \]得证.

        再证充分性,考虑数学归纳法,对$n$作归纳,当$n=1$时,$\bm{A}=\left(a\right),a>0,f=ax_1^2$正定,于是$\bm{A}$正定.设阶数小于$n$的时候成立,下面证明$n$阶的情形.\[
            \bm{A}=\begin{pmatrix}
                \bm{A}_{n-1} & \bm{\alpha} \\
                \bm{\alpha}' & a_{nn}
            \end{pmatrix}
        \]则由归纳假设$\bm{A}_{n-1}$正定.考虑分块对称初等变换\begin{align*}
            \bm{A}\longrightarrow\begin{pmatrix}
                                     \bm{A} & \bm{O}                                          \\
                                     \bm{O} & a_{nn}-\bm{\alpha}'\bm{A}_{n-1}^{-1}\bm{\alpha}
                                 \end{pmatrix}
        \end{align*}这是合同变换,是第三类分块对称初等变换,不改变行列式的值即\[
            \left|\bm{A}\right|=\left|\bm{A}_{n-1}\right|\left(
            a_{nn}-\bm{\alpha}'\bm{A}_{n-1}^{-1}\bm{\alpha}
            \right)\Longrightarrow a_{nn}-\bm{\alpha}'\bm{A}_{n-1}^{-1}\bm{\alpha}>0
        \]因为$\bm{A}_{n-1}$正定,故存在非异阵$\bm{C}_{n-1}\in M_{n-1}\left(\bbr \right)$使得\[
            \bm{C}_{n-1}'\bm{A}_{n-1}\bm{C}_{n-1}=\bm{I}_{n-1}
        \]于是\begin{align*}
              & \begin{pmatrix}
                    \bm{C}_{n-1} & \bm{O} \\
                    \bm{O}       & 1
                \end{pmatrix}
            \begin{pmatrix}
                \bm{A} & \bm{O}                                          \\
                \bm{O} & a_{nn}-\bm{\alpha}'\bm{A}_{n-1}^{-1}\bm{\alpha}
            \end{pmatrix}
            \begin{pmatrix}
                \bm{C}_{n-1} & \bm{O} \\
                \bm{O}       & 1
            \end{pmatrix}                                              \\
            = & \begin{pmatrix}
                    \bm{I}_{n-1} & \bm{O}                                          \\
                    \bm{O}       & a_{nn}-\bm{\alpha}'\bm{A}_{n-1}^{-1}\bm{\alpha}
                \end{pmatrix}
        \end{align*}即正惯性指数为$n$,故$\bm{A}$正定.
    \end{proof}
}
\exa{}{}{
    四元二次型\[
        f=x_1^2+4x_2^2+4x_3^2+3x_4^2+2tx_1x_2-2x_1x_3+4x_2x_3
    \]正定,求参数$t$的范围.\begin{solution}
        相伴的实对称阵\[
            \bm{A}=\begin{pmatrix}
                1  & t & -1 & 0 \\
                t  & 4 & 2  & 0 \\
                -1 & 2 & 4  & 0 \\
                0  & 0 & 0  & 3
            \end{pmatrix}
        \]容易求得$-2<t<1$时正定.
    \end{solution}
}
\pro{正定阵的性质}{正定阵的性质}{
    设$n$阶实对称阵$\bm{A}$为正定阵,则:\begin{enumerate}[label=\arabic*)]
        \item $\bm{A}$的所有主子阵均正定

        \item $\bm{A}$的所有主子式均大于零.特别地,$\bm{A}$的主对角元全大于零
        \item $\bm{A}$中绝对值最大的数必定在主对角线上
    \end{enumerate}\begin{proof}
        \begin{enumerate}[label=\arabic*)]
            \item 取$i_1,i_2,\cdots,i_k$行、列组成的主子阵$\bm{B}$,令\[
                      f_{\bm{B}}\left(
                      x_{i_1},x_{i_2},\cdots,x_{i_k}
                      \right)\coloneqq f\left(
                      0,\cdots,0,x_{i_1},0,\cdots,0,x_{i_2},\cdots,x_{i_k},0,\cdots,0
                      \right)
                  \]于是由$f$正定知$f_{\bm{B}}$正定,于是$\bm{B}$正定.
            \item 由$(1)$可得.
            \item 设$\bm{A}$的绝对值最大的数在$\bm{A}$的非对角线元素$a_{ij}\left(i<j\right)$上,考虑二阶主子式\begin{align*}
                      \bm{A}\begin{pmatrix}
                                i & j \\
                                i & j
                            \end{pmatrix} & =\begin{pmatrix}
                                                 a_{ii} & a_{ij} \\
                                                 a_{ji} & a_{jj}
                                             \end{pmatrix}         \\
                                            & =a_{ii}a_{jj}-a_{ij}^2 \\
                                            & \leqslant0
                  \end{align*}与$(2)$矛盾.\qedhere
        \end{enumerate}
    \end{proof}
}
\cor{}{正定实对称阵的特征值全是整数}{
    设$\bm{A}$为$n$阶正定实对称阵,则$\bm{A}$的特征值全为正数.\begin{proof}
        设$\bm{0}\neq\bm{\xi}=\left(a_1,a_2,\cdots,a_n\right)\in\bbc ^n$则\[\overline{\bm{\xi}}'\bm{\xi}=\sum_{i=1}^{n}\left|a_i\right|^2>0\]任取$\bm{A}$的特征值$\lambda_0$及其特征向量$\bm{\xi}$.因为$\bm{A}$正定,则存在非异阵$\bm{C}\in M_n\left(\bbr \right)$使得\[
            \bm{A}=\bm{C}'\bm{IC}
        \]因为\begin{align*}
            \bm{A\xi}                             & =\lambda_0\bm{\xi}                                         \\
            \overline{\bm{\xi}}'\lambda_0\bm{\xi} & =\overline{\bm{\xi}}'\bm{A\xi}                             \\
                                                  & =\overline{\bm{\xi}}'\bm{C}'\bm{C}\bm{\xi}                 \\
                                                  & =\left(\overline{\bm{C\xi}}\right)' \left(\bm{C\xi}\right) \\
        \end{align*}由$\bm{\xi}\neq\bm{0}\Longrightarrow\overline{\bm{\xi}}'\bm{\xi}>0,\bm{C\xi}\neq\bm{0}\Longrightarrow\left(\overline{\bm{C\xi}}\right)' \left(\bm{C\xi}\right)>0$于是\[
            \lambda_0=\frac{
                \left(\overline{\bm{C\xi}}\right)' \left(\bm{C\xi}\right)
            }{\overline{\bm{\xi}}'\bm{\xi}}>0
        \]
    \end{proof}
}
\pro{正定阵的若干等价结论}{正定阵的若干等价结论}{
    设$\bm{A}$为$n$阶正定实对称阵,则下列等价:
    \begin{enumerate}[label=\arabic*)]
        \item $\bm{A}$正定
        \item $\bm{A}$合同于$\bm{I}_n$
        \item 存在非异阵$\bm{C}\in M_n\left(
                  \bbr
                  \right)$使得\[
                  \bm{A}=\bm{C}'\bm{C}
              \]
        \item $\bm{A}$的所有主子式均大于$0$
        \item $\bm{A}$的所有顺序主子式均大于$0$
        \item $\bm{A}$的特征值均大于$0$
    \end{enumerate}\begin{proof}
        $(1)\Longleftrightarrow (2)$由\cref{thm:正定阵与负定阵}即得,$(2)\Longleftrightarrow (3)$由\cref{thm:正定阵与负定阵}即得,$(1)\Longrightarrow (4)$由\cref{prop:正定阵的性质}可知,$(4)\Longrightarrow (5)$显然,而$(5)\Longrightarrow (1)$由\cref{thm:正定阵的判定}.$(1)\Longrightarrow (6)$由\cref{prop:正定阵的性质}即得,$(6)\Longrightarrow (1)$后续证明.
    \end{proof}
}
\subsection{半正定阵}
\lem{半正定阵的判定}{半正定阵的判定}{
    设$\bm{A}$是$n$阶实对称阵,则$\bm{A}$半正定当且仅当存在$\bm{C}\in M_n\left(\bbr \right)$使得$\bm{A}=\bm{C}'\bm{C}$.特别地,$\left|\bm{A}\right|\geqslant 0.$\begin{proof}
        先证必要性.由\cref{thm:正定阵与负定阵}知$\bm{A}$合同于$\begin{pmatrix}
                \bm{I}_r & \bm{O} \\
                \bm{O}   & \bm{O}
            \end{pmatrix}$,其中$r=\rank \left(\bm{A}\right)$,即存在非异阵$\bm{B}\in M_n\left(\bbr \right)$使得\[
            \bm{A}=\bm{B}'\begin{pmatrix}
                \bm{I}_r & \bm{O} \\
                \bm{O}   & \bm{O}
            \end{pmatrix}\bm{B}
        \]令$\bm{C}=\begin{pmatrix}
                \bm{I}_r & \bm{O} \\
                \bm{O}   & \bm{O}
            \end{pmatrix}\bm{B}$注意到$\begin{pmatrix}
                \bm{I}_r & \bm{O} \\
                \bm{O}   & \bm{O}
            \end{pmatrix}^2=\begin{pmatrix}
                \bm{I}_r & \bm{O} \\
                \bm{O}   & \bm{O}
            \end{pmatrix}$于是$\bm{A}=\bm{C}'\bm{C}.$

        再证充分性,设$\bm{A}=\bm{C}'\bm{C},\forall \bm{0}\neq\bm{\alpha}\in \bbr ^n$\[
            \bm{\alpha}'\bm{A\alpha}=\bm{\alpha}'\bm{C}'\bm{C\alpha}=\left(\bm{C\alpha}\right)'\left(\bm{C\alpha}\right)
        \]其中$\bm{C\alpha}=\left(
            a_1,a_2,\cdots,a_n
            \right)'$则\[
            \bm{\alpha}'\bm{A\alpha}=\sum_{i=1}^{n}a_i^2\geqslant 0\qedhere
        \]
    \end{proof}
}
\clm{}{}{
    事实上,可以用正定阵逼近半正定阵.
}
\lem{}{正定与半正定}{
    设$\bm{A}$是$n$阶正定实对称阵,则$\bm{A}$正定当且仅当$\forall t\in\bbr ^+,\bm{A}+t\bm{I}_n$正定.\begin{proof}
        先证必要性.$\forall \bm{0}\neq\bm{\alpha}\in\bbr ^n$\begin{align*}
            \bm{\alpha}'\left(
            \bm{A}+t\bm{I}_n
            \right)\bm{\alpha}=\bm{\alpha}'\bm{A\alpha}+t\bm{\alpha}'\bm{\alpha}
        \end{align*}其中$\bm{\alpha}'\bm{A\alpha}\geqslant 0,\bm{\alpha}'\bm{\alpha}>0,t>0$,于是$\bm{\alpha}'\left(\bm{A}+t\bm{I}_n\right)\bm{\alpha}$正定.

        再证充分性.$\forall\bm{0}\neq\bm{\alpha}\in\bbr ^n$\[
            \bm{\alpha}'\left(
            \bm{A}+t\bm{I}_n
            \right)\bm{\alpha}=\bm{\alpha}'\bm{A\alpha}+t\bm{\alpha}'\bm{\alpha}>0
        \]恒成立,取$t\to0^+$即得$\bm{\alpha}'\bm{A\alpha}\geqslant 0.$
    \end{proof}
}
\exa{}{}{
    矩阵$\bm{A}=\diag \left\{
        1,0,-1
        \right\},\left|\bm{A}_1\right|=1,\left|\bm{A}_2\right|=\left|\bm{A}_3\right|=0$却是不定阵.
}
\lem{半正定的判定}{半正定的判定}{
    设$\bm{A}$是$n$阶实对称阵,则$\bm{A}$半正定当且仅当$\bm{A}$的所有主子式均大于等于零.\begin{proof}
        先证必要性.$f\left(x_1,x_2,\cdots,x_n\right)=\bm{x}'\bm{Ax}$半正定,取$\bm{A}$的$i_1,i_2,\cdots,i_r$行、列成主子阵.构造\[
            f_{\bm{B}}\left(
            x_{i_1},x_{i_2},\cdots,x_{i_r}
            \right)=f\left(
            0,0,\cdots,0,x_{i_1},0,\cdots,0,x_{i_2},\cdots,x_{i_r},0,\cdots,0
            \right)
        \]于是由$f$半正定知$f_{\bm{B}}$半正定即$\bm{B}$半正定于是\[
            \left|\bm{B}\right|=\bm{A}\begin{pmatrix}
                i_1 & i_2 & \cdots & i_r \\
                i_1 & i_2 & \cdots & i_r
            \end{pmatrix}>0
        \]

        再证充分性.考虑摄动\begin{align*}
            \left|\bm{A}+t\bm{I}_n\right| & =t^n+c_1t^{n-1}+\cdots+c_n \\
        \end{align*}其中$c_i$为$\bm{A}$的所有$i$阶主子式之和从而$c_i\geqslant 0,\forall 1\leqslant i\leqslant n.$于是$\forall t\in\bbr ^+,\left|\bm{A}+t\bm{I}_n\right|>0.$设$\bm{A}_k$为$\bm{A}$的顺序主子阵,于是$\bm{A}_k$的主子式均大于等于零.另做上述过程,有\[
            \left|
            \bm{A}_k+t\bm{I}_k
            \right|>0,\forall 1\leqslant k\leqslant n,t\in\bbr ^+
        \]于是$\bm{A}+t\bm{I}_n$正定$\forall t\in\bbr ^+$.根据\cref{lem:正定与半正定}知$\bm{A}$半正定.
    \end{proof}
}
\lem{}{特征值判定版正定}{
    设$\bm{A}$是$n$阶实对称阵,则$\bm{A}$半正定当且仅当$\bm{A}$的所有特征值均大于等于零.\begin{proof}
        设$\bm{A}$的特征值$\lambda_1,\lambda_2,\cdots,\lambda_n.$由\cref{lem:正定与半正定}知$\bm{A}+t\bm{I}_n,\forall t\in\bbr ^+$正定即$\lambda_i+t>0,\forall 1\leqslant i\leqslant n,t\in\bbr ^+$于是$\lambda_i\geqslant 0,\forall 1\leqslant i\leqslant n.$
    \end{proof}
}
\pro{半正定的等价结论}{半正定的等价结论}{
    假设$\bm{A}$为$n$阶实对称阵,则下列结论等价:\begin{enumerate}[label=\arabic*)]
        \item $\bm{A}$半正定
        \item $\bm{A}$合同于$\begin{pmatrix}
                      \bm{I}_r & \bm{O} \\
                      \bm{O}   & \bm{O}
                  \end{pmatrix}$
        \item $\exists \bm{C}\in M_n\left(\bbr \right),\bm{A}=\bm{C}'\bm{C}$
        \item $\bm{A}$的所有主子式均大于等于$0$
        \item $\bm{A}$的所有特征值均大于等于$0$
    \end{enumerate}
}
\lem{}{对角元为零则行列全为零}{
    设$\bm{A}=\left(a_{ij}\right)_{n\times n}$为半正定阵,若$a_{ii}=0$,则$\bm{A}$的第$i$行与第$i$列元素均为零.\begin{proof}
        任取$a_{ij}\left(i\neq j\right)$所在的二阶主子式$\begin{vmatrix}
                a_{ii} & a_{ij} \\
                a_{ji} & a_{jj}
            \end{vmatrix}=-a_{ij}^2$,由\cref{prop:半正定的等价结论}知结果为大于等于零即$a_{ij}=0\left(\forall i\neq j\right).$
    \end{proof}
}
\lem{}{成零}{
    若$\bm{A}$半正定,且$\exists \bm{\alpha}\in\bbr ^n$使得$\bm{\alpha}'\bm{A\alpha}=0$则$\bm{A\alpha}=\bm{0}.$\begin{proof}
        由\cref{prop:半正定的等价结论}知存在$\bm{C}\in M_n\left(\bbr \right)$使得$\bm{A}=\bm{C}'\bm{C}$.于是\begin{align*}
            0 & =\bm{\alpha}'\bm{A\alpha}                                                              \\
              & =\bm{\alpha}'\bm{C}'\bm{C\alpha}                                                       \\
              & =\left(\bm{C\alpha}\right)'\left(\bm{C\alpha}\right)\Longrightarrow\bm{C\alpha}=\bm{0}
        \end{align*}则$\bm{A\alpha}=\bm{C}'\bm{C\alpha}=\bm{0}.$
    \end{proof}
}