\newpage
\section{惯性定理}
\subsection{实二次型、实对称阵}
\thm{秩是合同关系的不变量}{秩是合同关系的不变量}{
    矩阵的秩是合同关系下的一个不变量.\begin{proof}
        因为对于实对称阵$\bm{A}$,存在非异阵$\bm{C}$使得$\bm{A}$合同于\[
            \bm{C}'\bm{AC}=\diag \left\{
            d_1,d_2,\cdots,d_n
            \right\}
        \]因为对换主对角元是合同变换,故可以把所有零元兑换到最后即\[
            \bm{C}'\bm{AC}=\diag \left\{
            d_1,d_2,\cdots,d_r,0,\cdots,0
            \right\}
        \]于是\[
            \rank \left(\bm{A}\right)=\rank \left(\bm{C}'\bm{AC}\right)=r
        \]即秩是合同关系下的一个不变量.
    \end{proof}
}
考虑对应的二次型\[
    f\left(
    x_1,x_2,\cdots,x_n
    \right)=d_1x_1^2+d_2x_2^2+\cdots+d_rx_r^2
\]并进一步假设前$p$个主对角元大于零即$d_1,d_2,\cdots,d_p>0,d_{p+1},\cdots,d_r<0$,则令\[
    \begin{cases*}
        y_1=\sqrt{d_1}x_1              \\
        y_2=\sqrt{d_2}x_2              \\
        \cdots\cdots\cdots\cdots       \\
        y_p=\sqrt{d_p}x_p              \\
        y_{p+1}=\sqrt{-d_{p+1}}x_{p+1} \\
        \cdots\cdots\cdots\cdots       \\
        y_r=\sqrt{-d_r}x_r             \\
        y_{r+1}=x_{r+1}                \\
        \cdots\cdots\cdots\cdots       \\
        y_n=x_n
    \end{cases*}
\]于是\[
    f=y_1^2+y_2^2+\cdots+y_p^2-y_{p+1}^2-\cdots-y_r^2
\]即$\bm{A}$合同于\[
    \diag \left\{
    1,1,\cdots,1;-1,-1,\cdots,-1;0,0,\cdots,0
    \right\}
\]其中有$p$个$1$和$r-p$个$-1$,$n-r$个$0$.
\dfn{规范标准型}{规范标准型}{
$n$元实二次型$f\left(
    x_1,x_2,\cdots,x_n
    \right)$化为的\[
    f=y_1^2+y_2^2+\cdots+y_p^2-y_{p+1}^2-\cdots-y_r^2
\]称为实二次型的规范标准型.
}
\thm{惯性定理}{惯性定理}{
    设$n$元实二次型$f\left(
        x_1,x_2,\cdots,x_n
        \right)$的两个规范标准型为\begin{align*}
         & y_1^2+y_2^2+\cdots+y_p^2-y_{p+1}^2-\cdots-y_r^2 \\
         & z_1^2+z_2^2+\cdots+z_k^2-z_{k+1}^2-\cdots-z_r^2
    \end{align*}则$p=k.$\begin{proof}
        即证\begin{align*}
            f & =y_1^2+y_2^2+\cdots+y_p^2-y_{p+1}^2-\cdots-y_r^2          \\
              & =z_1^2+z_2^2+\cdots+z_k^2-z_{k+1}^2-\cdots-z_r^2\tag{$*$}
        \end{align*}

        设$\bm{x}=\bm{Cy}=\bm{Bz},\bm{B},\bm{C}$为非异阵.于是$\bm{z}=\bm{C}^{-1}\bm{By}=\left(c_{ij}\right)_{n\times n}\bm{y}.$\[
            \begin{cases*}
                z_1=c_{11}y_1+c_{12}y_2+\cdots+c_{1n}y_n \\
                z_2=c_{21}y_1+c_{22}y_2+\cdots+c_{2n}y_n \\
                \qquad\qquad\cdots\cdots\cdots\cdots     \\
                z_n=c_{n1}y_1+c_{n2}y_2+\cdots+c_{nn}y_n
            \end{cases*}
        \]

        考虑反证法,先设$p>k$.思路是通过取值使得$\left(*\right)$式产生矛盾.不妨取$z_1=z_2=\cdots=z_k=y_{p+1}=\cdots=y_n=0$得到\[
            \begin{cases*}
                c_{11}y_1+c_{12}y_2+\cdots+c_{1n}y_n=0 \\
                c_{21}y_1+c_{22}y_2+\cdots+c_{2n}y_n=0 \\
                \qquad\qquad\cdots\cdots\cdots\cdots   \\
                c_{k1}y_1+c_{k2}y_2+\cdots+c_{kn}y_n=0 \\
                y_{p+1}=0                              \\
                \qquad\qquad\cdots\cdots\cdots\cdots   \\
                y_n=0
            \end{cases*}\tag{$**$}
        \]这个齐次线性方程组共$n-p+k<n$个方程,即系数矩阵的秩严格小于$n$,故一定有非零解$y_1=a_1,\cdots,y_p=a_p,y_{p+1}=0,\cdots,y_n=0$即$\exists i\in\left[1,p\right],a_p\neq 0$于是\[
            y_1^2+y_2^2+\cdots+y_p^2=a_1^2+a_2^2+\cdots+a_p^2>0
        \]返回去考虑$\left(*\right)$式\begin{align*}
            0<      LHS & =a_1^2+a_2^2+\cdots+a_p^2 \\
            =RHS        & =-\left(
            z_k^2+z_{k+1}^2+\cdots+z_r^2
            \right)\leqslant 0
        \end{align*}得到矛盾.同理可证$p<k$的情况,故$p=k$.
    \end{proof}
}
\dfn{惯性指数}{惯性指数}{
设$f$的规范标准型为\[
    f=y_1^2+y_2^2+\cdots+y_p^2-y_{p+1}^2-\cdots-y_r^2
\]称$r$为二次型$f$的秩,称$p$为二次型$f$的正惯性指数,称$q=r-p$为二次型$f$的负惯性指数,$s=p-q$为二次型$f$的符号差.
}
\clm{}{}{
    \cref{def:惯性指数}中四者知其二即可求得另外两者.

    \cref{def:惯性指数}中定义也用于对应的实对称矩阵.
}
\thm{合同关系的全系不变量}{合同全系不变量}{
    秩与符号差(或者正负惯性指数)是实对称矩阵在合同关系下的全系不变量.\begin{proof}
        设实对称阵$\bm{A},\bm{B}$的秩$r$,符号差$s$,正惯性指数$\displaystyle p=\frac{1}{2}\left(r+s\right)$,负惯性指数$q=\displaystyle \frac{1}{2}\left(r-s\right)$,即$\bm{A},\bm{B}$均合同于\[
            \diag \left\{
            1,1,\cdots,1;-1,-1,\cdots,-1;0,0,\cdots,0
            \right\}
        \]则$\bm{A},\bm{B}$合同.
    \end{proof}
}
\dfn{合同标准型}{合同标准型}{
    设$\bm{A}\in M_n\left(\bbr \right)$的秩为$r$,符号差为$s$,正惯性指数$\displaystyle
        p=\frac{1}{2}\left(r+s\right)$,负惯性指数$q=\displaystyle \frac{1}{2}\left(r-s\right)
    $
    则合同标准型\[
        \diag \left\{
        1,1,\cdots,1;-1,-1,\cdots,-1;0,0,\cdots,0
        \right\}
    \]其中有$p$个$1$和$q$个$-1$,$n-r$个$0$.
}
\subsection{复二次型、复对称阵}
由于负数在复数域上可开方,故复二次型的规范标准型为\[
    f=z_1^2+z_2^2+\cdots+z_r^2
\]其中$r$为复二次型的秩.
\dfn{复二次型的规范标准型}{复二次型的规范标准型}{
    $n$元复二次型$f\left(
        x_1,x_2,\cdots,x_n
        \right)$化为的\[
        f=z_1^2+z_2^2+\cdots+z_r^2
    \]称为复二次型的规范标准型.
}
\thm{复二次型的全系不变量}{复二次型的全系不变量}{
    复二次型的全系不变量只有秩一个.
}
\thm{复对称矩阵的合同标准型}{复对称矩阵的合同标准型}{
    复对称矩阵的合同标准型为\[
        \diag \left\{
        1,1,\cdots,1;0,0,\cdots,0
        \right\}=\begin{pmatrix}
            \bm{I}_r & \bm{O} \\
            \bm{O}   & \bm{O}
        \end{pmatrix}
    \]其中有$r$个$1$,$n-r$个$0$.
}