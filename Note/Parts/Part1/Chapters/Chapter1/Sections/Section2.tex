\newpage
\section{行列式的展开与转置}
\subsection{展开}
\thm{}{行列式按列展开的任意性}{
    行列式可以按照任意一列展开.
    \begin{proof}
        设将$\left|\bm{A}\right|$按第$r$列展开,首先考虑经过$r-1$次相邻对换将第$r$列移到第一列的情形得到行列式$\left|\bm{B}\right|$,由\cref{cor:列形式的行列式的基础性质}的$(4)$知$\left|\bm{B}\right|=\left(-1\right)^{r-1}\left|\bm{A}\right|.$展开得\[
            \left|\bm{B}\right|=a_{1r}N_{11}-a_{2r}N_{21}+\cdots+(-1)^{n+1}a_{nr}N_{n1}
        \]注意到$N_{i1}=M_{ir}$,于是
        \begin{align*}
            \left|\bm{A}\right| & =\left(-1\right)^{r-1}\left|\bm{B}\right|                                                  \\
                                & =\left(-1\right)^{r-1}\left(a_{1r}N_{11}-a_{2r}N_{21}+\cdots+(-1)^{n+1}a_{nr}N_{n1}\right) \\
                                & =a_{1r}A_{1r}+a_{2r}A_{2r}+\cdots+(-1)^{n+1}a_{nr}A_{nr}
        \end{align*}于是证明了$\left|\bm{A}\right|$可以对任何一列展开.
    \end{proof}
}
\dfn{Kronecken符号}{Kronecken符号}{
    定义Kronecken符号$\delta_{ij}$为\[
        \delta_{ij}=
        \begin{cases*}
            1 & $,i=j;$     \\
            0 & $,i\neq j.$
        \end{cases*}
    \]
}
\thm{异乘变零定理}{列形式的异乘变零定理}{
设$n$阶行列式$\left|\bm{A}\right|$且$1\leqslant r,s\leqslant n$,则
\[
    a_{1r}A_{1s}+a_{2r}A_{2s}+\cdots+a_{nr}A_{ns}=\delta_{rs}\left|\bm{A}\right|
\]\begin{proof}
    $r=s$显然成立.

    $r\neq s$时,不妨设$r<s$,构造
    \[
        \begin{vmatrix}
            a_{11} & \cdots & a_{1r} & \cdots & a_{1r} & \cdots & a_{1n} \\
            a_{21} & \cdots & a_{2r} & \cdots & a_{2r} & \cdots & a_{2n} \\
            \vdots &        & \vdots &        & \vdots &        & \vdots \\
            a_{s1} & \cdots & a_{sr} & \cdots & a_{ss} & \cdots & a_{sn} \\
        \end{vmatrix}=0
    \]按第$s$列展开\[
        a_{1r}A_{1s}+a_{2r}A_{2s}+\cdots+a_{nr}A_{ns}=0\qedhere
    \]
\end{proof}
}
\lem{}{单元素展开}{
\[
    \left|\bm{A}\right|=\begin{vmatrix}
        a_{11} & \cdots & a_{1r} & \cdots & a_{1n} \\
        \vdots &        & \vdots &        & \vdots \\
        0      & \cdots & a_{sr} & \cdots & 0      \\
        \vdots &        & \vdots &        & \vdots \\
        a_{n1} & \cdots & a_{sr} & \cdots & a_{nn}
    \end{vmatrix}=a_{sr}A_{sr}
\]\begin{proof}
    按第$r$列展开\[
        \left|\bm{A}\right|=a_{1r}A_{1r}+\cdots+a_{nr}A_{nr}
    \]因为$\forall i\neq s,A_{ir}$至少有一列全为零,于是$A_{ir}=0,\forall i\neq s.$
\end{proof}
}
\lem{}{按任意行展开}{
行列式可以按任意一行展开
\[
    \left|\bm{A}\right|=a_{r1}A_{r1}+a_{r2}A_{r2}+\cdots+a_{rn}A_{rn}
\]\begin{proof}
    进行拆分,$a_{r1}=a_{r1}+0+\cdots+0,a_{r2}=0+a_{r2}+\cdots+0,\cdots,a_{rn}=0+0+\cdots+a_{rn}$,于是
    \[
        \left|\bm{A}\right|=\begin{vmatrix}
            a_{11} & a_{12} & \cdots & a_{1n} \\
            \vdots & \vdots &        & \vdots \\
            a_{r1} & 0      & \cdots & 0      \\
            \vdots & \vdots &        & \vdots \\
            a_{n1} & a_{n2} & \cdots & a_{nn}
        \end{vmatrix}+
        \begin{vmatrix}
            0      & a_{12} & \cdots & 0      \\
            \vdots & \vdots &        & \vdots \\
            0      & a_{r2} & \cdots & 0      \\
            \vdots & \vdots &        & \vdots \\
            a_{n1} & a_{n2} & \cdots & a_{nn}
        \end{vmatrix}+\cdots+\begin{vmatrix}
            a_{11} & a_{12} & \cdots & a_{1n} \\
            \vdots & \vdots &        & \vdots \\
            0      & 0      & \cdots & a_{rn} \\
            \vdots & \vdots &        & \vdots \\
            a_{n1} & a_{n2} & \cdots & a_{nn}
        \end{vmatrix}
    \]由\cref{lem:单元素展开}知\[
        \left|\bm{A}\right|=a_{r1}A_{r1}+a_{r2}A_{r2}+\cdots+a_{rn}A_{rn}\qedhere
    \]
\end{proof}
}
\cor{异乘变零定理}{行形式的异乘变零定理}{
设$n$阶行列式$\left|\bm{A}\right|$且$1\leqslant r,s\leqslant n$,则\[
    a_{r1}A_{s1}+a_{r2}A_{s2}+\cdots+a_{rn}A_{sn}=\delta_{rs}\left|\bm{A}\right|
\]类似于列形式的\cref{thm:列形式的异乘变零定理},此行形式
$r=s$显然成立,$r\neq s$时,不妨设$r<s$,构造\[
    \left|\bm{C}\right|=\begin{vmatrix}
        a_{11} & a_{12} & \cdots & a_{1n} \\
        \vdots & \vdots &        & \vdots \\
        a_{r1} & a_{r2} & \cdots & a_{rn} \\
        \vdots & \vdots &        & \vdots \\
        a_{r1} & a_{r2} & \cdots & a_{rn} \\
        \vdots & \vdots &        & \vdots \\
        a_{n1} & a_{n2} & \cdots & a_{nn}
    \end{vmatrix}
\]由\cref{lem:按任意行展开},按第$s$行展开即可.
}
\rem{}{}{
    一个容易证明的结论是,$n$阶行列式展开式共有$n!$项.
}
\subsection{转置}
\dfn{转置}{转置}{
    设行列式\[
        \left|\bm{A}\right|=\begin{vmatrix}
            a_{11} & a_{12} & \cdots & a_{1n} \\
            a_{21} & a_{22} & \cdots & a_{2n} \\
            \vdots & \vdots &        & \vdots \\
            a_{n1} & a_{n2} & \cdots & a_{nn}
        \end{vmatrix}
    \]定义其转置\[
        \left|\bm{A}'\right|=\left|\bm{A}^T\right|=\begin{vmatrix}
            a_{11} & a_{21} & \cdots & a_{n1} \\
            a_{12} & a_{22} & \cdots & a_{n2} \\
            \vdots & \vdots &        & \vdots \\
            a_{1n} & a_{2n} & \cdots & a_{nn}
        \end{vmatrix}
    \]即$\left|\bm{A}\right|$的第$i$行就是$\left|\bm{A}'\right|$的第$i$列($\forall 1\leqslant i\leqslant n$).
}
\thm{}{行列式的转置不变性}{
    行列式转置后值不变即$\left|\bm{A}'\right|=\left|\bm{A}\right|$.\begin{proof}
        考虑数学归纳法,$n=1$时显然成立.设结论对$n-1$阶成立,下面证明$n$阶的情形.事实上,根据\cref{thm:行列式按列展开的任意性}和\cref{lem:按任意行展开},只要将$\left|\bm{A}\right|$和$\left|\bm{A}'\right|$分别按某一列(行)、行(列)展开即证.
    \end{proof}
}
\subsection{Cramer法则}
设线性方程组:
\[
    \left\{
    \begin{lgathered}
        a_{11}x_1 + a_{12}x_2 + \cdots +a_{1n}x_n = b_1\\
        a_{21}x_1 + a_{22}x_2 + \cdots +a_{2n}x_n = b_2\\
        \qquad\qquad\cdots\cdots\cdots\cdots\\
        a_{n1}x_1 + a_{n2}x_2 + \cdots +a_{nn}x_n = b_n
    \end{lgathered}
    \right.\tag{$\star$}
\]
一般记系数行列式
\[
    \left|\bm{A}\right|=\begin{vmatrix}
        a_{11} & a_{12} & \cdots & a_{1n} \\
        a_{21} & a_{22} & \cdots & a_{2n} \\
        \vdots & \vdots &        & \vdots \\
        a_{n1} & a_{n2} & \cdots & a_{nn}
    \end{vmatrix}
\]
\thm{Cramer法则}{Cramer法则}{
    当$\left(\star\right)$的系数行列式$ \det\left(\bm{A}\right) \neq 0$时,该方程组有且仅
    有一组解:
    \[
        x_i = \frac{ \det\left(\bm{A}_i\right)}
        { \det\left(\bm{A}\right)}\quad\left(
        i = 1,2,\ldots,n
        \right)
    \]
    其中的$ \det\left(\bm{A}_i\right)$
    为$ \det\left(\bm{A}\right)$用$\left(
        b_1,\ldots,b_n
        \right)^{\prime}$换掉第$i$列得到的行列式.
    \begin{proof}
        容易证明
        \begin{align*}
            \left|\bm{A}_i\right|
             & =
            \begin{vmatrix}
                a_{11} & \cdots & a_{1\left(i - 1\right)} & b_1 & a_{1\left(i + 1\right)} & \cdots & a_{1n} \\
                \vdots &        & \vdots                  &     & \vdots                  &        & \vdots \\
                a_{n1} & \cdots & a_{n\left(i - 1\right)} & b_n & a_{n\left(i + 1\right)} & \cdots & a_{nn}
            \end{vmatrix}                               \\
             & =\begin{vmatrix}
                    a_{11} & \cdots & a_{1\left(i - 1\right)} & a_{11}x_1 + \cdots +a_{1n}x_n & a_{1\left(i + 1\right)} & \cdots & a_{1n} \\
                    \vdots &        & \vdots                  &                               & \vdots                  &        & \vdots \\
                    a_{n1} & \cdots & a_{n\left(i - 1\right)} & a_{n1}x_1 + \cdots +a_{nn}x_n & a_{n\left(i + 1\right)} & \cdots & a_{nn}
                \end{vmatrix}
            \\
             & = x_i
            \begin{vmatrix}
                a_{11} & \cdots & a_{1n} \\
                \vdots & \ddots & \vdots \\
                a_{n1} & \cdots & a_{nn}
            \end{vmatrix}
            = x_i \left|\bm{A}\right|
        \end{align*}
        从而
        \[
            x_j = \frac{1}{\left|\bm{A}\right|}
            \left(
            b_1A_{1j}+\cdots
            +b_nA_{nj}
            \right),j = 1,\cdots,n
        \]
        因此$\forall 1\leqslant i
            \leqslant n$
        \begin{align*}
              & a_{i1}x_1+\cdots+a_{in}x_n                 \\
            = & \frac{a_{i1}}{\left|\bm{A}\right|}
            \left(b_1A_{11}+\cdots
            +b_nA_{n1}\right)                              \\
              & +\cdots+\frac{a_{in}}{\left|\bm{A}\right|}
            \left(
            b_1A_{1n}+\cdots+
            b_nA_{nn}
            \right)                                        \\
            = & \frac{b_1}{\left|\bm{A}\right|}
            \left(a_{i1}A_{11}+
            \cdots+
            a_{in}A_{1n}\right)                            \\
              & +\cdots+\frac{b_n}{\left|\bm{A}\right|}
            \left(
            a_{i1}A_{n1}+\cdots+
            a_{in}A_{nn}
            \right)                                        \\
            = & \frac{b_1}{\left|\bm{A}\right|}
            \cdot 0+\cdots+
            \frac{b_i}{\left|\bm{A}\right|}
            \cdot \left|\bm{A}\right|+\cdots+
            \frac{b_n}{\left|\bm{A}\right|}
            \cdot 0                                        \\
            = & b_i
        \end{align*}
        是原方程的解.
    \end{proof}
}