\newpage
\section{特别的行列式}
\subsection{Vandermonde行列式}
\exa{Vandermonde行列式}{}{
    Vandermonde行列式是指这样的行列式:
    \[
        V_n = \begin{vmatrix}
            1      & x_1    & \cdots & x_1^{n - 1} \\
            \vdots & \vdots & \ddots & \vdots      \\
            1      & x_n    & \cdots & x_n^{n - 1}
        \end{vmatrix}
    \]\begin{solution}
        不难得出其递推式:
        \[
            V_n = \prod _{ i  = 1 }^{n - 1}\left(x_n  - x_i\right) V_{n - 1}
        \]
        故:
        \[
            V_n  = \prod _{1 \leqslant i < j \leqslant n}
            \left(x _j - x_i\right)
        \]
    \end{solution}
}
\subsection{一个含组合数的行列式}
\exa{}{}{
    \[
        \begin{vmatrix}
            1      & 1            & \cdots & 1                  \\
            1      & \bmn{1}{2}   & \cdots & \bmn{1}{n}         \\
            1      & \bmn{2}{3}   & \cdots & \bmn{2}{n + 1}     \\
            \vdots & \vdots       &        & \vdots             \\
            1      & \bmn{n-1}{n} & \cdots & \bmn{n-1}{2 n - 2}
        \end{vmatrix}
    \]\begin{solution}
        因为有
        \[
            \bmn{m}{n}  - \bmn{m - 1}{n - 1} = \bmn{m}{n-1}
        \]
        故重复用每一行减去上一行得到\begin{align*}
            \left|\bm{D}\right|
             & =
            \begin{vmatrix}
                1      & 1             & \cdots & 1                    \\
                1      & \bmn{1}{2}    & \cdots & \bmn{1}{n}           \\
                1      & \bmn{2}{3}    & \cdots & \bmn{2}{n+1}         \\
                \vdots & \vdots        &        & \vdots               \\
                1      & \bmn{n-1 }{n} & \cdots & \bmn{n - 1}{ 2n - 2}
            \end{vmatrix}             \\
             & = \begin{vmatrix}
                     \bmn{1}{1}       & \bmn{1}{2}   & \cdots & \bmn{ 1}{n-1}      \\
                     \bmn{2}{2}       & \bmn{2}{3}   & \cdots & \bmn{2}{n}         \\
                     \vdots           & \vdots       &        & \vdots             \\
                     \bmn{n -1}{n -1} & \bmn{n-1}{n} & \cdots & \bmn{n  -1}{2n -3}
                 \end{vmatrix} \\
             & =\begin{vmatrix}
                    \bmn{2}{2}     & \cdots & \bmn{2}{n-1}    \\
                    \vdots         &        & \vdots          \\
                    \bmn{n-1}{n-1} & \cdots & \bmn{n-1}{2n-4}
                \end{vmatrix}                      \\
             & =\cdots                                                         \\
             & =\begin{vmatrix}
                    1 & \bmn{n-2}{n-1} \\
                    1 & \bmn{n-1}{n}
                \end{vmatrix}                                             \\
             & =1
        \end{align*}
    \end{solution}
}
\subsection{爪型行列式}
\exa{爪型行列式}{}{
    求形如
    \[
        \left|\bm{A}\right|
        =
        \begin{vmatrix}
            a_1    & b_2    & b_3    & \cdots & b_n    \\
            c_2    & a_2    & 0      & \cdots & 0      \\
            c_3    & 0      & a_3    & \cdots & 0      \\
            \vdots & \vdots & \vdots &        & \vdots \\
            c_n    & 0      & 0      & \cdots & a_n
        \end{vmatrix}
    \]
    的行列式的值其中$a_i\neq 0,\forall 2 \leqslant
        i\leqslant n.$
    \begin{solution}
        \begin{align*}
            \left|\bm{A}\right| & =
            \begin{vmatrix}
                a_1-\displaystyle\sum_{i = 2}
                ^{n}\cfrac{b_ic_i}{a_i} & b_2    & b_3    & \cdots & b_n    \\
                0                       & a_2    & 0      & \cdots & 0      \\
                0                       & 0      & a_3    & \cdots & 0      \\
                \vdots                  & \vdots & \vdots &        & \vdots \\
                0                       & 0      & 0      & \cdots & a_n
            \end{vmatrix} \\
                                & =
            \left(a_1-\sum_{i = 2}
            ^{n}\frac{b_ic_i}{a_i}\right)a_2a_3\cdots a_n               \\
                                & =a_1a_2\cdots a_n-
            \sum_{i = 2}^{n}a_2\cdots\widehat{a}_i\cdots
            a_nb_ic_i
        \end{align*}
    \end{solution}
}
\subsection{加和}
\exa{}{加和}{
证明:设
\[
    \left|\bm{A}\right|
    =\begin{vmatrix}
        a_{11} & \cdots & a_{1n} \\
        \vdots & \ddots & \vdots \\
        a_{n1} & \cdots & a_{nn}
    \end{vmatrix}
\]
则
\[
    \left|\bm{A}\left(
    t_1,t_2,\cdots,t_n
    \right)\right|
    =\begin{vmatrix}
        a_{11}+t_1 & a_{12}+t_2 &
        \cdots     & a_{1n}+t_n              \\
        a_{21}+t_1 & a_{22}+t_2 &
        \cdots     & a_{2n}+t_n              \\
        \vdots     & \vdots     &   & \vdots \\
        a_{n1}+t_1 & a_{n2}+t_2 &
        \cdots     & a_{nn}+t_n
    \end{vmatrix}
    =\left|\bm{A}\right|
    +\sum_{j=1}^{n}\left(
    t_j\sum_{i=1}^{n}
    A_{ij}
    \right)
\]
其中$A_{ij}$是$a_{ij}$的代数余子式.

特别地$
    t_1 = t_2 = \cdots = t_n = t
$时
得到
\[
    \left|\bm{A}\left(t\right)\right|
    =\left|\bm{A}\left(0\right)\right|
    +t\sum_{i,j=1}^{n}A_{ij}
\]
其中$A_{ij}$为$a_{ij}$在
$\left|\bm{A}\left(0\right)\right|$中的
代数余子式.\begin{proof}
    将上式第一列拆开
    \begin{align*}
        \left|\bm{A}\left(
        t_1,t_2,\cdots,t_n
        \right)\right|
         & =\begin{vmatrix}
                a_{11}+t_1 & a_{12}+t_2 &
                \cdots     & a_{1n}+t_n              \\
                a_{21}+t_1 & a_{22}+t_2 &
                \cdots     & a_{2n}+t_n              \\
                \vdots     & \vdots     &   & \vdots \\
                a_{n1}+t_1 & a_{n2}+t_2 &
                \cdots     & a_{nn}+t_n
            \end{vmatrix} \\
         & =\begin{vmatrix}
                a_{11} & a_{12}+t_2 &
                \cdots & a_{1n}+t_n              \\
                a_{21} & a_{22}+t_2 &
                \cdots & a_{2n}+t_n              \\
                \vdots & \vdots     &   & \vdots \\
                a_{n1} & a_{n2}+t_2 &
                \cdots & a_{nn}+t_n
            \end{vmatrix}+
        \begin{vmatrix}
            t_1    & a_{12}+t_2 &
            \cdots & a_{1n}+t_n              \\
            t_1    & a_{22}+t_2 &
            \cdots & a_{2n}+t_n              \\
            \vdots & \vdots     &   & \vdots \\
            t_1    & a_{n2}+t_2 &
            \cdots & a_{nn}+t_n
        \end{vmatrix}         \\
         & =\cdots                               \\
         & =\left|\bm{A}\right|
        +\sum_{j=1}^{n}\left(
        t_j\sum_{i=1}^{n}
        A_{ij}
        \right)\qedhere
    \end{align*}
\end{proof}
}
\exa{}{}{
    \cref{ex:加和}这一形式对形如
    \[
        \left|\bm{A}\right|=
        \begin{vmatrix}
            x_1    & a      & \cdots & a      & a      \\
            -a     & x_2    & \cdots & a      & a      \\
            \vdots & \vdots &        & \vdots & \vdots \\
            -a     & -a     & \cdots & -a     & x_n
        \end{vmatrix}
    \]
    的行列式极其方便,考虑$t=\pm a$,
    容易得到
    \[
        \left|\bm{A}\right|=\frac{1}{2}
        \left(\prod_{i=1}^{n}\left(x_i+a\right)+
        \prod_{i=1}^{n}\left(x_i-a\right)\right)
    \]
}
\exa{}{}{
设行列式
$\left|\bm{A}\right|
    =\left(a_{ij}\right)_n$,证明:
\[
    \begin{vmatrix}
        a_{11} & a_{12} & \cdots & a_{1n} & x_1    \\
        a_{21} & a_{22} & \cdots & a_{2n} & x_2    \\
        \vdots & \vdots &        & \vdots & \vdots \\
        a_{n1} & a_{n2} & \cdots & a_{nn} & x_n    \\
        y_1    & y_2    & \cdots & y_n    & z
    \end{vmatrix}=
    z\left|\bm{A}\right|
    -\sum_{i=1}^{n}\sum_{j=1}^{n}A_{ij}x_iy_j
\]\begin{proof}
    将行列式按最后一列展开,
    容易证明第$i$项$\left(1\leqslant i\leqslant n\right)$
    为$\displaystyle -\sum_{j=1}^{n}A_{ij}x_iy_j$,且最后一项为
    $z\left|\bm{A}\right|$,证毕.
\end{proof}
}
\subsection{Cauchy行列式}
\exa{Cauchy行列式}{Cauchy行列式}{
    求行列式
    \[
        \left|\bm{A}\right|
        =
        \begin{vmatrix}
            \left(a_1+b_1\right)^{-1}
                                      & \left(a_1+b_2\right)^{-1}
                                      & \cdots                    &
            \left(a_1+b_n\right)^{-1}                                          \\
            \left(a_2+b_1\right)^{-1} &
            \left(a_2+b_2\right)^{-1} & \cdots
                                      & \left(a_2+b_n\right)^{-1}              \\
            \vdots                    & \vdots                    &   & \vdots \\
            \left(a_n+b_1\right)^{-1} &
            \left(a_n+b_2\right)^{-1} &
            \cdots                    & \left(a_n+b_n\right)^{-1}
        \end{vmatrix}
    \]\begin{solution}
        记$D_n=\left|\bm{A}\right|$,用其余列减去第$n$列,提公因子后其余行减去第$n$行,再提公因子
        \begin{align*}
            D_n & =
            \begin{vmatrix}
                \dfrac{1}{a_1+b_1}     & \cdots
                                       & \dfrac{1}{a_1+b_{n-1}}
                                       & \dfrac{1}{a_1+b_n}                           \\
                \vdots                 &                            & \vdots & \vdots \\
                \dfrac{1}{a_{n-1}+b_1} &
                \cdots                 & \dfrac{1}{a_{n-1}+b_{n-1}} &
                \dfrac{1}{a_{n-1}+b_n}                                                \\[1em]
                \dfrac{1}{a_n+b_1}     & \cdots
                                       & \dfrac{1}{a_n+b_{n-1}}
                                       & \dfrac{1}{a_n+b_n}
            \end{vmatrix}                                     \\
                & =
            \begin{vmatrix}
                \dfrac{b_n-b_1}{\left(a_1+b_1\right)
                    \left(a_1+b_n\right)}
                                      & \cdots                     &
                \dfrac{b_n-b_{n-1}}{\left(a_1+b_{n-1}\right)
                \left(a_1+b_n\right)} &
                \dfrac{1}{a_1+b_n}                                                                 \\
                \vdots                &                            & \vdots               & \vdots \\
                \dfrac{b_n-b_1}{\left(
                    a_{n-1}+b_1
                    \right)\left(
                    a_{n-1}+b_n
                \right)}              & \cdots
                                      & \dfrac{b_n-b_{n-1}}{\left(
                a_{n-1}+b_{n-1}
                \right)\left(
                a_{n-1}+b_n
                \right)}              & \dfrac{1}{a_{n-1}+b_n}                                     \\[1em]
                \dfrac{b_n-b_1}{\left(
                    a_n+b_1
                    \right)\left(a_n+b_n\right)}
                                      & \cdots                     & \dfrac{b_n-b_{n-1}}{
                \left(a_n+b_{n-1}\right)
                \left(
                a_n+b_n
                \right)}              & \dfrac{1}{a_n+b_n}
            \end{vmatrix} \\
                & =
            \frac{
                \prod\limits_{i=1}^{n-1}
                \left(b_n-b_i\right)
            }{\prod\limits_{j=1}^{n}
                \left(a_j+b_n\right)}
            \begin{vmatrix}
                \dfrac{1}{a_1+b_1}         & \cdots & \dfrac{1}{a_1+b_{n-1}} & 1      \\
                \vdots                     &        & \vdots                 & \vdots \\
                \dfrac{1}{a_{n-1}+b_1}     & \cdots &
                \dfrac{1}{a_{n-1}+b_{n-1}} & 1                                        \\[1em]
                \dfrac{1}{a_n+b_1}         & \cdots &
                \dfrac{1}{a_n+b_{n-1}}     & 1
            \end{vmatrix}                                            \\
                & =
            \frac{
                \prod\limits_{i=1}^{n-1}
                \left(b_n-b_i\right)
            }{\prod\limits_{j=1}^{n}
                \left(a_j+b_n\right)}
            \begin{vmatrix}
                \dfrac{a_n-a_1}{\left(a_1+b_1\right)
                \left(a_n+b_1\right)} & \cdots             &
                \dfrac{a_n-a_1}{\left(
                    a_1+b_{n-1}
                    \right)\left(
                    a_n+b_{n-1}
                \right)}              & 0                                    \\
                \vdots                &                    & \vdots & \vdots \\
                \dfrac{a_n-a_{n-1}}{\left(
                a_{n-1}+b_1
                \right)\left(
                a_n+b_1
                \right)}              & \cdots             &
                \dfrac{a_n-a_{n-1}}{\left(
                a_{n-1}+b_{n-1}
                \right)\left(
                a_n+b_{n-1}
                \right)}              & 0                                    \\[1em]
                \dfrac{1}{a_n+b_1}    &
                \cdots                & \dfrac{1}{a_n+b_n} & 1
            \end{vmatrix}                       \\
                & =
            \frac{
                \prod\limits_{i=1}^{n-1}
                \left(a_n-a_i\right)\left(
                b_n-b_i
                \right)
            }{\prod\limits_{j=1}^{n}
                \left(a_j+b_n\right)\prod\limits_{k=1}^{n-1}\left(
                a_n+b_k
                \right)}D_{n-1}
        \end{align*}
        因此

        \[
            \left|\bm{A}\right|
            =
            \prod_{1\leqslant i < j\leqslant n}
            \left(a_j-a_i\right)\left(
            b_j-b_i
            \right)
            \Bigg/
            \prod_{i,j=1}^{n}\left(a_i+b_j
            \right)
        \]
    \end{solution}
}
\subsection{三对角行列式}
\exa{三对角行列式}{三对角行列式}{
\[
    D_n=\begin{vmatrix}
        a_1 & b_1 &        &        &         &         \\
        c_1 & a_2 & b_2    &        &         &         \\
            & c_2 & a_3    & \ddots &         &         \\
            &     & \ddots & \ddots & \ddots  &         \\
            &     &        & \ddots & a_{n-1} & b_{n-1} \\
            &     &        &        & c_{n-1} & a_n
    \end{vmatrix}
\]
按最后一列展开
\[
    D_n=a_nD_{n-1}-b_{n-1}c_{n-1}
    D_{n-2}\left(n\geqslant 2\right)
    ,D_0=1,D_1=a_1
\]
令$b_i=1,c_j=-1\left(
    1\leqslant i,j\leqslant n-1
    \right)$得到的行列式与连分数相关,记
\[
    \left(a_1a_2\cdots a_n\right)=
    \begin{vmatrix}
        a_1 & 1   &        &        &         &     \\
        -1  & a_2 & 1      &        &         &     \\
            & -1  & a_3    & \ddots &         &     \\
            &     & \ddots & \ddots & \ddots  &     \\
            &     &        & \ddots & a_{n-1} & 1   \\
            &     &        &        & -1      & a_n
    \end{vmatrix}
\]
易证
\[
    a_1+\cfrac{1}{a_2+\cfrac{1}{a_3+\cfrac{1}{\cdots+\cfrac{1}{a_{n-1}+\cfrac{1}{a_n}}}}}
    =
    \frac{\left(a_1a_2
        \cdots a_n\right)}{\left(
        a_2a_3\cdots a_n\right)}
\]
进一步令$a_k=1\left(
    1\leqslant k\leqslant n
    \right)$得到斐波拉契数列.
\[
    F\left(n\right) = F\left(n-1\right)+F\left(n-2\right)
\]
}
\subsection{行列式函数的导数}
\exa{}{}{
    设$f_{ij}\left(x\right)$是可导函数,记
    \[
        F\left(x\right)=
        \begin{vmatrix}
            f_{11}\left(x\right) & f_{12}\left(x\right) & \cdots & f_{1n}\left(x\right) \\
            f_{21}\left(x\right) & f_{22}\left(x\right) & \cdots & f_{2n}\left(x\right) \\
            \vdots               & \vdots               &        & \vdots               \\
            f_{n1}\left(x\right) & f_{n2}\left(x\right) & \cdots & f_{nn}\left(x\right)
        \end{vmatrix}
    \]
    证明:
    \[
        \frac{\rmd}{\rmd x}F\left(x\right)
        =
        \sum_{i=1}^{n}F_i\left(x\right)
    \]
    其中
    \[
        F_i\left(x\right)
        =
        \begin{vmatrix}
            f_{11}\left(x\right) & f_{12}\left(x\right) & \cdots & \dfrac{\rmd}{\rmd x}f_{1i}\left(x\right) & \cdots & f_{1n}\left(x\right) \\[1em]
            f_{21}\left(x\right) & f_{22}\left(x\right) & \cdots & \dfrac{\rmd}{\rmd x}f_{2i}\left(x\right) & \cdots & f_{2n}\left(x\right) \\
            \vdots               & \vdots               &        & \vdots                                   &        & \vdots               \\
            f_{n1}\left(x\right) & f_{n2}\left(x\right) & \cdots & \dfrac{\rmd}{\rmd x}f_{ni}\left(x\right) & \cdots & f_{nn}\left(x\right)
        \end{vmatrix}
    \]\begin{proof}
        考虑数学归纳法,对阶数$n$归纳,$n=1$时显然成立.设结论对$n-1$阶行列式成立,将行列式展开
        \[
            F\left(x\right)
            =
            f_{11}\left(x\right)
            A_{11}\left(x\right)
            +f_{21}\left(x\right)A_{21}\left(x\right)+
            \cdots+f_{n1}\left(x\right)A_{n1}\left(x\right)
        \]
        其中$A_{i1}$是$f_{i1}\left(x\right)$的代数余子式,两端求导并记$A_{ij}^k\left(x\right)$是对$A_{ij}\left(x\right)$第$k$列元素求导后得到的行列式,则
        \begin{align*}
            \frac{\rmd }{\rmd x}F\left(x\right)
             & =\frac{\rmd }{\rmd x}\left(
            \sum_{i=1}^{n}f_{i1}\left(x\right)A_{i1}\left(x\right)
            \right)
            =\sum_{i=1}^{n}f'_{i1}\left(x\right)A_{i1}\left(x\right)
            +\sum_{i=1}^{n}f_{i1}\left(x\right)A'_{i1}\left(x\right) \\
             & =F_1\left(x\right)+\sum_{i=1}^{n}f_{i1}\left(x\right)
            \left(\sum_{k=1}^{n-1}A^k_{i1}\left(x
            \right)\right)=F_1\left(x\right)+
            \sum_{k=1}^{n-1}\sum_{i=1}^{n}
            f_{i1}\left(x\right)A^k_{i1}\left(x\right)
        \end{align*}
        又$\displaystyle
            \sum_{i=1}^{n}f_{i1}\left(x\right)A_{i1}^k\left(x
            \right)=F_{k+1}\left(x\right)\left(
            1\leqslant k\leqslant n-1
            \right)$,故
        \[
            \frac{\rmd }{\rmd x}F\left(x\right)
            =
            F_{1}\left(x\right)+F_2\left(x
            \right)+\cdots+F_n\left(x\right)
            \qedhere
        \]
        利用组合定义也可证明:
        \[
            F\left(x\right)=
            \sum_{\left(k_1,k_2,\cdots,k_n\right)\in
                S_n}\left(-1\right)^{N\left(k_1,k_2,\cdots,k_n\right)}
            f_{k_11}\left(x\right)f_{k_22}\left(x\right)\cdots
            f_{k_nn}\left(x\right).
        \]
    \end{proof}
}
\subsection{含有三角函数的行列式}
\exa{}{}{
    求
    \[
        \left|\bm{A}\right|
        =
        \begin{vmatrix}
            1      & \cos \theta_1 & \cos2\theta_1 & \cdots & \cos\left(n-1\right)\theta_1
            \\
            1      & \cos \theta_2 & \cos2\theta_2 & \cdots & \cos\left(n-1\right)\theta_2
            \\
            \vdots & \vdots        & \vdots        &        & \vdots                       \\
            1      & \cos \theta_n & \cos2\theta_n & \cdots & \cos\left(n-1\right)\theta_n
        \end{vmatrix}
    \]
    \begin{solution}
        由De Moivre公式
        \[
            \cos k\theta +\rmi\sin k\theta =
            \left(\cos \theta + \rmi\sin \theta\right)^k
        \]及二项式定理并作三角恒等变换得
        \begin{align*}
            \left|\bm{A}\right|
             & =
            2^{\frac{1}{2}\left(n-1\right)\left(n-2\right)}
            \begin{vmatrix}
                1      & \cos \theta_1 & \cos^2\theta_1 & \cdots & \cos^{n-1}\theta_1
                \\
                1      & \cos \theta_2 & \cos^2\theta_2 & \cdots & \cos^{n-1}\theta_2
                \\
                \vdots & \vdots        & \vdots         &        & \vdots             \\
                1      & \cos \theta_n & \cos^2\theta_n & \cdots & \cos^{n-1}\theta_n
            \end{vmatrix} \\
             & =2^{\frac{1}{2}\left(n-1\right)\left(n-2\right)}
            \prod_{1\leqslant i<j
                \leqslant n}\left(\cos \theta_j-\cos \theta_i\right)
        \end{align*}
    \end{solution}
    同时利用和差化积容易证明
    \begin{align*}
          & \begin{vmatrix}
                \sin \theta_1 & \sin 2\theta_1 & \cdots & \sin n\theta_1 \\
                \sin \theta_2 & \sin 2\theta_2 & \cdots & \sin n\theta_2 \\
                \vdots        & \vdots         &        & \vdots         \\
                \sin \theta_n & \sin 2\theta_n & \cdots & \sin n\theta_n
            \end{vmatrix} \\
        = &
        2^{\frac{1}{2}
                \left(n-1\right)
                \left(n-2\right)}
        \prod_{i=1}^{n}\sin \theta_i \cdot
        \prod_{1\leqslant i<j
            \leqslant n}\left(\cos \theta_j-\cos \theta_i\right)
    \end{align*}
}
\subsection{其他}
\exa{}{}{
    计算
    \[
        D_n=\begin{vmatrix}
            x_1    & y      & y      & \cdots & y       & y      \\
            z      & x_2    & y      & \cdots & y       & y      \\
            z      & z      & x_3    & \cdots & y       & y      \\
            \vdots & \vdots & \vdots &        & \vdots  & \vdots \\\
            z      & z      & z      & \cdots & x_{n-1} & y      \\
            z      & z      & z      & \cdots & z       & x_n
        \end{vmatrix}
    \]
    \begin{solution}
        将第$n$列拆分得递推
        \[
            D_n=\left(x_n-y\right)D_{n-1}+y\prod_{i=1}^{n-1}\left(x_i-z\right)
        \]
        其转置同理有
        \[
            D_n=\left(x_n-z\right)D_{n-1}+z\prod_{i=1}^{n-1}\left(x_i-y\right)
        \]
        故
        $y \neq z$时
        \[
            D_n=
            \frac{1}{z-y}\left[
                z\prod_{i=1}
                ^{n}\left(x_i-y\right)-y
                \prod_{i=1}^{n}\left(x_i-z\right)
                \right]
        \]
        $y=z$时
        \[
            D_n=
            \prod_{i=1}^{n}\left(x_i-y\right)+y\sum_{i=1}^{n}\prod_{i\neq j}
            \left(x_j-y\right)
        \]
    \end{solution}
}