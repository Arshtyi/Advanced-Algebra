% ======================================================================
% LaTeX Macros for Advanced Algebra
% ======================================================================

% ======================================================================
% TEXT FORMATTING AND GENERAL COMMANDS
% ======================================================================

% ----- Text formatting commands -----
\newcommand{\dashline}{\text{---}}  % Dash for text
\newcommand{\st}{\textup{\bfseries{s.t.}}}  % "such that" abbreviation in bold

% ======================================================================
% MATHEMATICAL NOTATION
% ======================================================================

% ----- Mathematical notation with roman font -----
\newcommand{\rmi}{\ensuremath{\mathrm{i}}}  % Roman i (imaginary unit)
\newcommand{\rme}{\ensuremath{\mathrm{e}}}  % Roman e (base of natural logarithm)
\newcommand{\rmd}{\ensuremath{\mathrm{d}}}  % Roman d (differential operator)
\newcommand{\rmr}{\ensuremath{\mathrm{r}}}  % Roman r

% ----- Combinatorial notation -----
\newcommand{\bmn}[2]{\ensuremath{\mathrm{C}^{#1}_{#2}}}  % Combination notation C^n_m

% ======================================================================
% MATHEMATICAL SETS AND FONT STYLES
% ======================================================================

% ----- Mathematical sets using blackboard bold -----
\newcommand{\bb}[1]{\ensuremath{\mathbb{#1}}}  % Generic blackboard bold for any set

% ----- Number sets -----
\newcommand{\bbn}{\ensuremath{\mathbb{N}}}  % Natural numbers set
\newcommand{\bbz}{\ensuremath{\mathbb{Z}}}  % Integer numbers set
\newcommand{\bbq}{\ensuremath{\mathbb{Q}}}  % Rational numbers set
\newcommand{\bbr}{\ensuremath{\mathbb{R}}}  % Real numbers set
\newcommand{\bbc}{\ensuremath{\mathbb{C}}}  % Complex numbers set

% ----- Field and other mathematical sets -----
\newcommand{\bbf}{\ensuremath{\mathbb{F}}}  % General field F
\newcommand{\bbk}{\ensuremath{\mathbb{K}}}  % Field K

% ----- Calligraphic font for mathematical objects -----
\newcommand{\call}{\ensuremath{\mathcal{L}}}  % Calligraphic L (often used for linear spaces)

% ======================================================================
% MATHEMATICAL OPERATORS
% ======================================================================

% ----- Linear algebra operators -----
\DeclareMathOperator{\Image}{Im\,}   % Image of a mapping
\DeclareMathOperator{\Ker}{Ker\,}    % Kernel of a mapping
\DeclareMathOperator{\Null}{Null\,}  % Null space of a mapping
\DeclareMathOperator{\Span}{Span\,}  % Span of a set of vectors
\DeclareMathOperator{\rank}{rank\,}  % Rank of a matrix

% ----- Matrix operators -----
\DeclareMathOperator{\diag}{diag\,}  % Diagonal matrix operator
\DeclareMathOperator{\tr}{tr\,}      % Trace operator (lowercase)
\DeclareMathOperator{\Tr}{Tr\,}      % Trace operator (uppercase)

% ----- Mapping symbols -----
\newcommand{\longmapsfrom}{\ensuremath{\mathrel{\longleftarrow\!\shortmid}}}  % Long maps from symbol (reverse of \longmapsto)

% ----- Other mathematical operators -----
\DeclareMathOperator{\lcm}{lcm\,}    % Least common multiple operator